\documentclass[11pt, a4paper, oneside, draft]{book}

\usepackage{palatino, url}
\usepackage[ngermanb]{babel}
\usepackage[utf8]{inputenc}
\setlength{\parindent}{0cm}
\usepackage{textcomp}	% für \textmu also das Mü- bzw. Mikro-Symbol
\usepackage{makeidx}	% für ein "schöneres" Stichwortverzeichnis
\makeindex				% erzeugt das Stichwortvereichnis, blendet es jedoch nicht hier ein. Dafür am Ende den entspr. Code verwenden.

%
% \index{Eintrag}
% \index{Eintrag!Unterbegriff}
% erstellt eine Referenz auf eine Stelle im Text. Allerdings wird der Text in geschw. Klammern
% später nicht hier angezeigt, sondern das ist das Stichwort, das im Stichwortverzeichnis aufgeführt wird.
%
% \printindex 
% fügt das Stichwortverzeichnis an der Stelle ins Dokument ein, wo dieser Code verwendet wird.

\begin{document}

%%% Title declaration begin %%%

\title{Einführung in die maschinennahe, imperative, funktionale, relationale und objektorientierte Programmierung\\-\\EMIFROP 0.26}
\author{Markus Alpers\\B.Sc. und Ausbilder f. Industriekaufleute}
\date{\today}

\maketitle

%%% Title declaration end %%%

\tableofcontents

\subsection*{Hinweis bezüglich diskriminierender Formulierungen}

In diesem Text wurde darauf geachtet Formulierungen zu vermeiden, die diskriminierend verstanden werden können. Im Sinne der Lesbarkeit wurden dabei Formulierungen wie "`Informatiker und Informatikerinnen"'\\durch "`InformatikerInnen"' (mit großem i) ersetzt. An anderen Stellen habe ich Formen wie eine/einer durch eineR zusammengefasst. Hier berufe ich mich auf den Artikel "`Sprache und Ungleichheit"' der Bundeszentrale für politische Bildung, kurz BpB, vom 16. April 2014, insbesondere auf den Absatz "`Zum Umgang mit diskriminierender Sprache"', online abrufbar unter:\\

\url{http://www.bpb.de/apuz/130411/sprache-und-ungleich}\\
\url{heit?p=all}\\

Sollten Sie dennoch Formulierungen entdecken, die diesem Anspruch nicht entsprechen, möchte ich Sie bitten, mir eine entsprechende Nachricht zu senden, denn es ist mir wichtig, Ihnen mit diesem Buch eine wertvolle Unterstützung beim Start in die faszinierende Welt der Informatik zu bieten. Das sollte nicht durch verletzte Gefühle in Folge missverständlicher Formulierungen torpediert werden. \\

Sie erreichen mich unter \url{markus.alpers@haw-hamburg.de}. 

\subsection*{Hinweis zur Lizenz}

Dieses Buch wird in Teilen unter der Lizenz \emph{CC BY-SA 3.0 DE} veröffentlicht. Das bedeutet, dass Sie die entsprechenden Teile z.B. kopieren dürfen, so lange der Name des Autors erhalten bleibt. Sie dürfen diese auch in eigenen Werken weiterverwenden, ohne dafür z.B. eine Lizenzgebühr zahlen zu müssen. Dennoch müssen Sie auch hier bestimmte Bedingungen einhalten. Eine davon besteht darin, dass eine solche Veröffentlichung ebenfalls unter dieser Lizenz erfolgen muss. Sinn und Zweck solcher Lizenzen besteht darin, dass geistiges Eigentum frei sein und bleiben soll, wenn derjenige, der es erschaffen hat das wünscht. Und es ist mein Wunsch, dass so viele Menschen wie möglich von den Erklärungen in diesem Text profitieren.\\

Der vollständige Wortlaut der Lizenz ist auf folgender Seite nachzulesen. Dort erfahren Sie dann auch, welche Bedingungen einzuhalten sind:\\

\url{https://creativecommons.org/licenses/by-sa/3.0/de/}\\

Alle Teile des Buches, die ich unter der Lizenz \emph{CC BY-SA 3.0 DE}\\
veröffentliche enthalten am Anfang diesen Abschnitt "`Hinweise zur Lizenz"´. Wenn Sie einen Teil finden, in dem diese "`Hinweise zur Lizenz"´ nicht zu finden ist, dann dürfen Sie für den persönlichen Gebrauch dennoch Kopien davon anfertigen und Sie dürfen diese Kopien außerhalb von kommerziellen Projekten frei verwenden.\\

\subsection*{Hinweis zur Verwendbarkeit in wissenschaftlichen Arbeiten}

Bitte beachten Sie dabei aber, dass die Verwendung dieses Textes im Rahmen wissenschaftlicher Publikationen zurzeit aus anderen Gründen problematisch ist:

\begin{itemize}
	\item Wie viele andere Quellen, die frei im Internet verfügbar sind, wurde auch dieser Text bislang nicht durch einen nachweislich entsprechend qualifizierten Lektor verifiziert. Damit genügen Zitate aus diesem Band streng genommen noch nicht den Ansprüchen wissenschaftlicher Arbeiten.
	\item Weiterhin fehlen in diesem Buch häufig die für eine wissenschaftliche Arbeit nötigen Quellenangaben. Wann immer Sie in einer wissenschaftlichen Arbeit eine Behauptung aufstellen, müssen Sie diese durch einen Beleg oder Beweis untermauern. Ein solcher Belege/Beweis kann zum einen eine andere wissenschaftliche Arbeit sind, in der bewiesen wurde, dass die Behauptung den Tatsachen entspricht oder es muss ein eigenständiger Beweis für die Aussage sein.
\end{itemize}

Bitte beachten Sie außerdem, dass dieses Buch eine Konvention nutzt, die in wissenschaftlichen Arbeiten verpönt ist: Wenn in einer wissenschaftlichen Arbeit ein Begriff hervorgehoben wird, dann wird dazu kursive\\
Schrift verwendet. In diesem Buch verwende ich dagegen Fettdruck, da es vielen Menschen schwer fällt, einen kursiv gedruckten Begriff schnell zu finden und ich mir wünsche, dass Sie es möglichst effizient auch als Nachschlagewerk nutzen können.\\

\textbf{Dieses Buch ist} allerdings auch \textbf{nicht als wissenschaftliche Arbeit}, also als Ergebnis einer Forschung über Programmiersprachen, \textbf{sondern als Lehrwerk für den Einstieg in die Programmierung gedacht}. Es dient somit nicht dazu, Ihnen die theoretischen Grundlagen der Programmierung zu vermitteln, sondern dazu, ihnen eine fundierte Unterstützung beim Einstieg in die Programmierung anzubieten.
%
\subsection*{Zielgruppe und Vorwort}

Dieses Buch habe ich erstellt, um Studierenden der Studiengänge Media Systems (entspricht Medieninformatik an anderen Hochschulen) und Medientechnik an der HAW Hamburg den Einstieg ins Programmieren zu erleichtern. Deshalb finden sich hier teilweise Anmerkungen für die Studierenden der beiden Studiengänge, die aber in ähnlicher Form für Studierenden der Informatik und der Elektrotechnik gelten. Da es jedoch so formuliert ist, dass es für Studienanfänger ohne Programmiererfahrung geeignet ist, kann jede/r Studierende es gut nutzen, um sich in die Programmierung einzuarbeiten. Wenn die Version 1.0 abgeschlossen ist wird es eine \textbf{grundlegende Einführung} in die Konzepte (Paradigmen), der maschinennahen, der imperativen, der funktionalen, der relationalen und der objektorientierten Programmierung in den Ausprägungen protoypbasiert und klassenbasiert sein. Zusätzlich behandelt es den Einstieg in die Entwicklung verteilter Anwendungen.\\

Wie alle Lehrbücher für Studierende setzt es eines voraus: Wenn Sie es nutzen wollen, dann funktioniert das dann, und ausschließlich dann, wenn Sie zusätzlich zum Lesen zwei Dinge tun: Zum einen müssen Sie ständig kontrollieren, ob Sie jeden \textbf{neuen Begriff wirklich verstanden} haben und prüfen, wie er im Zusammenhang mit dem bisher Gelernten steht und zum anderen \textbf{müssen Sie tatsächlich programmieren}.\\

Was Sie hier nicht finden sind zum einen alle Varianten der Programmierung, die im Kern aus der Elektrotechnik entstanden sind oder für deren Verständnis Sie die Grundlagen kontinuierlicher Systeme beherrschen müssen. In der Informatik werden diese Bereich als \textbf{Technische Informatik}\index{Informatik!Technische Inf.} bezeichnet. Das schließt beispielsweise die Programmierung von Steuer- und Regelsystemen, also insbesondere \textbf{SPS}e\index{Systeme!SPS} und \textbf{FPGA}s\index{Systeme!FPGA} ein.\\

Damit sind wir auch schon bei einem ersten Missverständnis das zwischen InformatikerInnen einerseits und NaturwissenschaftlerInnen, IngenieurInnen und TechnikerInnen (kurz \textbf{INT-Akademiker}) existiert: Was in der Informatik als \textbf{Technische Informatik}\index{Informatik!Technische Inf.} bezeichnet wird ist alles, was die\\
übrigen drei als Informatik kennen. Diese gehen deshalb in aller Regel von der irrigen Vorstellung aus, das InformatikerInnen Programmierung meinen, wenn sie von \textbf{Praktischer Informatik}\index{Informatik!Praktische Inf.} reden. Tatsächlich haben beide (Praktische Informatik und Programmierung) kaum etwas miteinander zu tun. An dieser Stelle sei deshalb (vorrangig für Informatikstudierende) betont:\\

Dies ist eine \textbf{Einführung ins Programmieren, nicht in die Praktische Informatik}. Es wird zwar immer wieder Hinweise auf die Praktische und Theoretische Informatik geben, aber vorrangig ist und bleibt dies eine\\
Einführung ins Programmieren.\\

Die \textbf{systemnahe Programmierung}\index{Programmierung!systemnah} und die Programmierung von \textbf{Parallelprozessoren}\index{Programmierung!parallel} sowie die Implementierung von \textbf{Protokolle}n\index{Protokoll} für die Datenübertragung über Netzwerke entfallen ebenfalls. Dennoch werden Sie in diesem Buch zumindest einen Einblick in die Grundlagen der systemnahen Programmierung erhalten, da diese Systeme die Grundlage für alle Programmieransätze darstellen, die Sie hier kennen lernen können. Hier gilt dasselbe, was schon im letzten Absatz galt: INT-Akademiker kennen in aller Regel nur die systemnahe Programmierung und alle Konzepte, die sich direkt daraus ableiten lassen und die von InformatikerInnen mit dem Oberbegriff \textbf{Technische Informatik} bezeichnet werden. Es gibt jedoch auch Programmierkonzepte, die damit nicht mehr verständlich sind und für die es nötig ist, sich wesentlich grundlegender und abstrakter mit der Programmierung zu beschäftigen. Dazu kommen wir im zweiten Teil dieses Buches.\\

Fragen des \textbf{Software Engineering}\index{Software Engineering} werden zwar angerissen und es gibt Hinweise auf typische Missverständnisse, eine grundlegende Einführung ins Software Engineering kann dieser Band jedoch nicht ersetzen. Wie der Titel dieses Buches klar ausdrückt, geht es hier ums Programmieren. An den entsprechenden Stellen werden Sie aber entsprechende Hinweise auf Bücher und Themengebiete finden, damit Sie ggf. wissen, wonach Sie für weiterführendes Wissen suchen müssen. Ob sie sich nun zunächst in die Programmierung oder ins Software Engineering stürzen wollen, bleibt Ihnen überlassen; beides hat seine Vor- und Nachteile. In der Informatik\\
müssen Sie aber in jedem Fall beides durcharbeiten, um auch nur in\\
Ansätzen gute Software entwickeln zu können.\\

Der Grund ist simpel: Bei der \textbf{Programmierung}\index{Programmierung} geht es darum, Konzepte zur Lösung eines Programms in eine Sprache zu übersetzen, die ein Computer ausführen kann. Beim \textbf{Software Engineering}\index{Software Engineering} geht es dagegen darum, gute Konzepte zu entwickeln, die in Programmiersprachen übersetzt werden können. Wer nur eines von beidem beherrscht entwickelt häufig Programme, die niemandem nützen oder nützliche Konzepte, die niemand (in Form eines Computerprogramms) nutzen kann. Deshalb gibt es in jedem brauchbaren Kurs zur Programmierung Auszüge Teile, die eigentlich in den Bereich des Software Engineering gehören\footnote{Weshalb es nur wenige Kurse zur Programmierung gibt, die nach Ansicht dieses Autors brauchbar sind.}. Umgekehrt enthält jeder brauchbare Kurs zum Software Engineering Teile zur Programmierung\footnote{Weshalb auch hierfür aus Sicht dieses Autors nur wenig Brauchbares auf dem Markt ist.}.\\

Zusätzlich müssen Sie jedoch in jedem Fall noch die Grundlagen der Praktischen Informatik erlernen, um hochwertige Software zu erstellen. Diese können Sie im Bereich der Algorithmik (genauer \textbf{Algorithmen und Datenstrukturen}\index{Algorithmen und Datenstrukturen}, \textbf{Algorithmendesign}\index{Algorithmendesign} sowie \textbf{Algorithmik}\index{Algorithmik}) erlernen.\\

Aufgrund der häufigen \textbf{Änderungen bei aktueller Software} kann dieses Buch nur beschränkt Unterstützung bei Installations- und Konfigurationsfragen bieten. Hier bleibt zu hoffen, dass die Entwickler der einzelnen Sprache bzw. zusätzlicher Software eine ausreichende Dokumentation auf Ihrer Webpage bereitstellen.\\

Nochmal in anderen Worten: Ein häufiges Missverständnis besteht darin, dass Programmierung und Informatik bzw. Programmierung und Praktische Informatik miteinander verwechselt werden. Denn \textbf{Programmierung}\index{Programmierung} ist lediglich die Umsetzung einer Idee mit Hilfe einer Sprache, die einem Computer befiehlt, \textbf{was er tun soll}. Ob sie auch festlegt, \textbf{wie er das tun soll} ist eine ganz andere Frage. Einführungen in die Programmierung, die hier nicht deutlich werden sind der Grund für eine Vielzahl von Missverständnissen rund um die Programmierung.\\

Um überhaupt zu programmieren, müssen Sie also lediglich wissen, wie die Befehle und Befehlsstrukturen einer Sprache aussehen. Im Kern ist das also nichts anderes als das Erlernen einer gesprochenen Sprache. Doch so wie es selbst für das Erlernen nahe verwandter gesprochener Sprachen eben nicht ausreicht, nur die Übersetzung einzelner Wörter zu erlernen, genügt es für die kompetente Beherrschung von Programmiersprachen\\
nicht, sich nur grundsätzlich damit beschäftigt zu haben: So wie Sie eine gesprochene Sprache tatsächlich in Gesprächen benutzen müssen, um sie zu erlernen, müssen Sie eine Programmiersprache benutzen, indem Sie eine Vielzahl an Programmen damit entwickeln.\\

\textbf{Wichtig:}\\
Fähige (Medien-)InformatikerInnen sind nicht automatisch guten ProgrammiererInnen. Wenn Sie verstanden haben, was der Unterschied zwischen Informatik und Programmieren ist, dann wird es sie wundern, dass es\\
überhaupt Menschen gibt, die diese Aussage bezweifeln.\\

Die \textbf{Informatik}\index{Informatik} dagegen setzt sich mit der Frage auseinander, wie und ob eine bestimmte Idee besonders elegant und effizient umgesetzt werden kann. \textbf{Ob für die Umsetzung der Idee ein Computer nötig ist, ist zweitrangig.} Aber da Computer die Stärke haben, dass Sie langweilige Aufgaben mit einer für uns unfassbaren Geschwindigkeit ausführen, sind Sie das Werkzeug Nummer 1 für die Informatik. Zumindest ist nach Ansicht dieses Autors die Durchführung von 4 Milliarden Additionen pro Sekunde unfassbar schnell. Zum Vergleich: Würden alle Menschen auf dieser Welt gleichzeitig eine Addition zweier Zahlen mit bis zu 20 Stellen durchführen und für die Berechnung sowie das Aufschreiben nur zwei Sekunden brauchen, dann wären sie alle gemeinsam genauso schnell wie ein einzelner Prozessor, der in einem handelsüblichen Computer steckt.\\

Hier ein Beispiel, mit dem sich Informatikstudierende gegen Ende des Bachelorstudiums auseinander setzen: Sie fahren in den Skiurlaub und wollen mal das Skifahren ausprobieren. Nun könnten Sie die Skier kaufen oder mieten. Da Sie ja nicht wissen, ob Ihnen Skifahren wirklich Spaß macht, wäre es unsinnig, gleich am ersten Tag das Geld für den Kauf auszugeben. Aber auch am zweiten Tag wäre es nicht unbedingt sinnvoll, denn wer weiß, ob Sie am dritten Tag noch Lust dazu haben. Die Frage lautet also: Wann macht es Sinn, die Skier zu kaufen? Das ist ein Beispiel für einen Bereich, der in der Informatik als \textbf{online-Algorithmen}\index{Algorithmus!online} bezeichnet wird. Der zugehörige Bereich der Praktischen Informatik heißt \textbf{Algorithmendesign}\index{Algorithmendesign}.\\

Online-Algorithmen dienen beispielsweise dazu, die Verwaltung des Speichers einer Festplatte zu organisieren oder um die Mitarbeiter für Kassen in einem Supermarkt einzuplanen. Bei online-Algorithmen geht es immer um die Frage: Wie bereite ich mich am besten auf eine Situation vor, von der ich noch nicht genau weiß, wie sie aussehen wird? Und wie Sie sehen hat das zunächst einmal nichts mit Programmieren zu tun.\\

Sie möchten ein Beispiel, bei dem es um Programmierung und online-\\
Algorithmen geht? Dann haben Sie leider noch nicht verstanden, was der Unterschied zwischen Praktischer Informatik und Programmieren ist. Also weiter mit den online-Algorithmen: Denken Sie beispielsweise daran, was bei ebay kurz vor Ende einer Versteigerung passiert. Oder wie wäre es mit einem online-Händler wie amazon, kurz vor Weihnachten: Wann wie viele Menschen versuchen werden, auf eine einzelne Seite zuzugreifen, kann niemand wissen. Aber mit fähigen Informatikern kann jeder sich darauf so gut wie möglich vorbereiten. Hier sind wir übrigens auch schon ganz nahe an einem aktuellen Forschungsgebiet der (Medien-)Informatik: Bei \textbf{Big Data}\index{Big Data} handelt es sich um einen Bereich, in dem aus gigantischen Datenmengen versucht wird, Gruppen von Personen mit gemeinsamen Eigenschaften herauszufiltern und dann Prognosen über deren zukünftiges Verhalten zu schließen. Hier sind wir ganz nahe am Gebiet des \textbf{Datenschutzes}\index{Datenschutzes} mit ganz konkreten Auswirkungen auf das alltägliche Leben. Denn Big Data führt zum Teil zu absurden Abläufen: Wenn Sie beim Ausfüllen eines Kreditantrages online mehrfach Einträge ändern, dann wird alleine deshalb der Zinssatz erhöht oder der Antrag abgelehnt. Die Begründung stammt direkt aus dem Bereich angewandten Big Data und lautet so: Menschen mit niedrigem Bildungsniveau vertippen sich häufiger als Menschen mit hohem Bildungsniveau. Menschen mit hohem Bildungsniveau haben in aller Regel eine höheres Einkommen, längere Arbeitsverhältnisse und eine geringere Wahrscheinlichkeit, arbeitslos zu werden. Das lässt sich verkürzen zu: Menschen mit höherem Bildungsstand sind in aller Regel zuverlässiger bei der Rückzahlung von Krediten. Wenn wir jetzt noch die erste Aussage hinzunehmen, lautet die Schlussfolgerung nach den Prinzipien des Big Data: Wer sich öfter vertippt wird häufiger Probleme haben, einen Kredit zurückzuzahlen. Also wird der Zinssatz erhöht oder der Kredit gleich ganz verweigert. Und nein, das ist kein Scherz, sondern findet so Anwendung bei online-Finanzdienstleistern.\\

Wieder zurück zu den online-Algorithmen: In der BWL wird dieser Teil der Informatik als \textbf{Logistik}\index{Logistik} bezeichnet, wobei der Bereich der online-\\
Algorithmen deutlich mehr umfasst als nur Logistik. BWLern ist dabei in aller Regel leider nicht bewusst, dass es sich hier um einen Bereich handelt, der eine Kernkompetenz der Informatik ist. Das führt dazu, dass in der Forschung zur BWL teilweise Forschungen betrieben werden, um Probleme zu lösen, für die die Informatik längst eine Lösung bereit hält. Denken Sie bitte dennoch nicht in Kategorien wie Schuld: Es ist ein grundsätzliches Problem, dass zu viele Akademiker nur in den Kategorien der eigenen Disziplin denken und kaum den Austausch mit anderen Disziplinen suchen. Das ändert sich zwar in einigen wenigen Fällen, doch häufig kommt es dabei nicht zu einem vollwertigen Austausch, sondern es wird versucht, die jeweils andere Disziplin der eigenen anzupassen. Ein erster Schritt, um das nicht zu tun besteht darin, sich darüber auszutauschen, was Begriffe der jeweiligen Disziplin bedeuten. Ein "`gutes"´ Beispiel haben bereits kennen gelernt: Was INT-Akademiker als Informatik bezeichnen ist für Informatiker in aller Regel nur der Teilbereich der \textbf{Technischen Informatik}\index{Informatik!Technische Inf.}. So lange solche Missverständnisse nicht ausgeräumt sind und AkademikerInnen unterschiedlicher Disziplinen die Kompetenz des/der jeweils anderen nicht anerkennen, ist ein echter Austausch nicht möglich. Und dann sind auch umfassende Projekte mit Beteiligung unterschiedlicher wissenschaftlicher Disziplinen nicht möglich.\\
 
Sie studieren \textbf{Medientechnik}\index{Medientechnik} und fragen sich, was das mit Ihrem Studium zu tun hat? Ganz einfach: Auch in Ihrem Bereich werden Computer eingesetzt und Sie werden nicht immer eine fertige Lösung in Form eines Computerprogramms vorfinden. Also müssen Sie im Stande sein, einfache Probleme mit einem Computer selbst zu lösen. Und wenn die Probleme zu komplex werden, dann müssen Sie im Stande sein, InformatikerInnen zu erklären, worin das Problem besteht, denn sonst können die keine Lösung für Sie entwickeln. Wie wäre es beispielsweise mit einem Programm, mit dem Sie Ihre Schaltskizzen für das E-Technik-Labor erstellen können, die Sie außerdem online speichern und abgeben können, ohne sie als Mailanhang verschicken zu müssen? Wenn Sie die Veranstaltung "`Programmieren 1"' (entspricht Teil I dieses Buches) erfolgreich absolvieren, dann werden Sie im Stande sein, solche Programme selbst zu entwickeln.\\

Sie studieren \textbf{Media Systems}\index{Media Systems} und fragen sich, warum Sie einen Kurs mit dem Titel "`Einführung ins Programmieren"' (kurz PRG) belegen sollen,\\
wenn Sie doch schon die Veranstaltung "`Programmieren 1"' (kurz P1) belegen? In der Veranstaltung P1 lernen Sie die Entwicklung von Programmen mit einer imperativen und klassenbasierten objektorientierten Programmiersprache. In PRG konzentrieren wir uns dagegen auf einen anderen Bereich: Die Entwicklung verteilter Anwendungen. Wann immer Sie eine App nutzen, nutzen Sie eine verteilte Anwendung. Wann immer Sie ein Programm nutzen, das das Internet oder ein anderes Netzwerk voraussetzt, nutzen Sie eine verteilte Anwendung. Normalerweise ist das Stoff für ein Masterstudium, aber ich habe diese Veranstaltung so konzipiert, dass sie für Einsteiger ohne Vorkenntnisse geeignet ist. Denn so bekommen Sie einen Eindruck, wie all die verschiedenen Veranstaltungen Ihres Studiums ein sinnvolles ganzes ergeben.\\

Hier nochmals meine Bitte: Wenn Sie in diesem Skript Fehler finden (was bei einem Dokument dieser Länge unausweichlich der Fall ist), Sie weitergehende Fragen haben oder Ergänzungsvorschläge, dann senden Sie diese bitte an mich: \url{markus.alpers@haw-hamburg.de}

\subsection*{Work in Progress}

Der Begriff Work in Progress bedeutet, dass eine Arbeit noch nicht abgeschlossen ist und somit in Teilen unvollständig und in anderen Teilen unnötig detailliert ist. Das gilt zurzeit für dieses Buch: Zum einen habe ich verschiedene Passagen noch nicht abgeschlossen, zum anderen habe ich verschiedene Texte, die ich ursprünglich für unterschiedliche Kurse erstellt hatte hier zusammengefasst. Da diese Zusammenführung noch nicht abgeschlossen ist, wird es Passagen geben, in denen Sie nahezu identische Aussagen und Erklärungen erneut finden werden. Auch wenn Ihnen solche Passagen auffallen, schicken Sie mir bitte eine E-Mail.


\part{Einfache Einführung in die imperative Programmierung}

%\chapter[Das ist Programmieren (wirklich)]{Typische Irrtümer darüber, was Programmieren ist.}

Studierende im Studiengang \textbf{Media Systems}\index{Media Systems} besuchen unter anderem die Veranstaltungen Programmieren 1 und 2 sowie Informatik 3. Ziel dieser Veranstaltungen ist, dass Sie die Grundlagen zweier Arten der Programmierung erlernen. Diese werden als imperative bzw. prozdurale und klassenbasierter objektorientierte Programmierung bezeichnet.\\

Studierende der \textbf{Medientechnik}\index{Medientechnik} besuchen ebenfalls zwei Veranstaltungen mit dem Namen Programmieren 1 und 2. Die Inhalte entsprechen einer einfachen Zusammenfassung dessen, was Studierende in Media Systems in den Veranstaltungen "`Einführung ins Programmieren"´, "`Software Engineering"´ und "`Relationale Datenbanken"´ erlernen. Sie bekommen so einen kurzen Einblick in die Bereiche, mit denen sie immer wieder zu tun haben werden, die aber eigentlich Kernbereiche der Informatik sind. Der Grund dafür ist recht simpel: Sobald elektrotechnische Systeme (also der Kernbereich der Medientechnik) zu komplex werden, um sie ohne\\
zusätzliche Strukturierung zu nutzen, kommen wir in einen von zwei Bereichen: Nachrichtentechnik und Informatik. Beide können ohne\\
Verständnis der Elektrotechnik nur zum Teil verstanden werden, aber das gleiche gilt auch umgekehrt.\\

Aber bevor wir uns ansehen, was diese beiden Arten der Programmierung ausmacht, wo Schnittpunkte und wo Unterschiede vorliegen, sollten wir eine Frage klären: Was verstehen wir eigentlich unter dem Begriff "`Programmieren"´? Gerade diejenigen, die schon programmiert haben, sollten diesen Abschnitt lesen, denn Sie werden denken, dass Ihnen dieser Begriff klar ist. Einzig diejenigen, die bereits imperativ und (!) deklarativ programmiert haben, werden wissen, worin der Unterschied liegt und können ihn überspringen. (Verwechseln Sie aber bitte nicht die Deklaration einer Variablen mit der deklarativen Programmierung. Beide haben soviel miteinander gemein wie Schweinezucht mit Flugzeugbau.)

\section{Das ist an diesem Buch anders}

Es gibt eine Vielzahl an Einführungen ins Programmieren. Die meisten davon gehören in eine von zwei Kategorien:\\

\begin{itemize}
	
	\item \textbf{Variante a} richtet sich an Studierende an Universitäten und ignoriert weitgehend die konkrete Programmierung in einer Sprache. Der Fokus liegt hier vorrangig auf Aspekten der \textbf{Algorithmik}\index{Algorithmik}. Diese sind zwar außerordentlich wichtig, um fähigeR InformatikerIn zu werden, aber ohne eine Einführung in die konkrete Programmierung in einzelnen Sprachen ist sie kaum verständlich. Daran scheitern dann auch viele Studienanfänger. Und von denen, die nicht daran scheitern versteht nur ein Bruchteil, was Algorithmik ist. Am Ende gibts dann haufenweise Informatikabsolventen von Universitäten, die zwar\\
	ganz passabel programmieren können, deren Programme aber letztlich sehr schlecht strukturiert sind.\\
	
	\item \textbf{Variante b} behandelt dagegen nur die konkrete Programmierung in einer Sprache und in einer bestimmten Version, ohne dabei auf die allgemeinen Grundlagen einzugehen. Wer eine solche Einführung\\
	nutzt hat in aller Regel ein derart mangelhaftes Verständnis der\\
	grundlegenden Prinzipien, auf deren Basis die jeweilige Sprache entwickelt wurde, dass er/sie selbst mit großem Aufwand nicht im Stande ist, eine weitere Programmiersprache so zu erlernen, dass er/sie diese wirklich nutzen könnte. Ständig heißt es dann "`warum macht der das denn nicht,"´ und es wird über die vermeintlich schlechte andere Sprache geflucht. Dabei ist das Problem nicht die "`andere"´ Sprache, sondern die Tatsache, dass jemand mit dem Verständnis einer Programmiersprache versucht, eine andere Programmiersprache zu erlernen. Doch wenn diese andere Sprache alles genauso machen würde, wie die erste, dann wäre es komplett unsinnig, sie zu erlernen. (Medien-)informatikerInnen lernen deshalb vorrangig die Konzepte kennen, die in verschiedenen Programmiersprachen jeweils unterschiedlich eingesetzt werden.\\

\end{itemize}

Beide Ansätze ignorieren darüber hinaus, dass für viele Menschen sich mit der Programmierung beschäftigen wollen, das Innenleben von Rechnern unbekanntes Gebiet sind. Diese Einführung holt Leser dagegen an dem Punkt ab, an dem keine Vorkenntnisse nötig sind und führt sie kontinuierlich in das Themengebiet ein. Der erste Teil ist dabei so aufgebaut, dass ein Überblick über einige Möglichkeiten der Programmierung vermittelt werden. Erst wenn das geschafft ist, wenn also Leser ein Grundverständnis von verschiedenen Arten der Programmierung haben, beginnt mit dem zweiten Teil die eigentliche Einführung in die Grundlagen der Programmierung. Aber auch wenn Sie schon programmieren können (egal ob in HTML, Java, C oder welcher Sprache auch immer), sollten Sie Teil I des Buches durcharbeiten, weil hier bereits einige Konzepte eingeführt werden und anhand von Beispielen in einer oder mehreren Sprachen verdeutlicht werden.

\section{Zentrale Begriffe und Konzepte beim Programmieren}

Häufig werden die Begriffe Programmieren und Informatik in einen Topf geworfen, dabei haben Sie nicht wirklich viel gemeinsam. Damit Sie also wissen, was Ihnen dieses Buch im Rahmen eines Informatikstudiums bietet und was nicht, schauen wir uns einmal an, was Programmieren eigentlich ist und was Sie von Anfang an beachten sollten.

\subsection{Der Begriff des Programmierens}

Wenn Sie beispielsweise einen HDD-Rekorder programmieren, dann reden Sie zwar vom Programmieren,  gehen aber sicher nicht davon aus, dass Sie sich in einem Informatikstudium mit der Frage auseinander setzen, wie Sie einen solchen Rekorder programmieren können.\\

Interessanterweise können Sie diese Frage aber nach dem Besuch der Veranstaltung "`Informatik 3"  beantworten: Dort geht es um die Programmierung von Mikroprozessoren, also just der kleinen schwarzen Boxen, die seit Mitte der 80er Jahre praktisch jedes elektrische Gerät steuern. Na gut, die meisten Toaster noch nicht... Spätestens mit dem \textbf{IoT}, dem \textbf{Internet of Things} wird das aber kommen.\\

Aber was machen wir dann in Media Systems in Programmieren 1 und 2? Außerdem fehlt immer noch die Antwort auf die Frage, was Programmieren denn eigentlich ist. Von der Antwort auf die Frage, was das dann wiederum mit Informatik oder gar Medieninformatik zu tun hat, mal ganz zu schweigen.\\

Wenn wir (wie üblich) zunächst per deutscher Wikipedia suchen, dann erhielten wir am 27. April 2015 die Auskunft, dass es um das Erstellen von Computerprogrammen geht, was dabei wichtig ist und wer schon etwas darüber geschrieben hat. Aber die eigentliche Antwort auf die Frage, was wir tun, wenn wir programmieren, steht nicht dort.\\

Dabei ist das recht simpel: Wenn wir programmieren, dann teilen wir einem Computer schlicht mit, \textbf{dass er eine Reihe von Aufgaben erfüllen soll}. Und ja, damit ist auch das Drücken der Ruftaste auf einem Telefon eine Programmierung. Nochmal: Es geht darum, dass wir dem Rechner mitteilen, dass er etwas tun soll. Die Frage in welcher Form wir das tun ist davon vollkommen unabhängig und wird unter dem Oberbegriff des \textbf{Paradigma}s\index{Programmieren!Paradigma} geklärt.\\

\textbf{Kontrolle}

Sie sollten jetzt verstanden haben, dass wir den Begriff des Programmierens deutlich allgemeiner verwenden, als das üblicherweise von Programmierern getan wird. Wenn Sie denken, dass Programmieren beinhaltet, wie der Rechner Aufgaben ausführen soll, dann haben Sie eine zu beschränkte Vorstellung des Begriffs Programmieren.

\subsection[Paradigmen]{Programmierparadigmen – Wie Programme entwickelt werden können}

Sie wissen es jetzt bereits: Einen Computer zu programmieren bedeutet nicht, dass Sie ihm Schritt für Schritt erklärt, wie er eine Aufgabe lösen soll. Denn wenn wir über diese spezielle Art der Programmierung reden, dann nennen wir das \textbf{imperative Programmierung}\index{Programmierung!imperativ}: Hier erstellen Sie wie bei einem Kochrezept Zeile für Zeile eine Liste von Anweisungen, die beschreiben, wie der Rechner eine Aufgabe in Form einzelner Schritte lösen soll. Da das Programm aber nur aus den einzelnen Schritten besteht, ist später nicht mehr erkennbar, welche Aufgabe das Programm lösen soll.\\

Probleme tauchen hier immer dann auf, wenn ProgrammiererInnen ein Programm erstellt haben, das zu einem Ergebnis kommt, dieses Ergebnis aber nicht die gewünschte Aufgabe löst. Das liegt zum Teil an so subtilen und doch nicht trivialen Aspekten wie der Division einer Zahl durch eine andere Zahl mittels eines Computers.\\

Im Gegensatz zu dem, was Sie aus dem Deutschunterricht in der Schule als "`den Imperativ" kennen bedeutet imperative Programmierung also nicht nur, dem Computer Befehle zu erteilen, sondern auch ihm zu befehlen, \textbf{wie} er einen Befehl auszuführen hat.\\

Kommen wir damit zur obersten und unumstößlichen Regel bei der Programmierung: \label{rule001}\textbf{Nicht der Computer oder die Nutzer sind schuld, wenn etwas schief läuft, sondern ausschließlich die Entwickler.} Wenn Entwickler beispielsweise den Eindruck bei Käufern erzeugen, dass die Nutzung ganz simpel ist, dann ist das nicht die Schuld von Menschen, die sich auf diese Aussage verlassen. Na gut: Wenn Kunden mit Kommentaren kommen wie "`Das will ich nicht wissen," dann sind sie selbst schuld, aber auch nur dann... also leider fast immer...\\

Wenn Sie schon einmal programmiert haben, dann werden Sie jetzt wahrscheinlich einwenden, dass man doch nur imperativ programmieren kann. Und damit liegen Sie so falsch wie jemand, der denkt, dass \textbf{Programmieren}\index{Programmieren} und \textbf{Informatik}\index{Informatik} praktisch dasselbe wären, oder dass Programmieren und \textbf{Praktische Informatik}\index{Informatik!Praktische Inf.} dasselbe wären. Wie im Vorwort geschrieben haben beide kaum etwas gemeinsam, sondern das eine (Informatik) kann unter anderem dazu genutzt werden, um das andere (Programmieren) gut zu machen.\\

Mit der Informatik und dem Programmieren ist beispielsweise so, wie mit dem Energiesparen und dem Bau eines Hauses: Es ist möglich, ein energiesparendes Haus zu bauen, aber der Hausbau an sich hat mit Energieersparnis nichts zu tun. Und umgekehrt können Sie in wesentlich mehr Bereichen Energie sparen als nur beim Hausbau: Das eine (Energiesparen) hat etwas mit der Herangehensweise an eine Vielzahl von Bereichen zu tun, das andere (Hausbau) ist eine im Vergleich dazu nur in wenigen Bereichen einsetzbare Tätigkeit, für die Sie das erste aber sehr sinnvoll einsetzen können.\\

Wenn wir von imperativer Programmierung sprechen, dann fassen wir damit eine Reihe an Programmierparadigmen zusammen. Wenn Sie imperativ programmieren, dann wird das in bestimmten Fällen als \textbf{prozedurale Programmierung}\index{Programmierung!prozedural} oder auch als \textbf{strukturierte Programmierung}\index{Programmierung!strukturiert} bezeichnet. Die prozedurale Programmierung ist eine Methode, die dazu gedacht ist, um schlecht lesbaren Programmcode zu vermeiden. Meist ist mit imperativer Programmierung der Spezialfall der prozeduralen und der strukturierten Programmierung gemeint.\\

Bei der prozeduralen Programmierung zerlegen wir eine Aufgabe so lange in immer genauer definierte Anweisungen, bis wir eine Abfolge von Zeilen haben, die direkt einer Programmzeile einer Programmiersprache entspricht.\\

Bei der strukturierten Programmierung müssen wir außerdem bestimmte Vorgaben beachten, die verhindern, dass unser Programm unübersichtlich wird. So sind hier Sprünge innerhalb des Programms nur dann erlaubt, wenn wir dadurch einen anderen Programmteil überspringen. Ein\\
Rücksprung an eine beliebige frühere Stelle ist verboten. Es gibt zwar noch die sogenannten Schleifen und Rekursionen, doch die erlauben keinen beliebigen Sprung zurück zu irgen einem früheren Teil des Programms, sondern stellen nur die Möglichkeit zur Verfügung, Teile des Programms zu wiederholen, bevor das Programm weite Zeile für Zeile abgearbeitet wird. Dabei können wir allerdings durchaus Programmteile entwickeln, bei denen unter bestimmten Bedingungen andere Teile des Programms\\
übersprungen werden.\\

Leider wird häufig von der \textbf{objektorientierten Programmierung}\index{Programmierung!objektorientiert} gesprochen (auch diesem Autor passiert das immer wieder), dabei gibt es streng genommen keine objektorientierte Programmierung. Objektorientierung\\
ist strenggenommen ein Konzept des Software Engineering, das zwar in einigen Programmiersprachen direkt angewendet werden kann, aber in aller Regel handelt es sich bei dem, was "`objektorientierte"´ Sprachen anbieten nur um ein Konzept, das eher eine aktualisierte Form der strukturierten Programmierung ist: Umfangreiche Programmteile werden hier in sogenannte Module (bei Java beispielsweise als Klassen und Package bezeichnet) "`verpackt"´. Dadurch werden umfangreiche Programme übersichtlicher.\\

\textbf{Objektorientierung}\index{Objektorientierung} an sich geht einerseits weit darüber hinaus und hat andererseits mit der Lösung einer Aufgabe durch einen Computer eigentlich kaum etwas zu tun. Sie ist im Grunde ein Gegenentwurf zur Grundlage der imperativen Programmierung: Da bei dieser binäre Prozessoren mit Datenübertragungsleitungen und Speicher der konzeptionelle Ausgangspunkt sind, hat sie streckenweise starke Restriktionen, was die Umsetzung von Problemlösungen anbelangt. Die Objektorientierung ist also ein Entwurf, der die Softwareentwicklung von den strikten Restriktionen der imperativen Programmierung unabhängig macht. Sprachen, die tatsächlich die Objektorientierung integrieren sind deshalb für Entwickler mit Erfahrung in imperativer Programmierung kaum zu verstehen. Wenn wir uns der prototypbasierten Programmierung in JavaScript zuwenden, dann\\
werden Sie verstehen, warum das so ist.\\

Bevor wir zu einer anderen Art der Programmierung kommen, sollten noch einige Begriffe eingeführt werden: Ein \textbf{Algorithmus}\index{Algorithmus} ist eine Beschreibung dafür, \textbf{wie} ein Problem gelöst werden soll. Es ist allerdings noch kein \textbf{Computerprogramm}\index{Computerprogramm}. Wenn Sie basierend auf Algorithmen programmieren,\\
dann haben wir es immer (!) mit \textbf{imperativer Programmierung} \index{Programmierung!imperativ} zu tun. Wann immer also eine Einführung in die Programmierung mit dem Begriff des Algorithmus beginnt, handelt es sich nicht um eine allgemeine Einführung, sondern um eine Einführung in die imperative Programmierung. Ein Computerprogramm ist in diesen Fällen grundsätzlich die Umsetzung eines oder mehrerer Algorithmen in eine bestimmte Programmiersprache.\\

Manchmal wird in diesem Zusammenhang auch von \textbf{Pseudocode}\index{Pseudocode} gesprochen. Das ist ein Algorithmus, der so aufgeschrieben wurde, dass er einem (imperativen) Computerprogramm ähnelt. Deshalb ist die Lösung eines Problems in Pseudo-Code sehr praktisch: Angenommen, Sie suchen im Netz nach einer effiziente Lösung für ein Problem, über das Sie bei Ihrer aktuellen Programmieraufgabe stolpern. Wenn Sie die Aufgabe in einer bestimmte Programmiersprache lösen sollen und im Netz finden Sie Code für eine andere Programmiersprache, dann müssen Sie beide Sprachen beherrschen, um die Lösung in Ihre Sprache zu übersetzen. Ist die Lösung dagegen in Pseudocode gegeben, dann können Sie sie umsetzen, so lange Sie die eigene Programmiersprache grundlegend beherrschen.\\

Dann wäre da noch der Unterschied zwischen Hardware und Software:

\begin{itemize}
	\item \textbf{Hardware}\index{Hardware} ist die Gesamtheit aller Computerprogramme und sonstiger Bestandteile von Computern, die als "`anfassbare"´ Komponenten vorliegen,
	
	\item \textbf{Software}\index{Software} ist die Gesamtheit all der Computerprogramme, die ausschließlich in Form von Daten in Rechnersystemen unterwegs sind. 
	
	\item Richtig gelesen: \textbf{Hardware ist genauso sehr ein Programm bzw. Bestandteil von Programmen, wie die sogenannte Software.} Häufig werden diese Begriffe dagegen so erklärt, als wenn Hardware kein Programmteil wäre. Und das ist falsch; bis auf Dinge wie das Gehäuse eines Rechners dient praktisch die Gesamtheit der Komponenten aus denen ein Rechner zusammen gesetzt ist dazu, Daten zu speichern und zu verarbeiten. Die Speicherung und Verarbeitung eines Programms ist aber bereits ein Programmablauf. Und damit stellt auch die Hardware eine Sammlung von Programmen dar. Später werden wir uns über Server und Client oder auch über Backend und Frontend unterhalten. Genau wie die Unterteilung in Hardware und Software sind auch diese Unterteilungen für uns als (Medien-)\\
	InformatikerInnen vollkommen irrelevant.\\	
\end{itemize}

Hier sind wir auch direkt beim Zusammenhang zwischen Elektrotechnik, Nachrichtentechnik und Informatik: Alle drei beschäftigen sich zum\\überwiegenden Teil mit der Nutzung von Stromflüssen, um sinnvolle Aufgaben zu erfüllen. Allerdings übernehmen biologische Moleküle und Reaktionen zwischen diesen einen immer größeren Raum in allen drei Bereichen ein oder schaffen sogar vollständig neue Anwendungs- und Forschungsgebiete. In diesem Buch werden wir uns allerdings fast ausschließlich auf die Bereiche konzentrieren, in denen es um Anwendungen auf Basis fließender Ströme geht:\\

\begin{itemize}
	\item Die \textbf{Elektrotechnik}\index{Elektrotechnik} setzt einfache Schaltelemente ein, verbindet diese zu zum Teil hochkomplexen Schaltungen und findet vorrangig in der Peripherie von IT-Systemen Anwendungen.
	\item Die \textbf{Nachrichtentechnik}\index{Nachrichtentechnik} beschäftigt sich mit der Frage, wie beliebige Daten über verschiedene Medien und Distanzen oder Zeiträume transportiert werden können.
	\item Die \textbf{Informatik}\index{Informatik} beschäftigt sich dagegen mit der Frage, wie Daten zur Erzeugung von Daten genutzt werden können. Sie setzt die Nachrichtentechnik also zur Datenübertragung von Ort zu Ort und zur Speicherung von Daten ein. Die Elektrotechnik kommt hier insbesondere bei der Interaktion von Informatik-Systemen mit der Umgebung ein. 
	\item Wenn wir Informatik mit Nachrichtentechnik oder Elektrotechnik verbinden, landen wir direkt bei der \textbf{Technischen Informatik}\index{Informatik!Technische Inf.}.
\end{itemize}

Aber zurück zum eigentlichen Thema dieses Abschnitts und damit zu einer anderen Art der Programmierung: \\

Es gibt auch Programmiersprachen, in denen man die \textbf{Prämissen}\index{Prämisse} der Aufgabe beschreibt und den Rechner dann auffordert, eine mögliche Lösung zu nennen.  Eine Prämisse ist eine Voraussetzung für etwas bzw. bei der logischen Programmierung so etwas wie eine Bedingung, die in irgend einer Form Auswirkung darauf hat, welche möglichen Lösungen für unser Problem existieren. Dieser Ansatz der Programmierung wird als \textbf{logische Programmierung}\index{Programmierung!logisch} bezeichnet und gehört in den Bereich der \textbf{deklarativen Programmierung}\index{Programmierung!deklarativ}. Zur Erklärung:\\

Und vermutlich ploppen genau jetzt vor Ihrem inneren Auge die Fragezeichen auf: Wie soll das denn gehen?! Da wir uns zunächst mit verschiedenen Formen der imperativen Programmierung beschäftigen werden, sei hier nur ein Beispiel angeführt: \\

Jemand fragt Sie, ob Sie ihm den Weg zu einem Restaurant beschreiben können. Hier gibt es natürlich mehrere Möglichkeiten zu antworten. Wie sinnvoll eine Antwort ist, das hängt von den Prämissen ab, die für den Fragenden gelten, z.B.: Welches Verkehrsmittel will er nutzen? Welche Teile des Stadtplan kennen Sie? Usw. usf. \\

Eine Form deklarativer Programmierung besteht nun darin, dass Sie all diese bekannten Fakten (Straßennetz, Bedingungen des Fragenden, usw.) einprogrammieren. In der logischen Programmiersprache \textbf{PROLOG}\index{Programmiersprachen!PROLOG} geschieht das in Form sogenannter \textbf{Klausel}n\index{Klausel}. Diese Klauseln ähneln sehr den Relationen, die Sie in mathematischen Vorlesungen kennen lernen. Abschließend geben Sie eine Klausel ein, die z.B. die Frage repräsentiert, ob es einen Weg zum Ziel (hier dem Restaurant) gibt, ob es einen Weg einer bestimmten Länge dorthin gibt usw. Im Gegensatz zur imperativen Programmierung müssen Sie dem Computer dagegen nicht einprogrammieren, wie er nach diesem Weg suchen soll. Das erfolgt nach Regeln der Aussagenlogik und ist bereits als Teil der Programmiersprache festgelegt. Das Programm gibt dann eine mögliche Lösung an oder es gibt an, dass es keine solche Lösung gibt. Wenn Sie sich also bislang mit der Planung des Einsatzes von Mitarbeitern herumgeschlagen haben, dann erlernen Sie doch stattdessen die Programmierung in PROLOG, dann können Sie dieses Problem durch einen Computer lösen lassen.\\

Solche unterschiedlichen Ansätze der Programmierung werden auch als Paradigmen bzw. Programmierparadigmen bezeichnet. Langfristig werden Sie eine Vielzahl weiterer Paradigmen kennen lernen. Wenn Sie später einen Master in Informatik machen wollen, müssen Sie sich damit bereits\\
während des Bachelorstudiums beschäftigen.\\

\textbf{Kontrolle}

Was Sie sich an dieser Stelle merken sollten, sind zunächst zwei Punkte: 
\begin{itemize}
	\item dass es verschiedene Paradigmen gibt,
	\item dass Sie Programmiersprachen danach auswählen sollten, ob sie für das Paradigma passend sind, mit dem Sie gerade zu tun haben.
\end{itemize}

Wenn Sie das verstanden haben, dann haben Sie auch verstanden, warum Diskussionen sinnlos sind, in denen es darum geht, dass gewisse Sprachen nichts taugen. Für Betriebssysteme gilt ähnliches. Es ist allerdings durchaus möglich, dass frühere Versionen von Sprachen (genau wie Betriebssystemen) schlichtweg überholt sind und damit tatsächlich nicht mehr sinnvoll einsetzbar sind. Vor allem zeichnet es (Medien-)InformatikerInnen aus, dass sie im Stande sind, ein Paradigma auszuwählen, mit dem ein bestimmtes Problem effizient gelöst werden kann.\\

Das zweite, was Sie jetzt verstanden haben sollten ist, dass das was die meisten Menschen allgemein unter Programmierung verstehen die sogenannte prozedurale Programmierung ist, die zur Obergruppe der imperativen Programmierung gehört.

\subsection[Middleware, Framework, Bibliothek]{Man muss doch nicht immer das Rad neu erfinden – Middleware, Framework und Bibliothek}

Nahe verwandt mit Programmiersprachen sind die Begriffe \textbf{Middleware}\index{Middleware}, \textbf{Framework}\index{Framework} und \textbf{Bibliothek}\index{Bibliothek}. In der Urzeit der Programmierung entwickelte jeder Programmierer alles selbst. Dann wurden Softwarepakete entwickelt, die wie der Teil einer Programmiersprache verwendet werden können, aber im Grunde vollständige oder fast vollständige Programme sind. Je nachdem, wie umfangreich diese Pakete sind und was sie an funktionalem Umfang bieten, unterscheidet man zwischen den drei genannten Arten.\\

Der Begriff Middleware hat eine besondere Bedeutung, die Sie dann kennen lernen werden, wenn Sie eine Veranstaltung zur Nachrichten- oder Kommunikationstechnik besuchen. Dort steht er in aller Regel weniger für etwas, das direkt mit Programmierung zu tun hätte, sondern vielmehr für bestimmte Teile von Strukturen, Aufgaben und Hierarchien innerhalb eines Netzwerkes.\\

Im Rahmen dieses Buches werden wir zwischen den dreien nicht unterscheiden, die Begriffe werden hier nur eingeführt, damit Sie wissen, dass es da um Programmteile geht, die Sie in Ihrer Software nutzen können, ohne sie selbst entwickelt zu haben.\\

\textbf{Wichtig}

Java\index{Programmiersprachen!Java} beinhaltet zwar auch einen Teil, mit dem Sie imperativ programmieren können, aber \textbf{zum Großteil ist Java eine Middleware}. \\

\textbf{Kontrolle}

Es sollte Ihnen bewusst sein, dass Sie sich viel Zeit sparen können, indem Sie auf Middlewares, Frameworks und Bibliotheken zurückgreifen. Wie genau das geht, ist Teil aller Veranstaltungen, in denen Sie programmieren. Aber wie schon eingangs erläutert ist das Programmierung und nicht Praktische Informatik.

\subsection[IDE - Entwicklungsumgebung]{Damit Sie sich aufs Entwickeln konzentrieren können – IDEs / Entwicklungsumgebungen}

Auch \textbf{IDE}s (Integrated Development Environments bzw. Entwicklungsumgebungen)\index{Programmierung!IDE}\index{IDE} sind ein Mittel, mit dem Sie sich viel Arbeit sparen können. Eine IDE ist eine Zusammenstellung von Programmen, die Sie beim Programmieren unterstützen. Anfangs werden wir ohne eine IDE arbeiten, da Einsteiger häufig von den vielen Komfortfunktionen eher verwirrt als unterstützt werden.\\

\textbf{Kontrolle}

Sie sollten den Begriff IDE in Zukunft so selbstverständlich benutzen, wie andere Menschen den Begriff Brot.

\subsection{Dokumentation}\index{Dokumentation}

Nachdem Sie jetzt also einen ersten Eindruck davon haben, was Programmieren ist und wie Sie es sich erleichtern können, kommen wir zu einem entscheidenden Unterschied zwischen dem Alltag von professionellen ProgrammiererInnen und von HobbyprogrammiererInnen: Die Arbeit im Team.\\

Sie wissen wahrscheinlich, dass professionelle Software üblicherweise nicht von einzelnen Entwicklern in der Garage, sondern zum Teil von Hunderten von Mitarbeitern entwickelt wird. Und die müssen irgendwie miteinander arbeiten. Bevor wir hier auf die Details eingehen, folgt die zweite wichtige Regel fürs Programmieren:\\

\label{rule002}\textbf{Ein einsamer Wolf kam eine gute Software initiieren, aber dauerhaft können nur Teams daraus eine gute Software entwickeln.}\\

Der erste Schritt, um im Team erfolgreich Software zu entwickeln, ist die Dokumentation. Dokumentationen sind gleichzeitig der \\aufwändigste und vermeintlich wertloseste Teil der Arbeit im Team. Doch langfristig ist eine Softwareentwicklung ohne Dokumentation zum Scheitern verurteilt.\\

Stellen Sie sich einfach folgende Situation vor: Ein Kollege hat (irgendwann) einen Programmteil entwickelt, die in einem Spezialfall fehlerhaft funktioniert, der bislang nie auftrat. Zum Glück ist Ihnen das aufgefallen. Aber leider kann sich niemand mehr daran erinnern, wo genau sie in den 5000 Zeilen steckt und wer das damals eigentlich programmiert hat. Hätten Sie eine gute Dokumentation, dann würden Sie es jetzt nachschlagen. So können Sie nur beten, dass der Fehler niemals auffällt… Was natürlich bei der Steuerung eines AKWs keine gute Lösung ist.\\

\textbf{Kontrolle}

Wenn Sie in ein bestehendes Team einsteigen wollen, fragen Sie nach der Dokumentation. Gibt es keine oder ist sie sehr dünn, dann wird das Projekt scheitern, weil am Ende niemand mehr versteht, welche Funktion wo erfüllt wird. Allerdings werden Dokumentationen in aller Regel nicht ausgedruckt.

\subsection{SCM / Versionskontrolle}\index{Versionskontrolle}\index{SCM}

Damit Sie nun mit anderen gemeinsam an einer Software arbeiten können, gibt es Programme, die als Versionskontrollsysteme bezeichnet werden. Daneben ist auch die Bezeichnung \textbf{Software Control Management (SCM)} üblich. Bei diesen wird ihr Programm in einem sogenannten \textbf{Repository}\index{Repository} gespeichert, das auf einem Server platziert wird, den Sie über ein Netzwerk erreichen können.\\

Im Gegensatz zu dem, was sie wahrscheinlich bislang kennen gelernt haben, werden im Repository auch alle Änderungen (\textbf{Delta}s\index{Delta}) gespeichert, sodass Sie mittels einzelner Befehle unterschiedliche Versionen der Software einsehen und bearbeiten können.\\

Sie werden aktuell vorrangig auf zwei SCMs treffen: \textbf{Git}\index{Git} und \textbf{SVN}\index{SVN} (kurz für \textbf{Subversion}\index{Subversion}). Ohne auf die Details einzugehen: Bei Git haben Sie den Vorteil, dass Sie auch dann problemlos weiterarbeiten können, wenn Sie keinen Zugriff auf das Repository haben. Das ist bei SVN nur sehr beschränkt möglich.\\

Übrigens ist Git ein SCM, das von Linus Torvalds, dem Initiator und\\
höchsten Entwickler von \textbf{Linux}\index{Linux} angestoßen wurde. Mehr dazu auf \url{linux.com} und \url{git-scm.com}\\

\textbf{Kontrolle}

Ja, Sie können im stillen Kämmerlein vor sich hin arbeiten, aber das Thema Versionskontrolle müssen Sie im Hinterkopf haben. (Und das Thema Dokumentation natürlich erst recht.)

\subsection{Software Engineering / Softwareentwicklung}\index{Software Engineering}

Und wieder folgt ein Begriff, der Ihnen in Fleisch und Blut übergehen muss: \textbf{Software Engineering}\index{Software Engineering}, was als \textbf{Softwareentwicklung} übersetzt wird. Die Idee hier besteht im Gegensatz zu den Paradigmen weniger darin, dass Sie bestimmte Konzepte nutzen, um eine Aufgabenstellung zu lösen, sondern es geht um die Arbeitsteilung bei der Entwicklung eines großen Projekts. Die ursprüngliche Bedeutung des Begriffs lässt sich mit "`Sicherstellung hoher Qualität von Software" zusammenfassen, doch wenn wir uns damit beschäftigen, sind wir wieder einmal im Bereich der \textbf{Praktischen Informatik}\index{Informatik!Praktische Inf.}.\\

Wo es bei den Paradigmen um die Frage geht, wie wir eine möglichst optimale Lösung für unser Problem erreichen, stehen beim Software Engineering entsprechende Fragen im Mittelpunkt: Was will der Kunde eigentlich? Wie lange brauchen wir dafür? Können wir das überhaupt anbieten? Was wollen wir dafür berechnen? Wie oft müssen wir mit dem Kunden Besprechungstermine vereinbaren? Usw. usf.\\

Streng genommen handelt es sich hier also um eine Kernkompetenz der Betriebswirtschaftslehre (BWL), die als Projektmanagement bezeichnet wird. Aber im Gegensatz zum \textbf{Projektmanagement}\index{Projekt!-management} der BWL haben wir Werkzeuge zur Verfügung, mit denen wir räumlich getrennt gemeinsam an einer Software arbeiten können.\\

Einer der neuesten Vertreter dieser Spezies wird als \textbf{agil}e Softwareentwicklung\index{Software Engineering!agil} bezeichnet. Ältere Vertreter laufen unter Namen wie \textbf{Wasserfallmodell}\index{Software Engineering!Wasserfallmodell} und \textbf{V-Modell}\index{Software Engineering!V-Modell}. Aber keine Sorge, die Feinheiten des Software Engineering lernen Sie erst später im Studium kennen.\\

\textbf{Kontrolle}

Sie sollten wissen, dass es beim Software Engineering um Methoden geht, um Teamarbeit bei Softwareprojekten zu koordinieren. Dadurch wird eine hohe Qualität von Software in großen Projekten sichergestellt. Um hohe Qualität von Software geht es zwar auch bei der Praktischen Informatik, doch was Sie dort lernen können nützt Ihnen individuell bei der Entwicklung hochwertiger Software. Als letztes werfen wir nochmal kurz einen Blick auf die Programmierung: Um qualitativ hochwertige Software zu entwickeln müssen Sie einige Grundlagen verschiedener Programmierparadigmen lernen. Wenn Sie sehr gut in der Praktischen Informatik sind, ist es weitgehend belanglos, wie gut Sie eine bestimmte Programmiersprache beherrschen: Sie werden gute Software entwickln können. Wenn Sie dagegen kaum etwas von Praktischer Informatik verstehen, dann werden Sie selbst in Sprachen, die Sie sehr gut beherrschen bestenfalls mittelmäßige Software entwickeln.

\subsection[App-Entwicklung]{Für diejenigen, die die Programmierung von Apps erlernen wollen}

Der Begriff \textbf{App}\index{App!Entwicklung} ist im Grunde absurd. Eine App ist nichts anderes als ein Kunstwort, das geschaffen wurde, um Menschen, die nichts von Informatik oder Programmierung verstehen den Eindruck vorzugaukeln, dass es sich hier um eine besonders moderne oder hochwertige Anwendung handelt. Tatsächlich ist das Gegenteil der Fall: Der Begriff App ist vom Begriff Application abgeleitet, was nichts anderes als \textbf{Anwendung}\index{Anwendung} heißt. Eine Anwendung ist ein Programm, das für die Nutzung durch Menschen gedacht ist. Zwar wird in Bezug auf Anwendungen für mobile Endgeräte regelmäßig von Apps gesprochen, aber es gibt nichts und zwar überhaupt rein gar nichts, was an der Entwicklung solcher Anwendungen anders wäre als bei der Entwicklung von Anwendungen für andere Endgeräte.\\

Im Gegensatz zu anderen Anwendungen sind Apps jedoch immer sytemabhängig; Sie können also keine App entwickeln, die auf iPhone und Android unverändert lauffähig ist: Für jedes System sind umfangreiche Anpassungen nötig, die einzig deshalb nötig sind, weil die Entwickler der Geräte die Softwareentwickler davon abhalten wollen, für mehrere Systeme zu entwickeln. Gerade wenn Sie sich die Entstehungsgeschichte von Java ansehen werden Sie feststellen, dass die Systemabhängigkeit bei der App-Entwicklung mit Java (beispielsweise bei Android) geradezu absurd ist.\\

Wer die Entwicklung von Apps lernen will zeigt damit also im Regelfall, dass er/sie so viel von Softwareentwicklung versteht wie jemand, der\\glaubt, dass das Rauchen von Zigaretten, der Konsum von Alkohol bzw. Drogen oder der Kauf von Waren auf Ratenzahlung etwas mit Freiheit zu tun hätte: Er/Sie ist auf einen Werbeslogan hereingefallen und bedankt sich noch dafür, dass hohe Kosten für etwas zu zahlen sind, das das Versprechen nicht einhält und meist noch andere Kosten nach sich zieht. \\

Leider müssen wir diesen Begriff aber im Regelfall nutzen, weil Konsumenten und nicht-Informatiker auf genau dieses Marketingkonstrukt hereinfallen und gar nicht begreifen, wie unsinnig der Begriff App ist. Da diese Menschen uns für unsere Arbeit bezahlen müssen wir uns hier quasi dumm stellen und die "`asiatische Lösung"' wählen: Immer schön nicken und lächeln. Womit wir bei einem der Gründe wären, warum viele IT’ler und NT’ler im Servicebereich regelmäßig schlechte Laune haben, wenn sie mit nicht-IT’lern über IT reden (müssen): Sie werden von Nutzern damit konfrontiert, dass die ach so tollen Apps eben längst nicht das leisten, was die NutzerInnen sich aufgrund der Marketingbotschaften einbilden.\\

Schauen Sie sich dazu die Geschichte von Apple in den letzten zwanzig Jahren an, denn die ist direkt mit der Einführung des Begriffs der App verbunden: Gegen Ende der 90er stand Apple kurz vor dem Konkurs. Also entwickelten Jobbs \& Co. mobile Endgeräte, für die im Ein- bis Zweijahresrhythmus Nachfolger mit geringfügigen Verbesserungen bezüglich der Funktionalität erschienen. Vielmehr wurden hier jeweils signifikante ästhetische Änderungen durchgeführt, die Käufern den Eindruck vermittelten, die Nutzung dieser Geräte sei gleichbedeutend damit, zur \\technologisch-gesellschaftlichen Elite zu gehören. Gleichzeitig wurde wie in der Mode der Eindruck vermittelt, dass nur die jeweils aktuelle Baureihe genügen würde, um diesen vermeintlichen Status beizubehalten. Als dann nach einigen Baureihen absehbar war, dass dieses Vorgehen beim iPod nicht mehr die nötige Kundenbindung erzeugen würde, wurde mit dem iPhone bzw. dem iPad eine Kombination von iPod, Mac und Handy entwickelt, bei der die gleiche Strategie erfolgreich verfolgt wurde. \\

Wenn Sie über den Begriff \textbf{mobiles Endgerät}\index{Endgeräte!mobil} irritiert sind: Damit werden vorrangig in der Nachrichtentechnik all die Geräte bezeichnet, die an ein Netzwerk angeschlossen sind, die aber relativ frei bewegt werden können. Es geht hier also um Handys, Smartphones, Laptops mit WLAN, usw. Denn auch wenn die Nutzung eines Smartphones mit "`Internet"'-zugang für Sie genauso selbstverständlich sein dürfte wie die Nutzung eines Rechners, der per Kabel ans Internet angeschlosse ist, ist die Technik und Technologie, durch die insbesondere kabllose Anschlüsse realisiert werden alles andere als simpel. LTE ist dabei der erste erfolgreiche Versuch, verschiedene Arten von Vernetzung zu kombinieren.\\

Mittlerweile ist aber auch im Bereich von Smartphones die maximale Ausreizung des Marktes erreicht. Deshalb wurden inzwischen die \textbf{Wearables}\index{Endgeräte!Wearables} entwickelt: Kleine Geräte, die nur in Verbindung mit der aktuellsten Baureihe eines bestimmten Smartphones nutzbar sind. Während bislang die Ästhetik der Geräte selbst im Vordergrund stand, um eine vermeintlich elitäre gesellschaftliche Position und modisches Bewusstsein der BesitzerInnen zu vermitteln, steht bei den Wearables das Konzept im Vordergrund, KäuferInnen würden durch das sichtbare Tragen dieser Geräte zusätzlich sportliche Fitness, Flexibilität und Belastbarkeit beweisen. Mittlerweile steht Apple hier allerdings in direkter Konkurrenz zu Samsung. Die Koreaner haben dabei den großen Vorteil, dass Sie aus dem Land kommen, das weltweit seit Jahren die modernste Internetinfrastruktur besitzt. Deutschland ist in dieser Hinsicht selbst im EU-weiten Vergleich bestenfalls Mittelmaß, was ein essentieller Grund dafür sein dürfte, dass Siemens schon vor Jahren aus der Entwicklung von mobilen Endgeräten ausgestiegen ist.\\

Es sei an dieser Stelle allerdings noch angemerkt, dass Geräte von Apple häufig tatsächlich einen funktionalen Mehrwert gegenüber denen der Konkurrenz besitzen. Nur wird dieser Mehrwert eben von vielen Kunden gar nicht genutzt: Beispielsweise sind die Schnittstellen derart hochwertig, dass die Verbindung mehrerer Geräte im Regelfall problemlos ohne Zutun von Nutzern funktioniert. Microsoft versucht seit einigen Jahren ebenfalls in diesen Marktbereich einzudringen, um zur Konkurrenz im Bereich mobiler Endgeräte aufzuschließen. Das neueste Surface und Nokias neuste Modelle der Lumia-Reihe stellen in Verbindung mit dem Marketingkonzept rund um Windows 10 und dem augmented reality Sytem HoloLens den ersten erfolgreichen Versuch von Microsoft dar, die Grenzen zwischen verschiedenen Systemen aufzubrechen und neue Systemtypen ins Angebot zu integrieren. Wenn Sie sich dann allerdings die Situation des Herstellers von Blackberry Smartphones ansehen oder die Gründe für Apples wirtschaftlichen Einbruch in den 90er Jahren, werden Sie feststellen, dass eine Konzentration auf hochwertige Geräte für den Einsatz im professionellen Bereich wahrscheinlich keine Basis für dauerhaften wirtschaftlichen Erfolg darstellt.\\

Nun fragen Sie sich vielleicht, was all das mit einem Hochschulkurs zur Programmierung zu tun hat. Die Antwort ist simpel: Nach Ihrem Studium (egal ob Sie nach dem Bachelor noch einen Master und ggf. eine Doktorarbeit anfertigen oder direkt ins Berufsleben bzw. die Forschung starten) werden Sie entweder für einen Arbeitgeber tätig werden oder selbständig sein. Und wo auch immer das sein wird, gilt eine grundsätzliche Tatsache: Wenn Sie Software (oder auch Hardware) entwickeln, dann hängt Ihre Zukunft davon ab, ob diese Software tatsächlich deutlich mehr Geld einbringt, als Ihre Entwicklung gekostet hat. Schließlich müssen kontinuierlich neue Hardware angeschafft, Lizenzen erworben, Miete gezahlt werden usw. Das soll kein Appell gegen Indie Games oder Open Source sein, sondern es geht hier um Faktoren, die Ihre Zukunft bestimmen werden. Denn egal wohin Sie im Anschluss gehen, müssen Sie sich bewusst sein, dass Geld nicht einfach so entsteht, wie beispielsweise die Planer von Projekten wie der HafenCity, der Elbphilharmonie, von Stuttgard 21, Flughafen BER usw. usf. denken, sondern dass Sie nur dann dauerhaft ein zumindest ausreichendes Einkommen erreichen, wenn Sie Software entwickeln, die sowohl bestimmte Qualitätsstandards erreicht als auch von einer gewissen Anzahl von Kunden gekauft wird. Sollten Sie also in einem Unternehmen tätig werden, in dem entweder die Qualität der Software oder die Änderungen im Umfeld ignoriert werden, dann sollten Sie sich schnell nach einer neuen Aufgabe umsehen. Diese ignorante Strategie des Managements kann einige Jahre funktionieren, aber früher oder später wird sie das nicht mehr tun. Und ja, das gilt auch für den öffentlichen Dienst wie beispielsweise in öffentlichen Hochschulen, auch wenn wir hier nicht über wenige Jahre sprechen, sondern über Zeiträume, die eher in Richtung von zehn bis zwanzig Jahren gehen.\\

Zwei konkrete Beispiele möchte ich an dieser Stelle nennen: Zynga galt in den ersten Jahren, in denen Facebook international erfolgreich wurde als das erfolgreichste Unternehmen im Bereich der Browsergames\index{Games!Browsergames}. Werfen Sie dazu einen Blick in die Making Games-Ausgaben von 2012. Beispielsweise stammt das Spiel Farmville von diesem Entwickler. Zynga existiert zwar noch, aber mittlerweile hat das Unternehmen vier Fünftel seines Wertes verloren und viele Mitarbeiter wurden entlassen. Noch schlimmer sieht es beim ursprünglich größten Konkurrenten von Blizzard-Activision im Bereich MMORPGs\index{Games!MMORPG} aus: Sony Online Entertainment (kurz SOE) war u.a. mit Everquest II einer der erfolgreichste Entwickler von MMORPGs bis World of Warcraft erschien. Bei nachfolgenden Titeln versuchte SOE dann stets besonders hohe Qualitätsmaßstäbe zu setzen, was regelmäßig misslang. Im Februar 2015 wurde das Unternehmen an einen Finanzdienstleister verkauft und existiert heute unter dem Namen Daybreak Game Company. In beiden Fällen haben Sie es mit Unternehmen zu tun, die eine Zeitlang sehr erfolgreich waren, die es aber nicht geschafft haben, sich den Änderungen des Marktes anzupassen. Und die Auswirkung bestand darin, dass viele Mitarbeiter arbeitslos wurden.\\

Umgekehrt gibt es aber auch Unternehmen, die eine ausgewählte Kundengruppe bedienen, damit kontinuierlichen Erfolg haben aber nur gelegentlich in Spieletestzeitschriften auftauchen. Der isländische Entwickler CCPGames beispielsweise betreibt seit mehr als 10 Jahren das MMORPG Eve online, das seit Jahren kontinuierlich von mehr als 300.000 Kunden monatlich bezahlt wird. Zwar gab es hier durchaus Einbrüche bei den Nutzerzahlen, doch das Unternehmen reagierte auf Kritik und konnte so weiterhin bestehen.

\section{Informatik versus Programmierung, Studium und Arbeit}

Nachdem Sie jetzt einen ersten Einblick in einige Bereich bekommen haben, die bei der Programmierung eine Rolle spielen, schauen wir uns jetzt an, wo Informatik und Programmierung zusammenhängen.

\subsection{Informatik und Programmierung}

InformatikerInnen oder der Informatik nahe stehende Akademiker sind im Stande, ein Problem unabhängig von einer bestimmten Programmiersprache auszuformulieren und sich für ein Paradigma entscheiden, mit dem sie dann das Problem lösen können.\\

Der Bereich der \textbf{Praktischen Informatik}\index{Informatik!Praktische Inf.} widmet sich u.a. der Frage, wie Sie das Problem effizient lösen können und hat mit der \textbf{Programmierung}\index{Programmierung}, also der Umsetzung dieser Lösung in einer Programmiersprache nichts zu tun. Wenn Sie zusätzlich die Grundlagen der \textbf{Theoretischen Informatik}\index{Informatik!Theoretische Inf.} gemeistert haben, können sie außerdem erkennen, ob ein Problem überhaupt lösbar ist. Quereinsteiger versuchen dagegen häufig Programme zu entwickeln, die gar nicht lösbar sind. Und erst wenn sie wissen, dass und mit welchen Mitteln ein Problem idealerweise lösbar ist, entwickeln sie eine Lösung und setzen diese dann in einer gut geeigneten Sprache um. Außerdem sind sie im Stande, mit Dutzenden oder Hunderten anderer InformatikerInnen und ProgrammiererInnen gemeinsam Software zu entwickeln.\\

Auch hier nochmal der Hinweis: NaturwissenschaftlerInnen, TechnikerInnen und IngenieurInnen kennen diese Unterteilung in aller Regel nicht. Sie erlernen das Programmieren in einer oder mehrerer Sprachen, die für Ihren Bereich voll und ganz ausreichen. Dazu lernen Sie meist noch die Grundlagen dessen, was InformatikerInnen als \textbf{Technische Informatik}\index{Informatik!Technische Inf.} zusammenfassen, gehen aber davon aus, dass es sich hier um Informatik als ganzes handelt. Der Austausch mit diesen INT-AkademikerInnen ist deshalb meist schwierig: Sie begreifen oftmals nicht, dass sie die beiden Bereiche der Praktischen und Theoretischen Informatik mit Ihren Kenntnissen nicht erfassen können und gehen zusätzlich davon aus, dass Praktische Informatik dasselbe wie Programmieren ist. Und das ist nicht einmal ansatzweise richtig.\\

\textbf{ProgrammiererInnen} können all das nicht oder nur zu einem geringen Teil. Sie beherrschen eine oder mehrere Programmiersprachen und quetschen eine Lösung in diese Sprachen, egal ob das Ergebnis überhaupt brauchbar ist oder nicht. Manches, was Media Systems Studierende schon im Studium kennen lernen erkennen reine Programmierer erst nach Jahren oder Jahrzehnten. Aber unterschätzen Sie sie nicht: Dafür kennen ProgrammiererInnen häufig Kniffe, die man eben erst dann kennen lernt, wenn man eine Programmiersprache in- und auswendig gelernt hat. Und das ist immer wieder sehr wertvoll. \\

Leider werden InformatikerInnen und ProgrammiererInnen auch häufig deshalb in einen Topf geworfen, weil es den Ausbildungsberuf zum/zur "`FachinformatikerIn"' gibt. Dort liegt der Fokus im Gegensatz zum Informatikstudium klar auf der Programmierung. Im Anglo-Amerikanischen Raum gibt es noch den Studiengang Computer Science, der einem FH-Studium der Informatik hierzulande ähnelt.\\

Sie können sich den Unterschied zwischen Informatik und Programmierung auch dadurch veranschaulichen, dass Sie an Bauarbeiter und Bauingenieure denken: Ein Bauingenieur weiß, wie ein von ihm geplantes\\Gebäude errichtet werden muss, damit es z.B. nicht unter dem eigenen Gewicht zusammen bricht. Aber er wird im Regelfall viele Arbeitsschritte nicht selbst durchführen können. Der Bauarbeiter dagegen wird sehr genau wissen, wie die Arbeitsschritte richtig auszuführen sind. Dafür kennt er sich beispielsweise nicht mit den Kräften aus, die bei einem modernen Hochhaus auftreten. Sie beide haben unterschiedliche Fähigkeiten, aber nur gemeinsam können sie großes erreichen.

\subsection{Informatik: Uni versus HAW (FH)}

Im Gegensatz zu Ihnen beschäftigen sich Universitätsstudierende in der Informatik praktisch durchgehend mit der Frage, wie eine bestimmte Aussage zu beweisen ist. Diese Fragestellung ist spannend, aber es geht hier um rein abstrakte Konzepte, die bei höchst komplexen Projekten massive Effizienzsteigerung bedeutet.\\

Deshalb sollten Sie sich bewusst machen: Gute UniversitätsabsolventInnen sind Ihnen im Regelfall deutlich überlegen, wenn Sie es bei einer Aufgabenstellung mit sehr großen Datenmengen zu tun haben.

\subsection{Informatik und Programmieren im Beruf}

Schon beim Studienbeginn fragen viele Studierende, wie denn Ihre Aussicht auf dem Arbeitsmarkt ist. Hier gibt es zwei grundsätzliche Varianten: Zum einen können Sie nach dem Bachelor-Abschluss ein Masterstudium anstreben. Das ist gerade in der Elektrotechnik und der Informatik kein allzu großes Problem, weil hier viele Studienabsolventen bereits mit einem Bachelor-Abschluss spannende Stellen bekommen. Die Bewerbungssituation ist also entspannter, als beispielsweise bei den Sozialwissenschaften. Wenn Sie diesen Weg verfolgen, können Sie langfristig (ein Doktortitel ist da leider immer noch eine übliche Voraussetzung, wenn es Ihnen nicht genügt, als Assistent zu arbeiten) eine akademische Forschungskarriere anstreben.\\

Wenn Sie dagegen das Studium aufgenommen haben, um anschließend direkt in den Arbeitsmarkt zu starten, dann sollten Sie die Zeit neben dem Studium nutzen, um eigene Projekte zu realisieren. Denn das ist bei vielen Arbeitgebern eine gute Möglichkeit, um die eigenen Fähigkeiten nachzuweisen.\\

Doch das beantwortet natürlich nicht die Frage, welche Kenntnisse Arbeitgeber eigentlich erwarten, bzw. was Sie (neben dem Studium) an Kenntnissen erwerben müssen, um für eine bestimmte Tätigkeit gut vorbereitet zu sein. Dieser Abschnitt soll Ihnen da eine kleine Unterstützung bieten. Natürlich treffen die folgenden Aussagen nicht auf alle Arbeitgeber zu und wie die Situation in drei Jahren aussieht (also zu dem Zeitpunkt, zu dem Sie das Ende Ihres Studiums anstreben), ist noch eine andere Frage. Als Grundlage für die folgenden Aussagen dienten zwei Quellen: Zum einen aktuelle Stellenausschreibungen (August 2015), zum anderen Gespräche mit HR’lern im Hamburger Raum.\\

Die einfachste Antwort darauf, was von Ihnen häufig erwartet wird ist die am wenigsten hilfreichste: Es wird von Ihnen erwartet, dass Sie \textbf{backend}\index{backend} developer oder \textbf{frontend}\index{frontend} developer sind oder sonst eine Stelle ausfüllen können, die in Unternehmen zu besetzen ist. Das hat aber mit einem Studium (z.B. der Informatik) nichts zu tun, denn was Sie hier lernen (können) sind grundlegende Techniken und Methoden, die in beiden Bereichen angewendet werden. Leider gibt es immer wieder HR’ler (ja selbst in IT-Unternehmen), denen das nicht klar ist und die Sie dann mit Fragen konfrontieren, die Sie kaum beantworten können. Machen Sie sich in diesen Fällen nichts draus, wenn‘s mit der Stelle nichts wird; da gibt es bessere Angebote.\\

Die nächste Antwort hilft auch nicht weiter: Selbstverständlich sollen Sie... naja, so ziemlich alles können. Ums kurz zu machen, auch von solchen Unternehmen sollten Sie Abstand halten, denn dort wird von Ihnen im Grunde erwartet, dass Sie die Kenntnisse mitbringen, die Ihrem Vorgesetzten fehlen. Und glauben Sie mir, das wollen Sie nicht.\\

\textbf{Kommen wir jetzt also zu hilfreichen Antworten.}\\

Zunächst sollten Sie, nachdem Sie alle (!) Leistungsnachweise der ersten drei Semester erworben haben (ja, damit sind insbesondere Mathe 1 und 2 gemeint), sich ein wenig Zeit nehmen und prüfen, welche Bereiche Ihnen besonders liegen, bzw. mit welchen Konzepten der verschiedenen Veranstaltungen Sie am meisten anfangen konnten.\\

Hier ein Einwurf: Viele Studierende, die in Mathe 1 (oder einem ähnlich anspruchsvollen Fach) durchfallen, versuchen im nächsten Semester dennoch alle Veranstaltungen zu bestehen. Manche "`schieben" auch das, was noch nachzuholen ist in ein späteres Semester, weil Sie denken, dass das der passende Ansatz wäre. Beides hat wenig Aussicht auf Erfolg: Erledigen Sie zuerst das, was zuerst im Studienplan vorgesehen ist. Und wenn Sie dann zwei Veranstaltungen haben, zwischen denen Sie sich entscheiden müssen, dann entscheiden Sie sich für diejenige, die Sie für schwerer halten. Wenn Sie dann merken, dass Sie nirgends richtig vorankommen, dann streichen Sie die jeweils leichteste Veranstaltung, damit Sie für die übrigen Veranstaltungen mehr Zeit haben. Denn wenn Sie so agieren, haben Sie gute Aussicht darauf, mit nur einem zusätzlichen Semester einen guten Abschluss zu erlangen.\\

Jetzt also wieder zurück zur Situation, wenn Sie alles aus den ersten drei Semestern bestanden haben. Sie haben dann ein grundlegendes Verständnis für zwei Bereiche der Softwareentwicklung: FPGA-basiert und imperativ bzw. objektorientiert.\\

Wenn Sie sich jetzt im Bereich der maschinennahen Softwareentwicklung bzw. bei der Entwicklung von \textbf{FPGA}s\index{System!FPGA} wohler fühlen, dann sprechen Sie die Kollegen Edeler und Behrens an; diese können Ihnen die spannenden Möglichkeiten dieser Bereiche aufzeigen.\\

Wenn Sie sich dagegen eher bei \textbf{Desktoprechner}n und \textbf{Smartphone}s zu Hause fühlen, dann sind Sie im Bereich der Web- bzw. der Anwendungsentwicklung richtig. Sie können nun Spezialkenntnisse erwerben, um z.B. eine guter Entwickler für Android- oder iPhone-Apps zu werden. In dem Fall wäre Herr Plaß der Ansprechpartner Ihrer Wahl.\\

Um Software für mobile und/oder vernetzte Systeme entwickeln zu können, nutzen Sie die Kenntnisse aus PRG, P1 und P2 sowie Software Engineering, RDB, NWI und Kryptologie für Media Systems aus und setzen Sie diese Kenntnisse in Projekten um.\\

Sie wollen es genauer wissen? Gut: Aktuell (Herbst 2015) gibt es vorrangig Stellengesuche nach Leuten mit Kenntnissen in den folgenden Bereichen: 

\begin{itemize}
	\item Programmierung in \textbf{C}\index{Programmiersprachen!C} bzw. \textbf{C++}\index{Programmiersprachen!C++}\\
	Diese Stellen sind meist eher etwas für reine Informatiker, da hier umfangreiche Kenntnisse im Bereich der Algorithmik nötig sind, die in Ihrem Studienplan leider vollständig fehlen, obwohl sie ein der Grundlagen jeder Informatik und damit auch der Medieninformatik sind.
	
	\item Programmierung in \textbf{.NET}\index{Programmiersprachen!.NET} und angrenzenden Bereichen\\
	Das sind Spezialisten, die sich in die Softwareentwicklung mit Sprachen vertieft haben, die von Microsoft entwickelt wurden und ausschließlich auf Windows-Rechnern und –Smartphones genutzt werden können.
	
	\item Programmierung in \textbf{Objective-C}\index{Programmiersprachen!Objective-C} oder \textbf{C\#}\index{Programmiersprachen!C\#}\\
	Hier wird‘s schon interessanter, weil Sie zwar diese Sprachen in Ihrem Studium nicht kennen lernen, diese Sprachen aber eine sehr große Ähnlichkeit mit Java aufweisen.
	
	\item Programmierung in \textbf{Java für Server}\\
	Einerseits lernen Sie im Studium zwar Java, andererseits belegen Sie keine Kurse zu Betriebssystemen bzw. zur Serverwartung. Und genau das bräuchten Sie dafür.
	
	\item Programmierung in \textbf{HTML}\index{Programmiersprachen!HTML}, \textbf{CSS}\index{Programmiersprachen!CSS} und \textbf{PHP}\index{Programmiersprachen!PHP} oder \textbf{JavaScript}\index{Programmiersprachen!JavaScript}\\
	Hier sind Sie genau richtig, wenn Sie im Department Medientechnik der HAW Hamburg studieren, denn die Grundlagen erlernen Sie im ersten Semester im Kurs "´Einführung ins Programmieren"` bzw. "`Programmieren 1". Deshalb können Sie sie leicht ausbauen.
\end{itemize}

Bei den aktuellen Stellenanzeigen wird dann noch nach weiteren Kenntnissen gefragt. Damit Sie nachvollziehen können, was darunter verstanden wird und Sie in den noch folgenden Semestern die Möglichkeit haben, sich jeweils einzuarbeiten, folgen ein paar Erklärungen:

\begin{itemize}
	\item \textbf{HTML5}\index{Programmiersprachen!HTML5} und \textbf{CSS3}\index{Programmiersprachen!CSS3}\\
	Zunächst sind das die beiden Sprachen, in denen Sie Webpages programmieren (HTML) und das Design festlegen (CSS). Allerdings ist noch hinzuzufügen, dass die Änderungen in HTML5 (gegenüber der Version 4) so umfangreich sind, dass es eigentlich eine komplett neue Programmiersprache ist. So ist vieles, was früher mit Hilfe von PHP oder JavaScript programmiert werden musste Teil von HTML5. Aber das ist nur ein kleiner Teil von HTML5, der im Grunde auch als HTML 4.1 hätte veröffentlicht werden können. Dagegen beinhaltet die Version 5 mit den sogenannten \textbf{Microdata}\index{Microdata}\index{Semantik!Microdata} eine umfangreiche und standardisierte Semantik, womit es eine der ersten vollwertigen semantischen Programmiersprachen sein dürfte. Das ist gleichbedeutend mit derart fundamentalen Änderungen bei der Entwicklung von Anwendungen, das die Auswirkungen ähnlich wie beim WWW erst in zehn bis zwanzig Jahren spürbar werden dürften. Aber sie dürften ähnlich weite Kreise ziehen wie die Entwicklung des WWW selbst.\\
	\\
	Im Gegensatz dazu ist CSS nichts anderes als eine Sammlung von Befehlen, mit denen sich die Anzeige von Elementen innerhalb einer Webanwendung anpassen lässt. Es gibt dort sicher einiges, was Sie sich ansehen könnten, aber für MedieninformatikerInnen ist es ungefähr so wichtig, die Feinheiten der CSS-Programmierung zu beherrschen wie die genaue Bestimmung der Farbe der eigenen Tastatur. Sie halten das für vollkommen überflüssig und fragen sich, warum CSS dann überhaupt erwähnt wird? Na dann willkommen im Club: Der einzige Grund, sich als MedieninformatikerIn mit CSS zu beschäftigen ist der, dass es MediendesignerInnen gibt, die nicht einmal im Stande sind, eine Definition für einen Farbwert z.B. RGB-Werte in einen Computer einzugeben. Und dann wird diese Aufgabe eben an MedieninformatikerInnen delegiert...\\
	\\
	\textbf{Wichtig}\\
	Wenn ein Arbeitgeber explizit nach HTML und CSS fragt, aber sonst kaum nach Kenntnissen, die in dieser Aufstellung auftauchen, dann geht es in aller Regel um eine Stelle als Webdesigner und nicht um eine Stelle als Webdeveloper. (Letzteres wäre Ihr Gebiet, für ersteres sind Sie im falschen Studiengang.)

	\item \textbf{CSS Präprozessoren}\\
	Dieses Schlüsselwort weist eindeutig auf eine Stelle für Webdesigner hin.\\
	\\
	Beispiele: \textbf{Twig}\index{Framework!Twig}, \textbf{Gulp}\index{Framework!Gulp}, \textbf{SASS}\index{Framework!SASS}, \textbf{LESS}\index{Framework!LESS}.
	
	\item \textbf{XHTML}\index{XHTML}\\
	ist eine Kombinaion aus HTML und XML. Diese Kombination ist im Grunde durch HTML5 und Microdata oder andere Formen des semantic web in den meisten (aber nicht allen Bereichen) überflüssig geworden. Über das semantic web reden wir gleich in der ersten Veranstaltung des Kurses "`Einführung in die Programmierung"´.
	
	\item \textbf{JavaScript}\index{Programmiersprachen!JavaScript} oder \textbf{PHP}\index{Programmiersprachen!PHP}\\
	Für die Einarbeitung in PHP sollten Sie nicht allzu lange brauchen, wenn Sie eine andere imperative Sprache (wie Java) beherrschen, weil PHP nur wenige Konzepte der imperativen Programmierung umsetzt. \\
	\\
	Bei JavaScript sieht das anders aus: Der Einstieg ist zwar nicht schwer, aber um gute Software zu entwickeln (was in PHP eher nicht möglich ist), werden Sie schon ein Jahr Übungszeit brauchen. Das liegt auch daran, dass JavaScript u.a. das funktionale  Programmierparadigma umsetzt. Entwickler, die nur die klassenbasierte Objektorientierung (z.B. aus Java) kennen sind damit in aller Regel überfordert, ohne es zu merken. Aufgrund der Möglichkeiten, die JavaScript bietet und sein Status als Standard-Programmiersprache für dynamische HTML5-Seiten, dürfte PHP in zehn Jahren kaum mehr von Belang sein, wenn die Entwickler der Sprache hier keine fundamentalen Änderungen einführen. Leider sieht es zurzeit nicht so aus, als wenn das bei PHP7 der Fall wäre. \url{https://github.com/php/php-src/blob/php-7.0.0RC1/UPGRADING} Es scheint so, als wenn hier lediglich Neuerungen eingeführt werden würden, die in anderen Sprachen wie Ruby längst üblich sind oder die einfach selbst programmiert werden können. (Stichworte: UTF-8-Unterstützung oder die Einführung eines expliziten Spaceship-Operators)

	\item \textbf{ECMAScript}\index{Programmiersprachen!ECMAScript}\\
	ist eine standardisierte Version u.a. von JavaScript. Wenn Sie JavaScript beherrschen, sollten Sie sich hier relativ schnell zurecht finden. Umgekehrt gilt dasselbe.
	
	\item \textbf{JavaScript ... frameworks}\\
	Beispiele: \textbf{AngularJS}\index{Framework!AngularJS}, \textbf{Backbone}\index{Framework!Backbone}, \textbf{EmberJS}\index{Framework!EmberJS}, \textbf{Grunt}\index{Framework!Grunt}, \textbf{jQuery}\index{Framework!jQuery}, \textbf{requireJS}\index{Framework!requireJS}, \textbf{AJAX}\index{Framework!AJAX}. 
	
	\item \textbf{PHP ... frameworks}\\
	Im Falle des \textbf{Zend}\index{Framework!Zend} framework handelt es sich bei Zend nicht um eine kleine Erweiterung von PHP, sondern um ein sehr mächtiges Werkzeug, das eine längere Einarbeitungszeit benötigt. Dieses setzt u.a. voraus, dass Sie PHP objektorientiert programmieren können. 
	
	\item \textbf{Flash}\index{Programmiersprachen!Flash} und \textbf{ActionScript}\index{Programmiersprachen!ActionScript}\\
	sind praktisch dasselbe. ActionScript ist eine Konkurrenzsprachen zu JavaScript, deren Rechte bei Adobe liegen. Da JavaScript als Standardsprache für dynamische Webpages in HTML5 festgelegt wurde und es im Gegensatz zu ActionScript kein Unternehmen gibt, das hier Rechte beanspruchen würde, dürfte ActionScript ähnlich wie PHP in zehn Jahren nur mehr eine Nischensprache sein.\\
	\\
	Die Abkürzung AS wird teilweise für ActionScript verwendet, aber nicht nur dafür: Im Bereich der Programmierung von \textbf{SPS}en\index{Systeme!SPS} gibt es eine Programmiersprache namens Ablaufsprache, die ebenfalls mit \textbf{AS}\index{Programmiersprachen!AS} abgekürzt wird. Im Englischen wird diese Sprache als SFC / Sequencial Function Chart bezeichnet. ActionScript und Ablaufsprache haben so viel miteinander gemein wie Michael Schuhmacher mit einem Chinesen, der Tai Chi macht.\\
	
	\item \textbf{Frontend}\index{Frontend} und \textbf{Backend}\textbf{Backend}\\
	stehen für zwei Bereiche, die bei PHP noch strikt getrennt waren. (Auf die Details kommen wir bei der Einführung in PHP zu sprechen.) Meist ist mit Frontend die Programmierung des View gemeint (auch dazu kommen wir bald) und damit geht es dann in aller Regel um eine Aufgabe für Webdesigner. Das muss aber nicht so sein; sehen Sie sich ggf. die Stellenausschreibung genau an und fragen Sie beim Arbeitgeber nach. Beim Backenddevelopment haben Sie dagegen nie etwas mit Gestaltung zu tun. Als Studierende der Medieninformatik wäre das also der Bereich, in dem Sie vorrangig tätig werden. Für die Entwicklung des Frontends brauchen Sie immer zumindest eineN MediendesignerIn mit Grundkenntnissen der Typographie, da nur diese ein wenigstens ausreichendes Verständnis für die Darstellung medialer Inhalte haben. Hier nochmal der entsprechende Hinweis: Medieninformatik und Mediendesign arbeiten zwar beide im weitesten Sinne mit Medien, aber ersteres ist eine Spezialisierung der Informatik, letzteres eine Spezialisierung des Designs. Und wer aus dem einen Bereich kommt, kann bestenfalls verstehen, worüber die SpezialistInnen des anderen Bereichs reden. Mehr nicht!\\
	\\
	Grundsätzlich ist die Trennung in Front- und Backend aber recht\\willkürlich, weil damit suggeriert wird, es gäbe eine klare Trennung, wo eigentlich keine ist. Denn an welcher Stelle (auf dem Rechner eines Nutzers oder auf einem Server im Netz) Sie bei einem Softwareprojekt welche Funktionalitäten und Komponenten realisieren, ist Ihre Entscheidung.

	\item \textbf{Versionsverwaltung}\\
	(siehe Abschnitt "`SCM / Versionskontrolle"')

	\item \textbf{Continuous Integration}\index{Continuous Integration}\\
	ist ein relativ neues aber sehr mächtiges Konzept, wird teilweise mit CI abgekürzt, was aber leider auch für andere Begriffe stehen kann. Im Grunde ist es eine massive Erweiterung der Versionsverwaltung. Denn hier werden auch Tests und anderes in die Entwicklung eingeführt, das Sie später in der Veranstaltung Software Engineering kennen lernen.\\
	\\
	Beispiel: \textbf{Jenkins}

	\item \textbf{Continuous Deployment}\index{Continuous Deployment}\\
	ist dann gewissermaßen die aktuelle Luxusklasse fürs Software Engineering. Denn hier wird eine Software auch nach der Auslieferung an Kunden kontinuierlich weiter entwickelt. An bestimmten Punkten werden dann die Neuerungen an den Kunden ausgeliefert, um beispielsweise Fehler zu bereinigen (\textbf{Patch}en\index{Patch}) oder neue Funktionalitäten in die Software einzuführen.

	\item \textbf{Responsive Design}\index{Responsive Design}\\
	hat streng genommen nichts mit Design zu tun. Hier geht es darum, Webanwendungen zu entwickeln, die auf den unterschiedlichsten Displays nutzbar sind. Dieses Thema werden wir in in "`Programmieren 1"´ (für Medientechnik) bzw. "`Einführung in die Programmierung"´ (für Media Systems) kurz behandeln. Es gibt eine Vielzahl an JavaScript Frameworks, die Ihnen hier viel Arbeit abnehmen, die aber leider fast ausschließlich für HTML 4.01 gedacht sind.

	\item \textbf{Strukturiertes Arbeiten}\\
	Ein Arbeitgeber, der danach fragt ist für Sie die passende Wahl... außer wenn Sie eigentlich im falschen Studium gelandet sind.

	\item \textbf{UX / UI (User Experience, User Interface)}\\
	sind Begriffe, die in den Bereich Webdesign und \textbf{Usability}\index{Usability} fallen.

	\item \textbf{Photoshop}\\
	gehört ins Webdesign.

	\item \textbf{Design Patterns}\index{Design Pattern}\index{Software Engineering!Design Pattern}\\
	Seitdem die objektorientierte Programmierung in vielen professionellen Unternehmen genutzt wird, haben sich einige de-facto Standards ausgebildet, die die Arbeit im Team deutlich erleichtern. Damit werden Sie sich in der Veranstaltung Software Engineering auseinander setzen. Eines davon werden Sie aber schon in dieser Veranstaltung kennen lernen.\\
	\\
	Beispiel: \textbf{MVC}\index{MVC}\index{Software Engineering!MVC}

	\item \textbf{Agile Softwareentwicklung}\index{agil}\index{Software Engineering!Agile Softwareentwicklung}\\
	setzt objektorientierte Softwareentwicklung voraus und ist ein aktuelles Buzz Word: Viele Menschen benutzen es (in der Softwareentwicklung) aber leider gibt es hier wie bei der Objektorientierung eine Reihe von Missverständnissen. So gibt es ein Unternehmen, in dem ernsthaft versucht wird, agile Softwareentwicklung zu nutzen ohne dabei zentrale Aspekte der Objektorientierung zu verwenden. Und das ist unmöglich. Auch dieses Konzept lernen Sie in der Veranstaltung Software Engineering kennen.\\
	\\
	Beispiele: \textbf{TDD} (Test-Driven-Development)\index{TDD}\index{Test Driven Development}\index{Software Engineering!TDD}, \textbf{Iterationen} im Projektmanagement, \textbf{SCRUM}\index{SCRUM}\index{Software Engineering!SCRUM}

	\item \textbf{Extreme Programming}\index{Extreme Programming}\index{Software Engineering!Extreme Programming}\\
	diese Art der Softwareentwicklung geht in eine andere Richtung als die bislang besprochenen Varianten. Hier wird im Regelfall auf eine strukturierte Vorgehensweise verzichtet, wenn dadurch Kunden\-wünsche so schnell wie möglich integriert werden können.\\
	\\
	Schlagwörter: \textbf{YAGNI}\index{YAGNI}\index{Software Engineering!YAGNI}
	
	\item \textbf{Skalierbarkeit}\index{Skalierbarkeit} \\
	wenn dieser Begriff auftaucht, geht es um die Entwicklung von Anwendungen oder Systemen, die möglichst gut damit umgehen sollen, wenn die Anzahl Nutzer oder Daten, die jeweils auf die Anwendung oder das System zugreifen oder von diesem genutzt werden zum Teil innerhalb kurzer Zeit stark ansteigen oder abfallen können bzw.  stark schwanken. Dieser Bereich kommt für Sie leider nicht in Frage, weil Sie dafür mindestens die Veranstaltungen "`\textbf{Algorithmen und Datenstrukturen}"\index{Algorithmen und Datenstrukturen} sowie "`\textbf{Algorithmendesign}"\index{Algorithmendesign} (\textbf{Praktische Informatik}\index{Informatik!Praktische Inf.} 1 und 2) erfolgreich abgeschlossen haben müssen, die in unserem Department nicht auf dem Lernplan stehen.
	
\end{itemize}

Hier noch ein paar Begriffe, die jeder Arbeitgeber in diesem Bereich erwartet, weshalb sie eigentlich überflüssig sind:

\begin{itemize}
	\item Motivation
	\item Belastbarkeit
	\item Teamfähigkeit
	\item Kenntnis von Webstandards
\end{itemize}

Und nun noch ein paar Begriffe, bei denen Sie von einer Bewerbung absehen sollten, wenn Sie als Softwareentwickler ins Webdevelopment gehen wollen: So wie es wenig Sinn macht, Webdevelopment und Webdesign von einer Person durchführen zu lassen, macht es auch keinen Sinn, Webdevelopment und Serveradministration von einer Person durchführen zu lassen: Entweder ist diese Person sehr fähig (und damit sehr teuer) oder beherrscht nur einen der beiden Bereiche und das wäre sehr problematisch:

\begin{itemize}
	\item \textbf{Linux-/UNIX-Administration}\\
	Nicht zu verwechseln mit Linux-/UNIX-Kenntnissen; die müssen Sie bis zum Ende des Studiums zumindest grundlegend erworben haben, wenn Sie Media Systems studieren. Medientechnikstudierende, die in irgend einer Form mit Rechnern arbeiten wollen, sollten hier aber zumindest lernen, wie sie Befehle über die sogenannte Konsole eingeben können.

	\item \textbf{Konfiguration von ...-Servern (z.B. Linux, Apache)} wird gefordert.\\
	Die Administration von Servern hat nichts mit Softwareentwicklung zu tun, wie Sie sie im Rahmen von Veranstaltungen wie Programmieren 1 und 2 kennen lernen. Hier geht es vielmehr um Aspekte der Rechteverwaltung, Abwehr von netzbasierten Angriffen durch konkurrierende Unternehmen oder das organisierte Verbrechen und ähnliches. Wenn ein Unternehmen also jemanden sucht, der sowohl als Webdeveloper als auch als Serveradministrator fähig ist, dann herrschen dort offenkundig sehr große Wissensdefizite vor oder die Bezahlung ist entsprechend der geforderten Kenntnisse. Das würde dann aber bedeuten, dass Sie erst mit mehrjähriger Praxis in diesem Bereich die nötigen Kenntnisse erworben haben werden.

	\item Formulierungen wie \textbf{"`Gängige Webtechnologien wie HTML, CSS und ... sind Ihnen nicht fremd"´} weisen eher darauf hin, dass dem zuständigen Mitarbeiter des Unternehmens diese Programmiersprachen und die ihnen zugrunde liegenden Konzepte sehr fremd sind.

	\item Ebenfalls abzuraten ist von Anzeigen, in denen Ihre Aufgaben wie folgt beschrieben werden: \textbf{"`Development of the next generation of ..."´} Denn wenn Sie das wirklich könnten, dann sollten Sie sich lieber einen Investor suchen, der Sie bei der Umsetzung Ihrer Ideen unterstützt. In 99\% aller Fälle geht es hier um eine Unternehmen, dessen "`Fachkräfte" derart wenig über die Entwicklungen im Bereich der Softwareentwicklung wissen, dass sie glauben, eine abgerundete Ecke sei bereits eine historisch bedeutsame Produktentwicklung. (Nichts gegen abgerundete Ecken... Die sehen schick aus... findet jedenfalls der Autor dieses Buches.)

	\item Problematisch ist es, wenn bei Programmiersprachen keine Versionsnummern angegeben werden. Vielleicht kennen Sie genau die geforderte Version nicht, aber darüber lässt sich im Bewerbungsgespräch immer reden. Dagegen macht es beispielsweise einen großen Unterschied, ob Sie tatsächlich HTML5 oder eigentlich nur HTML4.01 beherrschen. Und umgekehrt ist es auch für Ihre berufliche Zukunft relevant, ob Sie für Arbeitgeber tätig werden, dessen SoftwareentwicklerInnen diesen Unterschied kennen oder eben nicht.

	\item Ebenfalls kritisch ist es, wenn einerseits nach einem "`Junior ... Developer"´ gesucht wird, andererseits aber eine sehr umfangreiche Palette an Kenntnissen gefordert wird. Denn einerseits wird damit angegeben, dass ein Einsteiger gesucht wird, der also gerade erst seinen Abschluss gemacht hat, andererseits werden derart viele Kenntnisse gefordert, dass im Grunde drei Jahre Berufserfahrung dafür nötig sind.

\end{itemize}

Kommen wir nun zu einem Bereich, den Sie schlicht deshalb ignorieren sollten, weil das eine Spezialdisziplin für Wirtschaftsinformatiker bzw. Informatiker mit mehrjähriger Praxiserfahrung in Unternehmen einer Branche ist:

\begin{itemize}
	\item \textbf{ERP}\index{ERP} (kurz für \textbf{Enterprise Resource Planning)}\\
	umfasst die verschiedensten Programme, die in Unternehmen zum Einsatz kommen. Hier ist neben der Kenntnis der jeweiligen Software ein detailliertes Verständnis für Unternehmensstrukturen und die Feinheiten der Branche nötig, in dem das Unternehmen tätig ist.\\
	\\
	Beispiele: \textbf{SAP}\index{SAP}, \textbf{Microsoft Dynamics}
	
\end{itemize}

\section{Zusammenfassung}

Nach diesem Kapitel wissen Sie, dass es eine Vielzahl von Möglichkeiten gibt, den Begriff des Programmierens zu verstehen, und dass InformatikerInnen den Begriff des Paradigmas nutzen, um zwischen diesen Möglich\-keiten zu unterscheiden. Einige davon sind Teil dieses Buches. (Deshalb ja auch der Titel.) Viele andere werden Sie in den nächsten Jahren kennen lernen, als Teil Ihres Studiums oder auch bei anderer Gelegenheit.\\

Sie wissen, dass es neben reinen Programmiersprachen Dinge wie IDEs, Bibliotheken, Frameworks und Middlewares gibt, die Ihnen einen Teil Ihrer Arbeit abnehmen.\\

Dann haben Sie etwas über Teamarbeit gehört, über die drei Teilbereiche der Informatik: Praktische, Theoretische und Technische Informatik. Und Sie haben insbesondere etwas darüber gehört, warum keines dieser drei Teilgebiete das gleiche ist wie Programmierung. Anschließend ging es um die Unterschiede zwischen den Inhalten der Informatik an Uni und Fachhochschule. Abgeschlossen wurde das ganze mit einem kurzen Überblick über den Arbeitsmarkt und die Begriffe, die Ihnen dort begegnen werden.\\

Was Sie jetzt aber noch nicht wissen ist, wie Sie mit all diesen Dinge umgehen sollen, bzw. wie Sie sie nutzen können. Keine Sorge, genau darum geht es ja in dieser Veranstaltung. Also sein Sie nicht frustriert, wenn Ihnen dieses Kapitel zu oberflächlich erscheint; es ist lediglich ein erster Einblick. \\

An dieser Stelle ein Anmerkung, die in der Wissenschaft (also auch in der Informatik bzw. in Ihrem Studium) immer und überall gilt: Alles wird kontinuierlich weiter entwickelt und wer glaubt, dass er ein fähiger Wissenschaftler ist, ohne sich in seinem Bereich ständig auf dem aktuellen Stand zu halten, der macht etwas falsch. Dementsprechend sollten Sie sich stets vor Augen halten: Was Sie hier lernen, ist Wissen, dass in wenigen Jahren so nicht mehr aktuell sein wird. Und Sie werden kontinuierlich prüfen müssen, welche Aspekte sich geändert haben. Denn sonst werden Sie in spätestens zehn Jahren nicht mehr verstehen, worüber in der Informatik geredet wird. Und das ist auch ein zentrales Element wissenschaftliches Arbeit: Hinterfragen und selbst prüfen, was man gehört hat und kontinuierlich die bestehenden Systeme verbessern. \\

Das ist übrigens eine der Besonderheiten der (Medien-)informatik: Einerseits ist sie so abstrakt, dass sie auf viele so abschreckend wirkt wie ein Gemälde von Pablo Picasso, andererseits in einer derart schnellen\\Veränderung, dass Kenntnisse zum Teil innerhalb von Monaten veraltert sind. Während also Ihre Kommilitonen z.B. in Elektrotechnik 1 und 2 Kenntnisse erlangen, die so noch in fünfzig Jahren nahezu gleich sein werden, müssen Sie als angehende (Medien-)informatikerInnen sich ständig mit den Änderungen auseinander setzen, weil Sie sonst nichts mehr von dem verstehen, was in Ihrem wissenschaftlichen Bereich passiert.\\

Sollten Sie allerdings zu denjenigen gehören, die eigentlich \textbf{Mediendesign}\index{Mediendesign} studieren wollten und die keine Lust haben, sich mit Fragen der Informatik und der Programmierung zu beschäftigen, dann möchte ich Sie auf Ihre Prüfungsordnung bzw. das Modulhandbuch des Studiengangs Media Systems hinweisen: 110 der 180 Credit Points Ihres Studiums liegen im Bereich der Informatik. Es wäre zwar schön, wenn ich durch dieses Buch oder die von mir angebotenen Veranstaltungen Ihr Interesse am Informatikteil der Medieninformatik bzw. von Media Systems wecken kann, aber Media Systems, Medieninformatik und Medientechnik sind Bachelors of Science und da sind die gestalterischen Anteile naturgemäß deutlich dünner gesät, als bei einem Bachelor of Arts. Wo Ihre Design-Kommilitonen kreativ übers Papier streichen, um Gefühlen Ausdruck verleihen, konzipieren Sie mit mathematischen und computerisierten Abstrakta, um umfangreiche Problemstellungen zu lösen.\\

Nach diesen freundlich gemeinten Ermahnungen wünsche ich Ihnen viel Erfolg bei dieser Veranstaltung und in Ihrem Studium. Lassen Sie sich von Rückschlägen nicht entmutigen und tun Sie das, wofür der Begriff Studium  steht: Sich intensiv mit etwas beschäftigen.

\chapter[Das ist Programmieren (wirklich)]{Typische Irrtümer darüber, was Programmieren ist.}
%\chapter[Nachrichtentechnik und Programmierung]{Von der Nachrichtentechnik zur Programmierung}

Computer sind eine Kombination aus Bauteilen und Datenübertragungs-leitungen. In vielen Programmiersprachen brauchen ProgrammiererInnen sich nicht direkt um die Datenübertragung zu kümmern. Allerdings folgen aus der Datenübertragung einige Fakten, die wir bei der Programmierung in jeder imperativen Programmiersprache beachten müssen. Tun wir das nicht, dann sind Fehler die Folge, die wir ohne ein allgemeines Verständnis der Grundlagen von Datenübertragungen nicht verstehen und damit lösen können.

\section{Nachrichtentechnik – So kommen Nullen und \\ Einsen in den Rechner.}

Sie haben schon öfter gehört, dass ein Computer im Kern mit Nullen und Einsen arbeitet. Was den ein oder anderen vielleicht zu Spekulationen über die Qualität dieser Geräte gebracht haben könnte… Wer arbeitet schon freiwillig einen Großteil seiner Zeit mit Nullen zusammen?\\

Was Sie aber in aller Regel in einem Informatikstudium nicht hören, ist dass diese Folgen von Einsen und Nullen nur ein Hilfsmittel sind, eine \textbf{Repräsentation} dessen, was tatsächlich vorhanden ist. Doch wenn Sie etwas darüber hören, dann wird Ihnen in aller Regel erzählt, dass damit der Unterschied zwischen fließendem oder nicht fließendem Strom dargestellt wird. Das ist so aber in aller Regel falsch: Es gibt die unterschiedlichsten Formen von Datenübertragungen, deren Signale am Ende als Folgen von Nullen und Einsen \textbf{interpretiert} werden. Und nur die am einfachsten zu verstehende Form ist die, bei der Signale erzeugt werden, indem zwischen fließendem und nicht-fließendem Strom unterschieden wird. Diejenigen von Ihnen, die Medientechnik studieren werden sich damit wenigstens fünf Semester lang beschäftigen, auch wenn Ihnen das anfangs also in Veranstaltungen wie Elektrotechnik 1 und 2 gar nicht bewusst sein wird.\\

Es folgen zwei Beispiele dafür, wie Einsen und Nullen auch anders dargestellt werden können:
\begin{itemize}
	\item Am Karlsruher Institut für Technologie (KIT\footnote{\url{http://www.kit.edu}}) wird an der Möglichkeit geforscht, wie man Daten in Molekülen aus Lachs DNA speichern kann. Hier wurden Nullen mit 0,4 V und Einsen mit 0,9 V repräsentiert.
	
	\item Im Bereich der Speicherung mit DNA gibt es aber auch andere Ansätze. Wie Sie aus dem Biologieunterricht wissen, besteht DNA aus vier Molekülen, die als sogenannte Basensequenzen endlose Ketten bilden können. Und richtig, auch diese Basensequenzen kann man nutzen, um damit Einsen und Nullen darzustellen. Hier steht also gar kein Stromfluss für Nullen und für Einsen, weil sie chemisch und nicht elektrisch repräsentiert werden. DNA bietet dabei eine derart hohe Dichte pro Bit an, dass Sie den Inhalt von 50 Millionen BlueRay Disks in einer Kaffeetasse unterbringen können\footnote{\url{http://www.spektrum.de/news/auf-petabyte-pro-gramm/1182773}}. Na gut, vielleicht war es auch ein Übersetzungsfehler und im Originalartikel war die Rede von einem Kaffeebecher, aber spielt das eine Rolle? Denn das bedeutet, dass Sie die gesamte Videothek der Welt in einem Kaffeebecher herumtragen könnten. Und überlegen Sie jetzt, wie viel Raum eine solche Videosammlung in Form von BlueRays oder Festplatten einnehmen würde.
\end{itemize}

Die wissenschaftliche Disziplin, die sich unter anderem mit der Frage \\beschäftigt, wie man Nullen und Einsen übertragen kann, nennt sich\\ \textbf{Nachrichten- und Kommunikationstechnik}\index{Nachrichtentechnik} und ist ein Teilbereich der\\ \textbf{Elektrotechnik}\index{Elektrotechnik}. Während es bei der Datenübertragung innerhalb eines \\ Computers noch recht problemlos möglich ist, mittels Ein- und Ausschalten von Stromflüssen Signale zu erzeugen und zu interpretieren, ist das bei Übertragungen über längere Distanzen eine eher ineffiziente Methode. Deshalb werden hier Schwingungen manipuliert, um die Datenübertragung zu realisieren. Das wichtigste mathematische Werkzeug ist dabei die \textbf{Fourier-Transformation}\index{entryFourier-Transformation}\index{Mathematik!Fourier-Transformation} (kurz \textbf{FT}). Diese transformiert Nullen und Einsen in eine Schwingung, indem Winkelfunktionen auf eine bestimmte Weise beliebig häufig angewendet werden. (Je häufiger, desto präziser.)\\

In der Nachrichtentechnik spielen dann Begriffe und Ansätze eine Rolle, die kaum etwas mit dem gemein haben, was die Tätigkeit von InformatikerInnen bestimmt. Während die \textbf{Media Systems}\index{Media Systems} Studierenen sich also vorrangig damit beschäftigen Gesamtaufgaben in Teilaufgaben zu unterteilen und mittels abstrakter Konzepte die Teilaufgaben möglichst übersichtlich zu gestalten und sie dabei möglichst effizient zu lösen, nutzen \textbf{Medien- und NachrichtentechnikerInnen}\index{Medientechnik} Formeln und Funktionen, um Daten so effizient wie möglich über einen bestimmten Leitungstyp zu übertragen.\\

\textbf{Kontrolle}

Die Nachrichtentechnik entwickelt die Systeme, die es uns überhaupt erst ermöglichen, Informatik zu betreiben. Einsen und Nullen sind zwar die Basis der IT-Systeme, die wir programmieren, was aber tatsächlich bei der Datenübertragung im Computer oder im Internet passiert, um Einsen und Nullen zu übertragen oder besser gesagt um eine jeweils passende Repräsentation für die Übertragung von Einsen und Nullen zu finden, hat damit größtenteils nichts gemein.

\section{Codierung – Was steht wofür?}
Eben haben Sie gelesen, dass Nullen und Einsen nur die Interpretation von z.B. Stromflüssen sind. Und damit sind wir bei einem zentralen Begriff, mit dem die meisten InformatikerInnen sich eher ungern beschäftigen, weil es dabei ausschließlich um die Frage geht, wie etwas interpretiert wird. Der Bereich, von dem hier die Rede ist, wird als \textbf{Codierung}\index{Codierung} bezeichnet.\\

Vielleicht haben Sie schon etwas vom \textbf{ASCI}-Code\index{ASCII}\index{Codierung!ASCII} gehört. Der ASCI-Code ist nichts anderes als eine Tabelle, bei der jedem Buchstaben und verschiedenen anderen Zeichen eine Zahl zugeordnet wird. Wenn Sie nur mit iOS- oder Windows-Rechnern arbeiten, haben Sie mit der Codierung im Regelfall nichts zu tun, aber sobald Sie Daten zwischen Computern austauschen, müssen sie darauf achten. Denn im Hintergrund werden alle Daten, die Sie eingeben in Form von Codes gespeichert. Wenn Sie Daten auf einen anderen Rechner übertragen, dann werden die Codes ausgetauscht. Und nur wenn beide Rechner die gleiche Codierung verwenden empfängt der zweite Computer die Daten, die Sie eingegeben haben.\\

An dieser Stelle werden wir uns einen Begriff ansehen, der Ihnen immer wieder mit jeweils unterschiedlicher Bedeutung begegnen wird: \textbf{Interface}s\index{Interface} bzw. \textbf{Schnittstelle}n\index{Schnittstelle}. Ein Interface ist etwas, das die Verbindung zwischen zwei "`Dingen" herstellt. Wenn wir über den Datenaustauch zwischen zwei Rechnern reden, die beide unterschiedliche Codierungen verwenden, dann ist eine mögliche Bedeutung des Begriffs Interface der eines Übersetzers. Das Interface kennt dann die beiden Codierungen und wandelt die\\ übertragenen Daten von der einen Codierung in die andere um, damit der Empfänger Daten in einer Codierung erhält, die er versteht.\\

Hier schon einmal ein Beispiel für eine ganz andere Bedeutung des Begriffs Interface: Bei der Programmierung in objektorientierten Sprachen kann ein Interface eine abstrakte Definition für einen Programmteil sein, der noch programmiert werden muss. Diese Interpretation des Begriffs Interface ist dann sinnvoll, wenn ein Programm im Team entwickelt werden soll. Das Interface enthält dann die Festlegung von Namen für "`Befehle", die zwar noch nicht funktionieren, aber bei denen schon festgelegt ist, was sie später tun sollen. So können dann alle Mitglieder des Teams mit Ihren Aufgaben beginnen: Sie können zwar Ihren Programmcode noch nicht ausprobieren, aber da Sie bereits wissen, wie die Befehle lauten, die später eine bestimmte Funktion erfüllen werden, können Sie zumindest schon mit der Entwicklung neuer Programmteile beginnen.\\ 

Ähnlich wie für den Begriff des Programmierens gilt auch für den Begriff der Codierung, dass er eine Vielzahl unterschiedlicher Methoden zusammenfasst, die jeweils für einen bestimmten Anwendungsfall sinnvolle Lösungen anbieten. Mehr dazu können Sie in dem großartigen Buch \textbf{"`Information und Codierung"} von \textbf{R.W. Hamming}\index{Codierung!Hamming, R.W.}\footnote{Hamming ist gewissermaßen der Godfather der Nachrichtentechnik.}  nachlesen. Codierung ist eines der zentralen Themen der Nachrichtentechnik und wird einführend in Veranstaltungen zur \textbf{Technischen Informatik}\index{Informatik!Technische Inf.} behandelt.\\

Im Bereich der \textbf{maschinennahen Programmierung}\index{Programmierung!maschinennah} kommt dann noch dazu, dass es zwei unterschiedliche Reihenfolgen gibt, in der Daten im Speicher abgelegt werden können. Die unterscheiden sich jedoch nicht (!) dadurch, dass Sie einfach nur spiegelverkehrt wären. Auch hierüber müssen Sie Bescheid wissen, wenn Sie zwischen zwei Computer Daten austauschen wollen. Denn selbst wenn Sie sichergestellt haben, dass die Codierung des einen Computers richtig in die Codierung des anderen übersetzt wird, kann Ihnen die unterschiedliche Reihenfolge im Speicher noch alles zerschießen. Die meisten von Ihnen werden dieses Problem aber nicht haben, da die Prozessoren heutige Rechner von Apple und Microsoft die gleiche Reihenfolge bei der Datenspeicherung nutzen. Vor einigen Jahren, als Apple noch Prozessoren von Motorola verwendete, war es dagegen eines der zentralen Probleme beim Datenaustausch zwischen Rechnern mit dem Betriebssystem beider Rechner.\\

Mit Codierungen bekommen es bereits Programmiereinsteiger sehr schnell zu tun, auch wenn wir dann von \textbf{Datentyp}en\index{Datentyp} sprechen. Denn letztere basieren im Grunde auf der Codierung: Sobald Sie eine Zahl oder eine andere Zeichenfolge in einem Programm verwenden, kann es dazu kommen, dass Ihr Programm vermeintlich falsche Ergebnisse liefert. Aber wie schon aber geschrieben: Das ist nicht der Fehler der Rechners, sondern der Fehler des Entwicklers, also ggf. von Ihnen, da hier die Arbeitsweise des Rechners ignoriert wurde. \\

\textbf{Kontrolle}
Texte und andere Inhalte werden bei der Speicherung codiert. Das bedeutet in der Informatik, dass Sie als Folgen von Binärziffern im Speicher liegen. Je nachdem, wie oder was Sie programmieren, werden Sie detaillierte Informationen über verschiedene Codierungsverfahren\\ benötigen.

\section{Informatik und Nachrichtentechnik - Die zankenden Geschwister}

Wie beschrieben beschäftigen sich \textbf{Nachrichtentechnik}erInnen\index{Nachrichtentechnik} vorrangig mit der Frage, wie Daten mit möglichst geringem Energieaufwand so\\ übertragen werden können, dass sie am Ende der Leitung empfangen und verstanden werden können. \textbf{Informatik}erInnen\index{Informatik} dagegen beschäftigen sich vorrangig mit der Frage, wie Daten möglichst schnell verarbeitet werden können.\\

Aus diesen unterschiedlichen Perspektiven resultiert eine vollständig andere Herangehensweise an Aufgaben. Und leider führt das häufig zu einer Trennung zwischen NachrichtentechnikerInnen und InformatikerInnen bzw. zwischen Ihnen und Ihren Kommilitonen der Medientechnik bzw. in der Medieninformatik. Dabei ließen sich ungemein spannende Projekte realisieren, wenn Sie es nur schafften, diese Kluft zu überwinden. Somit verfolgt dieses Kapitel auch das Ziel, Ihnen ein wenig Verständnis für dieses ungemein spannende aber auch höchst anspruchsvolle Gebiet zu vermitteln.\\

Das was wir heute als Computer bezeichnen, ist in aller Regel eine Art Black Box, in der mehrere Hundert Komponenten unabhängig voneinander arbeiten und über verschiedene Arten von Datenübertragungswegen miteinander kommunizieren. Vermutlich sind Sie jetzt skeptisch, da Sie wissen, dass in einem Computer Komponenten wie das Mainboard, Festplatten, Grafikarten und ähnliches vorhanden sind. Der Begriff der Black Box gilt insbesondere bei Smartphones, wo sie im Gegensatz zu den meisten anderen Rechnern nicht einmal mehr eine Grafikkarte oder einzelne Laufwerke erkennen können. Dennoch ist es so, denn wenn Sie beispielsweise ein Mainboard als eine einzige Komponente betrachten, dann ignorieren Sie, dass es sich bereits hier um eine Ansammlung von Dutzenden Komponenten handelt. Und dann gibt es zusätzlich noch Technologien wie die \textbf{VLSI}\index{VLSI}, die very large scale integration, bei der es darum geht, eine sehr große Anzahl von eigenständigen Einheiten in einem Bauteil zu integrieren.\\

Wie Sie in den Veranstaltungen der \textbf{Technischen Informatik}\index{Informatik!Technische Inf.} lernen, besteht ein Prozessor (fachlich korrekt ein \textbf{Mikroprozessor}\index{Mikroprozessor}\index{System!Mikroprozessor}) mindestens aus den drei Komponenten Rechenwerk, Steuerwerk und Speicher sowie den Verbindungen dazwischen, die als Bus bezeichnet werden. Das ist eine Abstraktion, die einem Prozessor aus den 50er Jahren entspricht, die aber genau dem entspricht, was wir bei der \textbf{imperativen Programmierung}\index{Programmierung!imperativ} beachten müssen. Wenn Sie mehr über den Aufbau von Computern mit Mikroprozessoren wissen wollen, werfen sie beispielsweise einen Blick in den Band \textbf{"`Rechnerarchitektur"} von \textbf{Paul Herrmann}.\\

Wenn wir uns nun die Arbeitsweise eines Computers unvoreingenommen\footnote{Mit unvoreingenommen ist hier gemeint, dass Sie sich die Arbeitsweise eines Computers ansehen und nicht die Darstellung auf einem Display bzw. die Eingabemöglichkeiten in Form von Maus, Tastatur usw.} ansehen, dann ist die wichtigste Komponente aber nicht "`der" Prozessor, sondern die Vielzahl der Kommunikationswege zwischen all den Komponenten, aus denen er besteht. Die bekannteste Art dieser Kommunikationswege wird als \textbf{Bus}\index{Bus} bezeichnet. Womit wir bei der Antwort auf die Frage wären, warum anstelle der Bezeichnung Nachrichtentechnik häufig den Begriff der \textbf{Kommunikationstechnik}\index{Kommunikationstechnik} benutzt wird und warum die auch für \textbf{Informatik}erInnen\index{Informatik} so wichtig ist: Wenn die Komponenten unseres Rechners keine Daten austauschen könnten, könnten wir mit Ihnen exakt gar nichts anfangen. Ohne die Arbeit der Kommunikations- und Nachrichtentechnik wäre unsere Arbeit also gar nicht möglich.\\

Für Einsteiger in die imperative Programmierung scheint diese Aufteilung in Prozessor, Bus und Speicher irrelevant zu sein, doch das ist schlicht falsch: Die meisten Fehler und Missverständnisse von Einsteigern kommen schlicht dadurch zustanden, dass Ihnen häufig nicht erklärt wird, wie eine \textbf{Variable}\index{Variable} im Speicher abgebildet wird und was der \textbf{Datentyp}\index{Datentyp} damit zu tun hat. Damit Sie professionell Software entwickeln können müssen Sie aber die aus der genannten Dreiteilung erwachsenden Probleme und Lösungen kennen: Wenn Sie später scheinbar simple Dinge wie den Zugriff auf eine Datei (Speichern bzw. Laden) selbst programmieren müssen, dann müssen Sie zuerst verstanden haben, dass es eine Datenübertragung zwischen dem Laufwerk und dem Prozessor gibt. Als nächsten müssen Sie verstanden haben, dass Sie niemals auf das Laufwerk, sondern nur auf den Datenstrom zugreifen können, der zwischen Prozessor und Laufwerk existiert. Sie müssen weiterhin beachten, dass die Datenübertragung über den Datenstrom eine gewisse Zeit dauert. Dann müssen Sie verstanden haben, dass bei dieser Datenübertragung einiges schieflaufen kann, was das ist und was Sie ggf. tun müssen, damit die Datenübertragung trotzdem klappt. Weiterhin müssen Sie sich unter Umständen damit auseinander setzen, welche Codierung bei der Speicherung der Daten verwendet werden muss. In dem Fall müssen Sie die Codierung und Decodierung programmieren. Und es gibt noch eine Vielzahl weiterer Probleme, die bei der Nutzung eines Buses auftreten können. Für viele davon hat die \textbf{Nachrichtentechnik}\index{Nachrichtentechnik} Lösungen entwickelt, aber einige müssen Sie selbst im Rahmen der \textbf{Programmierung}\index{Programmierung} lösen. Und das können Sie dann und nur dann, wenn Sie intensiv mit all den Problemen beschäftigen, die z.B. bei der Programmierung eines Taschenrechners irrelevant sind. Aber keine Sorge, wir fangen ganz einfach an. So lange Sie die Inhalte dieses Buches konsequent durcharbeiten und in eigenen Programmen umsetzen, werden Sie all das und noch viel mehr im Laufe der Zeit beherrschen.\\

Umgekehrt machen jedoch auch die Kommunikations- und NachrichtentechnikerInnen häufig den Fehler, die Konzepte und Modellierungen der Informatik als größtenteils untauglich oder überflüssig zu betrachten. Dabei sind diese Ergebnisse der Informatik Lösungen zu genau den Problemen, die bei der Nutzung nachrichtentechnischer Systeme entstanden sind bzw. entstehen. Hier sei wieder einmal auf das große Gebiet der \textbf{Praktischen Informatik}\index{Informatik!Technische Inf.} verwiesen, dass TechnikerInnen in aller Regel nicht kennen.\\

Bitte beachten Sie, dass wir hier über Kommunikationstechnik reden; in den \textbf{Kommunikationswissenschaften}\index{Kommunikationswissenschaften} (Teilbereich der Sozial- und Geisteswissenschaften) geht es um die Kommunikation zwischen Menschen.\\

\textbf{Kontrolle}
Informatik und Nachrichtentechnik sind im Grunde wie zwei Seiten einer Medaille: Die eine kann ohne die andere nicht existieren. Im Regelfall sind beide für die Leistungen der jeweils anderen jedoch blind.

\section{Von der Nachrichtentechnik zu logischen Gattern}

Ein zweiter Bereich, mit dem sich vorrangig die Elektro- bzw. \textbf{Nachrichtentechnik}\index{Nachrichtentechnik} beschäftigt, ist der Aufbau von Komponenten, die logische Operationen ausführen. Diese bestehen seit mehr als fünfzig Jahren aus Kondensatoren. Wenn Sie sich nicht mehr an den Physikunterricht erinnern: Ein Kondensator ist ein Bauteil, über das mittels einer Eingangsleitung gesteuert werden kann, ob Strom weitergeleitet wird oder nicht. Hier sind wir dann auch in einem Bereich, in dem eine 1 gleichbedeutend ist mit fließendem Strom und eine 0 mit nicht fließendem Strom. \\

Solche logischen Gatter können Sie bei den \textbf{FPGA}s\index{FPGA}, den Field-Programmable Gate Arrays direkt programmieren\footnote{Diese Art der Programmierung ist Teil der Veranstaltung Informatik 2 für Media Systems. Das bedeutet leider auch, dass Sie am Studienende keine Kenntnisse der Praktischen Informatik haben werden. Dafür werden Sie im Bereich der Technischen Informatik deutlich mehr Kenntnisse besitzen als Ihre Kommilitonen von anderen HAWs bzw. FHs.}. Wenn Sie das tun, dann befinden Sie sich in einem der wenigen Gebiete, in dem Elektrotechniker und Informatiker dieselbe Sprache sprechen. Diese Technologie wurde erst Mitte der 80er Jahre entwickelt. Und obwohl wir hier von einer Programmiertechnik reden, werden Sie in Büchern über Programmierparadigmen kein Kapitel finden, dass das entsprechende Paradigma enthält.\\

Der Grund dafür ist ganz einfach: Diese Art der Programmierung ist eine direkte Umsetzung dessen, was in der Mathematik \textbf{boolesche Algebra}\index{Mathematik!boolesche Algebra} genannt wird. Und an dieser Stelle möchte ich eine Lanze für die \textbf{Mathematik}\index{Mathematik} brechen: Immer wieder fallen Formulierungen wie „Mathe ist doch nur eine Hilfswissenschaft.“ Das allerdings ist ein Ausdruck, der eher auf die Ignoranz der Person verweist, die ihn gebraucht. Denn tatsächlich gab es die bei Computern verwendete Logik lange vor dem ersten Computer. \\

Und Mathematik hat zunächst auch nichts mit Rechnen zu tun: Im alten Griechenland gab es eine Disziplin namens Philosophie. Die Philosophen versuchten die Welt und das was in ihr geschieht zu erklären. Daraus entstanden dann weitere Disziplinen wie die Naturwissenschaften, die sich auf bestimmte Teilgebiete der Welt und des Universums konzentrieren. Wenn Sie also bislang über Philosophie als eine Wissenschaft betrachtet haben, in der es darum geht, wie Menschen sich verhalten sollten, dann haben Sie hier eine zu beschränkte Vorstellung.\\

Es gab aber auch Forscher, die sich fragten, wie ein Universum aussehen würde, wenn sie den umgekehrten Weg wählen würden. Sie untersuchten deshalb, wie ein Universum aussehen würde, für das Eigenschaften willkürlich festlegt werden. Um das nochmal zu betonen: Im Gegensatz zu allen anderen Wissenschaften setzt sich die Mathematik nicht mit der Beschaffenheit eines einzigen (nämlich unseres) Universums auseinander: Sie geht wesentlich weiter! Sie stellt sich die Frage wie jedes nur denkbare Universum aussehen würde. Und wer so etwas als Hilfswissenschaft abtut, der... Nun ja, wie soll ich es höflich ausdrücken? \\

Wenn Sie jetzt genau gelesen haben, dann verstehen Sie, warum die Aussage wie "`das wurde mathematisch bewiesen" in den Naturwissenschaften häufig einen falschen Eindruck vermittelt: Sie können mathematisch alles beweisen, wenn Sie nur kreativ genug bei der Entwicklung von Annahmen sind. Ein mathematischer Beweis alleine genügt also nicht, um zu beweisen, dass etwas wahr ist. Sie müssen außerdem beweisen, dass die Voraussetzungen bzw. Annahmen wahr sind, die Sie für Ihre Beweisführung vorausgesetzt haben.\\

Ein aktuelles Beispiel sind die "`Beweise" rund um den Urknall. Die Physik ist zurzeit im Stande, zu beweisen, in welchem Zustand sich das Universum wenige Sekunden nach dem sogenannten Urknall befand. Was also direkt zum "`Zeitpunkt" des Urknalls passiert oder was möglicherweise davor "`war" ist zumindest momentan nicht beweisbar. Es gibt jedoch PhysikerInnen, die darauf beharren, das zu können. Sie legen dafür mathematische Beweise vor, die in sich schlüssig sind. Der Fehler liegt hier aber nicht in der mathematischen Beweisführung, sondern darin, dass sie Annahmen treffen, die zumindest noch nicht bewiesen sind. Wie Sie in "`\textbf{Mathematik 1}"\index{Mathematik} lernen ist es aber möglich, aus einer falschen Annahme wahre und falsche Schlussfolgerungen zu ziehen. Wenn also PhysikerInnen Annahmen für eine mathematische Beweisführung verwenden, die noch nicht bewiesen sind, dann kommen sie damit zu Aussagen, bei denen unklar ist, ob sie nun wahr oder falsch sind. Also haben die eingangs genannten Forscher nicht etwa den Urknall oder Abläufe rund um den Urknall bewiesen, sondern lediglich gezeigt, welche Abläufe dort stattgefunden haben, wenn bestimmte Annahmen wahr sind.\\

Jedenfalls sollte Ihnen damit klar sein, dass MathematikerInnen häufig schon Lösungen parat haben, wenn die zugehörigen Probleme noch gar nicht in der Praxis auftreten. Und nein, wir reden hier nicht von Monaten; einige Lösungsansätze, die in den letzten Jahrzehnten zum Einsatz kamen wurden bereits vor mehreren hundert Jahren von Mathematikern konzipiert. Wenn Sie also in einem mathematischen Lehrwerk Formulierungen lesen wie "`Es sei ein Universum mit den Eigenschaften...", dann wissen Sie jetzt, dass das wortwörlich gemeint ist: Es geht wirklich darum, ein Universum oder einen Teilbereich eines Universums zu beschreiben, das über bestimmte Eigenschaften definiert wird. Dieses Universum hat aber in aller Regel nicht das geringste mit dem zu tun, was Sie sich als ein Universum vorstellen.\\

Hier auch noch ein Exkurs: Häufig wird von radikalen (Mono-)theisten behauptet, Naturwissenschaftler würde behaupten, dass es Gott nicht gäbe. Das ist falsch: Naturwissenschaftler untersuchen, wie das Universum aussieht. Sie untersuchen dabei, welche Gesetzmäßigkeiten im Universum nachweisbar sind. Das hat aber nichts mit der Frage zu tun, ob es einen Gott gibt, der all das erschaffen hat.\\

Aber zurück zu den FPGAs: Wenn Sie die boolesche Algebra nicht beherrschen, können Sie auch kein FPGA programmieren. Unabhängig davon kommt die boolesche Algebra auch immer dann zum Einsatz, wenn Sie den Rechner etwas prüfen lassen wollen. Für Fortgeschrittene gibt es dann zwar noch Mittel wie die Fuzzy Logic, aber das überlassen wir mal lieber den MathematikernInnen und universitären InformatikerInnen.\\

\textbf{Kontrolle}\\
Eine erste Art der Programmierung besteht in der Umsetzung der booleschen Algebra, einer mathematischen Methode, um aus einfachen Ja-/Nein-Fragen komplexe Modelle zu entwickeln. Prozessoren basieren auf nichts anderem.

\section{Weiter zur Maschinensprache}

Wenn wir nun einen Prozessor nicht entwickeln, bzw. seine Arbeitsweise programmieren wollen, sondern ihn so nutzen wollen wie er ist, um Programme zu entwickeln, dann ist die erste Methode die \textbf{Maschinensprache}\index{Programmiersprache!Maschinensprache}. Danach folgt die \textbf{maschinennahe Programmierung}\index{Programmierung!maschinennah}.\\

Wie Sie wissen reden wir von Prozessoren mit einer gewissen \textbf{Bittigkeit}\index{Bittigkeit}. Heute sind die 32-Bit- und die 64-Bit-Prozessoren am bekanntesten. Diese Bittigkeit bedeutet nichts anderes, als dass der Prozessor bei jedem Rechenschritt eine Zahl mit genau dieser Anzahl an Bits addieren kann. Ein 32-Bit-Prozessor kann also bei jedem Rechenschritt eine Zahl zwischen 0 und \(2^{32}-1\) zu einer anderen Zahl in diesem Bereich addieren\footnote{\(2^{32 - 1}\) entspricht etwas weniger als 4,3 Millionen}. Wie Sie in der \textbf{Technischen Informatik}\index{Informatik!Technische Inf.} erfahren liegt darin auch der Grund, dass 32-Bit-Prozessoren nur einen Speicher mit rund 4 GB verwalten können: Für mehr Speicherbereiche haben Sie schlicht keine "`Hausnummern".\\

\textbf{Aufgabe}:\\
Berechnen Sie doch mal eben die maximale Größe für Speicher, die ein 64-Bit-Prozessor nutzen kann. Und suchen Sie nach einem anschaulichen Beispiel, dass diese Zahl verdeutlicht. (Anmerkung, die nötig ist, weil leider einige Studienanfänger das Potenzrechnen nicht beherrschen: \(2^{64}\) ist nicht (!) \(2 \cdot 2^{32}\). Die Antwort lautet also nicht "`etwas weniger als 8,6 Millionen".)\\

Die Maschinensprache besteht dementsprechend aus nichts anderem als Zahlen, die in dem Bereich liegen, der von der Bittigkeit des jeweiligen Prozessors abhängt. Ob eine Zahl nun ein Befehl darstellt, ob Sie als Zahl zu verwenden ist oder ob es eine Position im Speicher sein soll, kann nur wissen wer die Maschinensprache eines Prozessors beherrscht; es gibt keine Möglichkeit, einer Zeile eines Programms in Maschinensprache anzusehen, wofür sie steht.\\

\textbf{Kontrolle}\\
Maschinensprache ist so schlecht lesbar, dass Sie sie am besten gleich ignorieren. Verwechseln Sie aber bitte nicht die Maschinensprache (siehe dieser Abschnitt) mit der maschinennahen Programmierung (Assembler, siehe nächster Abschnitt).

\section{Maschinennahe Programmierung – Assembler}

Nachdem wir jetzt geklärt hätten, wie man einen Prozessor besser nicht programmiert, kommen wir zur ersten Methode, um einen Prozessor sinnvoll zu programmieren: Die maschinennahe Programmierung, die auch als \textbf{Assembler}\index{Programmiersprache!Assembler} oder Assembler Programmierung bezeichnet wird. Es ist die erste Variante der \textbf{imperativen Programmierung}\index{Programmierung!imperativ} und leider ist sie alles andere als anschaulich. Aber wenn Sie sich hiermit beschäftigen, dann zu \textbf{C}\index{Programmiersprache!C} und später zu \textbf{C++}\index{Programmiersprache!C++} oder \textbf{Java}\index{Programmiersprache!Java} weitergehen, werden sie verstehen, was die Vorteile dieser Sprachen sind und werden diese zu schätzen wissen.\\

Im Regelfall lernen Sie zu Beginn einer Veranstaltung, in der Sie die maschinennahe Programmierung kennen lernen, etwas über den Aufbau des Prozessors, den Sie maschinennah programmieren werden. Da tauchen dann Begriffe wie Pipeline, RISC und CISC und andere auf. Das kann Ihnen helfen, bestimmte Abläufe besser zu verstehen. Für den Einstieg genügt es aber, wenn Sie die Dinge kennen lernen, die Sie tatsächlich programmieren können.\\

Aber bevor wir auf die eingehen, sollten Sie wieder (wie zu Beginn dieses Buches) die Antwort auf die Frage erfahren, was \textbf{maschinennahe Programmierung}\index{Programmierung!maschinennah} eigentlich ist. Und die Antwort hierauf ist genau die, die allgemein mit dem Begriff Programmieren verbunden wird: Sie tippen Zeile für Zeile ein, was der Computer (besser gesagt der Prozessor) tun soll und der führt es dann in genau dieser Reihenfolge aus. Einzige Ausnahme: Sie können Sprünge ins Programm einbauen, die dafür sorgen, dass der Prozessor einen anderen Teil Ihres Programms ausführt, aber am zeilenweisen Ablauf ändert sich ansonsten nichts. Diese anderen Teile werden als \textbf{Subroutine}n\index{Subroutine} bezeichnet. Es sind Programmteile, die an verschiedenen Stellen im Programm ausgeführt werden sollen. Und sie werden einzig und allein deshalb ausgelagert, weil sie dadurch nur einmal programmiert, aber mehrfach ausgeführt werden können. Das reduziert die Fehleranfälligkeit und erleichtert die spätere Verbesserung des Programms.\\

In der maschinennahen Programmierung besteht nun jede Zeile aus einem sogenannten \textbf{Mnemon}\index{Mnemon} und zwischen keiner und mehreren Zahlen. Ein Mnemon ist einem Hilfswort, das einem Befehl des Prozessors entspricht. Im Gegensatz zur tatsächlichen Maschinensprache können Sie hier also Buchstabenkombinationen für Befehle benutzen. Aber ob eine Zahl eine Zahl oder eine Speicheradresse ist, das müssen Sie zusammen mit der Definition jedes Mnemons lernen.\\

Dennoch gibt es Gründe, wegen denen bestimmte Entwickler immer noch in Assembler programmieren und aus denen es tatsächlich Sinn macht, sich auf diese Art der Programmierung zu konzentrieren: Es gibt keine effizientere Möglichkeit, einen Computer zu programmieren und viele Prozessoren lassen sich nicht vollständig in anderen Sprachen programmieren. Der Bedarf an Effizienz muss allerdings sehr hoch sein, denn mit C als höherer Sprache können Sie immer noch recht maschinennah und effizient programmieren.\\

Das Standardwerk für die maschinennahe Programmierung stammt von \textbf{Donald E. Knuth} und hat den Titel "`\textbf{The Art of Computer Programming}". Es besteht aus sieben Bänden, von denen die ersten drei veröffentlich wurden. Band vier wurde in zwei Teile unterteilt, von denen der erste ebenfalls veröffentlicht wurde. Die übrigen Bände sind noch in Arbeit. Das mag so klingen, als wenn eigentlich ein anderes Buch als Standard gelten sollte, doch im Gegensatz zu allen mir bekannten Autoren untersucht Knuth in seinem Buch jedes grundlegende Problem, das bei imperativer Programmierung vorkommen kann. Insbesondere wird in seinem Band praktisch ab der ersten Seite klar, warum InformatikerInnen welche mathematischen Grundlagen beherrschen müssen.\\

Dennoch lassen sich die Bücher auch ohne diese mathematischen Grundlagen durcharbeiten: Knuth hat viele Passagen so verfasst, dass klar erkennbar ist, ob sie auch ohne mathematische Grundlagen verständlich sind. Das gleiche gilt für die Vielzahl an Aufgaben, die den Text begleiten. Die Musterlösungen machen beispielsweise rund ein Drittel des ersten Bandes aus. Damit kann jedeR Interessierte sich unabhängig von den persönlichen Kenntnissen weitgehend in das Themengebiet einarbeiten.\\

\textbf{Kontrolle}\\
Auch die maschinennahe Programmierung ist eher abstrakt. Die Zeilen eines solchen Programms bestehen aus Buchstabenkürzeln und Zahlen. Was das Programm bei der Ausführung an welcher Stelle tut ist sehr schwer zu erkennen.

\section{Zusammenfassung}
In diesem Kapitel haben Sie erfahren, dass die Nachrichtentechnik die Techniken und Technologien liefert, mit der InformatikerInnen Ihre Ergebnisse in die Praxis umsetzen können. Ohne Nachrichtentechnik gibt es keine IT-Systeme und letztlich würden Sie nicht nur dieses Buch nicht lesen, Sie würden wahrscheinlich etwas ganz anderes studieren, weil Computer und das Internet nicht existieren würden. Unter Umständen hätten Sie nicht einmal die Hochschulreife erreicht.\\

Sie haben außerdem einen ersten Einblick in das erhalten, was hinter den Nullen und Einsen steht, und dass das etwas ganz anderes ist, als das, was die meisten sich darunter vorstellen. Sie haben ebenfalls erfahren, dass wir auch die Einsen und Nullen in aller Regel nicht wahrnehmen, weil sie sich hinter den Ergebnissen der Codierung verbergen.\\

Abschließend haben Sie einen ersten Blick darauf erhascht, wie all das zu Programmiersprachen führt. Im nächsten Kapitel geht es dann um Formen der Programmierung, mit denen Sie in den nächsten Jahren voraussichtlich am häufigsten zu tun haben werden.
\chapter[Nachrichtentechnik und Programmierung]{Von der Nachrichtentechnik zur Programmierung}
%\chapter{Vorbereitung fürs Programmieren}

Im Grunde brauchen Sie zum Programmieren lediglich einen Texteditor sowie ein SDK bzw. einen Compiler oder einen Interpreter für die jeweilige Sprache. Ein Texteditor wird bei jedem beliebigen Betriebssystem mitgeliefert, während Sie sich SDK/Compiler/Interpreter häufig von der Seite des Entwicklers herunterladen können.\\

Sie werden im Folgenden des Öfteren die Bezeichnung \textbf{Werkzeug}\index{Werkzeug} (engl. \textbf{tool}\index{tool}) lesen. Beim Programmieren ist damit im Regelfall ein Programm gemeint, das Sie in irgendeiner Form bei der Softwareentwicklung unterstützt.\\

Da Sie bei der Suche nach Werkzeugen fürs Programmieren über eine Vielzahl an Begriffen stolpern werden, folgt hier eine auszugweise Aufstellung häufiger Bezeichnungen, sortiert nach Bedeutung:\\

\begin{itemize}
	\item Mit \textbf{Quellcode}\index{Quellcode} bezeichnet man den Programmtext, den Sie eingegeben haben.
	\item Ein \textbf{Interpreter}\index{Interpreter} ist eine Software, der Ihren Quelltext Zeile für Zeile übersetzt und kontinuierlich jede übersetzte Zeile sofort ausführt.
	\item Ein \textbf{Compiler}\index{Compiler} ist eine Software, der das von Ihnen verfasste Programm vollständig übersetzt und dabei eine beschränkte Anzahl an Fehlertypen erkennen kann. Personen, die ein sehr beschränktes Verständnis von Programmierung haben sind deshalb häufig überzeugt, dass kompilierte Sprachen sicher, interpretierte Sprachen dagegen unsicher \\wären. Tatsächlich handelt es sich schlicht um zwei Paradigmen, die unterschiedliche Vor- und Nachteile haben. Und wenn wir von \textbf{IT-Sicherheit}\index{Sicherheit!IT-Sicherheit}\index{IT-Sicherheit} sprechen, dann hat das nichts aber auch rein gar nichts mit kompilierten Sprachen zu tun.
	\item Ein Debugger ist eine Software, die Ihren Quelltext auf Fehler untersucht.
	\item Ein \textbf{SDK}\index{SDK} (kurz für Software Development Kit bzw. Softwareentwicklungskit) ist eine Softwaresammlung, die neben einem Interpreter \\und/oder Compiler eine Reihe weiterer Programme beinhaltet, die Sie bei der Entwicklung von Programmen unterstützt. SDKs werden jeweils für eine bestimmte Sprache entwickelt. Gelegentlich wird auch der Begriff \textbf{Toolchain}\index{Toolchain} verwendet. Im hier verwendeten Sinne entspricht eine Toolchain einem SDK.
	\item Eine \textbf{IDE}\index{IDE} (kurz für Integrated Development Environment bzw. Integrierte Entwicklungsumgebung) ist ein Sammlung von Programmen, die meist unabhängig von einer bestimmten Sprache zum Programmieren nutzbar sind. Im Gegensatz zu einer SDK beinhalten IDEs in aller Regel einen eigenen Editor, der u.a. Syntaxerkennung bietet. Mehr noch: Viele IDEs sind so entwickelt, dass es möglich ist, eine Vielzahl von SDKs in ihnen zu nutzen und damit eine Vielzahl von Programmiersprachen in Ihnen zu nutzen.
\end{itemize}

Die nachfolgenden Werkzeuge gehören in den fortgeschrittenen Bereich, mit dem Sie im Rahmen dieses Buches nichts zu tun haben werden. Sie sollten Sie allerdings kennen, damit Sie wissen, wonach Sie später suchen müssen, um Probleme zu vermeiden, die sehr viel Zeit kosten können. Ihre Beherrschung unterscheidet professionelle Software Entwickler von Quereinsteigern.\\

\begin{itemize}
	\item \textbf{Deployment Support/Tools}\index{Deployment!support}\index{Deployment!tools} sind Werkzeuge, die Sie dabei \\unterstützen, ein Programm zu verteilen. Stellen Sie sich dazu vor, dass Sie ein Programm entwickeln, das von allen 15.000 Mitarbeitern Ihres Unternehmens genutzt wird. Ohne Deployment Support/Tool müssten Sie nun an jeden Arbeitsplatz gehen, die alte Version deinstallieren, u.U. den Rechner neustarten und dann die neue Version einspielen. Und wenn Sie fertig sind, gehen Sie in Rente. Mit Deployment Support/Tools dagegen lassen Sie lediglich die Änderungen (auch als \textbf{Delta}s\index{Delta} bezeichnet) automatisiert von einem zentralen Server aus an die Rechner verteilen und dort installieren.
	\item \textbf{Refactoring}\index{Refactoring} bezeichnet eine Überarbeitung einer Software. Hier geht es aber nicht darum, einzelnen Fehler zu beheben, sondern darum, strukturelle Änderungen in eine Software einzuarbeiten. (Hier wären wir bei einem weiteren Aspekt des Software Engineering angelangt.)
\end{itemize}

Die folgenden Abschnitte enthalten jeweils die Informationen, die sie zur Vorbereitung der Programmierung auf einem Windows-Rechner benötigen, da auf den Rechnern unseres Departments Windows zum Einsatz kommt. Für die Linux- und MacOS-User unter Ihnen gibt es hier leider keine bzw. nur wenige Hinweise. Nutzen Sie bitte über die üblichen Quellen im Netz.

\section{Assembler und C – Vorbereitung für die maschinennahe und imperative Programmierung}

In den Veranstaltungen zur maschinennahen und imperativen Programmierung unseres Departments kommen zurzeit \textbf{ARM Cortex-M0 Prozessoren} zum Einsatz. Bei diesen Prozessoren können Sie von der Webpage von IAR ein kostenloses SDK herunterladen. Im Rahmen der Veranstaltung Informatik 3 (für Media Systems) erhalten Sie außerdem Informationen darüber, wie Sie Ihr System einrichten können und wie Sie die dort verwendete IDE beziehen können. Diese stammt von der Fa. Keil, heißt \textmu Vision und wird auch als \textbf{MDK}\index{MDK}\index{SDKs!MDK} bezeichnet. Weiterhin benötigen Sie für die Entwicklung von Software für einen ARM Prozessor ein Entwicklerboard, auf dem der Prozessor bereits montiert ist, sowie eine debugging Unit. In beiden Fällen handelt es sich um Hardware, die Sie im Labor für Ihre Versuche erhalten.\\

Wenn Sie so lange nicht warten wollen und wesentlich weniger Geld ausgeben wollen, um sich in die Programmierung von ARM-Prozessoren einzuarbeiten, dann möchte ich Ihnen einen Computer mit ARM-Prozessor empfehlen, der von der Universität in Cambridge extra für Studienanfänger entwickelt wurde. Es handelt sich um den \textbf{Raspberry Pi} (kurz RasPi oder Pi), der in der Version 2 für rund 45,- € zu haben ist. (Maus und Tastatur sowie eine MicroSD-Card als Laufwerk und ein passendes Ladekabel fehlen hier noch, aber für knapp 70,- € sollten Sie ein komplettes System bekommen.) Das praktische am RasPi ist, dass Sie ihn in die Tasche stecken können.\\

Wenn Sie dagegen die maschinennahe Programmierung auf einem Intel- oder AMD-Prozessor mit IAx86-Architektur\footnote{Das sind die Prozessoren, die zurzeit üblicherweise in Desktop-Rechnern verbaut werden.} durchführen wollen, können Sie sich die \textbf{GCC} (kurz für \textbf{GNU Compiler Collection} ) kostenlos herunterladen und installieren. Hierbei handelt es sich um eine Sammlung von Compilern für die verschiedensten Sprachen. Im Gegensatz zu vielen anderen Compilern erhalten Sie die GCC unter der sogenannten \textbf{GNU GPL} (Lizenz). Kurz gesagt bedeutet das, dass Sie hierauf kostenfreien Zugang haben, dass die Entwickler Ihnen darüber hinaus aber noch wesentlich mehr Rechte einräumen. Die GCC sind in verschiedenen Paketen enthalten, die eine etwas komfortablere Installation erlauben.\\

\textbf{Windows}-Nutzern sei hier das \textbf{MinGW}\index{MinGW} empfohlen, das zusätzlich zum GCC auch ein minimalistisches GNU installiert. Mit letzterem können Sie unter Windows auf der Konsole wie in einer Linux-Umgebung arbeiten. Alternativ können Sie auch auf Microsofts \textbf{Visual Studio}\index{IDEs!Visual Studio}\index{Visual Studio} Professional oder Ultimate zurückgreifen, das Sie als Studierende kostenlos über das Dreamspark Programm erhalten. Davon gibt es noch die Community Edition, die grundsätzlich kostenlos von Microsoft angeboten wird. Beachten sie aber bitte, dass Sie in diesem Fall unter Umständen Software entwickeln, die ausschließlich auf Windows Rechnern lauffähig ist. Für den Aufruf des GCC-Assembler Compilers müssen Sie abschließend noch die PATH-Variable von Windows um das /bin-Verzeichnis Ihrer MinGW-Installation erweitern.\\


\textbf{MacOS}-Nutzer brauchen eine derart umfangreiche Software nicht, da Ihr Betriebssystem bereits die entsprechenden Werkzeuge an Bord hat. Allerdings müssen Sie immer noch einen Assembler Compiler installieren, wozu Sie am besten ebenfalls auf die GCC zurückgreifen. Einsteiger werden zuvor noch \textbf{XCode}\index{XCode}\index{IDEs!XCode} installieren müssen. Dabei handelt es sich um eine IDE, die Sie über den App Store herunterladen können. Suchen Sie anschließend im Netz nach einer Anleitung, wie Sie die GCC in XCode einbinden können. Unter MAC OS X wählten Sie 2010 im XCode\\ \verb|Menü -> Preferences -> Downloads| den Eintrag "`Command Line Tools" und hatten nach dem Abschluss des Downloads die GCC vollständig installiert. Die Installation sollte also auch Einsteigern problemlos möglich sein. Genau wie bei Microsofts Visual Studio sollten Sie allerdings daran denken, dass Sie bei der Nutzung von XCode unter Umständen Software entwickeln, die ausschließlich auf Apple-Rechnern lauffähig ist.\\

Warum es hier keine Informationen für \textbf{Linux}-User gibt? Weil Linux-User im Regelfall selbständig genug sind, um sich diese Informationen selbst zu beschaffen. Sollten Sie Linux-Neuling sein, empfehle ich Ihnen die Linux Einführung auf der Plattform edX. Dieser Kurs kann wie die meisten Kurse auf edX auch kostenlos belegt werden und wird im Gegensatz zu den meisten Kursen kontinuierlich angeboten; Sie sind hier also nicht an einen bestimmten Zeitraum gebunden. Er wurde von der FSF entwickelt und ist nach anfänglichem Marketing für die FSF eine fundamentale und sehr nützliche Einführung in das offene und freie Betriebssystem. Leider kommt GNU ein wenig zu kurz, aber Sie erhalten zumindest die entsprechenden Links.\\

Wenn Sie einen RasPi erworben haben, dann gibt es eine Einführung in Linux im sogenannten \textbf{Raspberry Pi Educational Manual} \\ \url{http://vx2-downloads.raspberrypi.org/Raspberry_Pi_} \\ \url{Education_Manual.pdf}, das zwar für die erste Version des RasPi entwickelt wurde, aber auch bei der aktuellen Version genutzt werden kann, da diese weitgehend abwärtskompatibel ist. Darin ist neben der Einführung in die Grundlagen verschiedener Arten der Programmierung auch eine Einführung in die Administration von Linux enthalten. Die sollten Sie als RasPi-Nutzer auf jeden Fall durcharbeiten.\\

Außerdem sollten Sie noch ein gutes Lehr- und Nachschlagewerk für die Assemblerprogrammierung erwerben. Hier gibt es allerdings auch einige Bände, die Sie legal kostenlos herunterladen dürfen. Grundsätzlich möchte ich Ihnen für diese Fälle die folgende Seite ans Herz legen:\\ \url{http://hackershelf.com/topics/} \\ Suchen Sie hier bitte unter dem Suchbegriff \textbf{assembly}, nicht assembler, da im Englischen die Bezeichnung nicht Assembler Language sondern Assembly Language lautet.\\

Persönlicher Tipp: Greifen Sie dort zu "`\textbf{Programming from the Ground up}" von Jonathan Bartlet.\\

Sie wollen wissen, was mit GNU gemeint ist? Gute Frage. Aber die Antwort lautet: Lesen Sie es selbst nach: \url{https://www.gnu.org/} Es geht hier letztlich um eine der wichtigsten Institutionen, die Sie als angehende InformatikerInnen kennen sollten: Die \textbf{FSF}\index{FSF} (kurz für Free Software Foundation). Ob nun die \textbf{GI}\index{GI} (kurz für Gesellschaft für Informatik), \textbf{IEEE}\index{IEEE}, die FSF oder \textbf{ITU-T}\index{ITU-T} die absolute Spitzenposition bezüglich der Bedeutung für InformatikerInnen einnimmt, kann ich Ihnen nicht sagen, aber sie sind allesamt außerordentlich wichtig. Beschäftigen Sie sich deshalb damit. \\

\textbf{Ergänzungen zu C und C++}\\

Leider befand sich zum Zeitpunkt, als diese Zeilen geschrieben wurden noch kein für Einsteiger empfehlenswertes Buch auf hackershelf.com. Allerdings gibt es für die Programmierung in C ein Standardwerk, das u.a. vom Entwickler der Sprache verfasst wurde: "`\textbf{The C Programming Language}" von \textbf{Ritchie} und \textbf{Kernighan}.

\section{Pascal - Eine weitere imperative Sprache}

Um in Pascal zu programmieren installieren Sie bitte den \glqq{}Free Pascal Compiler\grqq{} sowie die IDE \glqq{}Geay\grqq{}.

\section{Scheme - Eine funktionale Programmiersprache}

In einführenden Kursen wird Scheme regelmäßig als interpretierte Sprache vermittelt. Ein bekannter kostenloser Interpreter ist \glqq{}Petite Chez Scheme\grqq{}. (Bitte nicht mit Chez Scheme verwechseln; dabei handelt es sich um eine kostenpflichtige Version, die neben dem Interpreter einen Compiler beinhaltet.)

\section{PROLOG - Logische Programmierung}

Im Gegensatz zu den bisher aufgeführten Sprachen ist PROLOG eine interpretierte Sprache. Ein kostenloser Interpreter ist \glqq{}SWI-PROLOG\grqq{}, den Sie bitte ebenfalls installieren.

\section{Java - Vorbereitung für die Programmierung in Java}

\textbf{Wichtig für Wiederholer}: Stellen Sie sicher, dass Sie auf jedem Rechner die gleiche Version von Java installiert haben. Zwar ist Java weitgehend abwärtskompatibel, aber die Entwickler der Sprache erklären immer wieder einzelne Bestandteile für veraltet. Wenn Sie dann noch eine alte JDK verwenden, bedeutet das, dass Sie ggf. aktuellen Code nicht programmieren können und veraltete Vorgehensweisen vom Compiler akzeptiert werden. Gerade bei Hausaufgaben führt das dann schnell zu Problemen.\\

Für Java gibt es zunächst zwei Anbieter von SDKs. Zum einen können Sie eines der offiziellen SDKs von Oracle herunterladen, zum anderen gibt es das OpenJava, das unter einer anderen Lizenz entwickelt wird.\\

Auf der Webpage des Entwicklers Oracle gibt es Java2SE, Java2EE, Java for Mobile und noch ein gutes Dutzend weiterer Java Downloads. Das was Sie für den Einstieg brauchen ist \textbf{Java2SE}. Zum Grund dafür kommen wir am Ende dieses Abschnitts, für die Installation brauchen Sie es nicht zu wissen.\\

Nachdem Sie sich mit den Nutzungsbedingungen einverstanden erklärt haben, können Sie auf eine Reihe an Downloads zugreifen, die sich dadurch unterscheiden, ob Sie nun einen 32- oder 64-Bit-Prozessor haben, welches Betriebssystem Sie nutzen und ob Sie die Installationsdatei \\ vollständig (offline) oder nur zum geringen Teil (für eine online-Installation) herunterladen wollen. Wenn Sie sich die Zeit nehmen, genau hinzusehen, dann sollten Sie hier die für Ihr System optimale Version wählen können. Im Grunde ist es bei Windows, Linux oder MacOS aber egal, welche Version (32/64 Bit bzw. online/offline) Sie wählen, so lange das Betriebssystem und der Prozessor stimmen.\\

Einen Buchtipp für die Programmierung in Java kann an dieser Stelle nicht gegeben werden, was auch daran liegt, dass mit jeder neuen Version wieder essentielle Änderungen in die Sprache eingeführt werden. Dazu kommt, dass Java durch seinen Middleware-Charakter ohnehin derart umfangreich ist, dass kein Buch existieren kann, das auch nur größtenteils vollständig ist. Dazu ist das Framework schlicht zu umfangreich. Sie glauben mir nicht? Dann werfen Sie einen Blick in die Java API; das ist die vollständige Java Dokumentation: \url{http://docs.oracle.com/javase/8/docs/api/} \\ Und dort sind lediglich all die Klassen enthalten, die von Oracle selbst angeboten werden. Am besten setzten Sie in Ihrem Browser auch gleich ein Lesezeichen auf diese Seite, denn beim Lernen von und Arbeiten mit Java werden Sie wohl nichts so oft brauchen wie diese Seite. Sollte inzwischen die nächste Version von Java veröffentlicht worden sein, dann ändern Sie bitte die entsprechende Ziffer im Link ab.\\

Bei dieser Gelegenheit lernen Sie auch schon die nächste wichtige Abkürzung kennen: \textbf{API}\index{API} steht kurz für \textbf{Application Programming Interface}, was als \textbf{Programmierschnittstelle}\index{Programmierschnittstelle} übersetzt wird. Eine API galt als ein typisches Kennzeichen objektorientierter Programmiersprachen, doch das ist falsch: Eine Vielzahl an Lösungen wurden bereits von anderen Entwicklern programmiert und intensiv getestet, sodass Sie sich darauf konzentrieren\\ können, die Spezialfälle zu programmieren, die bei Ihrer Software auftreten. Tatsächlich finden Sie so etwas jedoch selbst für Assembler. Der Begriff \textbf{Klassenbibliothek}\index{Klassenbibliothek} wird des Öfteren als Synonym für eine API verwendet. An sich steht dieser Begriff aber allgemeiner für Sammlungen von Klassen, die Lösungen für häufig auftretende Probleme beinhalten.\\

Da der Begriff der \textbf{Klasse}\index{Klasse} untrennbar mit Java verbunden ist, hier eine kurze Erklärung, die zwar viel zu simpel ist, aber für den Anfang genügen soll: In Java bestehen Programme wie in alle objektorientierten Programmiersprachen aus sogenannten Modulen, die in Java als \textbf{Package}s und Klassen bezeichnet werden. Jedes dieser Module beinhaltet einen Programmteil, der für sich genommen bereits ein einen sinnvollen Teil des Gesamtprogramms erfüllen kann. Um bei besonders großen Projekten mehr Übersicht zu erhalten werden dort mehrere Klassen in einem Package gesammelt. Packages können Sie sich in Java wie Verzeichnisse bei einem Betriebssystem vorstellen. Genau wie dort kann ein Package mehrere Packages enthalten, die sowohl Klassen als auch Packages enthalten können.\\

Anschließend müssen Sie u.U. unter Windows (ähnlich wie bei der Vorbereitung für die maschinennahe Programmierung) noch die PATH-Variable erweitern. In diesem Fall müssen Sie dort das Verzeichnis angeben, in dem u.a. die Datei javac liegt. Prüfen Sie anschließend über den Befehl javac –v in der Eingabeaufforderung, ob Sie die PATH-Variable richtig gesetzt haben. Anmerkung für einen späteren Umstieg auf Linux: Dort heißt die „Eingabeaufforderung“ \textbf{Terminal}\index{Terminal}, \textbf{Konsole}\index{Konsole} oder \textbf{shell}\index{shell}, wird aber auch als bash, sh, z-sh und ähnliches bezeichnet, wobei das teilweise konkrete Programmnamen sind, die ein Terminal bzw. eine Konsole starten.\\

Jetzt aber zur Antwort auf die Frage, was denn die verschiedenen Java-Versionen unterscheidet. Hier werden wir nicht zu sehr ins Detail gehen, daher nur so viel: 

\begin{itemize}
	\item Warum steht da eine 2 in Java2SE?\\
	Obwohl mit jeder neuen Version (Java ist im Moment, in dem diese Zeilen geschrieben werden bei Version 8 angelangt) essentielle\\ Änderungen an der Sprache durchgeführt werden, entschieden sich die Entwickler nach Version 2 die Zahl 2 im Namen zu behalten. Das ist alles.
	\item Wir brauchen ja das SDK. Aber was ist die JRE?\\
	Sie wissen bereits, dass Sie ein Softwareentwicklungskit zum Entwickeln von Software nutzen können. Dieses beinhaltet eine Laufzeitumgebung (in diesem Fall die JRE, kurz für Java Runtime Environment), die dazu dient, Java Programme auf einem Computer laufen zu lassen, auf dem kein JDK installiert ist.\\
	Laufzeitumgebungen gibt es für eine Vielzahl von Programmiersprachen. Es handelt sich dabei um so etwas wie eine Übersetzungs-\\software, die es dem Betriebssystem ermöglicht, Programme in der jeweiligen Sprache auszuführen.\\
	Und richtig verstanden: Kein Programm wird direkt von einem Betriebssystem ausgeführt, sondern das Betriebssystem startet gegebenenfalls die Laufzeitumgebung für eine Sprache, in der das zu startende Programm dann ausgeführt wird. Ein häufiger Irrtum von Nutzern besteht dagegen darin, anzunehmen, dass beispielsweise unter Windows Dateien mit der Endung .exe von Windows selbst ausgeführt werden; vielmehr gilt auf für Dateien im .exe-Format das gleiche, was für alle Programm gilt, die ausgeführt werden sollen: Es muss auf dem Rechner eine Laufzeitumgebung (oder ein Interpreter) für die entsprechende Sprache installiert sein, sonst kann das Programm nicht ausgeführt werden. Einzige Ausnahme: Ein Programm liegt in Maschinensprache vor. Dann und nur dann kann es direkt vom Prozessor ausgeführt werden.\\
	\item Was bedeutet dieses SE in Java2SE? Was ist dieses Java2EE? Und was sollen all die anderen Versionen?
	\begin{itemize}
		\item \textbf{Java SE} (kurz für Standard Edition) ist gewissermaßen der Kern der Sprache/der Middleware. Es enthält alles, was Sie benötigen, um Java Programme zu entwickeln bzw. um sie auf einem Rechner laufen zu lassen.
		\item \textbf{Java EE} (kurz für Enterprise Edition) ist eine erweiterte Fassung von Java SE, die in Unternehmen eingesetzt werden kann, bei denen es wenigstens einen zentralen Server gibt, über den die Kommunikation von Java Anwendungen koordiniert wird.
		\item \textbf{Java ME} (kurz für Micro Edition) ist dagegen eine höchst effiziente Version, die auf Geräten mit geringen Kapazitäten zum Einsatz kommen kann. (Alte Rechner, Smartphones, usw.)
		\item \textbf{Java FX} (kurz für Effects) beinhaltet Klassen, mittels derer Internetapplikationen entwickelt werden können.
	\end{itemize}
\end{itemize}

Nachdem Sie das JDK heruntergeladen und installiert haben, brauchen Sie noch eine Möglichkeit, um Java Programme einzugeben oder zu ändern, denn das JDK enthält keinen eigenen Editor. Hier genügt für die ersten Schritte ein \textbf{einfacher Texteditor}. Verwechseln Sie das bitte nicht mit einer \textbf{Textverarbeitung}: Ein Texteditor zeigt Ihnen den eingegebenen Text standardmäßig ohne irgendwelche Hervorhebungen wie Fettdruck und ähnliches. Im Gegensatz dazu zeigt Ihnen eine Textverarbeitung Texte mit Hervorhebungen, bei denen dann der Teil des Quelltexts ausgeblendet wird, über den diese Hervorhebungen erzeugt werden.\\

Standardmäßig hat jedes Betriebssystem mindestens einen Texteditor an Bord. Bei Windows finden Sie diesen unter Zubehör bzw. indem Sie bei Windows 8 in der "`Kachelansicht" schlicht das Wort Editor eintippen. Bei älteren Windows-Versionen können Sie das Suchfenster im Windowsmenü nutzen.\\

Als ersten Schritt zu mehr Komfort gibt es Texteditoren mit \textbf{Syntaxerkennung}\index{Syntaxerkennung}. Beispielsweise den Notepad++. Solche Texteditoren heben automatisch syntaktische Fehler hervor, indem sie z.B. eine rote Linie unterhalb fehlerhaften Stellen anzeigen. Sie ändern dabei aber im Gegensatz zu Textverarbeitungen nichts am Quellcode.\\

Fortgeschrittene werden Ihnen jetzt empfehlen, doch sofort zu einer IDE wie \textbf{Eclipse}\index{IDEs!Eclipse} zu greifen, da die doch so viel besser zum Programmieren seien. Doch bitte lassen Sie das vorerst: Wie schon zuvor geschrieben haben selbst kostenlose IDEs inzwischen einen derartigen Umfang an Komfortfunktionen, dass sie für Einsteiger eher verwirrend als hilfreich sind. Es ist aber richtig: Nach spätestens sechs Monaten sollten Sie auf jeden Fall damit beginnen, eine IDE zu nutzen.

\section{HTML, PHP und MySQL – Vorbereitung für die Entwicklung verteilter Anwendungen}

Wenn Sie eine Webpage entwickeln wollen, dann benötigen Sie dazu folgende Dinge:

\begin{enumerate}
	\item Einen einfachen Texteditor wie \textbf{Notepad++}.
	\item Ein Softwarepaket, das Ihnen die nötige Infrastruktur für eine webbasierte Anwendung bietet. Beispiele: \textbf{XAMPP} oder \textbf{EasyPHP}
	\item Eine Software, die es Ihnen ermöglicht, Dateien auf einen Webserver zu übertragen. Ein einfaches kostenloses Programm ist \textbf{FileZilla}.
\end{enumerate}

Im Gegensatz zur imperativen Programmierung müssen Sie sich hier nur um wenige Dinge kümmern, obwohl hier tatsächlich wesentlich mehr Installationsschritte zu erledigen sind, als das bei C, C++ oder Java der Fall ist. Denn all die nötigen Schritte übernimmt hier die Installationsroutine von XAMPP bzw. EasyPHP.\\

Da Sie alle drei Programme im Netz legal kostenlos beziehen können und die Installation keine fortgeschrittenen Kenntnisse erfordert, sei dazu an dieser Stelle nichts weiter gesagt.\\

Sie sollten allerdings noch wissen, welche Sprachen und Strukturen hier zum Einsatz kommen. Wie Sie zu Beginn des Kapitels über verteilte Anwendungen erfahren werden, werden Sie diese Punkte später in Veranstaltungen wie "`Einführung in relationale Datenbanken" ausführlich kennen lernen.

\begin{enumerate}
	\item Zunächst benötigen Sie für eine Webanwendung einen Server, der die Daten der Seite vorhält und über das Internet erreichbar ist.\\
	An dieser Stelle empfehle ich Ihnen die Installation von EasyPHP. EasyPHP wird (genau wie XAMPP) Ihren Rechner so vorbereiten, dass der Apache Server auf Ihrem Rechner läuft und als Webserver genutzt werden kann. Aus Sicherheitsgründen sollten Sie diesen Server aber nur so lange laufen lassen, wie Sie ihn zum Testen benötigen. Denn wenn Sie keine fortgeschrittenene Kenntnisse in IT-Sicherheit und Serveradministration haben, werden Sie sonst mit höchster Sicherheit Sicherheitslücken auf Ihrem Rechner einrichten.\\
	\item Dann benötigen Sie eine Sprache, mittels derer Sie die Elemente Ihrer Webanwendung definieren können. Über diese Sprache regeln Sie also, wie die einzelnen Bestandteile der Webpage zu einzelnen Ansichten verbunden werden. Hier greifen wir zu \textbf{HTML}\index{Programmiersprache!HTML}, für das Sie im Gegensatz zu C, C++ oder Java keinen Compiler oder Interpreter installieren müssen: Den HTML-Interpreter finden Sie bereits in Form eines Webbrowsers auf praktisch jedem Rechner vor.\\
	HTML hat allerdings einen entscheidenden Nachteil für unsere Zwecke: Die Sprache ist durchgehend statisch. Wenn Sie also Änderungen auf der Seite einführen wollen, die beispielsweise von den Eingaben des Nutzers abhängen, dann können Sie das mit HTML alleine nicht erreichen.\\
	\item Deshalb folgt als nächstes eine imperative Sprache, wobei wir hier \textbf{PHP}\index{Programmiersprache!PHP} wählen, das im Gegensatz zu C und ähnlichen dynamisch typisiert ist. Sie werden sich erinnern: Das bedeutet, dass der Interpreter der Sprache vieles für Sie übernimmt, dass Sie also weniger Details selbst programmieren müssen. Dafür müssen Sie aber umso genauer bzw. umso mehr darüber Bescheid wissen, wie der Interpreter genau arbeitet.\\
	PHP wird wird von vielen Webservern interpretiert. In unserem Fall übernimmt das der Apache Server, der Teil der Easy-PHP- bzw. der XAMPP-Installation ist.\\
	Mittlerweile wird allerdings JavaScript mehr und mehr zu einem \\ernstzunehmenden Konkurrenten von PHP, weil es wesentlich mehr Möglichkeiten bietet und eine Vielzahl Beschränkungen dort nicht existieren. In HTML5 ist es als Standard für die Programmierung der Funktionalität einer Webanwendung eingestellt, sodass Sie hier zum Testen nicht unbedingt einen Webserver benötigen.\\
	\item Außerdem brauchen wir noch eine Möglichkeit, um Daten langfristig zu speichern. Denn der Apache Server stellt zwar unsere Webpage bereit, HTML realisiert mit CSS die Darstellung und PHP realisiert die dynamische Nutzbarkeit der Anwendung, aber wenn wir nur diese Sprachen nutzen, können wir Nutzereingaben nicht dauerhaft speichern und wieder verwenden.\\
	Gerade wenn wir eine Webanwendung entwickeln, über die wir Verträge abschließen wollen (egal, ob es dabei um ein browserbasiertes Spiel, einen Webshop oder was auch immer geht), genügt das nicht. Und das führt uns direkt zu einer Datenbank. Datenbanken werden über eigene Sprachen angesprochen, die häufig ein SQL vorkommt. SQL steht kurz für \textbf{Structured Query Language}, übersetzt sind das also standardisierte Abfragesprachen. Und hier ist \textbf{MySQL}\index{Programmiersprache!MySQL} die Sprache, die wir im Kurs verwenden.\\
	Aber auch hier brauchen Sie wieder keinen Interpreter oder Compiler herunterladen; diese Aufgabe übernimmt bereits XAMPP bzw. EasyPHP für Sie.
\end{enumerate}

Trotz der komfortablen Installation z.B. durch EasyPHP müssen Sie sich all diese Komponenten merken, weil Sie bei einer professionellen Entwicklung all diese Komponenten selbst installieren und konfigurieren müssen. Und schließlich geht es im Studium darum, dass Sie sich auf eine professionelle Tätigkeit vorbereiten.

\section{Textverarbeitung mit LaTeX}

Um umfangreiche wissenschaftlichen Texte zu erstellen wird häufig die Textverarbeitung LaTeX verwendet, die aus Sicht der Programmierung eine Markup Language ist. Ein guter Einstieg, um etwas über den Hintergrund zu erfahren und Teil der Community zu werden ist \url{http://www.latex-project.org/} . Installieren Sie bitte die Software \verb|TeXstudio|, um mit der Programmierung in LaTeX zu beginnen.

\section{Alle - Vorbereitung für die Teamarbeit}

Wie im ersten Kapitel aufgeführt ist die Arbeit an einer Software heute eine Aufgabe, die in Teams durchgeführt wird. Und hier müssen Sie Werkzeuge nutzen, die über einen Server koordiniert werden, um auch nur ansatzweise effizient zu sein. Wenn sie dagegen die Vorstellung haben, Ihren Teil der Software zu entwickeln, dann eine Kopie davon an Ihre Teamkollegen zu verschicken und sich anschließend die Kopien der Arbeitsergebnisse der Kollegen zu besorgen, dann sollten Sie sich schleunigst von diesem Dilletantentum verabschieden! So arbeiten ausschließlich Personen, die nichts aber auch gar nichts von Softwareprojekten verstehen.\\

Die aktuell effizienteste Möglichkeit, um ein solches Softwareprojekt über eine SCM zu verwalten heißt \textbf{Git}\index{Git}. Und diese Möglichkeit ist nicht nur außerordentlich effizient und praktisch, sondern obendrein noch kostenlos. Deshalb werden wir Sie hier von Anfang an einsetzen, damit Sie am Ende Ihres Studiums nicht einmal ansatzweise auf die Idee verfallen, Änderungen an Softwareprojekten per Kopie zu verteilen.\\

Im Arbeitsleben werden Sie allerdings häufig auf eine ältere Form des SCM treffen. Die Bezeichnung dafür lautet \textbf{SVN}\index{SVN} bzw. \textbf{Subversion}\index{Subversion}. Subversion hat deutliche Nachteile gegenüber Git, ist aber immer noch weit verbreitet, weil viele Nutzer schlicht nicht willens sind, sich in die Änderungen einzuarbeiten, die mit einem solchen Umstieg verbunden sind. Media Systems Studierende sollten sich deshalb in jedem Fall später (z.B. in der Veranstaltung Software Engineering) auch in Subversion einarbeiten.\\

Wie schon im ersten Kapitel beschrieben besteht der große Vorteil von Git darin, dass Sie auf das Repository keinen Zugriff haben müssen, um an der Software zu arbeiten; Subversion geht im Kern davon aus, dass Sie sich in einem Unternehmensnetzwerk befinden und deshalb ständigen Zugriff auf das Repository haben.\\

Neben Git gibt es noch eine Webplattform mit dem Namen \textbf{GitHub}. Git ist an Software das einzige, das Sie installieren müssen, um ein SCM zu beginnen. Für die Speicherung des Repository können Sie dann einen beliebigen Rechner nutzen, so lange dieser über das Internet oder ein anderes Netzwerk regelmäßig erreichbar ist. GitHub selbst ist ein Service, der Ihnen anbietet, dass Sie dort Ihre Repositories lagern. Ähnlich wie andere Cloud-Dienste gilt auch hier, dass Sie bei einem kostenlosen Account keine privaten Repositories erhalten. U.U. gibt es hier aber für Studierende Sonderangebote.\\

Wichtig: Sollten Sie GitHub (oder einen ähnlichen Dienst) nutzen und Ihre Repositories öffentlich sein, dann bedeutet das auch, dass Ihre Projekte für jedermann/-frau frei zugänglich sind. Nutzen Sie dagegen Git mit einem Server, für den Sie Zugangsbeschränkungen erlassen können, dann gilt das natürlich nicht (automatisch).\\

Das praktische bei Git ist, dass Sie gar keinen Server für die Repositories brauchen, um anzufangen: Sobald Sie die Software heruntergeladen (\url{https://git-scm.com/} ) und installiert haben, können Sie jedes beliebige Verzeichnis unter die Versionskontrolle stellen und können damit arbeiten, als gäbe es bereits ein Repository. Sobald Sie den nötigen Web- bzw. Cloudspace haben, können Sie Ihr/e Projekt/e dort in ein Repository einbinden, bzw. aus ihrem/n Projekt/en heraus ein Repository anlegen.\\

Die Einführung in die Arbeit mit Git auf der Webpage git-scm.com ist sehr gut, weshalb an dieser Stelle keine weiteren Erläuterungen erfolgen. Nur so viel: Arbeiten Sie sich dort ein, damit Sie von Beginn an damit arbeiten können.

\section{Alle – Nutzung des Netzlaufwerks zur Speicherung eigener Daten}

Leider wissen viele Studierende nicht, dass ihnen in der Hochschule ein Netzlaufwerk zur Verfügung steht, auf dem Sie alle Daten speichern können und speichern deshalb Ihre Ergebnisse "`auf dem Rechner" an dem Sie in der Hochschule sitzen. Dabei werden die Daten aber gerade nicht auf dem Rechner gespeichert, an dem sie sitzen, sondern in ihrem Rechnerprofil. Und jedes Mal, wenn Sie sich in der Hochschule an einem Rechner anmelden wird dieses Profil über das Netzwerk auf den Rechner übertragen, an dem Sie sich anmelden. Wir hatten schon Studierende, die deshalb eine halbe Stunde und länger an ihrem Rechner saßen, bis sie endlich den Desktop sahen. Aber das liegt nicht etwa daran, dass die Rechner oder das Netzwerk langsam wären, sondern einzig und alleine daran, dass diese Studierenden zum Teil 30 und mehr Gigabyte an Daten in Ihrem Rechnerprofil gespeichert hatten. Nutzen Sie deshalb bitte für alle Arbeiten am Rechner Ihr Netzlaufwerk. Denn die Daten, die dort gespeichert werden werden erst dann auf Ihren aktuellen Rechner übertragen, wenn Sie sie benötigen. Und keine Sorge: Die Daten werden nicht über das WLAN übertragen, sind also im Regelfall so schnell auf Ihrem Rechner, als hätten Sie sie auf einem USB-Stick gespeichert.\\

An dieser Stelle auch ein Hinweis auf die Nutzung des WLANs: Leider verstehen viele Studierende nicht, dass jeder Aufruf einer Webpage eine gewisse Datenmenge über das Netzwerk überträgt, über das sie mit dem Internet verbunden sind. Und wenn nun fünfzig Studierende gleichzeitig ein Dutzend YouTube-Videos starten oder den neuesten Client für World of Warcraft herunterladen, dann bedeutet das eben, dass das Netz bis an die Grenze ausgelastet ist. Das verstehen Sie nicht? Nun, es ist genau wie im Straßenverkehr: Wenn jede/r mit dem Auto zur Arbeit fährt, dann kommt eben keine/r mehr richtig voran. Und dieser Vergleich entspricht genau dem, was beim Surfen im Netz das Problem ist: Die meisten Nutzer verhalten sich, als wenn sie alleine auf der Welt wären und als wenn Ressourcen unbegrenzt vorhanden wären. Wenn es dann einen Engpass gibt, dann heißt es immer: Das Netz ist schlecht, die Stadt soll mehr Straßen bauen, usw. usf. Doch die Ressourcen sind begrenzt. Ja, das gilt auch in den LTE-Netzen, egal was der Verkäufer Ihres Handy-Providers Ihnen versprochen hat.


\section{Änderungen}

\begin{itemize}
	\item 9. März 2016: Hinzufügen des Abschnitts zur Vorbereitung für die Programmierung der Markup Language/Textverarbeitung LaTeX.
\end{itemize}

\chapter{Vorbereitung fürs Programmieren}
%\chapter{Ausgewählte Programmiersprachen}
In diesem Kapitel stelle ich Ihnen einige Programmiersprachen vor und erkläre, wo die Unterschiede liegen.\\

Der Begriff der höheren Programmiersprachen wird heute eigentlich nur noch am Rande verwendet, weil die Abgrenzung zur maschinennahen Programmierung (und genau dafür steht der Begriff) zum Normalfall geworden ist. Sollten Sie also über diesen Begriff stolpern und sich fragen, was denn eine höhere Programmiersprache ist, dann merken Sie sich einfach: Es ist eine Abgrenzung, die für uns weitestgehend irrelevant geworden ist.

\section{Nach B kam C}
Die mangelnde Verständlichkeit von maschinennahen Programmen führte dazu, dass Programmiersprachen entwickelt wurden, die Befehle und Zeilenstrukturen beinhalteten, die leichter lesbar waren.\\

Die Zeile \\

\verb~if (a < b) then print "a ist kleiner als b"~\\

dürfte auch von Menschen lesbar sein, die lediglich über grundlegende Englischkenntnisse, aber kaum über Computerkenntnisse verfügen.\\

Im Gegensatz dazu dürfte die Zeile\\

\verb~CMP R6 MSP~\\

selbst bei denjenigen unter Ihnen für Stirnrunzeln sorgen, die bereits Projekte in Java oder C++ entwickelt haben. (Hier handelt es sich um eine maschinennahe Programmzeile, die bei einem ARM-Prozessor einen Vergleich zwischen zwei Zahlen durchführt.)\\

Eine dieser höheren Sprachen wurde von ihrem Entwickler \textbf{Dennis Ritchie}\index{Wichtige Personen!Ritchie, Dennis} schlicht \textbf{C}\index{Programmiersprache!C} genannt. Es gab vorher unter anderem eine Sprache namens B, die als Vorlage für C diente. Die Informatiker der Anfangszeit waren weniger an Marketing interessiert, weshalb Bezeichnungen wie Java, Ruby, Python usw. erst ab den 90er Jahren üblich wurden. Vorher wurden häufig einzelne Buchstaben oder Abkürzungen wie im Falle der Sprache \textbf{PROLOG}\index{Programmiersprache!PROLOG} genutzt, was schlicht für programmable logic steht.\\

\textbf{C}\index{Programmiersprache!C} ist bis heute eine sehr wichtige Sprache, weil sie dafür entwickelt wurde, um \textbf{Betriebssystem}e\index{Betriebssystem} zu entwickeln und dabei möglichst wenig maschinennah programmieren zu müssen. Zusätzlich können Sie mit ihr grundsätzlich jede Form imperativer Programme entwickeln. Für Sprachen, die wie C für alle möglichen Zwecke eingesetzt werden können, wird die Bezeichnung \textbf{general purpose programming}\index{general purpose programming}\index{Programmierung!general purpose programming} (kurz GPP) verwendet.\\

Das Buch "`\textbf{The C programming language}" von \textbf{Kernighan}\index{Wichtige Personen!Kernighan} und \textbf{Ritchie}\index{Wichtige Personen!Ritchie} ist eines der ersten Bücher zur Einführung in die Programmierung mit C. Es ist eher schwer zu nutzen, aber wenn Sie sich durch diesen Band durchgearbeitet haben, dann beherrschen Sie die Grundlagen der imperativen Softwareentwicklung, die InformatikerInnen beherrschen müssen.\\

Wenn Sie in einer Statistik nachsehen, wie viele Programmierer C nutzen, dann werden Sie feststellen, dass diese einen immer geringeren Anteil aller Programmierer ausmachen. Das hat damit zu tun, das C für die Entwicklung verteilter Anwendungen relativ wenig Unterstützung anbietet. Aber denken Sie deshalb nicht, C sei belanglos geworden; es gibt schlicht wesentlich mehr Bereiche, in denen heute programmiert wird, als in den 70er Jahren, in denen C entwickelt wurde.\\

Eine \textbf{verteilte Anwendung}\index{Anwendung!verteilt} ist nichts anderes als ein Programm, das auf mehreren miteinander vernetzten Rechnern aktiv ist. Die Probleme, die dabei durch die Kommunikation zwischen den Rechnern entstehen sind eines der anspruchsvollsten Themen, mit denen Sie sich auseinander setzen können. \\

\textbf{Kontrolle}\\
Höhere Programmiersprachen sind Programmiersprachen, die für Menschen leichter lesbar sind, als das bei Assembler der Fall ist. C ist hier einer der wichtigsten Vertreter, auch wenn es insbesondere bei der Entwicklung von verteilten Systemen eher nicht eingesetzt wird.

\section{C++ : C mit Objektorientierung}
Auch wenn höhere Programmiersprachen übersichtlicher und verständlicher als maschinennahe Programmiersprachen sind, ändert das nichts daran, dass irgendwann der Punkt erreicht ist, an dem auch sie nicht genug Übersichtlichkeit bieten. Vielen Informatikern war das bereits in der Frühzeit der Programmierung klar. Seit Mitte der 50er Jahre wurden deshalb immer neue Konzepte erarbeitet, die dann die Basis für verschiedene Sprachen bildeten, die mehr Strukturierungsmöglichkeiten beinhalten, als das bei C der Fall ist.\\

Für die Zwecke dieser Einführung soll es genügen, wenn Sie wissen, dass C++ der Sprache C entspricht, aber zusätzlich Möglichkeiten zur objektorientierten Programmierung bietet. Wenn nun von \textbf{objektorientierter Programmierung}\index{Programmierung!objektorientiert} die Rede ist, dann gibt es das Problem, dass es hierfür zwei Interpretationen gibt, die zu gänzlich unterschiedlichen Programmierstilen führen:

\subsection{Objektorientierung nach Alan Kay}
\textbf{Alan Kay}\index{Wichtige Personen!Kay, Allen} war ein Forscher am \textbf{MIT}\index{Wichtige Institutionen!MIT}, der den Begriff der Objektorientierung mitprägte aber diese Bezeichnung später als einen großen Fehler bezeichnete. Denn was er meinte war eine Programmierung, bei der der Fokus auf dem \textbf{Nachrichtenaustausch zwischen virtuellen Objekten} liegt. Wohlgemerkt, der Fokus liegt auf dem Nachrichtenaustausch, nicht auf den Objekten selbst. \\

Wenn Sie sich jetzt daran erinnern, was das wichtigste bei einem Computer ist (die Datenübertragung zwischen den Komponenten des Rechners), dann verstehen Sie auch, warum dieses Konzept der logische Schluss ist. Wenn Sie dann noch an den Aufbau des Internet denken, dann können Sie sich vorstellen, wie grundsätzlich und vorausschauend dieses Konzept ist. Und nochmal: In diesem Bereich kommen wir nur dann zu sinnvollen und effizienten Programmen, wenn \textbf{Informatik}erInnen\index{Informatik} und \textbf{Nachrichtentechnik}erInnen\index{Nachrichtentechnik} zusammen arbeiten.\\

\textbf{Aufgabe}:\\
Können Sie jetzt nachvollziehen, warum Kay die Bezeichnung Objektorientierung als großen Fehler bezeichnet hat? \\

Dieser Begriff suggeriert, dass die virtuellen Objekte das wichtige sind und lassen naive ProgrammiererInnen die Bedeutung des Nachrichtenaustauschs vergessen. Da wäre der Begriff des \textbf{Message Sending}\index{Message Sending}\index{Objektorientierung!Message Sending} wesentlich passender gewesen. Aber so ist das eben, wenn ein neues Konzept entwickelt wird; da wird eine einprägsame Bezeichnung genutzt und dann gerät alleine dadurch das eigentliche Konzept in Vergessenheit.\\

Aus diesem Grund sind auch praktisch alle Programmiersprachen, die im Internet zum Einsatz kommen für dieses Einsatzgebiet praktisch nicht geeignet: Für die Probleme beim Nachrichtenaustausch, namentlich zeitliche Verzögerungen und Verluste bieten sie im Regelfall nur beschränkte Lösungsmöglichkeiten, was dementsprechend eher zu mittelmäßigen Programmen führt. Und hier gilt wieder, dass InformatikerInnen und NachrichtentechnikerInnen leider kaum zusammen arbeiten. Täten Sie das, dann würden just die Probleme wesentlich besser gehandhabt werden, die beim programmierten Nachrichtenaustausch auftreten: Die NachrichtentechnikerInnen würden die Probleme bei der Datenübertragung sinnvoll lösen und die InformatikerInnen würden die Probleme bei der Softwareentwicklung sinnvoll lösen.\\

Eine Sprache, die Message Sending bzw. Objektorientierung nach Kay umsetzt, heißt \textbf{Erlang}\index{Programmiersprache!Erlang}. Es handelt sich hier um eine Sprache, die mehrere Paradigmen unterstützt. Richtig gelesen: Es gibt Sprachen, die mehrere \textbf{Paradigmen}\index{Paradigma}\index{Programmierung!Paradigma} umsetzen. Und tatsächlich ist das bei den meisten Sprachen der Fall. \textbf{Java}\index{Programmiersprache!Java} war beispielsweise bis zur Version 7 eine rein imperative und klassenbasiert objektorientierte Sprache. Seit Version 8 beinhaltet Sie mit der funktionalen Programmierung aber auch ein Konzept der deklarativen Programmierung. Die Version 9 wird kein neues Paradigma einführen, sondern es wird eine massive Restrukturierung geben, die den Speicherbedarf von Java-Programmen deutlich reduzieren wird, die aber auch bei der Programmierung in Java Folgen haben wird.

\subsection{Objektorientierung nach Lieschen Müller}
Damit kommen wir jetzt zu dem, was heute üblicherweise unter Objektorientierung verstanden wird: \\

Anstelle eines Programms entwickeln wir im Kern lauter kleine Programme, die jeweils einen gewissen Funktionsumfang anbieten und grundsätzlich als \textbf{Klasse}n\index{Klasse}\index{Objektorientierung!Klasse} bezeichnet werden. Soll eine bestimmte Funktionalität genutzt werden, erhält die Klasse, die sie enthält einen entsprechenden Befehl, was dann als Methodenaufruf bezeichnet wird. Bitte beachten Sie: Ein Methodenaufruf ist mehr als nur ein einfacher Befehl, aber zu den Unterschieden kommen wir bei der Einführung in die imperative Programmierung, wenn wir uns die sogenannten Funktionen ansehen.\\

Und auch wenn Klassen, Methoden und Methodenaufrufe zentrale Themen der objektorientierten Programmierung sind, hat dieses Verständnis ungefähr so viel mit Objektorientierung zu tun, wie das Verleimen zweier Holzleisten mit dem Tischlerhandwerk: Es gibt noch wesentlich mehr, was Sie verstanden haben müssen, um wirklich zu verstehen, was Objektorientierung ist.\\

Das ist auch der Grund, warum man mit der Einführung in die Objektorientierung bereits eine vollwertige Vorlesung für ein oder zwei Semester füllen kann.\\

\textbf{Kontrolle}\\
Wenn C Programme zu umfangreich werden, kann man auf C++ zurückgreifen, da es den gleichen Umfang an Befehlen und Strukturen bietet, aber mit der Objektorientierung weitere Strukturierungsmöglichkeiten anbietet. Java ist ebenfalls in diesem Sinne eine objektorientierte Sprache. Beachten Sie bitte, dass bei allen dreien Objektorientierung nicht im Sinne von Alan Kay umgesetzt und angewendet wird, auch wenn das durchaus möglich wäre.\\

Darüber, was das im Detail bedeutet und wozu es gut ist, haben Sie jetzt noch nichts erfahren. Zerbrechen Sie sich da also bitte nicht den Kopf. Es braucht im Regelfall mehrere Jahre, um diese Punkte verinnerlicht und weitgehend verstanden zu haben.

\section{Java – C++ ohne maschinennähe}
C und damit C++ bieten wie beschrieben die Möglichkeit recht nah an der Maschine zu programmieren, auf der ein Programm laufen soll. Das bringt einen großen Nachteil mit sich: Angreifer können durch geschickt entwickelte Programme in laufende Prozesse eingreifen. Außerdem muss ein C bzw. C++ Programm individuell auf jeden Prozessor zugeschnitten werden, auf dem es laufen soll. Bei der Vielzahl an Prozessoren, die heute in mobilen Endgeräten zum Einsatz kommt ist aber genau dieser letzte Punkt ein ernstes Problem: Nicht nur müssen die Entwickler die Software für jeden Prozessor anpassen, sie müssen insbesondere die Details jedes dieser Prozessoren kennen, sonst entwickeln Sie im besten Falle ineffiziente, im schlimmsten Fall leicht angreifbare Programme. Und wer möchte schon, dass die neueste App ein Einfallstor für Viren und Trojaner wird?!\\

Deshalb wurde u.a. \textbf{Java}\index{Programmiersprache!Java} entwickelt: Zu einem Zeitpunkt, zu dem nicht nur für Frau Merkel das Internet Neuland war (Anfang der 90er Jahre) entwickelte ein Team bei \textbf{Sun Microsystems}\index{Wichtige Unternehmen!Sun} diese neue Sprache zusammen mit einem passenden mobilen Endgerät. Um Entwicklern die Umgewöhnung zu erleichtern, wurden viele Konventionen und Regeln in Java so umgesetzt, wie das bereits in C und C++ der Fall war.\\

Wenn Sie also bislang dachten, \textbf{Apple}\index{Wichtige Unternehmen!Apple} sei das Unternehmen, das (mit dem iPhone) das erste Smartphone entwickelt hat, dann liegen Sie schlicht falsch. Hier wie in mehreren anderen Fällen hat Apple (genau wie \textbf{Microsoft}\index{Wichtige Unternehmen!Microsoft}) ein Konzept, das andere bereits zuvor ausgearbeitet hatten schlicht zum richtigen Zeitpunkt in einem Produkt umgesetzt und es zum Verkauf angeboten, als es ausreichend Menschen gab, die bereit waren, dafür Geld auszugeben. Das gleiche gilt für grafische Nutzeroberflächen und die Bedienung eines Computers mit der Mouse. Die wurden ebenfalls nicht von Apple entwickelt, sondern von einem Unternehmen namens Xerox Parc. Allerdings kam dort (im Gegensatz zu Steve Jobs, der das Gelände besuchte) niemand auf die Idee, dass mit so etwas Geld verdient werden könnte.\\

Übrigens ist auch die Möglichkeit, ein Javaprogramm unverändert auf unterschiedlichen Systemen zu nutzen ein Kriterium der Objektorientierung. Dabei spricht man von \textbf{Portabilität}\index{Portabilität}\index{Objektorientierung!Portabilität}. \\

Insbesondere bei Entwicklern, die vorrangig in C oder C++ aber auch in anderen imperativen Sprachen entwickeln, herrscht bis heute das Vorurteil vor, Java sei eine viel zu langsame Sprache und deshalb überflüssig, ja generell sei \textbf{Objektorientierung}\index{Objektorientierung} unsinnig.\\

Hier sollten Sie sich merken, dass es im Regelfall keine unsinnigen Sprachen gibt; \textbf{Sprachen werden entwickelt, um einen bestimmten Zweck zu erfüllen.}\index{Programmiersprache!Zweck einer Sprache} Ist dieser Zweck tatsächlich nützlich und ist die Sprache sinnvoll und für den Zweck effizient konzipiert, dann wird sie im Regelfall einige Jahrzehnte verwendet. Wer dann einer solchen Sprache die Sinnhaftigkeit abspricht zeigt damit lediglich, dass er den Zweck nicht versteht. Und natürlich kann Java nicht die Geschwindigkeit einer Sprache wie C++ erreichen: Java übernimmt die Arbeit, jedes Programm auf einer möglichst großen Anzahl von Rechnern und Smartphones laufen zu lassen. Das bedeutet einen teilweise höheren Aufwand und damit laufen diese Programme in Java langsamer als in C++, wenn es fähige C++-ProgrammiererInnen in C++ umsetzen. Andererseits müssen diese eben auch sehr fähig sein und umfangreiche Kenntnisse über die Unterschiede zwischen rund dreißig Betriebssystemen und Prozessoren kennen und sich kontinuierlich in neue Systeme einarbeiten. Da das kaum jemand leisten kann, der als Entwickler bezahlbar ist, werden die meisten Spiele nur für ein System entwickelt oder sie sind nicht gut auf die einzelnen Systeme angepasst.\\

Einige von Ihnen werden jetzt einwenden, dass es doch von \textbf{Steam}\index{Wichtige Unternehmen!Steam}\index{Games!Steam} eine Plattform gibt, auf der Spiele unabhängig vom System laufen. Hier gilt das gleiche, was schon bei Java gilt: So lange diese Spiele nicht individuell für jede Plattform entwickelt werden, kann auch nicht die volle Palette an Möglichkeiten genutzt werden, die das System bietet. Also werden einige Spiele auf dieser Plattform langsamer laufen als wenn Sie speziell an das System angepasst wären.\\

Die meisten Spieleentwickler nutzen heute allerdings keine Programmiersprache mehr, sondern sogenannten \textbf{Game Engines}\index{Game Engines}\index{Games!Game Engines}\index{Programmierung!Game Engines}. Das sind Softwarepakete, die bereits eine Vielzahl an \textbf{Bibliotheken}\index{Bibliothek}\index{Programmierung!Bibliothek} beinhalten, sodass die Mitglieder von Entwicklerstudios sich nur noch auf den Ablauf des Spiels konzentrieren müssen und ein Team von Designern für die Grafik und den Sound benötigen. Um mit einer Game Engine zu arbeiten brauchen Sie deshalb kein Informatikstudium mehr abschließen. Im Gegenteil: Da diese Softwarepakete genau das übernehmen, was fähige InformatikerInnen tun, gibt es in der Spielebranche nur wenige Stellen für vollwertige InformatikerInnen. Im Gegenteil: Als Absolvent z.B. von Media Systems ist die Nutzung einer Game Engine eigentlich ein Rückschritt: Das System übernimmt nicht nur vieles, was Sie sonst umsetzen müssten, es verhindert auch vieles, das Sie kennen und schätzen gelernt haben.\\

Dagegen müssen Sie anspruchsvolle Aufgaben als (Medien-)InformatikerIn erfüllen können, um eine Game Engine zu entwickeln oder Ihren Funktionsumfang zu erweitern. In Ihrem Studium können Sie das später ausprobieren: Sie werden eine Game Engine namens \textbf{Blender}\index{Blender}\index{Games!Blender} kennen lernen. Diese wird von den meisten Studierenden abgelehnt, da die Nutzeroberfläche nicht wie die von vielen Game Engines oder Programmpaketen für Computergrafik aussieht. Tatsächlich ist Blender die wahrscheinlich beste Game Engine für (Medien-)InformatikerInnen: Da Sie hier alles und ohne Beschränkung erweitern oder verändern können und da Blender vollständig kostenlos und frei verfügbar ist, können Sie genau das tun, was Sie später als professionelle EntwicklerIn tun müssten. (Zum Vergleich: Wenn Sie keine akademische Lizenz erhalten, dann zahlen Sie für Engines wie Maya 3D mehrere tausend Euro. Doch selbst wenn Sie die Software erhalten, dürfen Sie daran nahezu nichts ändern.)\\

\textbf{Kontrolle}\\
Java ist eine imperative und klassenbasierte objektorientierte Programmiersprache, deren Programme leicht auf andere Systeme portiert werden können. Es unterstützt zusätzlich seit Version 8 die funktionale Programmierung und damit ein deklaratives Paradigma.

\section{Verteilte Anwendungen}
Die drei Sprachen, mit denen wir uns bislang beschäftigt haben setzen nicht voraus, dass unser Rechner sich in einem Netzwerk befindet, und dass es möglich ist, Daten mit den anderen Rechnern dieses Netzwerks auszutauschen. Nun wissen Sie aber, dass heute annähernd jedes computerbasierte System (also auch Smartphones) zumindest zeitweilig vernetzt ist. Wie Sie durch die einleitenden Kapitel wissen, wurden Rechner im Regelfall schon immer vernetzt und nur im Heimbereich hatten Nutzer einen Computer ohne Netzzugang. (Hieraus resultiert auch die veraltete Unterteilung in Heimcomputer und PCs.)\\

Wenn wir nun ein Programm entwickeln wollen, das vernetzte Rechner nutzen soll oder sogar nur bei vernetzten Rechnern einsetzbar sein soll, dann müssen wir die Strukturen, die sich daraus ergeben auch in unseren Programmen integrieren. Ein solches Programm wird übrigens als \textbf{verteilte Anwendung}\index{Anwendung!verteilt} bezeichnet, vor allem wenn es genau genommen aus mehreren individuell agierenden Programmen besteht. Wir müssen dann (siehe Alan Kay und die Objektorientierung) beachten, dass Daten zwischen Rechnern transportiert werden müssen. Und das bedeutet, dass wir eine Absicherung für die Fälle schaffen müssen, in denen Daten nicht das Ziel (also einen anderen Rechner) erreichen oder in denen das Ziel aus irgendwelchen Gründen nicht versteht, was es mit diesen Daten tun soll. Wir müssen insbesondere bei Verbindungen über das Internet auch beachten, dass die Datenübertragung einen Zeitversatz hat, und dass wir keine genaue Zeitabstimmung zwischen den Rechnern realisieren können. Die genauen Ursachen und möglichen Auswirkungen verstehen Sie, wenn Sie Veranstaltungen zum Thema \textbf{Netzwerk}e\index{Netzwerk} und \textbf{Nachrichtentechnik}\index{Nachrichtentechnik} belegen. Es folgen in Kürze einige Beispiele, um Ihnen einen ersten Eindruck zu vermitteln.\\

Aus der dafür nötigen Denkweise resultieren auch zwei Begriffe: Server und Client. Die naive Vorstellung lautet hier, dass ein Server ein Rechner im Netz ist, der eine bestimmte Funktionalität anbietet, und dass ein Client ein anderer Rechner im Netz ist, der vom Server eine solche Leistung anfordert. Das ist allerdings nicht richtig; ein \textbf{Server}\index{Server} ist lediglich ein Programm, das eine bestimmte Funktion anbietet, und ein \textbf{Client}\index{Client} ist ein Programm, das eine Funktion abruft. Sie können also auf einem Rechner verschiedene Server und Clients betreiben, wobei bei Betriebssystemen in aller Regel eine Vielzahl an Servern und Clients aktiv ist. (Hier gibt es noch andere Programmarten über die wir aber erst im Rahmen von Veranstaltungen wie "`Betriebssysteme" reden werden.) Einsteiger, die aus der Apple- oder Microsoftwelt kommen sind häufig bei der Installation von Linux überrascht, dass Sie Mailserver und andere Server installieren können, aber Sie wissen jetzt, warum das so ist.\\

Dennoch werden häufig einzelne Rechner als Server oder Client bezeichnet. Das ist insbesondere dann kein Problem, wenn ein solcher Rechner ausschließlich eine entsprechende Funktion im Netz übernimmt. Aber Sie wissen jetzt, dass Sie kein Netz benötigen, wenn Sie ein netzbasiertes Programm entwickeln wollen, weil Sie ja auf einem Rechner sowohl den Server als auch den Client betreiben können. Und ja: Sie können dann einen Datenaustausch zwischen Client und Server auf Ihrem Rechner praktisch genauso durchführen, als wenn beide auf unterschiedlichen Rechnern installiert und über ein Netz verbunden wären. Deshalb können Sie z.B. eine Webanwendung, die später im Internet nutzbar sein soll auf Ihrem Rechner entwickeln und sie dort auch testen, selbst wenn keine Internetverbindung vorhanden ist.\\

Machen Sie sich in solchen Fällen aber bewusst, dass Sie dann die zentrale Fehlerquelle bei verteilten Anwendungen ausblenden: Da Sie keine Daten über das Netz austauschen, wissen Sie nicht, ob die Anwendung am Ende auch tatsächlich so funktioniert, wie Sie sich das vorstellen: Die Datenübertragung kostet Zeit und diese Zeit ist deutlich höher, wenn die Daten über ein Netzwerk übertragen werden, als wenn Sie innerhalb eines Rechners übertragen werden.

\subsection{Entwicklung von Webanwendungen – MySQL und PHP versus Ruby on Rails}
Bevor wir an dieser Stelle weiter machen, hier ein wichtiger Hinweis: Bis vor wenigen Jahren entwickelten die meisten Softwareentwickler Anwendungen, die auf einem System liefen und die eine Netzwerk nur nutzten, um Nachrichten darüber auszutauschen. \textbf{Webanwendung}en\index{Anwendung!Webanwendung} haben wie alle \textbf{verteilten Anwendungen}\index{Anwendung!verteilt}, mindestens einen Server- und einen Clientteil, die tatsächlich auf getrennten Rechnern aktiv sind. Die einfachste Form von Webanwendungen kennen Sie wahrscheinlich unter dem Namen Internetseiten. Doch das sind nicht einfach nur Dokumente mit Bildern und Videos, die Sie sich auf Ihren Rechner bzw. Ihr Smartphone herunterladen können, sondern es sind immer öfter komplette Anwendungen. Allerdings werden diese Anwendungen zum Teil auf dem Server und zum Teil auf dem Client ausgeführt. Wenn Sie sich intensiver mit diesem Bereich beschäftigen, werden Sie Sprachen wie \textbf{PHP}\index{Programmiersprache!PHP} kennen lernen, die ausschließlich als Server bzw. auf einem Webserver genutzt werden können. Allerdings ist dieser Ansatz veraltet: Aktuelle Sprachen wie \textbf{JavaScript}\index{Programmiersprache!JavaScript} können sowohl als Server als auch als Client eingesetzt werden. Das ist allerdings für Einsteiger bzw. Erstsemester meist nur schwer umsetzbar, da Sie hier bewusst und gut begründet entscheiden müssen, auf welche Programmteile Nutzer Zugriff haben dürfen.\\

Wenn Sie also denken, die Programmierung in \textbf{HTML}\index{Programmiersprache!HTML} (der am häufigsten eingesetzten Sprache für Webanwendungen) sei langweilig und man könnte damit nur einfache Internetseiten programmieren, dann liegen Sie falsch; das war in der Version 4.01 so, die Ende 1999 veröffentlicht wurde. Mit der Version 5, die im Herbst 2014 veröffentlicht wurde, ist dieses Thema endgültig passé: Basierend auf HTML 5 ergänzt um Sprachen wie PHP oder JavaScript können Sie Anwendungen entwickeln, die genau das gleiche leisten wie eine beliebige Anwendung, die Sie auf einem einzelnen Rechner nutzen können. Der Begriff Webanwendung ist im Grunde ein Synonym für den Begriff der verteilten Anwendung.\\

Zu Beginn dieses Kapitels haben Sie erfahren, dass im Grunde keine Sprachen existieren, die die zentralen Probleme angehen, die bei der Datenübertragung im Netz aufkommen. Der Grund besteht darin, dass für die meisten InformatikerInnen eben das System im Mittelpunkt steht, auf dem eine Software ausgeführt wird. Die Kommunikationswege dazwischen und der Zeitfaktor bei der Übertragung werden im Grunde immer nur als lästiges Übel angesehen oder gleich gänzlich ignoriert. Das gleiche gilt für die Nutzung von Webanwendungen durch Menschen.\\

Ein anschauliches Beispiel konnten Sie bei ebay in der Anfangszeit erleben: Damals hatten die Entwickler ignoriert, dass kurz vor Abschluss einer Auktion besonders viele Aufrufe und Gebote für ein Angebot erfolgten. Also konnten die Server gar nicht alle Angebote "`sofort" verarbeiten. Dementsprechend wurde es zu einer Art Glücksspiel, ein Gebot kurz vor Versteigerungsschluss abzugeben: Unter Umständen wurde Ihr Gebot scheinbar ignoriert, weil es erst nach Auktionsende von den ebay-Servern verarbeitet werden konnte. Dieser Fehler im System wurde inzwischen soweit als möglich bereinigt. Dies ist außerdem ein Beispiel für die Bedeutung des Begriffs \textbf{Skalierbarkeit}\index{Skalierbarkeit}\index{Software Engineering!Skalierbarkeit}.\\

Damit wieder zurück zu den eingangs genannten Sprachen: \textbf{Ruby on Rails}\index{Programmiersprache!Ruby on Rails}\index{Framework!Ruby on Rails} ist nicht die Sprache \index{Ruby}\index{Programmiersprache!Ruby} selbst, sondern ein Framework namens Rails, das Ruby so erweitert, dass Sie damit \textbf{Webanwendungen}\index{Anwendung!Webanwendung} entwickeln können.\\

Eine zweite Möglichkeit (und deutlich älter), um Webanwendungen zu entwickeln besteht in der Kombination aus drei Sprachen: \textbf{MySQL}\index{MySQL}\index{Programmiersprache!MySQL}\index{Datenbank!MySQL} ist eine Sprache, mit der Sie Datenbanken nutzen können und \textbf{PHP}\index{PHP}\index{Programmiersprache!PHP} ist eine imperative Sprache, mit der Sie die Funktionalität von Elementen einer Webanwendung programmieren können. Dazu kommt noch \textbf{HTML}\index{HTML}\index{Programmiersprache!HTML}, was eine \textbf{Markup Language}\index{Markup Language}\index{Programmierung!Markup Language} ist. Markup Languages sind Programmiersprachen mittels derer sich die Struktur von Anwendungen unabhängig von der Darstellung und der Funktion programmieren lassen. HTML ist eine Markup Language mit der sich die Struktur einer Webanwendung und seit Version 5 die Bedeutung der Inhalte programmieren lässt.\\

Viele Softwareentwickler reden in Bezug auf Markup Languages vom sogenannten \textbf{Script}en\index{Script}\index{Programmierung!Script}. Für diesen Begriff gibt es keine präzise Definition. Wenn ProgrammiererInnen ihn benutzen, dann geht es in aller Regel um etwas, das zwar programmiert werden muss, damit eine bestimmte Aufgabe erfüllt wird, das aber vom jeweiligen Entwickler keinerlei logisches Denkvermögen erfordert. Sie müssen im Falle vom Scripten also nur verschiedene Befehle aneinander reihen oder ineinander verschachteln, ohne sich weiter Gedanken darüber zu machen, wie die miteinander interagieren: Sie tun es schlicht nicht. Teilweise wird auch bei Konfigurationsdateien vom scripten gesprochen, obwohl hier (im Gegensatz zu HTML4.01 oder \LaTeX) sehr viel Grundlagenwissen nötig ist. Wenn Sie beispielsweise nicht genau wissen, was der Unterschied zwischen SSH und SSL ist, dann sollten Sie von der Konfiguration eines Servers die Finger lassen.\\

Haben Sie im ersten Semester eine Veranstaltung zum Webpage Development besucht, dann können Sie eine Webpage entwickeln, denn das ist gar nicht so schwer. Aber viele Konzepte und Abläufe werden Ihnen kaum klar werden. Wenn Sie allerdings die nötige Zeit investieren, dann werden Sie sich basierend auf dieser Veranstaltung die Grundlagen erarbeiten können, um eine vollwertige Webanwendung zu entwickeln.\\

Ein Tipp für den Fall, dass Ihnen jemand zu \textbf{Ruby on Rails}\index{Ruby on Rails}\index{Programmiersprache!Ruby on Rails}\index{Framework!Ruby on Rails} rät: Ja, es ist richtig, dass Rails ein großartiges Framework ist, mit dem Sie selbst komplexe Anwendungen für multinationale Konzerne entwickeln können. Aber Sie merken es schon an der Formulierung: Es ist für Einsteiger schlicht zu komplex und die Vielzahl an Optionen, mit denen Sie von Beginn an konfrontiert werden, lenkt Sie von den Punkten ab, die Sie als Einsteiger verinnerlicht haben müssen. Hier würde ich eher empfehlen, dass Sie sich in \textbf{HTML5}\index{HTML5}\index{Programmiersprache!HTML5} und \textbf{JavaScript}\index{Programmiersprache!JavaScript} einarbeiten. Leider sind aber die meisten Anleitungen im Netz (selbst wenn dort die Rede von HTML 5 ist) immer noch Einführungen in HTML 4, bei denen praktisch alles ignoriert wird, was an Version 5 so großartig ist. Teilweise werden hier sogar Techniken vermittelt, die bereits in der Version 4 als schlampig galten. Der Grund ist recht simpel: Wie so oft erklären dort Menschen die Programmierung, die zwar die alte Version beherrschen, aber die schlicht zu faul oder dumm sind, um zu erkennen, dass Version 5 keine kleine Erweiterung um ein paar nette Effekte ist, sondern eine vollständige Überarbeitung, bei der Aspekte berücksichtigt wurden, die nirgends in Version 4 auftauchen und auch nichts mit dem zu tun haben, was in Version 4 vorhanden ist.\\

Sie fragen, was eine Datenbank ist? Wie so oft, wenn \textbf{Informatik}erInnen\index{Informatik} es mit gleichartigen Daten oder Abläufen zu tun haben, entwickeln Sie entsprechende Strukturen, die letztlich dazu dienen, Fehler zu reduzieren und Abläufe effizienter zu gestalten. \textbf{Datenbank}en\index{Datenbank} sind eine weitere Lösung, die so entstanden ist: Wann immer es um große Mengen gleichartiger oder gleichartig strukturierter Daten geht, die nach verschiedenen Kriterien untersucht oder geändert werden müssen, wird eine Datenbank verwendet. \textbf{Relationale Datenbanken}\index{Datenbank!relational} bestehen dabei aus Tabellen, bei denen jede Spalte einem Kriterium entspricht und jede Zeile einem sogenannten \textbf{Datensatz}\index{Datensatz}\index{Datenbank!Datensatz}. Beispielsweise würde bei einer relationalen Kundendatenbank jede Zeile einem Kunden entsprechen und Einträge in den Spalten wären nach Aspekten wie Name, Vorname, Anschrift, usw. unterteilt.\\

Bei Webanwendungen dienen Datenbanken (wie bei allen Anwendungen) verschiedenen Zwecken. Zum einen wäre da die klassische Kundendatenbank. Dann gibt es Datenbanken, in denen Einträge auf den Webanwendungen verwaltet werden. Auch die Speicherung jedes Klicks und jeder Taste, die NutzerInnen gedrückt haben wird mit einer Datenbank realisiert. Aber es gibt noch wesentlich mehr Einsatzmöglichkeiten. Wie oben genannt geht es schlicht darum, große Mengen gleichartiger Daten in einer strukturierten Form aufzubewahren, um möglichst schnell darauf zuzugreifen.\\

\textbf{Kontrolle}\\
Beginnen Sie beim Webapplication Development mit einer Einführung in MySQL und PHP sowie HTML5. Wenn Sie die objektorientierte und funktionale Programmierung beherrschen, dann können Sie auch JavaScript anstelle von PHP verwenden. Allerdings ist hier ein häufiger Fehler, dass dann HTML nur noch dazu genutzt wird, die JavaScript-Anwendung zu starten, sodass sie in einem beliebigen Browser genutzt werden kann. Ruby on Rails ist ein sehr mächtiges Werkzeug aber für Einsteiger nur beschränkt empfehlenswert. Dazu kommt, dass Rails häufig in Kombination mit weiteren Frameworks verwendet wird, was den Einstieg zusätzlich erschwert. Außerdem ist es leider noch immer nicht so effizient wie JavaScript.

\section{Konzepte bei der Programmierung}
Wie schon mehrfach angeführt gab und gibt es zu jeder Zeit eine Vielzahl von Programmiersprachen, die jeweils für bestimmte Zwecke ausgelegt sind. Zum Teil sind die Unterschiede nur in wenigen Details begründet. Um Ihnen einen kleinen Überblick darüber zu verschaffen, was es noch für Sprachen gibt und wofür diese nützlich sind, folgt eine kleine und dementsprechend unvollständige Aufstellung. Zum Teil werden hier weitere Begriffe eingeführt, die für die Programmierung insgesamt wichtig sind.

\subsection{Dynamisch versus statisch – Ruby und Python versus C und Java}
\textbf{Ruby}\index{Programmiersprache!Ruby} und \textbf{Python}\index{Programmiersprache!Python} sind Programmiersprachen, die in Konkurrenz zu Java stehen. Der auffälligste Unterschied besteht darin, dass Ruby \textbf{schwach typisiert}\index{Typisierung!schwach} ist. (Alternativ spricht man auch von \textbf{dynamischer Typisierung}\index{Typisierung!dynamisch}.) Im Gegensatz dazu sind \textbf{C}\index{Programmiersprache!C} und \textbf{Java}\index{Programmiersprache!Java} \textbf{stark bzw. statisch typisiert}\index{Typisierung!stark}\index{Typisierung!statisch}. Beide Varianten haben Vor- und Nachteile. Und leider neigen die meisten Entwickler dazu, die Variante als schlecht zu bezeichnen, die sie als zweites kennen lernen. Wer das tut hat aber leider nichts mit professionellen \textbf{Informatiker}Innen\index{Informatik} zu tun, selbst wenn er/sie in einer Sprache bzw. einem \textbf{Paradigma}\index{Programmieren!Paradigma} wirklich gut ist.

\subsection{Typisierung von Daten}
Bislang haben wir lediglich über Codierung gesprochen aber nicht über Typisierung. Wie Sie bereits wissen werden alle möglichen Daten, die sie vom Computer verarbeiten lassen in einer anderen Form gespeichert als die, in der sie angezeigt werden. Wenn wir nun von statischer oder starker Typisierung sprechen, dann bedeutet das, dass Sie bei einem Wert, den Sie programmieren so etwas Ähnliches wie eine Codierung festlegen. Der Typ eines Wertes, den Sie so vergeben wird entsprechend als \textbf{Datentyp}\index{Datentyp} bezeichnet.\\

Einer der ersten Fälle, in denen Sie mit Typisierung zu tun bekommen ist das Rechnen mit ganzzahligen und ganzrationalen Zahlen. Diese werden nämlich vom Rechner unterschiedlich gespeichert: Fließkommazahlen werden nicht in der Form gespeichert, die Sie aus dem Mathematikunterricht kennen, aber eine Einführung in diese Materie überlasse ich den Kollegen der \textbf{Technischen Informatik}\index{Informatik!Technische Inf.}. Jetzt aber ein Beispiel für die möglichen Varianten, wie eine Programmiersprache mit ganzen Zahlen umgehen kann:\\

Wenn Sie die Zahl 5 programmieren, als Typ der Zahl ganzzahlig (\textbf{Integer}\index{Datentyp!Integer}) festlegen und anschließend durch 2 teilen (oder jede andere Zahl ungleich +/- 5 oder +/- 1), dann ergibt sich bekanntlich eine ganzrationale Zahl. Je nach Programmiersprache gibt es nun unterschiedliche Möglichkeiten, was dabei passiert:\\

\begin{enumerate}
	\item Bei statisch typisierten Sprachen erfolgt in aller Regel eine Fehlermeldung, denn Sie haben definiert, dass Ihre Zahl ganzzahlig ist. Also muss das Ergebnis ebenfalls ganzzahlig sein. Es gibt dennoch Möglichkeiten, um eine solche Aufgabe in einer solchen Sprache lösen zu lassen. Dazu sind jedoch zusätzliche Programmzeilen nötig.
	
	\item Wenn keine Fehlermeldung erfolgt, ist das das Ergebnis häufig nicht das, was Sie erwarten. In den Fällen, wo die Sprache vorsieht, dass das Ergebnis als ganzzahliger Wert gespeichert wird, wird nun entweder auf- oder abgerundet.\\
	
	\textbf{ABER! Ob auf- oder abgerundet wird, das hat nichts mit den Rundungsregeln zu tun, die Sie aus der Schule kennen:} Es gibt also vier Möglichkeiten, wie eine Programmiersprache damit umgeht, wenn Sie eine Division von zwei ganzzahligen Werten einprogrammieren und bei der keine Ganzzahl berechnet wird.
	
	\begin{enumerate}
		\item In der Sprache wird stets abgerundet: \\
		5 : 2 = \textbf{2}.\\
		-5 : 2 = \textbf{-3},\\
		denn -2,2 abgerundet ergibt -3 und nicht -2!

		\item In der Sprache wird stets aufgerundet:\\
		5 : 2 = \textbf{3}\\
		-5 : 2 = \textbf{-2},\\
		denn -2,2 aufgerundet ergibt -2 und nicht -3!
		
		\item In der Sprache sind die Rundungsregeln enthalten, die Sie aus dem Mathematikunterricht der Schule kennen. Das ist der seltenste Fall.
		
		\item Die Sprache gibt in solchen Fällen eine Fehlermeldung oder eine \textbf{Exception}\index{Exception} aus. Als EntwicklerIn müssen Sie dann das Programm entsprechend korrigieren.
		
		\textbf{Wichtig}:\\
		Eine Exception ist KEINE Fehlermeldung. Es ist vielmehr ein Komfortfaktor einzelner Programmiersprachen, der sie darauf hinweist, dass Ihr Programm in bestimmten Fehlern nicht so ablaufen wird, wie Sie das wahrscheinlich erwarten. Je nach Komfort der jeweiligen Sprache gibt es auch Exceptions, die auf Situationen hinweisen, die durchaus wie gewünscht verlaufen können, wo Ihnen die Programmiersprache also quasi den Tipp gibt, zu prüfen, ob Sie dieses Verhalten so haben wollen oder ob das nicht doch ein logischer Fehler ist. In Java haben Sie sogar die Möglichkeit, eigene Exceptions zu programmieren, um bestimmte Ausnahmen ganz bewusst anders verarbeiten zu lassen als das sonst der Falle wäre.
	\end{enumerate}
	
	\item Bei \textbf{dynamisch typisierten Sprachen}\index{Typisierung!dynamisch} wird der Datentyp je nach Bedarf automatisch von der Programmiersprache angepasst. Hier programmieren wir also in aller Regel nicht den Datentyp. Generell steht der Begriff des \textbf{Typecasting}\index{Typecasting}\index{Typisierung!Typecasting} für eine solche Anpassung, die in einigen statisch typisierten Sprachen möglich ist.\\
	
	Typecasting gibt es noch für andere Datentypen, es ist also nicht nur auf die Umwandelung des Datentyps bei einer Zahl beschränkt, sondern bei allen denkbaren virtuellen Objekten.\\
	
	Aber auch beim Typecasting ist das Ergebnis nicht automatisch das, was Sie denken. Das hat wiederum mit der Speicherung von Daten zu tun. Wie Sie wissen basiert die Speicherung von Daten in einem Computer auf Zahlen der Basis 2. Und damit basiert die Speicherung von Nachkommastellen auf Potenzen von \(\frac{1}{2}\). In Informatikveranstaltungen werden Sie dazu einige Beispiele rechnen, hier seien nur zwei genannt: \(\frac{5}{2} = 2 + 1/2\). Binär lässt sich das in der Form \([10,1]_2\) darstellen: \((1 \cdot 2) + (0 \cdot 1) + (1 \cdot \frac{1}{2})\). Versuchen wir einmal, die Zahl 0,3 als Binärzahl darzustellen: \\
	
	\((0 \cdot 1) + (0 \cdot \frac{1}{2}) + 0,3\) \\
	\(= (0 \cdot 1) + (0 \cdot \frac{1}{2}) + (1 \cdot \frac{1}{4}) + 0,05 \)\\
	\(= (0 \cdot 1) + (0 \cdot \frac{1}{2}) + (1 \cdot \frac{1}{4}) + (0 \cdot \frac{1}{8}) + 0,05 \)\\
	\(= (0 \cdot 1) + (0 \cdot \frac{1}{2}) + (1 \cdot \frac{1}{4}) + (0 \cdot \frac{1}{8}) + (0 \cdot \frac{1}{16}) + 0,05 \)\\
	\(= (0 \cdot 1) + (0 \cdot \frac{1}{2}) + (1 \cdot \frac{1}{4}) + (0 \cdot \frac{1}{8}) + (0 \cdot \frac{1}{16}) + (1 \cdot \frac{1}{32}) + 0,01875 \)\\
	\(= ...\)\\
	
	Aber es gibt kein endgültiges Ergebnis. Dementsprechend kann weder der Computer noch die Programmiersprache eine Zahl wie 0,3 richtig darstellen. Er könnte sie als \(3 * 10^{-1}\) darstellen, aber da das nur eine begrenzte Anzahl an Spezialfällen aller möglichen ganzrationalen Zahlen löst, ist das keine Lösung, die wir immer nutzen können.) Also wird eine Zahl wie 0,3 nur annäherungsweise gespeichert. Und wenn sie dann ausgegeben wird, kann so etwas wie 0,3000010002 oder 0,298991 herauskommen. Auch hierfür gibt es Lösungen, die aber auch wieder bedeuten, dass Sie zusätzliche Zeilen programmieren müssen.\\
	
	Wenn wir mit der Programmierung in \textbf{C}\index{Programmiersprache!C} beginnen, werden Sie häufig mit solchen Fällen zu tun haben. Sie werden dann (wie auch sonst grundsätzlich bei der Programmierung) Lösungen entwickeln müssen, damit der Rechner die Zahlen verwendet und speichert, die für Ihre Aufgabenstellung eine richtige Lösung darstellen.\\
	
\end{enumerate}

Dies ist allerdings kein Beispiel dafür, was \textbf{Informatik}erInnen\index{Informatik} von ProgrammiererInnen unterscheidet: Gute ProgrammiererInnen wissen, dass es solche Probleme gibt, sie kennen die Ursachen und sie entwickeln Programme so, dass diese Probleme in allen Varianten gelöst werden. (Sonst sind es Dilletanten, was leider auf viele Quereinsteiger zutrifft.) Allerdings lernen InformatikerInnen in Ihrem Studium verschiedene systematische Methoden, um solche Probleme effizient anzugehen. Der entsprechende Bereich heißt \textbf{Praktische Informatik}\index{Informatik!Praktische Inf.}, wobei die entsprechenden Veranstaltungen in aller Regel unter Titeln wie \textbf{Algorithmen und Datenstrukturen}, \textbf{Algorithmendesign} und \textbf{Algorithmik} angeboten werden.\\

Aber zurück zu Ruby: Wie gesagt handelt es sich hier um eine dynamisch typisierte Programmiersprache, während C, C++ und Java statisch typisiert sind. Es spielt keine Rolle, welche der beiden Varianten Ihnen lieber ist, das einzige, was eine Rolle spielt ist, dass Sie langfristig lernen, beide Formen der Typisierung zu beherrschen.\\

Eine weitere Programmiersprache, die neben Ruby in den letzten Jahren immer bekannter wurde und dynamische Typisierung bietet, ist \textbf{Python}\index{Programmiersprache!Python}.\\

\textbf{Kontrolle}\\
Es gibt dynamisch und statisch typisierte Sprachen. Der Unterschied besteht darin, dass bei den dynamisch typisierten Sprachen die Programmiersprache Methoden hat, um den Datentyp eines Wertes automatisch anzupassen. Das ist komfortabel, aber es ist nicht intuitiv. Denn während Sie bei einer statisch typisierten Sprache selbst programmieren müssen, wie ein Typcasting ausgeführt wird, müssen Sie bei jeder dynamisch typisierten Sprache genau wissen, wie diese einzelne Sprache das "`automatische" Typecasting durchführt. Wenn Sie das eine oder das andere nicht beherrschen, dann programmieren Sie den Computer nicht, um das zu tun, was er tun soll, sondern Sie programmieren Ergebnisse, die schlichtweg falsch sind. (Und glauben Sie mir, das wollen Sie nicht bei der Steueranlage eines Flugzeugs…)

\subsection{Funktionen – Dynamisch versus statisch}
Dieser Abschnitt dürfte auch für die meisten fortgeschrittenen unter Ihnen eine Überraschung beinhalten: Nicht nur Datentypen, sondern auch Funktionen können bei einzelnen Sprachen dynamisch während der Laufzeit eines Programms geändert werden.\\

\textbf{Wichtig}:\\
Bitte denken Sie jedoch nicht, dass das Programmieren einer Funktion eine Form der \textbf{funktionalen Programmierung}\index{Programmierung!funktional} ist: Genau wie bei der \textbf{imperativen Programmierung}\index{Programmierung!imperativ} nutzen Sie dort etwas, das als Funktion bezeichnet wird, um Abläufe aus dem Programm auslagern. Diese Art der Funktionsdefinition sorgt dafür, dass Sie den Inhalt der Funktion an beliebigen Stellen Ihres Programms verwenden können. Bei der funktionalen Programmierung ist eine Funktion dagegen eine Umsetzung des sogenannten \textbf{Lambda-Kalkül}s\index{Lambda-Kalkül}\index{Programmierung!Lambda-Kalkül}, das wir uns erst bei der Einführung in die \textbf{deklarative Programmierung}\index{Programmierung!deklarativ} ansehen werden.\\

Doch für die Einsteiger zunächst die Erklärung, was eine Funktion ist: Eine \textbf{Funktion}\index{Funktion} fasst in der Programmierung mehrere Programmzeilen zusammen und lässt sich über einen Bezeichner von beliebigen Stellen eines Programms aus aufgerufen werden.\\

Die meisten Programmierer kennen Funktionen dagegen nur als ein Mittel der Programmierung, das sich nicht ändern kann, während das Programm läuft. Das ist aber ein Irrtum, der darauf basiert, dass das bei Programmiersprachen wie C, C++ oder Java so ist.\\

Tatsächlich werden Funktionen während des Programmablaufs wie alle anderen Daten eines Programms im Speicher des Rechners abgelegt und bei Bedarf von dort geladen. Und weil der Speicher eines Rechners zu beliebigen Zeiten geändert werden kann, ist es natürlich auch grundsätzlich möglich beliebige Teile eines Programms abzuändern, das dort abgelegt wurde.\\

\textbf{Kontrolle}\\
Auch wenn die meisten bekannten Sprachen das nicht können, ist es grundsätzlich möglich, dass auch Funktionen eines Programms dynamisch programmiert werden. 

\section{First-class Objects}
Die meisten Programmierneulinge lernen das Programmieren mit Variablen kennen und entwickeln dabei die Vorstellung, dass eine Variable nur einen Wert oder eine Menge an Werten (z.B. die sogenannten Arrays) sein kann. Tatsächlich kann aber auch eine Funktion der Wert einer Variablen sein. Ist das bei einer Programmiersprache der Fall, dann können Sie nicht nur eine Funktion mit einem Wert aufrufen, sondern Sie können eine Funktion quasi wie einen beliebigen Wert an eine andere Funktion übergeben. Das bedeutet, dass Sie dann die Möglichkeit haben, den Ablauf einer Funktion während eines Programmablaufs dynamisch anpassen können.\\

Diejenigen von Ihnen, die bereits imperativ programmiert haben werden jetzt behaupten, dass das doch klar sei, weil so "`schon immer" der Wert einer Funktion an eine Variable übergeben wurde. Damit zeigen Sie, dass Sie den Absatz missverstanden haben: Dort steht, dass auch eine Funktion als ganzes und eben nicht nur der Wert, den sie berechnet in einer Variablen gespeichert werden kann. Warum das so ist werden wir uns ansehen, wenn wir klären, was genau eine Variable eigentlich ist.\\

Als Sammelbegriff für alles, was einer Funktion übergeben werden kann wird der Begriff des \textbf{first-class Object}\index{first-class Object} verwendet. Wenn also die Rede davon ist, dass in einer Programmiersprache Funktionen first-class Objects sind, dann bedeutet das nichts anderes, als dass Sie in dieser Sprache eine Funktion genauso als Objekt an eine andere Funktion übergeben können, wie Sie das mit einer Variablen gewohnt sind.\\

\textbf{Kontrolle}\\
Im Gegensatz zur meist anzutreffenden Überzeugung von ProgrammiererInnen spricht eigentlich nichts dagegen, auch Funktionen als Argumente an Funktionen zu übergeben. Und wenn eine Programmiersprache das unterstützt, dann reden wir davon, dass in dieser Sprache Funktionen first-class objects sind.

\section{Zusammenfassung}
In diesem Kapitel haben Sie einen Überblick erhalten, wie die drei Sprachen C, C++ und Java zusammen hängen und wo die Unterschiede liegen. Sie haben eine Vielzahl an Begriffen kennen gelernt, die bei der Programmierung von Hochsprachen von Belang sind. \\

Sie haben verstanden, dass es keine beste Sprache oder sinnlose Sprachen gibt, sondern dass Sprachen jeweils für eine bestimmte Problemstellung entwickelt wurden. Deshalb gibt es dann auch ganz unterschiedliche Arten (Paradigmen) des Programmierens und hier haben Sie konkrete Fälle kennen gelernt, um den Begriff des Paradigmas mit Leben zu füllen.\\

Sie wissen jetzt, dass Sie es gelegentlich mit einem bestimmten Programmierparadigma zu tun haben und teilweise lediglich mit einem Spezialfall, der so nur in einer einzelnen oder bei einigen wenigen Sprachen umgesetzt wird. Diese Kenntnisse sind ein weiterer Unterschied zwischen einem dilletantischen Quereinsteiger und einem ernstzunehmenden Softwareentwickler.\\

Danach haben Sie etwas über Konzepte erfahren, die C-, C++- und Java-ProgrammiererInnen nicht verstehen und die beispielsweise in Ruby, Python und JavaScript zum Einsatz kommen.
\chapter{Ausgewählte Programmiersprachen}

%\chapter[Grundlagen verteilter Anwendungen]{Grundlagen der Entwicklung verteilter Anwendungen}

\textbf{Verteilte Anwendungen}\index{Verteilte Anwendungen} sind Programme, die aus einer Vielzahl von einzelnen und unabhängigen Programmen bestehen, die jeweils auf unterschiedlichen Rechnern als Server bzw. Clients in einem Netzwerk Teilaufgaben erfüllen. Auch wenn Sie unabhängig voneinander agieren, sind Sie doch als Ganzes eine gemeinsame Funktionalität bzw. ein gemeinsames Programm. Erst wenn dieses letzte Kriterium gilt, wird das Ganze als eine verteilte Anwendung bezeichnet.\\

Wenn Ihnen das zu abstrakt klingt, dann stellen Sie sich ein Team von verschiedenen MitarbeiterInnen vor, die an verschiedenen Standorten für ein Unternehmen tätig sind. Die verschiedenen Server und Clients entsprechen den MitarbeiterInnen. Wenn Sie mit den Begriffen Server und Client Schwierigkeiten haben, dann lesen Sie bitte die entsprechenden Abschnitte der vorhergehenden Kapitel.\\

Die einfachste Möglichkeit, um eine verteilte Anwendung zu entwickeln besteht in einer Kombination, bei der HTML für die Struktur der Anwendung verwendet wird. Zusätzlich benötigen wir dann noch eine Sprache um die Funktionalität der Elemente dieser Anwendung zu programmieren und abschließend eine Sprache, um die langfristige Speicherung von Nutzerdaten sicher zu stellen. Das können Sie als Erstsemester mit entsprechendem Arbeitsaufwand gut schaffen.\\

\glqq{}Echte\grqq{} verteilte Anwendungen werden dagegen erst im Masterbereich eines Informatikstudiums grundlegend behandelt, weil Sie für den Einstieg in diesen Bereich eine Vielzahl an Kursen absolviert haben müssen, die Teil eines Bachelorstudiums der Informatik und verwandter Studiengänge ist. Hier seien die wichtigsten dieser Veranstaltungen und Themenbereiche genannt:

\begin{itemize}
	\item Diskrete Mathematik und Graphentheorie
	\item Grundlagen der technischen Informatik
	\item Nachrichten- und Kommunikationstechnik
	\item Netzwerke und Internetsicherheit
	\item Imperative Programmierung
	\item Objektorientierte Softwareentwicklung
	\item (Relationale) Datenbanken
	\item Mobile Apps und Responsive Design
	\item Medienrecht
\end{itemize}

Wie eingangs beschrieben können Sie einfache verteilte Anwendungen wie beispielsweise eine Webanwendung auch ohne fundierte Kenntnisse dieser Bereiche entwickeln. Leider kann es Ihnen passieren, dass Sie nach dem Studienabschluss in einem Unternehmen tätig werden, wo Sie es mit vermeintlichen Fachkräften zu tun haben, deren Kenntnisse der oben genannten Bereiche zumindest teilweise mangelhaft sind. Diese wissen dann z.B., dass es so etwas wie \glqq{}das Internet\grqq{} gibt, wie sie eine Webanwendung programmieren können und ähnliches, aber ihnen fehlt das fundamentale\\ Verständnis für die zugrunde liegenden Technologien. Das führt dann langfristig zu massiven Schwierigkeiten, denn sobald neue Technologien auf dem Markt erscheinen (HTML 5 ist da ein sehr gutes Beispiel) sind diese \glqq{}Fachkräfte\grqq{} nicht im Stande die neuen Technologien sinnvoll in bestehende Projekte zu integrieren und was recht bald dazu führt, dass Sie einen neuen Arbeitgeber benötigen.\\

Bevor Sie hier weiterlesen: Haben Sie die einleitenden Kapitel zumindest überflogen und den Abschnitt zur Vorbereitung für die Entwicklung von verteilten Anwendungen durchgearbeitet? Wenn nicht, dann tun Sie das bitte, bevor Sie hier weitermachen. Im laufenden Kapitel wird zum einen vorausgesetzt, dass Sie die Inhalte der einleitenden Kapitel kennen und insbesondere die nötigen Installationen abgeschlossen haben. Unklare Fachbegriffe können sie anhand der Anhänge zu jedem Kapitel leicht nachschlagen.\\

Wenn Sie ein anderes Paket als EasyPHP nutzen, dann werden einige Ausgaben des Rechners bei Ihnen anders aussehen als hier geschildert. Eine zusätzliche Hilfe für diese Fälle kann leider im Rahmen dieses Kurses nicht geboten werden.\\

Einführende Programmierveranstaltungen bereiten Sie in aller Regel darauf vor, Programme für einen Computer zu entwickeln. Das ist in dieser Veranstaltung anders: Hier lernen Sie, Programme zu entwickeln, die über das World Wide Web auf allen möglichen Endgeräten laufen können. Mit Endgeräten ist alles gemeint, was im Kern ein Computer ist, aber nicht unbedingt als solcher zu erkennen. Ein Beispiel sind Smartphones.\\

\section{Internet, WWW und HTML}

In einer Veranstaltung zur \textbf{Nachrichten- oder Kommunikationstechnik}\index{Nachrichtentechnik}\index{Kommunikationstechnik} werden Sie lernen, dass das, was allgemein als \glqq{}das Internet\grqq{} bezeichnet wird in Wahrheit etwas ist, das mit dem Begriff \glqq{}World Wide Web\grqq{} bezeichnet wird. Der Begriff \textbf{Internet}\index{Internet} bezeichnet dagegen Verbindungen zwischen zwei Netzwerken auf globaler Ebene. Ein Beispiel hierfür wären die transatlantischen Kabel, die die Verbindung zwischen den Kommunikations- und Datennetzen in Europa und Amerika sicherstellen.\\

Somit ist das Internet also die Voraussetzung für das World Wide Web: Ohne diese Leitungen und Funkstrecken (z.B. über Satellit) könnten wir keine Daten in ferne Länder übertragen oder von dort erhalten. Und Psys Gangnam Style wäre nie so berühmt geworden. Das WWW ist damit die wahrscheinlich umfangreichste verteilte Anwendung, die es weltweit gibt.\\

Doch was ist nun das \textbf{World Wide Web}\index{World Wide Web} (kurz \textbf{WWW})? Hier handelt es sich um Dateien, die auf Computern rund um die Welt gespeichert sind. Diese Daten sind in bestimmten Formaten verfasst und dienen dazu, mit einem entsprechenden Programm, Texte und multimediale Inhalte anzuzeigen. Das Format in dem diese Dateien verfasst sind ist eine von mehreren Programmiersprachen, die in aller Regel auf die beiden Buchstaben ML endet. Dieses ML steht für Markup Language. Eine \textbf{Markup Language}\index{Markup Language} wird von den meisten Softwareentwicklern nicht als \glqq{}echte\grqq{} Programmiersprache anerkannt. Sie können mit einer ML nämlich lediglich Elemente oder sogenannte Container definieren, die Texte oder Verweise auf beliebige Daten beinhalten. Interaktive Programme (also Anwendungen, die auf die Eingaben von Nutzern so reagieren, dass sich ihre Inhalte ändern) sind in HTML wie in beliebigen anderen Markup Languages nicht realisierbar.\\

An dieser Stelle müssen wir uns drei Dinge klarmachen:\\

\begin{enumerate}
	\item Die Inhalte (also Texte, Bilder, Hintergrundmusik, Videos usw. usf.) sind NICHT Teil einer Markup Language. Vielmehr gibt es innerhalb einer Markup Language verschiedene Möglichkeiten, um auf den Speicherort zu verweisen, an dem diese Inhalte sich befinden. Das kann auch durch eine weitere Programmiersprache passieren. Allerdings ist es meist möglich, Texte direkt innerhalb einer Markup Language einzutragen.
	\item Markup Languages sind statisch, weil Sie ausschließlich dazu dienen, Strukturen zu definieren. Alles dynamische wird in einer zweiten Programmiersprache programmiert.
	\item In einer Markup Language benutzen wir sogenannte Tags (sprich Täg), um zu definieren, was für eine Art von Element wir definieren wollen. In HTML5 benutzen wird beispielsweise das Tag \verb|<main>|, um einen Bereich zu definieren, der zentral für unsere Webanwendung ist. Wie dieser Bereich dann später angezeigt wird, ist für die Programmierung in HTML vollkommen irrelevant.
\end{enumerate}

Allerdings ist die Behauptung, dass eine Markup Language keine Programmiersprache ist schlichtweg falsch. Richtig ist dagegen, dass es weder eine imperative noch eine deklarative Programmiersprache ist. Richtig ist aber auch und insbesondere, dass Markup Languages außerordentlich wichtig sind, wenn wir es mit großen Softwareprojekten zu tun bekommen. In der objektorientierten Softwareentwicklung wird beispielsweise die \textbf{UML}\index{Programmiersprache!UML}\index{UML} (kurz für Unified Markup Language) genutzt, um übersichtlich darzustellen, aus welchen Komponenten ein Softwareprojekt besteht. Die UML ist aber noch zu wesentlich mehr zu gebrauchen, doch das würde jetzt vom Thema ablenken.\\

Markup Languages sind also ein wichtiges Werkzeug, wann immer wir ein umfangreiches Softwareprojekt realisieren wollen. Sie helfen uns, Fehler zu vermeiden und klare Strukturen im Projekt zu schaffen, wodurch die Teamarbeit deutlich erleichtert wird.\\

Programme, die Dateien in einer Markup Language, nämlich HTML anzeigen können kennen Sie, denn Sie nutzen Sie täglich: Es sind Webbrowser. Aber damit die Dateien von anderen Computern auf Ihrem Endgerät angezeigt werden können ist es noch nötig, dass sie von dort auf Ihr Endgerät übertragen werden. Wenn Sie also bislang dachten, dass es einen echten Unterschied zwischen einem Download und dem Besuch einer Webanwendung (bzw. einer Webpage) gibt, dann wissen Sie es jetzt besser: Alles, was in Ihrem Webbrowser angezeigt wird liegt mindestens so lange im Speicher des Endgerätes, wie es angezeigt wird. Denn für einen Computer macht es im Grunde keinen Unterschied, ob Daten \glqq{}nur\grqq{} im (RAM-)Speicher oder als sogenannte Datei auf einem Speicher wie einer Festplatte vorliegen.\\

\subsection{HTTP und HTTPS}

Damit Daten auf Ihr Endgerät übertragen werden gibt es verschiedene Verfahren, von denen Sie als Nutzer eines Gerätes nichts sehen. Diese Verfahren werden in Form sogenannter Protokolle vereinbart. Wie die einzelnen Protokolle aussehen, das ist Teil der Veranstaltung Netzwerke und Internetsicherheit. Für diesen Kurs genügt es, dass Sie wissen, was ein \textbf{Protokoll}\index{Protokoll} ist. Eines dieser Protokolle übersehen Sie praktisch jedes Mal, wenn Sie eine Webanwendung aufrufen: Es handelt sich um das \textbf{HTTP}\index{HTTP}\index{Protokoll!HTTP}, das \textbf{HyperText Transfer Protocol}. Das ist ein Protokoll, das einzig dazu entwickelt wurde, um die Übertragung von Dateien zu organisieren, die Teil des WWW sind.\\

Übrigens, wenn in der Adresszeile des Browsers nicht HTTP, sondern \textbf{HTTPS} steht, dann bedeutet das, dass der Datenaustausch verschlüsselt durchgeführt wird. Vielleicht erinnern Sie sich daran, dass Ihr Browser Sie mit einer Warnung genervt hat, wonach eine Webpage ein ungültiges Zertifikat genutzt hat. Ein solches Zertifikat ist bei der verschlüsselten Daten-\\übertragung essentiell: Es ist der Schlüssel, mit dem Ihr Browser die Daten verschlüsselt, die an eine bestimmte Seite übertragen wird. In einer Veranstaltung zur Netzwerksicherheit werden Sie lernen, was dabei im Hintergrund passiert und warum Browser manchmal ein gültiges Zertifikat nicht anerkennen.\\

Wichtig: Das, was Sie hier kennen lernen gilt genauso für Anwendungen im WWW. Mit den selben Grundlagen, mit denen Sie eine Webanwendung entwickeln, können Sie also auch ein Spiel oder eine beliebige andere Anwendung entwickeln, die auf jedem Endgerät läuft, das vernetzt ist und auf dem ein Webbrowser läuft. Unterschiede beim Betriebssystem wie zwischen \textbf{Android}\index{Android} und \textbf{iOS}\index{iOS} oder \textbf{Windows}\index{Windows} und \textbf{Linux}\index{Linux} sind dann vollkommen belanglos. \\

\textbf{Kontrolle}

\begin{itemize}
	\item Sie wissen jetzt, was der Unterschied zwischen Internet und WWW ist.
	\item Sie wissen, dass eine Webanwendung lediglich eine Ansammlung von Dateien ist, die über das Internet auf Ihren Rechner übertragen werden.
	\item Sie wissen, dass es Standards für diese Übertragung gibt, die in Form sogenannter Protokolle veröffentlicht und genutzt werden.
	\item Sie wissen, dass diese Dateien in einer von vielen Sprachen verfasst sind, die als Markup Languages bezeichnet werden.
	\item Sie wissen, dass die beiden Buchstaben ML im Namen einer Sprache für Markup Language stehen können.
\end{itemize}

\textbf{Ausblick}\\

Jetzt werden Sie mehr über die Programmiersprachen erfahren, die neben Markup Languages bei Webanwendung und Webanwendungen zum Einsatz kommen.\\

\section{Funktionalitäten unserer Webanwendung}

Der folgende Satz aus der Architektur ist für uns als SoftwareentwicklerInnen essentiell:\\

\textbf{Form follows function!}\\

Das bedeutet, dass zuerst die Funktion definiert werden muss, bevor es um gestalterische Fragen geht. Aber wie Sie ohne schon erkennen konnten kommt vor der Funktion die Struktur. Und zur Funktion gehört dann noch die Speicherung und Wiederverwendung von Nutzerdaten. Im Gegensatz dazu konzentrieren sich MediendesignerInnen auf das Design.\\

Zusammengefasst folgt daraus für uns:\\

\begin{enumerate}
	\item Zuerst legen wir die Struktur fest,
	\item dann legen wir für jedes Element der Struktur die Funktion(en) fest, die dieses Element anbieten soll.
	\item Abschließend legen wir fest, welche Nutzereingaben wie verarbeitet und ob sie gespeichert bzw. wiederverwendet werden sollen.
\end{enumerate}

Dagegen gilt hier:\\

\textbf{Design ist für uns irrelevant.}\\

Wenn Sie sich dagegen mit Design beschäftigen wollen, dann wechseln Sie bitte zu einem Studium des Medien- und/oder Kommunikationsdesigns; dort lernen Sie das, was Sie suchen. Hier sind Sie dann schlicht und ergreifend am falschen Platz.\\

Denn wir werden uns hier grundsätzlich mit dem Entwurf und der Entwicklung von Webanwendungen beschäftigen. Üblicherweise wird bei vergleichbaren Kursen und Tutorien im Netz damit begonnen, einen Webshop in HTML4.01 zu entwickeln, selbst wenn die Anbieter behaupten, es handle sich um eine Einführung in HTML5. Im Gegensatz dazu erhalten Sie hier einen ersten Einblick in aktuelle Methoden bei der Entwicklung von Webanwendungen.

\subsection{Erste Schritte zur Webanwendung}

Und der erste Schritt dazu hat etwas mit Stift und Papier zu tun: 

\begin{enumerate}
	\item Notieren Sie, was Ihre Webanwendung tun soll.
	\item Streichen Sie alles durch, was angibt, wie sie es umsetzen soll.
	\item Notieren Sie dann, welche Elemente Nutzer für welche Funktion angezeigt bekommen sollen. (Auch das Anzeigen von Texten ist eine Funktion.)
	\item Streichen Sie alle konkreten Textentwürfe durch. (\glqq{}Guten Tag, lieber Nutzer\grqq{} ist ein konkreter Text, \glqq{}Hier Begrüßung des Nutzers einblenden\grqq{} dagegen nicht.)
	\item Streichen Sie alles durch, was definiert, wie diese Dinge angezeigt werden sollen. (Zur Erinnerung: Wir machen Medieninformatik, nicht Mediendesign.)
\end{enumerate}

Wenn Sie jetzt gleich zum nächsten Abschnitt weitergehen, dann machen Sie etwas falsch, denn bevor Sie nicht ausführlich überlegt haben, was Sie auf Ihrer Seite unterbringen wollen, brauchen wir uns über die Programmierung oder das Design keine Gedanken zu machen.\\

\subsection{Projektdokumentation und Arbeitsumgebung}

Und damit sind wir beim wichtigsten Arbeitsmittel beim Projektstart: Papier. Für eigene Notizen im Format A4 oder A5, für Gruppenarbeiten sollten sie auch Bögen im Format A3 oder besser noch A2 besorgen. Klebeband, Textmarker und Stifte brauchen Sie natürlich auch. Schließlich wollen wir keine Papierflieger bauen.\\

Das mag in der Zeit von Digitalisierung und Vernetzung seltsam klingen, aber da Papier nun wirklich kein Luxusgut ist und es nunmal leichter ist, mehrere Blätter auf einen Tisch zu legen, um zu vergleichen, was den Mitgliedern eines Teams am besten gefällt, führt an dieser Stelle kein Weg an Stift und Papier vorbei. Idealerweise nutzen Sie dabei wenigstens ein Dutzend unterschiedlicher Farben. So können Sie beispielsweise jedem Mitglied Ihres Teams eine Farbe zuordnen. Dann ist erkennbar, wer welche Einträge vorgenommen hat.\\

Bei der Programmierung werden Sie noch mit einem Repository arbeiten, aber für den Moment bleiben wir bei Stift und Papier.\\

Für alle, die bereits mit Software gearbeitet haben, die eine solche Arbeit z.B. via Tablet ermöglicht und für diejenigen, die das gerne tun würden: Der einzige Grund, aus dem ich diese Methoden hier nicht unterstütze ist der, dass es heute noch keine günstigen Arbeitstische gibt, die diese Methoden für Teams unterstützen. Bei rund einhundert Studierenden pro Semester reden wir hier über wenigstens 10 entsprechende Tische, bzw. deutlich mehr als 100.000,- € zuzüglich der Kosten für zwei Räume à 50 \(m^2\), in denen diese Tische stehen. Denn Sie werden schlicht und ergreifend eine große Arbeitsfläche benötigen; da stellen Tablets aufgrund Ihrer beschränkten Formate eine zu große Beschränkung dar. Ansonsten bin ich hier voll und ganz auf Ihrer Seite und hoffe, dass die nötige Hardware sobald als möglich zu bezahlbaren Preisen auf den Markt kommt. Sie wollen ein Beispiel für das sehen, was ich meine? Dann sehen Sie sich den Film \grqq{}Casino Royale\grqq{} mit Daniel Craig an. Dort können Sie sehen, wie eine professionelle digitale und vernetzte Arbeitsumgebung für Teams aussehen kann. Allerdings fehlen dort Interfaces mit Tastatur und Maus, die fürs Programmieren unbedingt nötig sind.

\subsection{Erste Übung}

Die folgende Aufstellung ist für MS-Studierende gedacht, die in diesem Semester die Leistungsnachweise für PRG und Projekt 1 erwerben wollen. Für diese gilt:\\

\textbf{Alle Schritte müssen in der genannten Reihenfolge bis zum 25. März 2016, 12.00 Uhr erledigt sein, sonst ist keine Teilnahme am Projekt 1 in diesem Semester mehr möglich.}\\

Sollte die Veranstaltung in zwei Gruppen durchgeführt werden, dann gilt für die erste Gruppe die gerade genannte Frist, für die zweite Gruppe eine Frist bis zum 1. April 2016, 12.00 Uhr. (Nein, das ist kein Aprilscherz.)

\begin{enumerate}
	\item Bevor Sie in einem Team mit einem Projekt beginnen, überlegen Sie für sich, was für ein Softwareprojekt Sie gerne umsetzen würden. Wichtig ist hier nicht, ob Sie sich vorstellen können, es umzusetzen, sondern dass Sie sich ein interaktives Programm überlegen, das Sie schon immer haben wollten.
	\begin{itemize}
		\item Beschreiben Sie, was dieses Programm tut, bzw. wie Nutzer damit arbeiten, spielen oder wie auch immer interagieren.
		\item Streichen Sie alles, was mit Design und konkreten Inhalten zu tun hat.
		\item Es soll kein Roman dabei herauskommen: Kürzen Sie Sätze mit mehr als 10 Wörtern. Streichen Sie alle Adjektive. (Interessanterweise gibt es einen Literaturnobellpreisträger, der in seinen Romane die zweite dieser Vorgaben umgesetzt hat: Ernest Hemmingway)
		\item Formulierungen wie \glqq{}Es ist eine AAA-roguelike X4-Sim mit\\ Shooter-RPG-Elementen.\grqq{} gehören ins Marketing und zeigen, dass Sie nicht im Stande sind auszudrücken, was Ihre Anwendung ausmacht. 
	\end{itemize}
	\item Wenn Sie Ihre Projektidee in ein oder zwei Stunden wie gerade beschrieben grob skizziert haben, dann arbeiten Sie sich an Ihrem Rechner in Git ein. (Beachten Sie dabei alles, was im Kapitel \glqq{}Vorbereitung fürs Programmieren\grqq{} steht.)
	\item Erstellen Sie in Ihrem Repository ein Verzeichnis mit dem Namen Dokumentation. Speichern Sie darin Ihren Projektvorschlag in einer Datei mit dem Namen \verb|Projektidee.txt| . Wenn Ihnen nichts eingefallen ist, dann tragen Sie das in der Datei ein.
	\item Erstellen Sie (wenn noch nicht passiert) ein Repository bei GitHub, mergen Sie Ihr lokales Repository damit (bzw. pushen Sie eben Ihr lokales Repository auf GitHub) und tragen Sie mich als Collaborator ein. Mein Username bei GitHub lautet \verb|MarkusAlpers| .
	\item Melden Sie sich bei Helios für die Leistungsnachweise PRG und Projekt 1 an.\\
	Sollte die Anmeldung zurzeit nicht möglich sein, senden Sie mir bitte eine kurze Nachricht, damit ich in der Verwaltung alles nötige veranlassen kann.
	\item Senden Sie mir an meine Hochschuladresse\\ \verb|markus.alpers@haw-hamburg.de| eine E-Mail \textbf{von Ihrer Hochschulmailadresse} (private Mailadressen werden nicht angenommen) mit folgenden Angaben:
	\begin{itemize}
		\item Als Bezug: 
		\item \verb|Anmeldung Projekt 1 (MS)|
		\item Als Inhalt die folgenden Zeilen (nur ausfüllen, also nicht explizit Nachname als Wort angeben, sondern Ihren Nachnamen usw. eintragen):
		\item \verb|Nachname, Vorname, Matrikelnummer|
		\item \verb|Ihren Usernamen bei GitHub|
		\item \verb|Den Namen Ihres Repositories|
	\end{itemize}
\end{enumerate}

\subsection{Mehr zu den Leistungsnachweisen für alle}

Programmieren ist wie Fahrradfahren: Manche schaffen es auf Anhieb loszufahren und können relativ schnell an Wettrennen teilnehmen, andere brauchen Jahre, um auch nur den Einstieg zu schaffen. Es hat nichts mit auswendig lernen oder anderen Arten des \glqq{}Lernens\grqq{} im Sinne von \glqq{}büffeln\grqq{} zu tun, sondern beim Programmieren geht es (genau wie bei mathematischen Verfahren oder handwerklichen Tätigkeiten) fast ausschließlich darum, es wieder und wieder anzuwenden. Wenn Sie sich also nicht zu Hause hinsetzen und jede Woche mehrere Stunden selbst programmieren, dann ist Ihre Anwesenheit in diesem Kurs sinnlos und Sie werden keinesfalls einen Leistungsnachweis erhalten.\\

Umgekehrt hat Programmieren auf Hochschulniveau sehr viel mit Planung und Konzeption zu tun. Wenn Sie also die Inhalte der Veranstaltung ignorieren und einfach drauflos programmieren, wird auch das nicht funktionieren. Viele Studierende suchen dazu im Netz nach Lösungen, die sie nicht verstehen. Am Ende sagen Sie dann, dass Programmieren langweilig oder sinnlos ist. Dabei könnten Sie mit dem, was wir in dieser Veranstaltung besprechen praktisch alles selbst programmieren, was Sie an fertigen Programmen kaufen können.

\subsection{Mehr zu den Leistungsnachweisen für MS-Studis}

Wenn Sie einen Abschluss in Media Systems erwerben wollen, dann gibt es unter anderem die beiden Leistungsnachweise \verb|Projekt 1| und\\ \verb|Einführung ins Programmieren|. Diese beiden Leistungsnachweise erhalten Sie, wenn Sie das Projekt 1 erfolgreich abgeschlossen haben. Nachdem Sie die eben genannten Bedingungen erfüllt haben wird die gesamte Kommunikation zum Projekt über Ihr Repository erfolgen. Der weitere Ablauf wird so erfolgen, dass ich Ihren individuellen Leistungsstand regelmäßig (mindestens alle vier Wochen einmal) anhand dessen kontrolliere, was im Repository steht und Ihnen dort individuelle Hinweise gebe, wie Sie weiterarbeiten können.\\

Laut Modulhandbuch sollen Sie im Projekt nachweisen, dass Sie die Inhalte der Veranstaltung erfolgreich in einer Gruppe umgesetzt haben. Als Arbeitsaufwand sind 80 Stunden vorgesehen. Das schließt nicht die Vor- und Nachbearbeitung der Vorlesung bzw. des Seminars ein. Damit ergeben sich, dass Sie sich außerhalb der Veranstaltung rund 10 Stunden pro Woche mit den Themen der Veranstaltung und der Durchführung des Projekts beschäftigen müssen.\\

Häufig werde ich aufgefordert, Ihnen genau Angaben dazu zu machen, was Sie machen sollen. Das ist aber so konkret nicht möglich: Programmieren hat etwas mit der Fähigkeit zu tun, die eigenen Vorstellungen zu abstrahieren und sie in einer Form auszudrücken, die ein Computer ausführen kann. Ob Sie das schaffen lässt sich aber nicht daran messen, ob Sie die Umsetzung in fünf oder in 500 Zeilen durchführen. Es ist im Grunde so ähnlich wie mit Fahrradfahren: Wenn jemand mehrere hundert Meter mit einem Fahrrad ohne Stützräder gefahren ist, dann ist klar, dass er/sie es kann. Würde ich das aber auf den Leistungsnachweis übertragen, dann würden viele von Ihnen, die eine ausreichende Leistung erbracht haben keinen Leistungsnachweis erhalten. Und das wäre unnötig demotivierend. Deshalb lege ich keine solche messbare Grenze fest.\\

Es gibt aber noch einen weiteren Grund: Für den Leistungsnachweis müssen Sie eine gewisse Qualität erreichen, weil Sie MedieninformatikerInnen sind. Würde ich eine Liste aller möglichen Qualitätskriterien aufstellen, dann wäre das Ergebnis ein Katalog von mehreren hundert Seiten, den Sie erst dann verstehen würden, wenn Sie mehrere Jahre programmiert hätten. Einen Teil davon finden Sie allerdings in diesem Buch. Und selbst wenn Sie davon nur einen Teil umsetzen, stehen Ihre Chancen gut, den Leistungsnachweis zu erwerben.

\subsection{Mehr zum Leistungsnachweis für MT-Studis}

In Ihrem Studiengang gibt es für die Veranstaltungen \verb|Programmieren 1| und \verb|Programmieren 2| einen benoteten Leistungsnachweis, der nach Bestehen eines Projekts ausgestellt wird, das Sie in PRG2 durchführen. Dieses Projekt machen Sie voraussichtlich bei Herrn Wagener. Zur Durchführung kann ich deshalb nichts sagen. Bei mir erlernen Sie die Grundlagen, über die es eine schriftliche Prüfung am Ende dieses Semesters geben wird. Sie werden also anders als in der Schule nur eine Prüfung für die gesamte Veranstaltung schreiben.

\section{Technische Grundlagen verteilter Anwendungen}

Sie wissen bereits, dass eine verteilte Anwendung aus mehreren Programmen besteht, die über ein Netzwerk\footnote{oder über ein Netzwerk von Netzwerken wie das beim \glqq{}Internet\grqq{} der Fall ist}  miteinander kommunizieren. In den nachfolgenden Abschnitten und Kapiteln dieses Buches werden wir Schritt für Schritt alles besprechen, das Sie benötigen, um Sie Ihre erste verteilte Anwendung starten können: Eine Webanwendung, zu der es Nutzerkonten gibt und deren Inhalte sich abhängig von den Eingaben der Nutzer ändern können. \\

Wie Sie schon aus dem Abschnitt zur Vorbereitung auf die Entwicklung von verteilten Anwendungen wissen, gliedert sich dieser Teil des Buches in die folgenden Bereiche:\\

\begin{itemize}
	\item Festlegung, welche Funktionalitäten unsere Webanwendung enthalten soll.
	\item Betrieb eines Webservers und Zugriff durch einen Webbrowser.
	\item Einführung in die Markup Language HTML, um statische Webanwendung zu entwickeln.
	\item Bereitstellung dieser Webanwendung auf dem Webserver.
	\item Erweiterung der Anwendung durch die Nutzung von PHP, sodass wir eine dynamische Webanwendung erhalten.
	\item Einbindung einer Datenbank, die auf dem Webserver bereit gestellt wird, um so die Speicherung von Daten über individuelle Nutzer zu realisieren.
\end{itemize}

Wenn Sie diese sechs Punkte beherrschen, dann verstehen Sie die\\ grundsätzlichen Abläufe, die bei jeder verteilten Anwendung vorkommen. Der einzige Unterschied zwischen Ihnen und einem professionellen Entwickler solcher Anwendungen besteht in zwei Punkten: Zum einen die langjährige Erfahrung, zum anderen ein wesentlich besseres Verständnis, wie diese Abläufe stattfinden und wie hochkomplexe Projekte sinnvoll realisiert werden können.

\section{Erste Gehversuche mit Server und Client}

Nachdem wir die Planung der Webanwendung begonnen haben, fahren wir also damit fort, den Webserver zu nutzen bzw. zu prüfen, ob er soweit funktioniert, wie das für unsere Zwecke nötig ist:\\

Prüfen Sie bitte, ob EasyPHP den Server gestartet hat. Wenn das der Fall ist, öffnen Sie bitte einen Browser und geben dort in der Adresszeile (nicht im Suchfeld einer Suchmaschine!) das Wort localhost ein. Groß- und Kleinschreibung ist hier nicht von Belang.\\

Nun sollte eine Seite angezeigt werden, auf der am oberen Rand einige wenige Einträge mit Titeln wie \glqq{}version\grqq{} erscheinen. Im Zentrum der Seite sollten drei Verzeichnisse angezeigt werden. Wenn Sie diese anwählen, werden Sie feststellen, dass Sie allesamt leer sind.\\

Wiederholen wir an dieser Stelle die Frage, was denn eigentlich der \textbf{Server}\index{Server} und was der \textbf{Client}\index{Client} ist (momentan haben wir jeweils nur einen): Der Server ist ein sogenannter Apache Server. Dabei handelt es sich um ein Programm, das mit Hilfe von Dateien, auf die es Zugriff hat, verschiedene Dateien generieren kann, die über ein beliebiges Netzwerk übertragen und von einem Browser als etwas angezeigt werden, das wir unter der Bezeichnung Webanwendung bzw. Webpage kennen. Das Programm, das also die Daten bzw. Dateien vom Server nutzt (eben der Client) ist dementsprechnd der Browser, den wir nutzen.\\

Wenn es keine Datei gibt, die eine Webanwendung generiert, dann erzeugt ein Webserver Daten, die vom Client so angezeigt werden, wie Sie das von Ihrem Rechner beim Dateibrowser kennen: Da werden einerseits Dateinamen mit Endungen und andererseits Verzeichnisnamen angezeigt. Und für jeden dieser Einträge gibt es dann noch ein Symbol. Aber auch hier haben wir es wieder mit HTML-Dateien zu tun, denn sonst könnte der Browser sie nicht anzeigen.\\

Aber selbst wenn der Server auf eine Datei Zugriff hat, die definiert, wie eine Webanwendung aussehen soll, bedeutet das nicht, dass der Server basierend auf dieser Datei eine Webanwendung-Ansicht generiert. Vielmehr müssen Sie ihn so konfigurieren, dass er das tut.\\

Für den Fall, dass Sie sich wundern: Die Bezeichnung \textbf{localhost}\index{localhost} ist eine Art Alternativbezeichnung für eine bestimmte IP-Adresse. Darüber können wir einige Server nutzen, die auf unserem Rechner aktiv sind.\\

Was eine \textbf{IP-Adresse}\index{IP!IP-Adresse} ist? Ach ja richtig, Sie hatten ja noch keine Veranstaltung über Netzwerke... Eine IP-Adresse ist in Netzwerken (und damit auch im Internet) so etwas wie eine Telefonnummer für Computer. Diese IP-Adressen sind also eine Möglichkeit, damit Computer über Netzwerke Daten austauschen können. Denn ohne IP-Adresse hätten Sie ja keine Möglichkeit, einen bestimmten Rechner anzusprechen. Egal, ob Sie das wollen oder nicht: Dieser Datenaustausch passiert, wenn Sie einen Rechner, ein Smartphone oder welches vernetzte Gerät auch immer einschalten\footnote{Wie das im Detail funktioniert und wann genau eine Datenübertragung begonnen wird ist Teil jeder Veranstaltung zu IT-Netzwerken.}. Im Gegensatz zu Telefonnummern sind IP-Adressen aber nicht automatisch fest einem Computer zugeordnet, sie sind also im Regelfall dynamisch zugeordnet.\\

Wenn sie nun im Netz surfen, dann geben Sie im Regelfall nicht die IP-Adresse eines Rechners ein, sondern den \glqq{}Namen\grqq{} des Standorts einer Webanwendung. Solche Namen haben wie alles einen Fachbegriff: Die \textbf{URL}\index{URL} (kurz für Unified Remote Location) einer Webanwendung der HAW Hamburg ist beispielsweise www.haw-hamburg.de wobei Sie im Regelfall das www. am Anfang weglassen können. Eine andere URL der HAW Hamburg lautet beispielsweise www.mt.haw-hamburg.de. Der Begriff URL ist zwar gebräuchlich, aber im Kern falsch, denn auch wenn das L für Location bzw. Standort steht, können Sie aus der URL nichts über den Standort eines Servers aussagen. Der einzige Standort, den Sie aus einer URL ablesen können ist der Standort des Verwaltungsgremiums, das für die Vergabe von URLs zuständig ist. (In Deutschland ist das die DENIC.) Präziser ist da die Bezeichnung \textbf{URI}\index{URI} (kurz für Unified Remote Identifier).\\

Leider gibt es keine unterschiedliche Bezeichnung für eine einzelne Ansicht einer Webanwendung oder die Gesamtheit aller Webanwendung, die unter einer URL zu finden sind. Sie müssen also jeweils überlegen, ob nun eine einzelne Seite oder die Gesamtheit aller Seiten einer Webanwendung gemeint ist.\\

Und jetzt kommt etwas, dass viele Menschen nicht wissen: Wenn Sie eine solche URI in einen Webbrowser eingeben, dann schlägt dieser in einem Verzeichnis nach, welche IP-Adresse zu dieser URL gehört. Ein solches Verzeichnis ist ein Programm (hier spricht man auch von einem \textbf{Dienst}\index{Dienst}), denn es muss ja im Stande sein, eine Antwort zu senden. Und wie nennen wir solche Programme? Richtig, es ist ein Server, der als Domain Name Server, kurz \textbf{DNS}\index{DNS} Bezeichnet wird. Warum dieser Server so heißt, was Domänen sind, wie er funktioniert usw. usf. ist auch wieder Teil jeder Veranstaltung über Netzwerke. Ob ein Dienst nun ein einzelnes Programm oder eine verteilte Anwendung ist, soll uns an dieser Stelle nicht weiter interessieren. Der Begriff wird vorrangig im Bereich der \textbf{Nachrichten- und Kommunikationstechnik}\index{Nachrichtentechnik}\index{Kommunikationstechnik} verwendet, während wir als (Medien-)InformatikerInnen je nach Situation von Programm, verteilter Anwendung, Server, Client oder ähnlichem sprechen.\\

Wie bei allen Servern gilt auch hier: Dieser Server kann sich auf Ihrem Rechner befinden oder auf einem beliebigen Rechner irgendwo im Netz. Als Frau von der Leyen vor einigen Jahren forderte man müsse bestimmte Seiten im Netz sperren, indem man bei Suchanfragen ein Stoppschild einblenden würde, bewies Sie damit, dass sie nicht einmal die einfachsten Strukturen des Internet kannte. Denn um diese Stoppschilder zu realisieren, hätte jeder DNS-Server weltweit angepasst werden müssen. Und nur wer keine Ahnung vom Unterschied zwischen nationalem und internationalem Recht hat, kann auf die Überzeugung verfallen, dass beispielsweise die Administratoren eines DNS-Servers in Timbuktu sich bei ihrer Arbeit nach den Gesetzen der Bundesrepublik Deutschland richten. Vielleicht glaubt er oder sie auch an den Weihnachtsmann; wir beschäftigen uns hier mit der Informatik und ihren Auswirkungen in der Realität, deshalb wissen wir es besser.\\

Aber zurück zu unserem einfachen Webserver, bzw. zum Aufruf localhost im Webbrowser. Der Begriff localhost ist auch eine URI. Das bedeutet, dass ein Webbrowser im Regelfall auf einem DNS-Server nachsieht, welche IP-Adresse zu dieser Adresse gehört. Im Gegensatz zu anderen URIs gibt es aber für localhost eine Konvention: Dieser URL ist eine statische IP-Adresse zugeordnet: Es ist 127.0.0.1 Geben Sie diese Zahlenkombination einmal in Ihrem Browser in die Adresszeile ein.\\

Wenn Ihr Browser jetzt eine Übersicht von Internetseiten präsentiert, wie das bei Google der Fall ist, dann hat Ihr Browser entweder gar keine Adresszeile mehr, was leider immer öfter der Fall ist oder Sie haben die 127.0.0.1 in das Suchfenster eingetragen. Im zweiten Fall sollten Sie nochmal ernsthaft in sich gehen und überlegen, ob Sie nicht doch lieber irgend etwas anderes studieren wollen, denn dann haben Sie einen sehr hohen Nachholbedarf, was die Grundlagen der Bedienung eines Computers angeht.\\

Nochmal zur Erinnerung bezüglich der Begriffe statisch und dynamisch: Wenn wir in der Informatik davon reden, dass etwas statisch ist, dann handelt es sich dabei um eine Konvention, bzw. eine Vereinbarung und nicht um eine Art Naturgesetz. Denn im Gegensatz zu dem, was jemand mit dem Begriff \glqq{}statisch\grqq{} verbinden könnte, sind alle Daten dynamisch: Ein Computer kann nur mit Werten arbeiten, die er in die Register seines Prozessors lädt. Und der Inhalt eines solchen Registers kann nicht statisch sein. Das wiederum entspricht auch letztlich jedem Naturgesetz: Das einzig konstante ist die Veränderung.\\

\textbf{Wichtig}: Diese Art der IP-Adressen ist als \textbf{IPv4}\index{IP!IPv4} bekannt, wobei das v für Version steht. Der nächste genutzte Standard für IP-Adressen lautet \textbf{IPv6}\index{IP!IPv6} und wird immer häufiger eingesetzt. Bei Webanwendungen ist aber zurzeit IPv4 noch Standard. Die Unterschiede zwischen den Versionen 4 und 6 sind auch wieder Teil jeder Veranstaltung zu IT-Netzwerken.\\

\textbf{Kontrolle}\\

\begin{itemize}
	\item Sie wissen, wie Sie auf Ihren Rechner einen Webserver starten und Sie wissen auch, warum hier nicht automatisch eine Webanwendung erzeugt wird.
	\item Sie wissen, dass auf Ihrem Rechner sowohl Client als auch Server laufen können und Sie wissen, wie man diese Art von Server bzw. Client bei Webanwendungen nennt, bzw. welche Programme Sie dafür nutzen.
	\item Sie wissen, wie Sie auf die Einstiegsseite kommen, die der Webserver generiert.
	\item Sie kennen auch die Aufgabe eines DNS Servers, wissen was die\\ Abkürzungen IP und URL bzw. URI bedeuten und wie diese zusammenhängen oder eben nicht zusammenhängen.\\
	(Anmerkung: Die Bezeichnung DNS Server ist im Grunde doppelt gemoppelt, denn das S in DNS steht ja bereits für Server, aber dennoch ist es üblich, diese Server so zu nennen.)
\end{itemize}

\subsection{Eine erste Webanwendung mit PHP}

Erstellen Sie bitte das folgende Programm mit Hilfe eines Editors und speichern Sie es im Verzeichnis \url{C:/Program Files (x86)/} \\ \url{EasyPHP-DevServer-14.1VC11/data/localweb} unter dem Namen \verb|phpinfo.php| . Wenn Sie nicht EasyPHP installiert haben oder das Verzeichnis von EasyPHP bei der Installation geändert haben, müssen Sie\\ prüfen, wo Sie die Datei speichern müssen. Es ist auch möglich, dass der Name des Verzeichnisses sich bei einer späteren Version von EasyPHP \\ändern wird.\\

\begin{verbatim}
<html>
<body>
<?php
phpinfo();
?>
</body>
</html>
\end{verbatim}

Rufen Sie nun die Webanwendung auf. In der Ansicht sollte neben den drei Verzeichnissen auch der Dateiname phpinfo.php erscheinen. (Sonst haben Sie beim Speichern einen Fehler gemacht.) Wählen Sie diesen Dateinamen einmal mit der Maus an, so wie Sie es bei einem beliebigen Link auf einer Webanwendung tun würden.\\

Sie erinnern sich an die Aussage vom Anfang, wonach die Entwicklung verteilter Anwendungen umfangreiche Kenntnisse aus vielen Bereichen\\ benötigt? Die Seite, die Ihnen jetzt angezeigt wird, zeigt Ihnen die Konfiguration (also die Einstellung) aller Komponenten an, die alleine bezüglich der Darstellung von Elementen auf einer Webanwendung relevant sind. Das ist aber nur ein Teilbereich aller Konfigurationen, die bei der Entwicklung und beim Betrieb einer Webanwendung oder einer anderen verteilten Anwendung relevant sind. Und? Sind Sie schockiert über die Masse an Dingen, die Sie nicht verstehen? Gut, denn dann werden Sie hoffentlich nicht so überheblich wie die breite Masse an Entwicklern, die glauben, dass Sie Meister des Netzes sind, nur weil sie im Stande sind, eine Webanwendung zu entwickeln und zu betreiben. Denn dafür brauchen Sie so gut wie nichts von dem zu verstehen, was Ihnen gerade angezeigt wird. Aber bedenken Sie: All diese Einstellungen sind wichtig und müssen richtig konfiguriert sein, damit Ihre Webanwendung richtig funktioniert und um es Angreifern so schwer wie möglich zu machen, Ihre Webanwendung zu manipulieren.\\

\textbf{Kontrolle}\\

Sie wissen jetzt grundsätzlich, wie Sie eine Webanwendung auf einen Server übertragen können, und dass diese Übertragung bereits genügt, damit diese von Nutzern aufgerufen werden kann.\\

Sie wissen jedoch bislang nichts darüber, wie Sie eine Webanwendung programmieren müssen, um eigene Inhalte darauf zu präsentieren. Wir haben auch noch nicht über mögliche rechtliche Folgen einer Veröffentlichung von Inhalten auf einer Webanwendung gesprochen oder darüber, welche rechtliche Folgen es haben kann, dass Sie überhaupt eine Webanwendung ins Netz stellen. Doch bevor wir über weitere Details der Programmierung von Webanwendung sprechen, sollten wir dafür fünf Minuten aufwänden.\\

\textbf{Ausblick}\\

Nachdem Sie jetzt die Voraussetzungen geschaffen haben, um auf Ihrem System Webanwendungen zu entwickeln, wenden wir uns den Details der Softwareentwicklung in den drei Bereichen zu, mit denen Sie zu tun haben werden: Modell, Steuerung und langfristige Speicherung. Für jeden dieser drei Teile verwenden wir hier eine eigene Programmiersprache: Für das Modell HTML in der Version 5, für die Steuerung PHP in der Version 5.6 und für die langfristige Speicherung MySQL in der Version 5.6. An dieser grundsätzlichen Aufteilung und der konzeptionellen Arbeit wird sich bei Webanwendungen nichts ändern, auch wenn Sie z.B. JavaScript anstelle von PHP oder MongoDB anstelle von MySQL verwenden.\\

Seit einigen Monaten ist PHP 7 auf dem Markt. Wenn Sie damit weiterarbeiten wollen, sollten Sie sich hier einarbeiten. Da ich mich zurzeit darum kümmere, dieses Buch fertig zu stellen habe ich mir die Änderungen noch nicht im Detail angesehen und werde das auch bis Ende des Semesters nicht tun.

\section{Programmierung von Webanwendungen}

Die bekannteste Markup Language ist \textbf{HTML}\index{HTML}\index{Programmiersprache!HTML}, die HyperText Markup Language. Seit dem Herbst 1999 wurde HTML in der Version 4.01 verwendet, seit Herbst 2014 steht HTML5 zur Verfügung. Beide Versionen haben nur gemein, dass HTML5-Dokumente auch HTML4-Code beinhalten können, und dass dieser wie bisher ausgeführt wird. Sonst nichts. Wer denkt, HTML5 sei nur eine Erweiterung von 4.01, hat es nicht verstanden. Es ist im Kern eine vollständig neue ML, die entwickelt wurde, um alle nur denkbaren multimedialen und auch interaktiven Inhalte verfügbar zu machen.\\

Die Kompatibilität zu Version 4 wurde eingeführt, damit es keine Probleme mit alten Webanwendung gibt. Insbesondere unterstützt HTML5 direkt die objektorientierte Softwareentwicklung und ist darauf ausgelegt im Verbund mit JavaScript jede Art von Programmen im Browser zu realisieren. Das bedeutet unter anderem, dass \textbf{Flash}\index{Flash} bzw. ActionScript nunmehr überflüssig geworden ist. \textbf{ActionScript}\index{Programmiersprache!ActionScript} ist die Sprache, in der Flash-Anwendungen programmiert werden und die zum Eigentum von Adobe gehört. Es ist eine Alternative zu JavaScript.\\

Wie gesagt definieren Sie in einer Markup Language lediglich Elemente bzw. Container, die Texte und Verweise auf Dateien enthalten. Die Darstellung des Inhalts dieser Container wird dagegen über sogenannte Cascading Style Sheets (kurz \textbf{CSS}\index{Programmiersprache!CSS}) festgelegt und über die Einstellungen des Browsers auf dem Rechner eines Nutzers.\\

An dieser Stelle ein Hinweis für diejenigen von Ihnen, die bereits mit HTML programmiert haben: (Alle anderen lesen bitte beim nächsten Absatz weiter, da Sie das hier noch nicht verstehen können.) Sie haben wahrscheinlich schon Attribute wie align genutzt. Das dürfen Sie unter HTML5 nicht mehr in dieser Form tun! Hier müssen Sie alles, was sich auf das Layout bezieht in CSS programmieren. Das mag ungewohnt sein, ist aber sinnvoll, weil auf diese Weise die Trennung zwischen der Definition von Elementen (was unter HTML passiert) und ihrer Darstellung (was unter CSS programmiert wird) eindeutig geklärt ist.\\

Um nun Programme wie beispielsweise Spiele im Browser zu entwickeln werden verschiedene Ansätze gewählt. Auch hier wurde durch die\\ Einführung von HTML5 ein Standard eingeführt: Während bei Version 4 keine Programmiersprache als Standard vorgesehen ist, sieht Version 5 die Anbindung von Programmen in der Programmiersprache JavaScript als Standard vor. JavaScript ist eine der mächtigsten Sprachen, die Sie zurzeit nutzen können, um Anwendungen auf einem Rechner oder in einem Browser zu entwickeln. Auch aus diesem Grund werden Sie JavaScript immer öfter auf Webanwendung finden. \\

Eine Sprache, die früher häufig eingesetzt wurde und deshalb auch heute noch weit verbreitet ist, ist \textbf{PHP}\index{Programmiersprache!PHP}. JavaScript ist zwar mächtiger und aus diesem Grund empfehlenswerter, aber es hat einen Nachteil für Sie, wenn Sie sich in die professionelle Teamarbeit einarbeiten wollen: Während Sie in PHP nichts programmieren können, das Sie in HTML oder CSS programmieren können, ist es möglich, nahezu eine vollständige Webanwendung in JavaScript zu entwickeln. Das führt dann bei Einsteigern schnell dazu, dass der JavaScript-Programmierer Teile übernimmt, die andere umsetzen sollten. Und die haben dann nichts mehr zu tun.\\

Ursprünglich wollten die Entwickler durch den Einsatz von Sprachen wie PHP und JavaScript nur erreichen, dass Webanwendung dynamisch werden. Eine \textbf{dynamische Webanwendung}\index{dynamisch} ist schlicht eine Webanwendung, deren Inhalte sich z.B. durch Eingaben von Nutzern ändern können. Die Möglichkeiten, die HTML5 zusammen mit JavaScript bietet gehen weit darüber hinaus.\\

Zu guter Letzt brauchen Sie als Entwickler einer Webanwendung noch eine Möglichkeit, um Daten langfristig zu speichern. Hier kommen die sogenannten \textbf{Datenbanken}\index{Datenbank} zum Einsatz: Eine Datenbank ist eine Ansammlung von Tabellen, in denen Daten in standardisierter Form abgespeichert werden. Im Gegensatz zu Dateien liegen Datenbanken dabei ständig im Speicher eines Rechners vor und können deshalb genutzt werden, ohne dass Sie zunächst von einer Festplatte oder einem USB-Stick geladen werden müssten.\\

Auch hier bekommen wir es mit Programmiersprachen zu tun, namentlich mit den sogenannten Query Languages (kurz QL) oder Structured Query Languages (kurz \textbf{SQL}), was übersetzt so viel wie strukturierte Anfrage-Sprache bedeutet. Denn nichts anderes tun diese Sprachen: Sie ermöglichen es einem Nutzer in standardisierter Form Anfragen an eine Datenbank zu stellen und geben die Antwort der Datenbank in einer Form aus, die für Menschen lesbar ist. Die bekannteste dieser Sprachen ist \textbf{MySQL}\index{Programmiersprache!MySQL}.\\

Wichtig: SQL steht dagegen nicht für sequel (zu Deutsch: Nachfolger). Es ist aber im englischsprachigen Raum durchaus üblich, Abkürzungen nicht zu buchstabieren, sondern sie wie ein Wort auszusprechen. Deshalb wird SQL gelegentlich als sequel bezeichnet.\\

Während PHP so entwickelt wurde, dass es die Nutzung von MySQL direkt unterstützt, benötigt JavaScript eine Erweiterung dafür. Eine recht neue Erweiterung heißt \textbf{Node.js}. Dieser Ansatz eine Sprache über Erweiterungen zu ergänzen, anstatt diese von vornherein in die Sprache zu integrieren, wirkt auf den ersten Blick komplizierter als die vermeintlich einfache Kombination von PHP und MySQL, aber es ist einfach nur ein Ansatz, der es ermöglicht, die Sprache selbst nicht zu umfangreich werden zu lassen. Was das im Detail bedeutet können Sie in Veranstaltungen zum Software Engineering lernen.

\subsection{MVC – Das Model View Controller Pattern}

Diese Aufteilung der Programmierung einer Webanwendung entspricht einem Entwurfs- und Architekturmuster, das auch in der Wirtschaft angewendet wird. Dieses Muster wird als MVC (kurz für Model View Controller) bezeichnet. Pattern ist schlicht das englische Wort für Muster.\\

Solche Modelle spielen bei der Entwicklung großer Softwareprojekte eine essentielle Rolle, denn indem wir sie nicht wie einen Monolithen, sondern als Kombination von einzelnen Komponenten entwickeln, können wir leichter Fehlerkorrekturen durchführen und Änderungen einbauen.\\ Außerdem können wir so die Arbeit leicht innerhalb eines Teams mit mehreren Dutzend Entwicklern verteilen.\\

Allerdings kommt es nur sehr selten vor, dass für jeden Teil eines Patterns auch eine eigenständige Programmiersprache genutzt wird. Das macht es dann gerade für Einsteiger schwer, zu verstehen, wo die Grenze zwischen Model und View und Controller liegt. Das ist bei der Kombination aus HTML5, CSS und PHP oder JavaScript anders: Hier können wir jedem Bestandteil des Patterns genau eine Sprache zuordnen, so wie das oben schon passiert ist. Die SQ-Sprache lassen wir hier mal außen vor; sie dient ja nur dazu, die langfristige Speicherung von Daten zu ermöglichen.\\

In unserem Fall ist HTML5 die Sprache für das Model, CSS ist die Sprache für den View und PHP bzw. JavaScript ist die Sprache für den Controller. Deshalb werden wir uns im Kapitel über HTML auch nur darum kümmern festzulegen, aus welchen Teilen wir eine Webanwendung zusammensetzen können. Aus dem gleichen Grund werden wir uns im Kapitel über CSS ausschließlich darum kümmern, wie die Elemente der Seite dargestellt werden sollen. Und im Kapitel über PHP bzw. JavaScript wird es dann ausschließlich darum gehen, wie wir Änderungen steuern können, die während der Nutzung der Webanwendung auftreten.\\

Diejenigen von Ihnen, die Media Systems studieren werden später in der Veranstaltung Software Engineering eine Vielzahl von Patterns kennen lernen. Leider bleibt dort kaum Zeit, sich einem dieser Design Patterns\\ ausführlich zu widmen. Aber so haben Sie jetzt schon die Möglichkeit, den Nutzen eines solchen Patterns kennen zu lernen.\\

\textbf{Kontrolle}\\

\begin{itemize}
	\item Sie wissen, dass Sie bei der Programmierung von Webanwendung vier Arten von Programmiersprachen nutzen, und dass Sie damit ein Design Pattern der Informatik umsetzen, den MVC:
	\begin{itemize}
		\item Markup Languages dienen dazu, Container zu definieren, die die Inhalte Ihrer Webanwendung enthalten oder die auf Dateien verweisen, die als Teil einer Webanwendung angezeigt werden.
		\item CSS dient dazu, die Darstellung der Inhalte zu gestalten.
		\item Sprachen wie JavaScript und PHP werden dazu genutzt, um dynamische Webanwendung bzw. Webanwendungen zu realisieren.
		\item SQL-Sprachen dienen dann dazu, um z.B. Nutzereingaben dauerhaft zu speichern, um Kataloge von Waren bereitzustellen oder andere große Mengen an Daten aufzubewahren.
	\end{itemize}
	\item Ihnen ist klar, dass Sie damit wesentlich mehr realisieren können als \glqq{}nur\grqq{} Webpages, auch wenn Sie noch nicht wissen, wie Sie das tun können.
	Ausblick
\end{itemize}

Jetzt werden Sie den nächsten Schritt der Web-Evolution kennen lernen: Das semantische Web.

\section{Das semantische Web}\index{semantisches Web}

Bis hierher haben Sie nur über Dinge gelesen, die Sie unter Umständen schon wussten. Doch nun kommen wir zu einem Thema, dass den meisten Menschen und leider auch vielen Webentwicklern nicht bekannt ist, obwohl es der nächste Schritt für die Nutzung des WWW ist. Die Rede ist vom semantischen Web. Ohne das Verständnis, was das semantische Web ist, können Sie mit HTML5, dem neuen Standard für Webanwendung nichts anfangen. Sehen Sie sich dazu bitte die folgenden Einleitung ein: \url{https://plus.google.com/+ManuSporny/posts/FPwMGWhgQYh?cfem=1}

Sehen wir uns das mal im Detail an:

\subsection{Syntax und Semantik}

Den einen dieser Begriffe haben Sie wahrscheinlich in der Schule kennen und hassen gelernt. Dabei stehen die beiden für ein ausgesprochen sinnvolles Konzept. Mit \textbf{Syntax}\index{Syntax} werden all die Aspekte einer Sprache bezeichnet, bei denen es darum geht, welche Buchstaben und andere Zeichen in welcher Reihenfolge notiert werden dürfen. (Denken Sie an so etwas wie Satzbau, Kommasetzung, usw.) Wenn also eine Sprache ohne Syntax auskommen könnte, gäbe es keine Absprache darüber, welche Wörter es gäbe und wie ein Satz aussehen kann. Und egal ob Sie diesen Stil nun mögen oder nicht, auch der folgende Satz folgt einer Syntax: \glqq{}Ey Alter, was geht’n?\grqq{}\\

Damit kommen wir zur \textbf{Semantik}\index{Semantik}. Mit Semantik bezeichnen wir alles, was uns sagt, welche Bedeutung etwas in einer Sprache hat. Es ist wichtig zu verstehen, dass zwei syntaktisch unterschiedliche Sätze in semantischer Hinsicht gleich sein können. So gibt es einen großen syntaktischen Unterschied zwischen \glqq{}Ey Alter, was geht’n?\grqq{} und \glqq{}Hallo Herr Schulz, wollen wir Essen gehen?\grqq{} aber semantisch ist dieser Unterschied eher gering. Das gilt genauso für Programmiersprachen: Vieles lässt sich in gänzlich unterschiedlicher syntaktischer Form programmieren und erfüllt doch die selbe Aufgabe.\\

Das führt uns auch nochmal zu der Antwort auf die Frage, warum bei es bei den Projekten der MS-Studierenden keine feste Vorgabe für den Umfang gibt: Da mit extrem unterschiedlichem Code identische Funktionalitäten zu realisieren sind und es nicht um die Form, sondern um die Funktion geht, sind Sie in der Wahl der Form recht frei. Da aber auch dir Funktion von Element zu Element unterschiedlich ist, sind Sie auch in der Wahl der Formen recht frei. Nur die Programmiersprachen und die zu verwendende Version sind vorgegeben.\\

Ein fähiger Softwareentwickler ist somit wie ein Diplomat oder ein Dichter: Er kann mit Sprachen umgehen wie ein Küstler. Und das nicht, weil er ihre Regeln gelernt hat, sondern weil er sie nutzt, um Ideen und Konzepte in eleganter Form auszudrücken. Deshalb ist der erste Schritt zu einem guten Programm auch nicht der Griff zur Tastatur, sondern zu Stift und Papier, um Konzepte und Relationen zu skizzieren, sie zu überarbeiten und ein sinnvolles Ganzes daraus zu gestalten. Erst wenn dieser Schritt abgeschlossen ist, beginnt die eigentliche Programmierung.\\

Wieder zurück zum Begriff der Semantik: Es gibt einen sehr großen Unterschied zwischen der Semantik bei gesprochenen Sprachen und bei Programmiersprachen: Die Semantik einer Zeile einer Programmiersprache ist immer eindeutig: Sie besagt je nach gewähltem Paradigma entweder was der Computer tun soll oder wie er es tun soll. Bis auf HTML5 gibt es aber noch keine Programmiersprache, die auch aussagt, was das ganze im Sinne der Kommunikation von Menschen bedeutet!\\

Stellen Sie sich dazu die folgende Situation vor, die Schultz von Thun in seinem Standardwerk über zwischenmenschliche Kommunikation präsentiert: Ein Mann sitzt am Steuer seines Wagens und seine Frau sagt zu ihm: \glqq{}Es ist grün.\grqq{} Dieser Satz lässt sich unterschiedlich interpretieren. Das ist das Problem mit der Semantik in gesprochenen Sprachen: Sie ist im Regelfall nicht eindeutig.\\

Die meisten Anfänger einer Programmiersprache verstehen deshalb nicht, dass es bei Programmiersprachen keine unterschiedliche Interpretation eines Programms gibt. Wenn es also bei einer Programmiersprache das\\ Äquivalent zu \glqq{}Es ist grün.\grqq{} gibt, dann hat dieser Programmbefehl genau eine Bedeutung. Daraus folgt auch, dass Informatiker in aller Regel versuchen, Dinge so eindeutig wie möglich zu formulieren. Es geht ihnen nicht darum, Dinge zu verkomplizieren, sondern darum, Missverständnisse zu vermeiden.\\

Woran Sie erkennen können, dass selbst erfahrene Programmierer häufig ignorieren, dass die Semantik von Programmiersprachen eindeutig ist?\\ Ganz einfach: Immer wenn Sie den Satz \glqq{}Warum macht der das denn nicht?!\grqq{} hören, ignoriert jemand diese fundamentale Tatsache. Aber da Programmierer eben auch nur Menschen sind, ist das vollkommen in Ordnung.\\

Aber was hat das mit dem WWW zu tun? Und wie passt hier der Begriff semantic web ins Bild? Dazu eine kleine Übung:\\

\textbf{Aufgabe}

\begin{itemize}
	\item Suchen Sie im Internet (per google.de und duckduckgo.com) nach dem Wort Musik. Vergleichen Sie die Ausgaben der beiden Suchmaschinen. 
	\item Überlegen Sie, warum Menschen mit den Ergebnissen dieser Suche unzufrieden sein könnten, egal welche Suchmaschine sie nutzen.
	\item Überlegen Sie, was Sie eingeben müssten, damit eine Suchmaschine Ihnen Angebote für CDs Ihrer Lieblingsband anzeigt.
\end{itemize}

\textbf{Kontrolle}\\

Sie haben eine erste Idee davon, dass das semantic Web etwas damit zu tun, welche Bedeutung Wörter und multimediale Inhalte im WWW haben.

\subsection{Semantik und das WWW}

Wichtig: Oben haben Sie eine Bedeutung des Wortes Semantik erfahren: Es ist die Bedeutung eines Programmbefehls, bzw. eines Teils eines Computerprogramms. Beim semantic Web geht es aber nicht darum, wie ein Computerprogramm auszuführen ist. Wenn Sie das verwirrt, denken Sie bitte daran, dass der Begriff der Semantik allgemein als „Bedeutung und Interpretation von etwas“ übersetzt werden kann. Wenn wir also vom semantic Web reden, dann geht es um die Bedeutung von etwas, das mit dem WWW zu tun hat, bzw. Teil des WWW ist.\\

Zurück zum eigentlichen Text:\\

Die Aufgabe oben war schwierig, aber durch die einleitenden Erklärungen zum WWW hatten Sie das Wissen, um die Ursache des Problems zu erkennen. Dort haben Sie erfahren, dass Webanwendungen in Markup Languages programmiert werden. Sie haben dort ebenfalls erfahren, dass Sie in diesen Sprachen lediglich Container definieren können. Und was ist das Problem mit Containern? Genau: Die meisten Menschen sehen nur die Verpackung aber nicht den Inhalt. Und das gleiche gilt für Container einer Markup Language: Suchmaschinen tun sich schwer damit, die Webanwendung zu finden, nach denen ein Nutzer sucht, weil sie keine semantischen, sondern ausschließlich syntaktische Informationen bekommen, wenn ein Benutzer eine Eingabe macht. Und umgekehrt stellen Webanwendung ebenfalls keine semantischen, sondern ausschließlich syntaktische Informationen zur Verfügung.\\

Es hat seit der Einführung von HTML4.01 mehrere Versuche gegeben, dieses Problem zu lösen. Ein Lösungsansatz besteht darin, sogenannte Labels zu programmieren. Damit befestigt der Entwickler einer Webanwendung gewissermaßen Schilden an die Container. Aber auch das ist ja wieder nur eine syntaktische Information, also löst es das Problem nicht.\\

\textbf{Kontrolle}\\

Wenn Computer Dateien in einer ML per HTTP austauschen, dass haben Sie nicht die geringste Ahnung, womit sie es zu tun haben. Also müssen wir einen Standard haben, mit dem wir semantische Informationen in ML-Dateien unterbringen können. Denn nur dann können die Computer nach diesen Informationen suchen.

\subsection{DOM und Microdata}

Wenn Sie sich schon einmal mit der Entwicklung von Webanwendung\\ beschäftigt haben, dann haben Sie vielleicht schon die Abkürzung DOM gesehen. Das Document Object Model (kurz DOM) ist die Basis für die Nutzung von JavaScript im Browser. Es geht hier schlicht darum, dass jeder Container einer ML von der Sprache JavaScript als ein eigenständiges Objekt behandelt werden kann. Die Sprache kann dann den Inhalt eines solchen Containers ändern. Über DOMs kann außerdem definiert werden, wie verschiedene Container voneinander abhängen. All das hilft uns aber bei der Suche nach einem semantic Web nicht weiter, weil es immer noch keine semantischen Informationen bereitstellt, sondern nur klärt, wie die Elemente einer Webanwendung voneinander abhängen.\\

Microdata sind nun gewissermaßen eine Erweiterung von DOMs, die mit HTML5 eingeführt wurden: Hier enthält ein Container Einträge, die in standardisierter Form angeben, welche Bedeutung ein Eintrag in einem Container hat. Wenn Sie also Microdata verwenden, dann bedeutet das, dass Sie eine Anschrift in HTML5 so einprogrammieren, dass der Browser erkennen kann, dass es eine Anschrift ist, welcher Teil die Straße ist, welche Textpassage der Name ist, usw. Ein Besucher dieser Webanwendung braucht dann nichts weiter zu tun, wenn er diese Anschrift in sein Adressbuch übernehmen will oder die Anschrift in einem Kartenprogramm suchen will: Durch die Microdata kann der Browser die entsprechenden Verlinkungen erzeugen und die nötigen Daten weitergeben. Mit Microdata erhalten wir also eine semantische Webanwendung.\\

Bitte denken sie nicht, dass das nur für Anschriften gilt. Auf der Seite \url{http://www.schema.org/docs/full.html} finden Sie einen\\ Überblick über alle Arten von Microdata, die Sie verwenden können. Bitte versuchen Sie nicht, all diese Typen auswendig zu lernen; wie so oft in der Informatik ist auswendig lernen unsinnig, da kontinuierlich Änderungen erfolgen und es viel wichtiger ist, zu wissen, wo Sie etwas finden, als dass Sie wissen, wie es im Detail aussieht.\\

Neben naheliegenden Inhalten wie Anschriften finden Sie hier auch Microdata für Veranstaltungen. Richtig gelesen: Wenn Sie innerhalb eines Containers einen Eintrag als solch einen Typ von Microdata programmieren, dann brauchen Sie auf der ganzen Seite nicht ein einziges Mal den Begriff Veranstaltung zu benutzen: Eine Suchmaschine kann alleine durch diese Typangabe erkennen, dass es sich um eine Veranstaltung handelt und nicht etwa um einen Leitfaden für die Planung und Durchführung einer solchen.\\

Neben den negativen Möglichkeiten sorgt das semantic Web auch dafür, dass Manipulationen durch Betreiber von Preisportalen und ähnlichen Webanwendung in Zukunft weniger erfolgreich sein werden: Da bei konsequenter Anwendung von HTML5 und Microdata eine Suchmaschine direkt erkennt, wo eine Seite ein Angebot z.B. für Flugreisen anbietet, werden Nutzer langfristig eine eigene Suche nach Flugreisen durchführen können. Deshalb werden dürften diese Portale mittelfristig wieder vom Markt verschwinden.\\

Und das Großartige bei all dem ist, dass es wie beim WWW kein gewinnorientierte Unternehmen gibt, dass für die Nutzung von HTML5 Geld verlangt.\\

\textbf{Kontrolle}

\begin{itemize}
	\item Sie wissen jetzt, dass Sie in HTML5 eine fast schon unüberschaubare Menge an Informationen so verpacken können, dass Browser wissen, um was für Daten es sich handelt. 
	\item Sie haben eine erste Vorstellung davon, wie umfangreich die Auswirkung des semantic Web ist.
\end{itemize}

\subsection{Zusammenfassung}

Sie haben jetzt einen ersten Überblick über Techniken und Technologien, die Sie nutzen können, um Webanwendungen entwickeln können. Wenn Sie dieser Einleitung aufmerksam gefolgt sind, dann haben Sie außerdem verstanden, dass es bei der Entwicklung einer Anwendung, eines Computerprogramms oder eine verteilten Anwendung nicht zuerst darauf ankommt, welche Programmiersprache Sie nutzen wollen, sondern welche Konzepte und Ideen Sie umsetzen wollen.\\

\section{Übertragung der Dateien auf einen Webserver im Netz}

Wenn Sie im Gegensatz zum hier beschriebenen Vorgehen bereits einen Webserver nutzen, der Ihre Webanwendung im Netz anbietet und über das Internet erreichbar ist, dann müssen Sie noch wissen, was für ein Programm Sie zur Datenübertragung nutzen können.\\

Hierzu müssen Sie zunächst wissen, was ein \textbf{Protokoll}\index{Protokoll} ist und welche Protokolle es gibt.\\

Protokolle sind Vereinbarungen darüber, wie bestimmte Abläufe geregelt werden. Klingt kompliziert? Ist es nicht. Ein praktisches Protokoll ist die Begrüßung: Es kann sein, dass Sie jemandem zur Begrüßung die Hand schütteln, ihr ein einfaches Hallo sagen, oder wie auch immer Sie die Begrüßung handhaben. Wenn Sie mit einer anderen Person eine feste Vereinbarung treffen sollten, wie die Begrüßung ablaufen soll, dann entspricht das dem, was wir als Protokoll in einem Netzwerk bezeichnen. Sie habens noch nicht verstanden? Dann sehen Sie ein paar Folgen Star Trek.\\

Wie Sie bereits wissen, gibt es verschiedene Arten, wie man Daten darstellen kann. Zur Erinnerung: In der \textbf{Nachrichtentechnik}\index{Nachrichtentechnik} wird das über die sogenannte \textbf{Codierung}\index{Codierung} festgelegt. In der Informatik werden üblicherweise nur Codes genutzt, die sich als Codetabelle darstellen lassen oder die mittels einer mathematischen Funktion erfolgen. Vereinfacht ausgedrückt definiert ein Protokoll unter anderem, mittels welcher Codierung Daten zwischen Server und Client ausgetauscht werden.\\

Ein Protokoll bekommen Sie fast immer angezeigt, wenn Sie im Netz unterwegs sind. Es handelt sich um das \textbf{HTTP}\index{HTTP}, das Hyper Text Transfer Protokoll. Übersetzt ist das also ein Protokoll, das festlegt, wie Hypertexte (was auch immer das sein mag) von einem Client angefordert und an ihn übertragen werden.\\

Nun also zum Begriff \textbf{Hypertext}\index{Hypertext}: Wie Sie wissen werden in Webanwendungen Inhalte angezeigt, von denen aus Sie über Links (Fachbegriff \textbf{Hyperlink}s\index{Hyperlink}) zu anderen Webanwendungen bzw. zu anderen Inhalten gelangen. Um zu betonen, dass ein Dokument aus Texten und aus Verbindungen zwischen entfernten Textpassagen besteht, wurde der Begriff des Hypertexts entwickelt. (Das ist heute eine Selbstverständlichkeit, aber versuchen Sie mal einen Link jemandem zu erklären, der bislang nur Bücher aber keine Webpages kennt. Denn bei der Nutzung eines Links springen Sie ja direkt an eine bestimmte Stelle, ohne erst nachschlagen und nach der entsprechenden Stelle im Buch suchen zu müssen.)\\

Nun aber zu zwei Protokollen, die Sie für die Datenübertragung auf einen Webserver kennen müssen: Das FTP und den SSL. Wenn Sie konzentriert gelesen haben, dann haben Sie bemerkt, dass HTTP es nicht ermöglicht, ganze Dateien beliebiger Dateiformate zu übertragen, sondern dass es nur dazu dient, die Inhalte von Webanwendungen anzufordern und zu \\übertragen.\\

Das \textbf{FTP}\index{FTP}\index{Protokoll!FTP}, kurz für File-Transfer-Protokoll löst nun genau diese Aufgabe: Es definiert, wie ein Client Dateien auf einen Server übertragen kann. Aus Sicherheitsgründen sollten Sie jedoch prüfen, ob dieses Protokoll für Ihre Zwecke ausreichend ist. Denn eine Datenübertragung per FTP ist öffentlich. Das bedeutet in Kurzform: Alle angeschlossenen Nutzer können sämtliche Daten kopieren, die Sie über ein Netzwerk übertragen. Ja! Natürlich gilt das auch für Ihre Passwörter. Und das gleiche Prinzip gilt auch für HTTP. Denn diese Dienste wurden entwickelt, um bestimmte Daten mit einer möglichst hohen Effizienz über Netzwerke zu übertragen und jede Form von Sicherheit reduziert im Regelfall die Effizienz einer Übertragung. In den\\ Äußerungen mancher politisch interessierten Personen zeigt sich hier das mangelnde Verständnis für die Technologie: Denn das hat nichts mit einem Wunsch nach allumfassender Überwachung zu tun, das hat etwas mit der Anwendung von Naturgesetzen zu tun.\\

Wollen Sie dagegen weitestgehend sicherstellen, dass niemand die\\ übertragenen Dateien ändern oder auch nur kopieren kann, dann sollten Sie eine Verbindung mittels \textbf{SSL}\index{SSL}\index{Protokoll!SSL}, kurz für Secure-Socket-Link wählen. Allerdings kann es sein, dass der Webserver diese Art des Datentransfers nicht unterstützt. Gerade bei Billigangeboten wird SSL-Unterstützung gerne als kostenpflichtiges Update verkauft. Ob SSL nun ein eigenes Protokol oder die Erweiterung eines Protokolls ist, lassen wir hier einmal außen vor.\\

Ähnlich wie SSL bietet auch \textbf{HTTPS}\index{HTTPS}\index{Protokoll!HTTPS} eine sicherere Möglichkeit, um im Netz Daten zu übertragen. Beachten Sie aber bitte: Eine absolute Sicherheit gibt es auch in Netzwerken nicht. Der Sinn von Netzwerken bestand sehr lange auch ausschließlich darin, Verbindungen zu ermöglichen, nicht darin, die Vertraulichkeit der übertragenen Daten sicherzustellen. Über Details zu diesen Themen hören Sie mehr in der Veranstaltung Netzwerke und Internetsicherheit.\\

\textbf{Kontrolle}\\

Sie wissen jetzt, dass Sie für die Übertragung von Dateien auf einen Webserver ein FTP- oder ein SSL-Programm benötigen. Sie wissen auch, dass Sie im Netz nicht nach FTP-Programm oder SSL-Programm suchen, weil Sie wissen, dass es für solche Programme eine andere Bezeichnung als das Wort Programm gibt.\\

Wie Sie ein solches Programm bedienen müssen ist nicht Teil dieser Veranstaltung, da auch hier erneut weitergehende Kenntnisse bei der Administration von Servern nötig sind, die mit der Entwicklung einer Webanwendung bzw. den Grundlagen der Programmierung nichts zu tun haben.


\chapter[Grundlagen verteilter Anwendungen]{Grundlagen der Entwicklung verteilter Anwendungen}
\chapter{Einstieg in HTML 5}

Dass Sie in diesem Kapitel die Grundlagen der Programmierung in HTML5 erlernen ist nach der Kapitelüberschrift klar. Aber Sie werden hier ebenfalls Konzepte kennen lernen, die wichtig sind, damit Ihre Seite tatsächlich ein Teil des semantic web ist. Dieses Kapitel ist auch für komplette Neueinsteiger in die Programmierung leicht zu bearbeiten, da Sie hier nur wenige abstrakte Grundlagen verinnerlichen müssen, die sonst in der Welt der Informatik und der Programmierung recht häufig vorkommen.\\

Wenn Sie mit Dateien arbeiten, sind Sie es in aller Regel gewöhnt, die Datei mit einem Programm zu öffnen. Die meisten von Ihnen kennen also nur die verschiedenen Ansichten, die einzelne Programme erzeugen, wenn Sie mit einem dieser Programme eine Datei öffnen. Bei einer Webanwendung sehen Sie beispielsweise verschiedene Elemente wie Bilder, Videos, Texte usw. Sie sehen dann aber in aller Regel nicht, wie diese Datei selbst aussieht. Doch genau darum geht es in diesem Kurs.\\

Wenn Sie das jetzt irritiert, dann machen Sie sich bitte folgendes klar: Wenn Sie eine pdf-Datei öffnen, dann generiert der pdf-Reader eine Ansicht. Das was Sie sehen ist also nicht die Datei selbst, sondern nur eine aus dieser Datei erzeugte Ansicht. Und alles, was Sie irgendwo von einem Computer angezeigt bekommen ist eine Interpretation einer oder mehrerer Dateien. Wenn Sie also lernen wollen, wie Sie Computer programmieren können, dann müssen Sie als erstes verstehen, dass Sie bislang immer nur eine Interpretation dessen gesehen (oder gehört) haben, was tatsächlich \glqq{}auf dem\grqq{} Computer gespeichert ist.\\

In HTML kümmern wir uns dagegen nicht darum, wie eine Webanwendung aussehen soll, sondern ausschließlich darum, aus welchen Bestandteilen die Webanwendung bestehen soll. In anderen Worten: In diesem ganzen Kapitel wird es nicht ein einziges Mal darum gehen, wie die Elemente Ihrer Webanwendung später aussehen werden. Wenn Sie hier also an irgend einer Stelle versuchen, die Darstellung zu programmieren, dann machen Sie einen Fehler.\\

Nachdem wir zuvor weitgehend geklärt haben, wie wir eine Webserver auf unserem Rechner in Betrieb setzen und kontrollieren können, kommen wir nun zum zweiten Teil dieser Einführung: Die Einführung in die Markup Language HTML. Am Ende dieses Abschnittes können Sie also statische Webanwendung erstellen. Danach werden wir über \textbf{CSS}\index{Programmiersprache!CSS}\index{CSS} sprechen: Cascading Style Sheets sind eine Möglichkeit, um auf unterschiedlichen Webanwendung gleiche Strukturen und Formate für Inhalte zu realisieren. Danach kommen wir zur Programmierung in PHP, womit Sie die Elemente Ihrer Webanwendung dynamisch und interaktiv programmieren können.\\

\textbf{Aufgabe}:\\

Unabhängig von dem, was in den einzelnen Abschnitten steht, programmieren Sie bitte für jedes Element, dass Sie auf Ihrer Webanwendung einbinden wollen einen eigenen HTML-Container, ohne sich über das Design Gedanken zu machen. Das ist der erste Schritt hin zu professionellen Anwendungen: So lange Sie es nicht schaffen, zuerst die Inhalte bzw. Inhaltsstruktur festzulegen und erst dann ein passendes Design auszusuchen sind Sie weit davon entfernt, professionelle/r InformatikerIn zu sein. Das bedeutet nicht, dass Design unwichtig wäre, sondern es geht darum, dass Sie sich auf Ihre zukünftige Aufgabe in Teams konzentrieren. Und so lange Sie nicht Design (z.B. Medien- und Kommunikationsdesign) studieren, bereiten Sie sich eben auch nicht darauf vor, als DesignerIn zu arbeiten. So lange Sie das nicht akzeptieren werden Sie ein Don Quichote der Medieninformatik bzw. der Medientechnik sein.\\

Ignorieren Sie also Aspekte wie Design, die Positionierung eines Elements auf der Seite, Hyperlinks, usw. usf. Auch für die einzusetzenden Text, Bilder usw. verwenden Sie bitte vorerst Platzhalter: Die Qualität Ihrer Arbeit als MedieninformatikerIn wird sich nicht in umfangreichen Texten und Bildergallerien zeigen, sondern in gut durchdachten und ausgearbeiteten Strukturen. Das Einfügen von Texten und Bildern sowie das Ausarbeiten eines guten Designs ist dann die Aufgabe anderer Mitarbeiter im Team.

\subsection{Das ist HTML}

\textbf{HTML}\index{HTML}\index{Programmiersprache!HTML}, kurz für Hyper Text Markup Language ist, wie es aus dem Namen hervorgeht eine \textbf{Markup Language}\index{Markup Language}. Teilweise wird auch von einer statischen Skriptsprache gesprochen. Statisch bedeutet hier, dass es mit HTML nicht möglich ist, Inhalte während der Nutzung zu ändern. In der Anfangszeit des \textbf{WWW}\index{WWW} genügte das auch, weil es schon eine großartige Bereicherung darstellte, Texte und Bilder direkt über eine Datenleitung in wenigen Sekunden oder Minuten empfangen zu können, die sonst per Post erst nach einigen Tagen oder Wochen im Haus gewesen wären. \\

Heute dagegen, wo selbst das Format von Displays kaum noch standardisiert ist, benötigen wir weitergehende Möglichkeiten. Diese werden unter dem Begriff \textbf{Responsive Design}\index{Responsive Design} zusammengefasst: Das Design einer Webanwendung passt sich automatisch dem Gerät an, auf dem es angezeigt wird. Wir reden wir hier also über etwas, das in den Bereich der Informatik bzw. der Softwareentwicklung fällt und nicht in den Bereich dessen, was üblicherweise unter Design im Sinne von kreativer Gestaltung verstanden wird.\\

Deshalb muss es hier nochmal betont werden: Wenn Sie HTML programmieren und festlegen, wie ein Element aussieht oder wo es angeordnet werden soll, dann erzeugen Sie damit in den meisten Fällen eine Webanwendung, die auf einer Vielzahl von Nutzergeräten grausig aussehen wird, egal wie gut Ihr Webdesign sonst sein mag. Also lassen Sie das.\\

Dieser Kurs behandelt jedoch nicht die Feinheiten der Usability, zu denen Responsive Design gehört. Vielmehr ist das eine der Spezialisierungen, die MedieninformatikerInnen im Masterstudium wählen können.

\subsection{Erster HTML-Quellcode}

Das wichtigste bei der Programmierung einer Webanwendung ist die Antwort auf die Frage, aus welchen Einheiten sie bestehen soll und welche Funktion jede dieser Einheiten übernehmen soll. Erst danach überlegen Sie sich idealerweise unterstützt durch Designer, mittels welches Designs diese Funktion erkennbar sein soll. Ein typischer Einsteigerfehler besteht darin, sich zunächst über das Design Gedanken zu machen und dann mit dem Programmieren anzufangen. Das ist deshalb ein Fehler, weil dabei in aller Regel wichtige Funktionalitäten vergessen werden oder (noch schlimmer) es für Nutzer nicht erkennbar ist, wie eine Funktionalität genutzt werden kann. Wird dann entdeckt, dass eine Funktionalität fehlt, die wichtig ist, dann muss häufig das Design komplett verworfen und neu entwickelt werden. Oder es herrscht die Überzeugung vor, dass der Kunde schon blöd genug sein wird, sich nicht daran zu stören. Und glauben Sie mir: Bei der Prüfung für Ihre Leistungsnachweise haben Sie keinen naiven Kunden vor sich.\\

Deshalb nun die Frage: Welche Funktion soll unsere erste Webanwendung erfüllen?\\

\textbf{Hinweis}:\\

Diese Webanwendung ist ausschließlich für diejenigen gedacht, die zu Hause dieses Buch durcharbeiten wollen. Das gleiche gilt für \emph{Hausaufgaben} und andere Übungen, die Sie hier finden. Es handelt sich hier nicht um Aufgaben, die Sie für einen Leistungsnachweis in Studienveranstaltungen von mir bearbeiten müssen. Vielmehr können Sie sich so unabhängig von meinen Studienveranstaltungen in das Thema einarbeiten.\\

Beginnen wir mit einer einfachen Seite, mit der wir einfach nur prüfen wollen, ob ein Text so angezeigt wird, dass wir mit der Darstellung zufrieden sind: Er sollte groß genug angezeigt werden, damit wir ihn lesen können. Außerdem sollte er an einer Position angezeigt werden, an der er von einem Nutzer unserer Seite bemerkt wird.\\

Das ist ziemlich viel Text, um daraus eine Prüfung abzuleiten, ob die Funktionalität erfüllt ist. Deshalb gibt es das Planungswerkzeug \textbf{Use Case}\index{Use Case}. Ein Use Case ist eine kurze Beschreibung, wer was mit einer Webanwendung tut und was dann passieren soll. Machen wir das doch gleich für unseren Fall:\\

Use Case 1: Der Nutzer liest einen Begrüßungstext. (Keine Interaktion.)\\

Sicher, das ist kein spannender Use Case, weil er keine Interaktion enthält. Vor allem können wir auch gar nicht prüfen, ob einzelne NutzerInnen tatsächlich lesen, was da angezeigt wird. Aber damit können wir arbeiten und nach der Programmierung prüfen, ob die Funktionalität erfüllt ist. Geben Sie den folgenden Quellcode in einen einfachen Editor (z.B. den kostenlosen notepad++) ein und speichern ihn unter dem Namen \verb|seite01.html| im bekannten Verzeichnis Ihres Webservers.

\begin{verbatim}
<html>
<head>
<title>Die erste HTML-Webanwendung</title>
</head>
<body>
Einführung in die Programmierung, Teil 1
</body>
</html>
\end{verbatim}

Gleich vorweg: Willkommensgrüße oder ähnliches haben auf einer Webanwendung nichts verloren. Entweder der Nutzer kann auf Anhieb erkennen, was hier angeboten wird oder wir haben als Entwickler etwas falsch gemacht.\\

Wenn Sie jetzt die Webanwendung aktualisieren (oder bei gestopptem Webserver zunächst den Webserver neustarten und anschließend wieder die Webanwendung localhost aufrufen), sehen Sie, dass das neue HTML-Skript als Datei angezeigt wird. Wählen Sie doch einfach den \glqq{}Link\grqq{} an und öffnen Sie damit die Webanwendung, die Sie gerade programmiert haben.\\

Es gibt zwei Dinge, die Ihnen auffallen, wenn Sie das Ergebnis im Browser genau prüfen (zumindest wenn Sie keinen Tippfehler im Quellcode haben): \\

\begin{itemize}
	\item Zum einen steht da nicht das Wort Einführung im Text, sondern so etwas wie EinfÄ3/4rung.
	\item Und dann steht auf der Registerlasche der Webanwendung der Schriftzug \verb|Die erste HTML-Webanwendung|.
\end{itemize}

Wenn Sie die einleitenden Kapitel aufmerksam gelesen haben, dann haben Sie sicher schon eine Idee, warum hier kein \verb|ü| in Einführung steht. Aber dazu gleich mehr. Zuvor schauen wir uns zwei Dinge an. Da wäre einmal die Frage, wie wir möglichst effizient programmieren könenn und dann steht die Frage an, welche Bestandteile unser Quellcode enthält.

\subsubsection{Tags und Container}

Der Schriftzug \verb|Die erste HTML-Webanwendung|, der als Titel der Seite im Webbrowser angezeigt wird, steht zwischen den \glqq{}Befehlen\grqq{} \verb|<title>| und \verb|</title>|. Solche \glqq{}Befehle\grqq{} werden in Markup Languages als Tags bezeichnet. (Aussprache wie tägs und nicht wie tags (im Sinne von mittags).)\\

Wenn wir wie bei \verb|<title>| und \verb|</title>| zwei Tags haben, von denen das eine den Anfang und das andere das Ende von einem Teil unserer Webanwendung definiert, dann bezeichnen wir die beiden als \textbf{öffnendes} bzw. \textbf{schließendes} Tag und beide gemeinsam mit allem, was zwischen ihnen steht als einen \textbf{Containter}\index{HTML!Container}\index{Container}.\\

Wir haben aber auch Container, die zwischen dem öffnenden und dem schließenden Tag keinen eigentlichen Inhalt haben. Um zu verdeutlichen, dass ein Container keinen solchen Inhalt hat, können wir ein öffnendes Tag auch direkt wieder schließen und brauchen kein schließendes Tag zu programmieren. Dazu schreiben wir als letztes Zeichen in einen öffnenden Tag das \verb|\|-Symbol.\\

\textbf{Kontrolle}:

\begin{itemize}
	\item Wie gesagt programmieren Sie in einer Markup Language ausschließlich die Struktur von Anwendungen und weder ihre Funktionalität noch ihre Gestaltung. In sofern war HTML4.01 keine reine Markup Language, sondern eine Mischung aus Markup und Design Sprache.
	\item Sie wissen jetzt, dass die Elemente, aus denen Sie eine Struktur in HTML programmieren Container heißen und das jeder Container aus einem in sich abgeschlossenen Tag oder aus einem öffnenden und einem schließenden Tag besteht.
\end{itemize}

\subsubsection{Dokumentteile auslagern}

Mit HTML-Dokumenten definieren Sie, aus welchen Bestandteilen eine Webanwendung besteht. Sie besteht dagegen nicht aus den konkreten Inhalten. Texte, Bilder, Sound und Videos sind nicht Teil von HTML. Allerdings können wir Texte durchaus direkt ins HTML einfügen, so wie wir das oben getan haben. Das ist allerdings eine Vorgehensweise, die bei größeren Projekten nicht empfehlenswert ist. Vielmehr sollten Sie auch Texte in eigenständigen Dateien speichern und sie durch den Webserver ins HTML einfügen lassen. Die einfachste Möglichkeit dafür ist eine PHP-Funktion, die Sie in das HTML-Dokument einfügen.\\

\begin{enumerate}
	\item Erstellen Sie dafür eine Datei mit dem Namen \verb|test_title_001.txt|, in der Sie die folgende Zeile eintragen und die Sie dann im gleichen Verzeichnis wie \verb|seite01.html| abspeichern:
	\begin{verbatim}
		echo("Die erste HTML-Webanwendung");
	\end{verbatim}
	\item Erstellen Sie jetzt eine Datei mit dem Namen \verb|test_begruessung_001.txt| mit dem Inhalt \verb|echo("Einführung in die Programmierung, Teil 1");|.
	\item Ändern Sie die Datei \verb|seite01.html| wie folgte ab und speichern Sie sie unter dem Namen \verb|seite01.php|:
	\begin{verbatim}
	<html>
	<head>
	<title><?php include(test_title_001.txt); ?></title>
	</head>
	<body>
	<?php include(test_begruessung_001.txt); ?>
	</body>
	</html>
	\end{verbatim}
	\item Rufen Sie jetzt im Browser \verb|localhost| auf und wählen Sie dort \verb|seite01.php| an.
\end{enumerate}

Wie Sie sehen sieht die Webpage genauso aus, als wenn Sie \verb|seite01.html| geöffnet hätten. Schauen wir uns die Änderungen und Ihre Auswirkungen einmal im Detail an:

\begin{itemize}
	\item Die Funktion \verb|echo()| ist eine Funktion der Programmiersprache PHP.
	\item Alles, was Sie zwischen zwei Anführungszeichen in die Klammer von echo() schreiben wird ausgegeben.
	\item Eine Programmzeile in PHP wird durch ein Semikolon beendet.
	\item Die PHP-Funktion \verb|include()| lädt den Inhalt einer anderen Datei und fügt ihn an der Stelle ein, wo die include()-Funktion steht.
	\begin{itemize}
		\item Sie können also mit include() beliebige Teile eines PHP-Programms in eigenen Dateien speichern.
		\item Das besondere dabei ist, dass Sie somit jeden Teil eines PHP-Programms, der mehrfach genutzt werden soll nur einmal programmieren müssen. (Wenn Sie die objektorientierte Programmierung kennen lernen werden Sie dafür eine andere Möglichkeit kennen lernen.)
	\end{itemize}
	\item Um PHP-Programme oder Programmteile in ein HTML-Dokument einzufügen müssen Sie es in ein Tag eintragen, das mit \verb|<?php| beginnt und mit \verb|?>| endet.
	\item Das Großartige dabei ist, dass Sie auf diese Weise gleich zwei Fliegen mit einer Klappe schlagen:
	\begin{enumerate}
		\item Sie können beliebigen PHP-Code ohne weitere Vorbereitung direkt in HTML einfügen.
		\item Sie können über den oben gezeigten Weg beliebigen HTML-Code aus anderen Dateien an beliebigen Stellen in verschiedene HTML-Dokumente einfügen.
	\end{enumerate}
\end{itemize}

Ausblick \textbf{für Fortgeschrittene}: 

\begin{itemize}
	\item Um Programme oder Programmteile in ein HTML-Dokument zu integrieren, die in JavaScript programmiert wurden, nutzen Sie das \verb|script|-Tag. Wenn Sie beispielsweise die JavaScript-Datei \verb|intro.js| an einer Stelle Ihres HTML-Dokuments einzufügen, lautet das HTML-Tag \verb|<script src=intro.js>|. In HTML4.01 musste deutlich mehr programmiert werden, um JavaScript-Code in HTML einzufügen.
	\item Wollen Sie dagegen JavaScript direkt in HTML programmieren, dann sieht das so aus: \verb|<script> ... Hier steht der JavaScript-Code ... </script>|. Auch hier gilt wieder: In HTML4.01 musste deutlich mehr programmiert werden, um JavaScript-Code in HTML einzufügen.
	\item \textbf{Hinweis}: Diejenigen von Ihnen, die bei mir den Leistungsnachweis \glqq{}Projekt 1 (MS)\grqq{} erwerben wollen sollten sich das genau merken: Ich brauche in aller Regel nur zwei Sekunden, um zu erkennen, ob Sie HTML4.01 oder HTML5 programmieren. Für den Leistungsnachweis gilt: Die Nutzung von HTML4.01 bedeutet automatisch, dass Sie (noch) nicht bestanden haben.
\end{itemize}

\subsection{Struktur eines HTML-Dokuments}

Damit bleibt die Frage, was die HTML-Tags eigentlich bewirken oder besser gesagt, wie sie von einem Browser interpretiert werden, um daraus die Ansicht zu generieren, die wir als NutzerInnen angezeigt bekommen.\\

\begin{itemize}
	\item Zu Beginn sehen Sie dort den öffnenden Container \verb|<html>|.\\
	Dieser gibt an, dass alles nachfolgende HTML-Code ist. In der letzten Zeile finden Sie dann das schließende Tag \verb|</html>|. Dieses besagt also schlicht, dass jetzt kein HTML-Code mehr folgt.
	\item Danach folgt der \verb|head|-Container.\\
	Wie Sie sehen befindet sich innerhalb dieses Containers der \verb|title|-Container, der definiert, unter welchem Titel die Webanwendung angezeigt wird. Hier werden Sie später auch allgemeine Formatierungen für eine einzelne Webanwendung programmieren.\\
	Grundsätzlich gilt, dass wir nicht unbedingt einen <head>-Container programmieren müssen. Aber es macht nur selten Sinn, ihn vollständig wegzulassen.
	\item Danach folgt der \verb|body|-Container.\\
	Dieser beinhaltet all das, was einer Ansicht der Webanwendung angezeigt wird.
\end{itemize}

\textbf{Anmerkung}:\\

Bei HTML wird häufig der Begriff HTML-Skript verwendet, weil es sich ja um kein Programm im dem Sinne handelt, dass hier ein Algorithmus umgesetzt wird. Das ist aber keine allgemeingültige Definition; Sie können also ruhig von einem HTML-Programm oder einer HTML-Seite sprechen, wenn Sie sich damit wohler fühlen. Der von mir angesprochene Unterschied zwischen Skript und Programm wird Ihnen spätestens dann deutlich, wenn Sie mit der imperativen Programmierung in PHP beginnen.\\

\textbf{Kontrolle}:\\

Sie wissen jetzt, dass jedes HTML-Skript drei Container beinhalten sollte, den html-, den head- und den body-Container. Sie haben richtig gelesen: Es sollte diese Container beinhalten, muss es aber nicht. Das HTML-Skript ließe sich also auch mit den folgenden Zeilen programmieren:\\

\begin{verbatim}
	<title>Die erste HTML-Webanwendung</title>
	Willkommen zur Einführung in die Programmierung.
\end{verbatim}

Aus Gründen der Lesbarkeit sollten Sie dennoch immer die etwas umfangreichere Struktur mit html-, head- und body-Container verwenden.

\subsubsection{HTML-Referenz: W3Schools}

Bis Anfang 2015 galt \textbf{SelfHTML}\index{SelfHTML} als die Standardreferenz für HTML. Leider kann ich hiervon mittlerweile nur noch abraten: SelfHTML ignoriert wie die meisten HTML 4.01-Veteranen alles, was HTML 5 zu einer echten nächsten Version von HTML macht. Wenn Sie sich an SelfHTML orientieren, dann können Sie auch gleich bei HTML 4.01 bleiben.\\

Die einzige mir bekannte Webpage, bei der eine klare Trennung zwischen HTML 4.01 und 5 durchgeführt wird ist zum Jahreswechsel 2015/16 (neben dem W3 Consortium, das die Standards festlegt) die Seite \textbf{W3Schools}\index{W3Schools}(\url{http://www.w3schools.com/default.asp}).

\paragraph{Übersicht über alle Tags}

Wir werden in diesem Kurs nur einen kleinen Teil aller Tags besprechen, die es gibt. Und wir werden hier auch jeweils nur einige wenige Anwendungen besprechen können. Damit Sie aber sinnvolle Webanwendung und Webanwendungen entwickeln können, müssen Sie wissen, wo Sie nachschlagen können, wenn Sie mehr Informationen zu einzelnen Tags suchen. Bei den W3Schools nutzen Sie dazu die folgende Unterseite: \url{http://www.w3schools.com/tags/}\\

Hier finden Sie eine vollständige Übersicht über die Tags in HTML. Außerdem finden Sie hier Hinweise darauf, was sich zwischen HTML4.01 und HTML5 geändert hat. Erfahrene HTML-Programmierer werden hier überrascht: Zwar sind alle Bereiche aus HTML entfallen, die mit dem Layout bzw. der Gestaltung zusammen hängen, aber dafür gibt es zum Teil Elemente aus HTML3, die in HTML4 entfernt und in HTML5 wieder hinzugefügt wurden.\\

\textbf{Wichtige Hinweise}:

\begin{itemize}
	\item Auf \url{www.w3schools.com} finden Sie zu jedem Tag Hinweise darauf, welcher Browser das jeweilige Tag unterstützt. Die Angaben dort werden jedoch nicht aktualisiert, sodass der Eindruck entstehen könnte, dass Sie HTML5 praktisch kaum einsetzen können. Das trifft aber nicht zu. Welche Tags tatsächlich von welchem Browser nicht automatisch unterstützt werden und wie Sie dafür sorgen können, dass ein Tag dennoch richtig angezeigt wird, erfahren Sie im Abschnitt zu den sogenannten Polyfills.
	\item Für diejenigen, die bereits in HTML4.01 programmiert haben hier noch ein essentieller Hinweis: Einige Container wie z.B. \verb|<div>|, die bislang sehr oft verwendet wurden, gibt es weiterhin, aber sie sind nur noch in den seltenen Spezialfällen einzusetzen, die mit den neuen Tags in HTML5 nicht abgedeckt sind.
	\item Leider wird an dieser Stelle häufig auf selfhtml verwiesen. Das ist eine Webpage, die sehr detailliert ist, wenn es um HTML4.01 geht. Bezüglich HTML5 ist die Seite leider zurzeit nicht nutzbar. Zu oft werden Sie dort nach dem Lesen eines Abschnitts vor der Frage stehen: Gilt das jetzt immer noch? Deshalb lautet die klare Empfehlung: Nutzen Sie die Seite der W3Schools.
\end{itemize}

\subsubsection{Was haben XML und XHTML mit Webanwendungen zu tun?}

Einer der ersten Versuche, um semantische Webanwendungen zu entwickeln wurde unter der Bezeichnung \textbf{XML}\index{Programmiersprache!XML} (kurz für extended markup language) veröffentlicht. Eine Kombination aus XML und HTML wird als \textbf{XHTML}\index{XHTML}\index{Programmiersprache!XHTML} bezeichnet.

\subsubsection{Validierung - Prüfung auf Fehler}

Bei der Eingabe von Programmcode machen wir unvermeidlich Fehler. Sie haben im vorigen Kapitel den Unterschied zwischen Syntax und Semantik kennen gelernt und wissen deshalb, dass es für einen Computer möglich ist, syntaktische Fehler zu erkennen, aber unmöglich semantische Fehler zu erkennen. Eine weitere Art von Fehlern, die ein Computer nicht erkennen kann hat etwas mit der Logik eines Programms zu tun: Wenn wir in einem Programm etwas einprogrammiert haben, dass der Computer ausführen kann, das aber nicht zu dem Ziel führt, zu dem wir gelangen wollen, dann kann der Computer das mit den bekannten Sprachen nicht erkennen.\\

Vor diesem Hintergrund ist die Behauptung, dass nur \textbf{streng typisierte Sprachen}\index{Typisierung!streng}\index{statisch} sicher seien kompletter Humbug: Sie bieten eine minimale Art von Sicherheit, die aber häufig irrelevant ist. Wer das nicht glauben mag, dem möchte ich einmal empfehlen, sich anzusehen, warum der erste Start der \verb|Ariane V| ein Totalausfall war. Was hat all die ach so großartige Typsicherheit beim Start dieser Rakete gebracht? Gar nichts! Ist Typsicherheit deshalb irrelevant? Nein. Sorgt sie dafür, dass Entwickler ein ungerechtigtes Gefühl von Sicherheit haben? Definitiv. Ist sie somit selbst ein Sicherheitsrisiko? Das können Sie sich selbst beantworten.\\

Kommen wir nun wieder zurück zur Programmierung von HTML-Dokumenten. Bislang kennen Sie keine Möglichkeit, um Ihr HTML-Dokument auf Fehlerfreiheit zu prüfen. Der Begriff der Fehlerfreiheit ist aber ungenau. Syntaktische Fehler können Sie ja recht leicht durch die Syntaxhervorhebung von notepad++ erkennen. Bei Markup Languages geht es uns aber um die Struktur einer Anwendung, also wäre es schön, eine Möglichkeit zu haben, um zu prüfen, ob die Struktur zumindest grundsätzlich den Vorgaben für HTML5 entspricht. Und das wird als \textbf{Validierung}\index{Validierung} bezeichnet.\\

Damit aber ein Browser oder Validator (Programm, das die Validität eines Dokuments prüft) erkennen kann, dass wir ein HTML5-Dokument erstellt haben, müssen wir noch eine Zeile an den Anfang des Dokuments einfügen. Diese Zeile gibt den sogenannten \textbf{Doctype}\index{Doctype}\index{HTML!Doctype} an. Neben der Arbeitserleichterung folgt hieraus auch, dass Sie sich rechtlich etwas absichern: Indem Sie eine validierbare Webanwendung programmieren zeigen Sie, dass Sie nach den aktuellen technischen Standards arbeiten. Und das kann sich im Falle eines Gerichtsverfahrens zu Ihren Gunsten auswirken.\\

Den \textbf{W3C-Validator}\index{Validator!W3C} finden Sie unter \url{http://validator.w3.org}. Hier geben Sie den Link auf die zu prüfende Seite ein und erhalten dann eine Ausgabe über die Validität Ihrer Seite. Alles was Sie tun müssen, um ein HTML-Skript als HTML5-Skript zu markieren ist das Hinzufügen einer kurzen Zeile am Anfang des Skripts (noch vor dem html-Container):\\

\begin{verbatim}
	<!doctype html>
\end{verbatim}

Und das ist alles. Wenn Sie sich nicht sicher sind, ob Sie wirklich HTML5 programmieren wollen, brauchen Sie sich keine Sorgen zu machen: HTML5-Skripte können Passagen enthalten, die wie bei einer früheren HTML-Version programmiert sind. Diese werden dann wie bei den früheren Versionen ausgeführt.\\

Wenn wir also unseren bisherigen Code (als HTML5-Code) validierbar machen wollen, dann sieht er so aus:

\begin{verbatim}
	<!doctype html>
	<html>
	<head>
	<title><?php include(test_title_001.txt); ?></title>
	</head>
	<body>
	<?php include(test_begruessung_001.txt); ?>
	</body>
	</html>
\end{verbatim}

Die erste Programmzeile innerhalb eines HTML-Dokuments gibt somit zwei Dinge an, die gemeinsam als \textbf{Doctype Definition}\index{Doctype Definition}\index{HTML!Doctype Definition} (kurz DTD) bezeichnet werden:

\begin{itemize}
	\item Dies ist ein HTML-Dokument.
	\item Die HTML-Version, in der das Dokument erstellt wurde.
\end{itemize}

Bei HTML 4.01 war diese Zeile recht lang und es gab mehr als nur eine Variante. Hier ein Beispiel für eine DTD in HTML4.01:\\

\verb|<!DOCTYPE HTML PUBLIC "-//W3C//DTD HTML 4.01 Transitional//EN">|\\

Da ist die Variante von HTML5 doch deutlich angenehmer, nicht zuletzt weil es nur genau eine Variante gibt.\\

Wenn Sie Ihr HTML-Dokument mit der DTD gespeichert und den Browser aktualisiert haben, werden Sie feststellen, dass sich die Anzeige nicht geändert hat. Deshalb hier nochmals der Hinweis: Diese Programmzeile teilt dem Webbrowser lediglich mit, dass es sich hier um eine HTML5-Seite handelt. Da es für NutzerInnen einer Webanwendung üblicherweise belanglos ist, wie diese Seite programmiert wurde, wird diese Information auch nicht angezeigt. Der Webbrowser hat damit aber eine wichtige Information erhalten, denn bei der Darstellung einer Seite richten sich Webbrowser unter anderem nach den Sprachen und Versionen, in denen Dokumente verfasst sind.

\subsubsection{Absätze}

Wenn Sie in Ihrer Webanwendung mehrere Absätze Text einfügen wollen, dann machen Sie das mit dem p-Container. Nutzen Sie bitte keinesfalls leere Container oder den \verb|<br \>|-Container, um leere Zeilen einzufügen. Wenn Sie solche Kniffe anwenden, dann programmieren Sie die Ansicht der Webanwendung. Das ist aber Mediendesign und hat in der Programmierung von HTML5 nichts zu suchen.

\begin{verbatim}
	<p> ... Erster Absatz ... </p>
	<p> ... Zweiter Absatz ... </p>
	usw.
\end{verbatim}

Pflegen wir das doch gleich in unsere \verb|.txt|-Dateien ein: Erweitern Sie den Inhalt der \verb|echo()|-Funktion um einige Absätze, die Sie genauso programmieren, als wenn sie direkt im HTML-Dokument stehen würden. (Wenn die Ausgabe nicht ganz so aussieht, wie Sie es erwarten, dann gedulden Sie sich noch ein wenig: Neben einem einfachen Syntaxfehler gibt es noch die Möglichkeit, dass Sie das HTML-Dokument nicht lokalisiert haben, oder dass Sie Sonderzeichen wie \grqq{} verwendet haben, die von der \verb|echo()|-Funktion nicht übernommen werden. Aber keine Sorge, um beide Fälle kümmern wir uns bald.

\subsubsection{Kommentare}

Bei längeren HTML-Dokumenten oder Containern, die nicht selbstredend sind, sollten Sie außerdem Kommentare einpflegen. Kommentare sind Codezeilen, die nicht ausgeführt werden, sondern ausschließlich dazu dienen, um anzuzeigen, was in einem Teil einer Programms oder Skripts passiert. Im Gegensatz zu anderen Containern sieht ein Kommentar wie folgt aus:\\

\verb|<!--| Kommentar, auch über mehrere Zeilen \verb|-->|\\

Bei Kommentar-Containern gibt es also keinen Abschluss durch den Slash, wie das bei anderen Tags der Fall ist.\\

In der Vergangenheit wurden spezielle Befehle, die nur für den Internet Explorer zugeschnitten waren in Form von Kommentaren einprogrammiert; sollten Sie also in fremdem HTML-Skripten Kommentare der Form \verb|if IE ...|treffen, dann wissen Sie jetzt, worum es dabei geht.

\subsubsection{Attribute}

Später werden Sie in öffnende Tags noch Attribute einfügen. Attribute legen Einschränkungen oder Erweiterungen fest, die für einen Container gilt und für alles, was sich darin befindet. Diese programmieren Sie ausschließlich ins öffnende Tag. Das schließende Tag beinhaltet weiterhin nur die Bezeichnung eines Containers, um dem Browser anzuzeigen, dass der Container hier \glqq{}zuende\grqq{} ist.\\

Wenn Sie zum Beispiel für den \verb|html|-Container: Wenn Sie hier das \verb|lang|-Attribut festlegen wollen, über das wir noch nicht gesprochen haben und diesem Attribut den Wert \verb|de| zuordnen, sieht das öffnende html-Tag so aus: \verb|<html lang=de>|. Das schließende html-Tag wäre dann aber immer noch \verb|</html>|. Also wäre es unsinnig, hier die Attribute zu programmieren: Attribute gelten immer für einen Container und gelten damit nicht mehr, sobald der Container geschlossen ist.\\

Übrigens legen Sie mit dem \verb|lang|-Attribut die sogenannte Internationalisierung fest. Was das im Detail bedeutet besprechen wir später.\\

\textbf{Für diejenigen, die bereits mit HTML4.01 programmiert haben}: Dort war es üblich, über Attribute die Gestaltung von Containern direkt über das \verb|style|-Attribut zu programmieren. Das passiert unter HTML5 nur noch indirekt über die beiden Attribute \verb|class| und \verb|id|. Bei diesen funktioniert es aber noch genauso wie unter HTML4.01.

\subsubsection{Aufgaben}

\begin{enumerate}
	\item Erstellen Sie jetzt die folgenden sechs HTML-Dokumente, programmieren Sie jeweils DTD und die drei Container:
	\begin{itemize}
		\item arbeitsweg.php
		\item meinCampus.php
		\item meineHobbies.php
		\item meineZiele.php
		\item meinBlog.php
		\item index.php
	\end{itemize}
	\item Erklären Sie in eigenen Worten, warum es für die Ansicht im Browser keinen Unterschied macht, ob die Dateiendung .html oder .php lautet.
	\item (Für Fortgeschrittene, benötigt Recherche im Netz) Erklären Sie in eigenen Worten, warum es für die Ansicht im Browser einen großen Unterschied macht, ob Sie HTML-Dokumente, in denen PHP-Code enthalten ist direkt vom Browser geöffnet oder von einem Webserver (also per localhost) an den Browser übertragen werden.
\end{enumerate}

\subsubsection{Zusammenfassung}

Sie kennen jetzt alles, was Sie benötigen, um eine einzelne Seite mit Text ins Netz zu stellen.

\section{Barrierefreiheit, Internationalisierung und Lokalisierung}

Die meisten Kurse zu HTML4.01 kommen (wenn überhaupt) erst ganz zum Schluss zu diesem Thema, dabei ist es für Webanwendungen essentiell: Dieser Abschnitt vermittelt Ihnen ein paar grundlegende Informationen dazu, wie Sie sicherstellen können, dass Ihre Webanwendung auf den verschiedensten Endgeräten und von den verschiedensten Nutzern genutzt werden kann. Dazu gehört aber vor allem etwas, das bereits mehrfach betont wurde: Programmieren Sie in HTML5 nur dann Design, wenn es anders unmöglich ist.\\

Der Grund für die Bedeutung dieses Kapitels ist so simpel wie ärgerlich:\\

Viele Entwickler vergessen schlicht, dass eine Webanwendung auf einem Smartphone anders dargestellt wird als auf einem Rechner mit einem 2k-Display, das horizontal ausgerichtet ist. Außerdem nutzen nicht nur hörbehinderte Personen Software, die Ihnen Webanwendung vorliest. Und zu guter Letzt bieten mehr und mehr Browser eine automatische Übersetzung von Texten an.\\

Damit Ihre Webanwendung unter all diesen Bedingungen immer noch gut aussieht, mussten Sie unter HTML4.01 eine ganze Menge Arbeit leisten, die bei HTML5 mit wenigen Befehlen erledigt werden kann. Wobei all die guten Vorsätze nichts Wert sind, wenn Sie im Team jemanden haben, der in HTML Design programmiert.

\subsection{Barrierefreiheit}

Hier geht es darum, eine Webanwendung so zu programmieren, dass Sie auch dann noch gut aussieht, wenn der Nutzer einen starken Zoom-Faktor nutzt oder sich die Webanwendung durch eine Software vorlesen lässt. Darüber hinaus geht es aber auch um Personen, die beispielsweise die Sprache der Webanwendung nicht gut verstehen.

\subsubsection{Aufgabe}

Öffnen Sie die Webpage der HAW mit Ihrem Smartphone und versuchen Sie nun, über die Links auf der Startseite auf die Seite des Departments Medientechnik zu gelangen.\\

Selbst wenn Sie gute Augen haben, werden Sie merken, dass diese Aufgabe sehr anstrengend ist. Und das liegt eben nicht daran, dass es allzu schwer ist, die passenden Links zu finden; vielmehr wurde bei der Webpage der HAW die Barrierefreiheit zum Teil ignoriert. Damit sollte Ihnen klar sein, dass Barrierefreiheit kein belangloses Thema für Randgruppen ist, sondern dass die Missachtung der Barrierefreiheit ein Problem ist, das Nutzer davon abschreckt eine Webanwendung zu nutzen.\\

Es gibt drei einfache erste Prüfungen, um die grundsätzliche Einhaltung der Barrierefreiheit zu prüfen:

\begin{enumerate}
	\item Es gibt einen Seitentitel, der kurz und zutreffend ist und die Seite von anderen Seiten unterscheidet.
	\item Jeder Inhalt in einem Container bekommt eine Überschrift. (Ausnahmen sind z.B. \verb|<p>|-Conatainer.)
	\item Für jedes multimediale Element gibt es eine alternative Bezeichnung, die Sie mit Hilfe des \verb|alt|-Attributs festlegen können. 
\end{enumerate}

Den ersten Punkt können Sie jetzt schon umsetzen, zum zweiten und dritten Punkt kommen wir, sobald wir die entsprechenden Container behandeln.\\

\textbf{Kontrolle}:

\begin{itemize}
	\item Sie verstehen, dass es bei Barrierefreiheit nicht nur um die Unterstützung von Menschen geht, die körperlich oder geistig beeinträchtigt sind.
	\item Sie wissen vielmehr, dass die Beachtung der Barrierefreiheit wichtig ist, um Nutzer nicht von der eigenen Webanwendung zu vertreiben.
	\item Ihnen ist klar, dass dazu wesentlich mehr nötig ist, als die drei Prüfungen, die Sie gerade kennen gelernt haben. (Für diesen Kurs soll das aber als kleine Einführung genügen.)
\end{itemize}

\subsection{Internationalisierung (kurz i18n) und Lokalisierung (kurz l10n)}

Wenn Sie sich fragen, warum die Abkürzung i18n lautet, hier ist die Antwort: Das englische Wort internationaliziation beginnt mit einem i, hat 18 Buchstaben und endet mit einem n. Auf ganz ähnliche Weise kommen sie von localization zu l10n.\\

Nach der Barrierefreiheit kümmern wir uns nun darum, dass der Browser mit den deutschen Umlauten zurechtkommt. Dazu müssen wir zwei Dinge erledigen:

\begin{enumerate}
	\item Zum einen müssen wir unsere Seite \textbf{internationalisieren}\index{Internationalisierung}. Das bedeutet schlicht, dass wir dem Browser mitteilen, dass unsere Seite auf Deutsch (bzw. in einer anderen Sprache) verfasst wurde. Dieser Schritt dient Browsern mit Übersetzungsalgorithmen, um unsere Seite automatisch \glqq{}richtig\grqq{} zu übersetzen. Sie werden später erfahren, wie wir einzelne Passagen unserer Webanwendung zu programmieren können, dass sie von der automatischen Übersetzung ausgenommen werden.
	\item Danach müssen wir noch die \textbf{Codierung}\index{Codierung} festlegen, damit der Browser weiß, dass wir z.B. deutsche Umlaute verwenden. Dieser Schritt wird als \textbf{Lokalisierung}\index{Lokalisierung} bezeichnet. Im Gegensatz zur Internationalisierung werden Sie die Auswirkung der Lokalisierung direkt auf der Webanwendung sehen.	
\end{enumerate}

Wie Sie Ihre Webanwendung internationalisieren und lokalisieren erfahren Sie direkt im nächsten Abschnitt.

\section{Der head-Container, Meta-Daten und Attribute}

Sie wissen, dass unser html-Container (der auch als \textbf{Wurzelelement} eines HTML-Dokuments bezeichnet wird) die beiden Container head und body enthält. Wenn Sie sich nun die Webanwendung ansehen, dann können Sie sich schon denken, was die beiden unterscheidet: Im head steht alles, was nicht im Fenster des Browsers angezeigt wird. Hier finden sich zusätzliche Informationen und allgemeine Definitionen, die für die gesamte Webanwendung gelten. Wenn es also eine Einstellung gibt, die für alle Container im \verb|<body>| gelten soll, dann werden Sie sie in aller Regel im \verb|<head>| programmieren. Beispielsweise wird hier die Lokalisierung festgelegt. Solche allgemeinen Definitionen, die für eine ganze Webanwendung gelten, werden auch als \textbf{Meta-Daten}\index{Meta-Daten} bezeichnet. Hier der entsprechende Container:

\begin{verbatim}
	<meta charset=utf-8 />
\end{verbatim}


Dieser meta-Container legt fest, dass als Codierung für unsere Webanwendung UTF-8 verwendet werden soll. Codierung ist Teil der Nachrichten- und Kommunikationstechnik, wird aber auch in einführenden Veranstaltungen der Technischen Informatik besprochen.\\

Außerdem sehen Sie, dass der Container am Ende einen / enthält. Wie Sie bereits wissen, gibt es dafür gibt es einen einfachen Grund: Wenn ein Container keinen Inhalt hat, dann können Sie ihn so abschließen und brauchen kein eigenständiges schließendes Tag programmieren. In HTML5 ist das nur dann nötig, wenn ein Container im Regelfall einen Inhalt hat. Hier können Sie also genauso gut die folgende Zeile programmieren:

\begin{verbatim}
	<meta charset=utf-8>
\end{verbatim}

Damit haben Sie auch Ihr erstes Attribut mit einer Wertzuweisung kennen gelernt: \verb|charset| ist das Attribut, \verb|utf-8| der Wert, der diesem Attribut zugeordnet wird.\\

Nun wollen wir unsere Webanwendung internationalisieren. Das Attribut für Sprache lautet \verb|lang|. Da wir aber nicht nur allgemein festlegen wollen, dass eine Sprache verwendet wird (das wäre ja sinnlos), müssen wir noch festlegen, dass die Sprache Deutsch sein soll. Und das tun wir, indem wir mit \verb|lang=de| de als Wert des Attributs festlegen. Im Gegensatz zur Lokalisierung ist lang ein Attribut des html-Containers.\\

Wenn Sie die nötigen Änderungen an Ihren HTML-Dokumenten durchgeführt haben, sieht das also so aus:

\begin{verbatim}
<!doctype html>
<html lang=de>
<head>
<meta charset=utf-8>
<title><?php include(test_title_001.txt); ?></title>
</head>
<body>
<?php include(test_begruessung_001.txt); ?>
</body>
</html>
\end{verbatim}

Sie wollen wissen, welche Codierung Sie für Sprachen wie Farsi (Afghanistan) oder Chinesisch brauchen? Genau dieselbe wie für Deutsch. Eine weitere Anpassung im Rahmen der Lokalisierung ist also nicht nötig. Aber Sie müssen die Internationalisierung in diesen Fällen über das lang-Attribut anpassen.\\

\textbf{Aufgabe}:\\

Nachdem Sie Ihr Dokument um Internationalisierung und Lokalisierung erweitert haben, laden Sie es erneut. Jetzt wird endlich das \verb|ü| im Browser als ein ü angezeigt.

\section{Mehr Grundlagen in HTML}

\subsection{Mehr Auslagerung von Code}

\textbf{Aufgabe}:\\

Oben haben Sie gelernt, wie Sie Texte aus Dateien mit Hilfe von PHP in eine HTML-Dokument einfügen können. Stellen Sie sich vor, Sie müssten 95 HTML-Dokumente programmieren und bei allen würde in den ersten drei Zeilen der folgende Code stehen: \verb|<html lang=de><head><meta charset=utf-8>|.\\

Programmieren Sie die Auslagerung dieses Code-Fragments in eine Datei, die per PHP ausgelesen und in ein HTML-Dokument eingefügt wird.\\

\textbf{Hinweis}:\\

Eine Auslagerung von so wenig Code ist meist nicht sinnvoll, weil es Ihnen den Überblick über den Code erschwert. Aber wenn wir zur Programmierung von PHP kommen werden Sie feststellen, dass Sie sich mit dieser Methode viel Tipparbeit ersparen können. Und das bedeutet sehr oft eine deutlich reduzierte Fehleranfälligkeit. Außerdem stellen Sie damit sicher, dass Code-Fragmente, die in mehreren HTML-Dokumenten identisch sein sollen auch dauerhaft identisch sind. Stellen Sie sich umgekehrt den Aufwand und die Fehlerwahrscheinlichkeit vor, wenn Sie in 95 HTML-Dokumenten eine kleine Änderung durchführen müssen.

\textbf{Kontrolle}

\begin{itemize}
	\item Sie wissen, dass jedes HTML-Dokument aus einer Doctype Declaration sowie den drei Containern html, head und body besteht.
	\item Sie haben eine erste Vorstellung davon, was Sie im <head> und was Sie im <body> programmieren.
	\item Sie wissen, dass Sie Text im Webbrowser anzeigen können, indem Sie ihn innerhalb des <body>-Containers eingeben.
	\item Sie verstehen noch nicht genau, wie Sie mit Hilfe von Attributen einzelne Container ändern können.
\end{itemize}

\subsection{Verwendung von Escape-Sequenzen}

Vor HTML5 galt die Aussage, dass Sie grundsätzlich davon ausgehen mussten, dass Sie Sonderzeichen einer Sprache mit sogenannten Escape-Sequenzen programmieren mussten. Diese werden Ihnen auch weiter in HTML-Quellcode begegnen, weil Sie ja beispielsweise eine spitze Klammer nicht als Text in HTML einfügen dürfen.\\

Standardmäßig beherrschen Webbrowser ausschließlich die Zeichen, die in der sogenannten ASCII-Tabelle aufgeführt sind. Für HTML4.01 bedeutet das: Nur Zeichen, die im englischen Alphabet vorkommen können direkt auf einer Webanwendung angezeigt werden. Alle anderen müssen mit einer sogenannten Escape-Sequenz ausgedrückt werden. Diese beginnt mit einem \verb|&|-Zeichen und endet mit einem Semikolon. Das deutsche ß mussten Sie unter HTML4.01 mit der Escape-Sequenz \verb|&szlig;| programmieren.\\

Weiterhin gab es noch die Möglichkeiten, Escape-Sequenzen mit der Nummer der Unicode-Tabelle zu programmieren. Beispielsweise steht \verb|&#x9fb9;| für ein chinesisches Schriftzeichen. Stellen Sie sich einmal vor, Sie müssten auf diese Weise eine Webanwendung Zeichen für Zeichen programmieren...\\

Auch hier zeigt sich, dass die Entwickler beim W3C mit HTML5 einen Standard veröffentlicht haben, der wirklich sinnvolle Änderungen einführt, die auch das Programmieren von Webanwendung deutlich vereinfacht, ohne dabei die Funktionalität oder das Layout von Seiten zu beschränken. Vielmehr ist das Gegenteil der Fall.\\

Es folgen einige Escape-Sequenzen, die Sie auch weiterhin benötigen werden:\\

\begin{verbatim}
	&amp;	& (kaufmännisches Und)
	&quot;	`` (quotation mark)
	&lt;	< (less than)
	&gt;	> (greater than)
\end{verbatim}

Nehmen wir dazu an, dass Sie innerhalb eines HTML-Dokuments den folgenden Satz angeben wollen:\\

\verb|Die Zeichenfolge <p> öffnet einen Absatz-Container in HTML|, \\

dann müssen Sie das so einprogrammieren:\\

\verb|Die Zeichenfolge &lt;p&gt; öffnet einen Absatz-Container in HTML.|\\

Unter HTML4.01 hätten Sie zusätzlich für das ö von öffnet die Escape-Sequenz \verb|&ouml;| eingeben müssen:\\

\verb|Die Zeichenfolge &lt;p&gt; &ouml;ffnet einen Absatz-Container in HTML.|

\subsection{HTML5: Anführungszeichen sind weitestgehend optional}

Wenn Sie sich ältere HTML-Kurse ansehen, werden Sie feststellen, dass bei Wertzuordnungen wie lang=de der zugeordnete Wert immer in Anführungszeichen steht: \verb|<html lang="de">| Das ist bei HTML5 nicht mehr zwingend vorgeschrieben. \\

Wenn Sie allerdings Werte zuordnen, die Leerzeichen beinhalten oder eine Webanwendung für veraltete Browser entwickeln wollen, dann sollten Sie in jedem Fall Anführungszeichen verwenden.

\subsection{Zusammenfassung}

Für den Rest dieses Kapitels werden wir nicht wieder über die Container \verb|<html>| und \verb|<head>| sprechen.

\section{Strukturen von HTML5-Dokumenten}

Auch wenn das offensichtlich sein sollte, sei hier betont: Alle Container, die in diesem Abschnitt besprochen werden können ausschließlich im body-Container eingesetzt werden.\\

All diese Elemente haben in HTML5 eine feste Bedeutung, die insbesondere im Rahmen der Barrierefreiheit von Belang ist. Bei HTML4.01 mussten Webentwickler sich noch mit div-Containern behelfen, denen Sie nicht-standardisierte Attribute zuordneten. Die Folge bestand darin, das Browser die in HTML-Dokumenten vorgegebenen Strukturen nicht erkennen konnten und somit weitgehend willkürlich die Ansicht generierten.\\

Das wiederum brachte viele \glqq{}professionelle\grqq{} EntwicklerInnen dazu, Unmengen an Zeit damit zu verschwenden, HTML-Dokumente mit tausenden Zeilen Code zu erweitern, die letztlich nur dafür sorgen sollten, dass die Ansicht wie gewünscht in jedem denkbaren Browser angezeigt wurde.\\

Deshalb werden Sie heute eine Vielzahl an Frameworks finden, die Ihnen diese Arbeit abnehmen, die aber gleichzeitig garantieren, dass Ihre Webanwendung auf bestimmten Endgeräten oder für bestimmte NutzerInnen grausig aussehen werden, weil z.B. die Barrierefreiheit ignoriert wird. Die Seite der HAW ist da nur ein Beispiel. Schauen Sie sich beispielsweise Seiten auf einem Rechner an, deren Design für Smartphones entwickelt wurde: Zum Teil füllen deren Überschriften ein Viertel des Bildschirms: Das ist mangelhaftes Webdesign! Nehmen Sie bento (einen Ableger von Spiegel online), um einen Eindruck zu bekommen, was hier gemeint ist: \url{http://www.bento.de/}\\

Also nochmal: Ignorieren Sie bitte das Design Ihrer Anwendung, denn so lange Sie nicht über jahrelange Erfahrung in diesem Bereich verfügen werden Sie sonst nur Anwendungen entwickeln, die auf einer Art von Endgerät gut aussehen, auf allen anderen dagegen grausig. Und das ist das Vorgehen von Amateuren.

\subsection{header, footer, main und aside}

Mit diesen vier Containern können Sie eine grobe Struktur Ihrer Webanwendung vorgeben.

\begin{itemize}
	\item Im \verb|<main>| sollen sich alle Inhalte befinden, die auf der Webanwendung zentral angezeigt werden sollen.
	\item Der \verb|<aside>|-Container ist dann für ergänzende Inhalte zu den Inhalten im <main> gedacht.
	\item Im \verb|<header>| (nicht zu verwechseln mit dem <head>) können Sie beispielsweise ein Unternehmenslogo oder ähnliches unterbringen.
	\item Der \verb|<footer>| dient dann dazu, um beispielsweise einen Verweis auf das Kontaktformular, das Impressum und ähnliches aufzunehmen.
	\begin{itemize}
		\item Der \verb|<nav>|-Container kann in jedem der anderen vier Container untergebracht werden. Er ist vorrangig dafür gedacht, um eine Navigationsleiste zu realisieren. (Wir haben noch nicht darüber gesprochen, wie Sie einen Link auf eine andere Seite programmieren, aber auch dazu kommen wir bald.)
	\end{itemize}
\end{itemize}

Wenn also ein Browser diese Container unterstützt, dann könnte das bedeuten, dass Header und Footer automatisch am oberen bzw. unteren Rand des Bildschirms eingeblendet werden, während NutzerInnern durch die Seite scrollen. Der Main-Bereich würde dann rund zwei Drittel der Breite und der Aside-Bereich ein Drittel des Bildschirms einehmen.\\

Bei Smartphones mit kleinen Displays würden dagegen Header und Footer wahrscheinlich automatisch minimiert werden und Main sowie Aside untereinander angezeigt werden, damit die Ansicht auf dem Gerät im Sinne der NutzerInnen aufgebaut wäre.\\

Es wären auch noch viele weitere Möglichkeiten denkbar, aber letztlich würden gut programmierte Browser diese Entscheidung anhand der Bauweise des Gerätes selbst durchführen. Als HTML-EntwicklerInnen könnten uns dann vollständig darauf konzentrieren, gut strukturierte semantische Webpages zu entwerfen, anstatt den Großteil unserer Zeit damit zu verschwenden jeden denkbaren Fall manuell zu programmieren.

\subsection{article und section}

Diese beiden Container kommen vorrangig im <main> und <aside> zum Einsatz. Das Konzept sieht wie folgt aus: Wenn Sie wie bei einem Buch die Inhalte Ihrer Webanwendung in Kapitel und Unterkapitel unterteilen wollen, dann beginnen Sie mit einem \verb|<article>|. Sobald ein Unterkapitel beginnt, fügen Sie innerhalb dieses Containers einen \verb|<section>|-Container ein. Wenn Sie dann noch weiter unterteilen wollen, fügen Sie an der entsprechenden Stelle des <section> einen <article> ein usw. usf. Wichtig  ist nur, dass Sie zum einen mit <article> beginnen, und dass Sie die beiden Container-Typen im Wechsel verwenden, wenn Sie jeweils eine weitere unter-Struktur (Unter-Unter-Kapitel zum Unter-Kapitel) beginnen wollen.

\subsection{h1 bis h5}

Alle Container, die Sie in den beiden vorigen Abschnitten kennen gelernt haben, sollen in HTML5 mit einer Überschrift (engl. heading) beginnen. Leider gibt es kein allgemeines <h>-Element, mit dem Sie eine Überschrift definieren können.\\

Vielmehr müssen Sie entsprechend der Struktur, die Sie z.B. durch article und section konstruiert haben den passenden Container (<h1> bis <h5>) auswählen. Durch diese Auswahl legen Sie fest, wie die jeweilige Überschrift angezeigt wird. <h1> ist eine Kapitelüberschrift, <h2> die Überschrift eines Unterkapitels, usw.

\subsection{p}

Wenn Sie einen article oder eine section in mehrere Absätze unterteilen wollen, dann nutzen Sie dafür den <p>-Container. Es gibt zwar auch noch den <div>, der bei HTML4.01 sehr oft verwendet wurde, aber alles, wofür der dort verwendet wurde wird in aller Regel durch die Container erfüllt, die in den beiden ersten Abschnitten aufgezählt wurden.\\

Der <div> ist der einzige Container, für den keine semantischen Merkmale festgelegt werden können. Alleine deshalb sollten Sie im Regelfall <p> verwenden. Bitte missverstehen Sie das nicht: Der Inhalt eines div-Containers kann selbstverständlich auch semantische Informationen enthalten, der Container selbst dagegen nicht.\\

\textbf{Für diejenigen, die bereits HTML4.01 programmiert haben}:\\

Wenn Sie Container wie \verb|<div class="main"> <div class="section">| usw. verwenden, ist klar, dass Sie nichts von dem verstanden haben, was hier wir bislang über HTML5 besprochen haben.\\

<h1> bis <h5>, <div> und <p> gab es bereits in HTML4.01. In HTML5 haben Sie aber zusätzlich die eine Vielzahl an Containern, die für die standardisierte Strukturierung einer Webanwendung genutzt werden sollen.

\subsection{Aufgabe}

Integrieren Sie die neuen Container in die bisherigen Dateien, auch wenn sie damit leer sind. Vergessen Sie dabei nicht, ggf. über PHP identische Teile unterschiedlicher HTML-Dokumente auszulagern.

\section{Polyfills}

Das englische Wort \textbf{Polyfill}\index{Polyfill} bedeutet schlicht Spachtelmasse. Es geht hier also um Programm-fragmente, die dazu dienen, um Lücken aufzufüllen. Sie werden immer wieder auf Möglichkeiten von HTML5 treffen, die von einzelnen Webbrowsern nicht oder nicht vollständig unterstützt werden. Wenn Sie dann einen Container programmiert haben, wird der im betreffenden Browser nicht wie gewünscht angezeigt und seine Elemente werden wie bei HTML4.01 relativ willkürlich platziert.\\

Aber das ist kein echtes Problem, denn für die meisten dieser Fälle gibt es eine Lösung, die Sie per Copy-Paste in Ihre Webanwendung kopieren können. Und diese Lösungen werden als Polyfill bezeichnet.\\

Wichtig ist hier nicht, dass Sie sich merken, für welche Fälle Sie ein Polyfill benötigen, denn das ändert sich ja kontinuierlich mit jeder neuen Version der verschiedenen Browser. Wichtig ist vielmehr, dass Sie sich merken, auf welcher Seite Sie prüfen können, ob Sie ein Polyfill benötigen und wo Sie es bekommen:

\begin{itemize}
	\item Auf \url{www.caniuse.com} können Sie für jeden HTML5-Container prüfen, in welchem Browser er unterstützt wird. Häufig finden Sie bei den einzelnen Containern auch Verweise auf Seiten, auf denen Sie ein passendes Polyfill finden.
	\item Auf \url{www.html5please.com} finden Sie eine Vielzahl an Polyfills.\\
	Merken Sie sich dazu den folgenden kurzen Dialog: 
	\begin{itemize}
		\item What \textbf{can I use}?
		\item \textbf{HTML5}, \textbf{please}.
	\end{itemize}
	Denn damit haben Sie sich schon fast die URLs der beiden Seiten gemerkt.
\end{itemize}

\subsection{Einbindung von Polyfills - <script>-Container}

Polyfills werden in der Sprache \textbf{JavaScript}\index{Programmiersprache!JavaScript}\index{JavaScript} programmiert. Sie brauchen sich jedoch keine Gedanken zu machen, wenn Sie diese Sprache nicht kennen. Denn Sie müssen lediglich einen \verb|<script>|-Container in Ihren \verb|<html>|-Container einfügen. Sämtlichen JavaScript-Code, den Sie verwenden wollen, müssen Sie lediglich in diesen \verb|<script>|-Container kopieren und schon wird er auf Ihrer Webanwendung angewendet.\\

Wichtig ist dabei nur, dass Sie jeweils genau die Bezeichnungen für Container verwenden, die bei HTML5 gelten. Denn genau wie CSS ändern Polyfills die Art, wie ein bestimmter HTML-Container dargestellt wird. Das geht aber nur, wenn die Container die richtige Bezeichnung haben. Sollten Sie sich also vertippen und anstelle des \verb|<aside>|-Containers einen <aseid>-Container programmieren, dann wird Ihnen ein Polyfill, dass die Darstellung des \verb|<aside>|-Containers sicherstellt nichts nützen.\\

Auch hier wieder ein Hinweis für fortgeschrittene HTML-Programmierer: In HTML4.01 mussten Sie für die Verwendung von JavaScript noch eine Zeile im \verb|<head>|-Container erstellen, die dem Browser mitteilt, dass \glqq{}Scripte\grqq{} in JavaScript (oder einer anderen Programmiersprache) erstellt werden. Das ist bei HTML5 nicht mehr nötig, weil hier JavaScript die Standardprogrammiersprache ist.

\section{Zusammenfassung}

Die folgenden beiden HTML-Dokumente zeigen, wie Sie mit den bisher vorgestellten HTML5-Elementen eine einfache Webanwendung entwickeln können. Wichtig: Bislang haben Sie noch keine Möglichkeit kennen gelernt, um

\begin{itemize}
	\item Links auf andere Seiten zu erstellen,
	\item Bilder und andere multimediale Inhalte einzufügen oder
	\item die Webanwendung semantisch zu machen.
\end{itemize}

Aber keine Sorge, all diese Punkte und noch viel mehr werden wir in Kürze behandeln.\\

Hier ein Beispiel, in dem der Anfang dieses Skripts als zwei HTML-Dokumente zusammengefasst ist. Die Absätze wurden deutlich gekürzt, damit Sie jeweils sehen können, wie die Programmierung aussehen kann.\\

\begin{verbatim}
<!doctype html> 
<html lang=de> 
<head> 
<meta charset=utf-8>
<title>PRG - Vorwort</title> 
</head> 

<body> 
<header>
<!-- Hier später das Logo der Webanwendung einfügen -->
</header>

<main>
<h1>Vorwort</h1>
<p>Ein häufiges Missverständnis besteht darin, dass Programmierung und Informatik miteinander verwechselt werden. ....</p> 
</main>

<footer>
<!-- Hier später den Link auf das Impressum einfügen. -->
</footer>

<aside>
<article>
<h1>Zusammenfassung</h1>
<p>Informatiker entwickeln Konzepte und Modelle, um Ideen möglichst effizient umzusetzen. Dabei spielt es keine Rolle, ob diese Konzepte mit einem Computer umsetzbar sind.
</p>
<p>Programmieren sind Menschen, die Konzepte und Modelle in eine Programmiersprache umsetzen.
</p>
</article>
</aside>
</body> 
</html>

<!doctype html> 
<html lang=de> 
<head> 
<meta charset=utf-8 />
<title>PRG - WWW und semantic Web</title> 
</head> 

<body> 
<header>
<!-- Hier später das Logo der Webanwendung einfügen -->
</header>

<main>
<article>
<h1>Vorwort</h1>
<p>Programmierveranstaltungen bereiten Sie in aller Regel darauf vor, Programme für einen Computer zu  ....</p> 
<section>
<h2>Programmierung von Webanwendung und Webapplicationen</h2>
<p>Die bekannteste Markup Language ist HTML, die HyperText Markup Language. ...</p>
<p>Wie gesagt definieren Sie in einer Markup Language lediglich  ...</p>
</section>
<section>
<h2>Das semantische Web</h2>
<p>Bis hierher haben Sie nur über Dinge gelesen, die Sie unter  ...</p>
<p>Sehen wir uns das mal im Detail an:</p>
<article>
<h3>Syntax und Semantik </h2>
<p>Den einen dieser Begriffe haben Sie wahrscheinlich in der Schule kennen und ...</p>
<p>Damit kommen wir zur Semantik. Mit Semantik bezeichnen wir ...</p>
<p>Aber es gibt einen sehr großen Unterschied zwischen  ...</p>
<p>Stellen Sie sich nun die folgende Situation vor, ...</p>
</article>
</section>
</article>
</main>

<footer>
<!-- Hier später den Link auf das Impressum einfügen. -->
</footer>

</body> 
</html>
\end{verbatim}

\subsection{Hausaufgabe}

Sie haben vorhin sechs Dateien (arbeitsweg.html usw.) erstellt. Erweitern Sie diese Seiten zum nächsten Mal wie folgt:

\begin{itemize}
	\item Internationalisieren und Lokalisieren Sie jede Seite.
	\item Erstellen Sie für jede Seite eine grundlegende Struktur, bei der Sie insbesondere sämtliche hier vorgestellten HTML5-Container sinnvoll verwenden.
	\item Füllen Sie Ihre Seiten mit Inhalten, die zum Seitentitel passen.
	\item Wenn Sie irgendwelche Inhalte wie Spiele, Videos, Bilder usw. einfügen wollen, dann tragen Sie an der entsprechenden Stelle einen Platzhalter ein. (Z.B. „Einzufügen: Bild von Horst“, „Hier mein geniales Gitarrensolo einbauen.“ usw. usf.) Wichtig: Es spielt keine Rolle, ob Sie sich zutrauen, die entsprechenden Inhalte selbst zu entwickeln. Lassen Sie einfach Ihrer Phantasie freien Lauf.
	\item Lassen Sie die Datei \verb|index.html| vorerst so, wie Sie ist.          
	\item Suchen Sie nach einem Polyfill und programmieren Sie es ein, damit der \verb|<aside>|-Container in den folgenden Browsern „richtig“ angezeigt wird: Firefox, Internet Explorer, Safari, Edge
\end{itemize}

\section{Hyperlinks}

Vorhin haben Sie gelernt, dass das Protokoll zur Übertragung von Webanwendung HyperText Transfer Protokoll heißt. Sie wissen bereit, dass das Wort Protokoll nur eine Bezeichnung für eine Vereinbarung darüber, wie etwas zu tun ist. In diesem Fall geht es also um eine Vereinbarung darüber, wie Hypertexte übertragen (transferiert) werden sollen. Also kommen wir kurz dazu, was denn nun wiederum Hypertexte sind und was die mit Webanwendung zu tun haben.\\

Die Antwort ist ganz einfach: Im Gegensatz zu einem Buch beinhaltet eine Webanwendung Kreuzverweise, denen Sie folgen können. Diese Verweise kennen Sie umgangssprachlich als Links. Eine Webanwendung ist also mehr als nur eine Ansammlung von Texten wie bei einem Buch. Und aus diesem Grund wurde das, was wir heute Webpage nennen, in der Zeit als \textbf{Hypertext}\index{Hypertext} bezeichnet, als das WWW gerade erst entwickelt wurde. Wenn Sie also in \textbf{HTML}\index{HTML}\index{Programmiersprache!HTML} programmieren, dann programmieren Sie Hypertext. Nur nennt das heute kaum noch jemand so. Im Namen des Protokolls HTTP lebt dieser Begriff wahrscheinlich noch sehr lange weiter.\\

Damit kommen wir zu den Links, die ursprünglich als Hyperlinks bezeichnet wurden. Das englische Wort Link bezeichnet ja allgemein eine Verbindung. Dementsprechend bezeichnet ein \textbf{Hyperlink}\index{Hyperlink} eine Verbindung zwischen zwei Hypertexten. Jetzt aber genug über Begriffe, schließlich wollen Sie wissen, wie Sie einen Link programmieren können. Warum der so heißt, wie er heißt, dürfte da für Sie nebensächlich sein.

\subsection{Anker}

Auch wenn wir hier in Hamburg sind, wo Sie am Hafen Anker in Hülle und Fülle finden, reden wir an dieser Stelle über Teile einer Webanwendung, wenn wir über einen Anker reden. \textbf{Anker}\index{Anker} sind bei HTML Container, auf die ein Link verweisen kann. In anderen Worten: Wenn Sie in Ihrem HTML-Dokument einen Anker definieren, dann können Sie von einer beliebigen anderen Stelle aus auf diese Stelle verweisen. Sie können dann also an einer beliebigen Stelle einen Link einprogrammieren, mit dem ein Nutzer genau bei einem bestimmten Anker landet.\\

Wie alles andere in HTML waren Anker ursprünglich vollwertige Container. Deshalb können sie auch als eigenständige Container programmiert werden. Sinnvoller ist es allerdings, Container mittels des id-Attributs als einen Anker zu programmieren. Hier ein Beispiel für einen Hypertext, in dem ein Überschrift-Container als Anker programmiert wurde:

\begin{verbatim}
...
<main>
<article>
<h1>Mein leckerstes Fleischgericht</h1>
<p>Dieses Rezept habe ich von meiner Oma, die ...</p>
<p>Als sie dann 1972 in ...</p>
<section>
<h2 id=blanchieren>Blanchieren und andere Zubereitungsarten</h2>
<p>Und dann sagte sie ...</p>
</section>
... 
\end{verbatim}

Auch wenn es naheliegend ist: In einem HTML-Dokument darf jedes \textbf{id}\index{id}-Attribut nur einmal verwendet werden. Das heißt nicht, dass Sie jeweils nur einen Anker programmieren dürfen, sondern dass Sie z.B. innerhalb eines HTML-Dokuments nur einmal \verb|id=blanchieren| einprogrammieren dürfen.

\subsubsection{Aufgabe:}

\begin{itemize}
	\item Programmieren Sie einige Anker in Ihre HTML-Dokumente.
\end{itemize}

\subsection{Links}

Es gibt drei wichtige Varianten von Links, die aber im Großen und Ganzen gleich programmiert werden.\\

Wenn Sie auf eine andere Seite verlinken wollen, dann nutzen Sie dazu den Container \verb|<a href ...>|. Das a am Anfang ist noch ein Überbleibsel aus der Zeit, als Anker in Form das \verb|<a>|-Containers programmiert wurden. Wie gesagt wird dafür heute das \verb|id|-Attribut verwendet.\\

Hier die drei genannten Varianten:

\begin{itemize}
	\item Sie wollen auf einen Anker verlinken, der sich im selben HTML-Dokument befindet, aus dem heraus Sie ihn verlinken wollen. Bsp.: Sie haben ein kleines Glossar auf Ihrer Seite, in dem Sie den Anker „blanchieren“ einprogrammiert haben. Ein Nutzer soll innerhalb dieses Glossars zum Anker springen können. Dann sieht der Link-Container so aus:\\
	\verb|<a href=blanchieren> Beliebiger Text, der auf der Webanwendung unterstrichen angezeigt wird und damit anzeigt, dass hier ein Link vorhanden ist. </a>|\\
	\item Sie wollen auf eine andere Webanwendung verlinken. Dann tragen Sie nach dem Gleichzeichen die vollständige URL ein. (Was eine URL ist, klären wir gleich.) Im folgenden Beispiel programmieren wir einen Link auf die Webanwendung des Departments Medientechnik:\\	
	\verb|<a href=http://www.mt.haw-hamburg.de>Zum Department Medientechnik</a>|\\
	\item Nehmen wir an, Sie wollen dagegen auf einen Anker auf einer anderen Seite verweisen. Dann geben Sie zunächst die URL der Seite an, gefolgt von einem Hash (das ist dieses Zeichen: \verb|#|) und gefolgt vom Namen des Ankers. Nehmen wir an, die Seite heißt \verb|meineRezepte.html| und der Anker heißt \verb|kohlroulade|. Dann könnte der Link so aussehen:\\	
	\verb|Zu meinem Rezept für <a href=meineRezepte.html\#kohlroulade>Kohlrouladen</a>.|
\end{itemize}

\subsubsection{Aufgabe:}

\begin{itemize}
	\item Programmieren Sie jetzt einige Links auf die verschiedenen Anker Ihrer Webanwendung.\\
	
	(Sie haben doch die letzte Aufgabe erfüllt, in der Sie Anker programmieren sollten, nicht wahr?)
\end{itemize}

\subsection{Verlinkungen als expliziter Download}

Bei HTML4.01 bewirkt das Anwählen eines Hyperlinks, dass der Browser versuchen wird, die Datei zu öffnen. Erst wenn er feststellt, dass es sich um ein Format handelt, dass er nicht öffnen kann, wird er einen Download anbieten. Mit HTML5 wurde für \verb|<a href>|-Container das \verb|download|-Attribut eingeführt. Darüber können Sie explizit angeben, dass ein Link heruntergeladen werden soll.\\

Wenn Sie diesem Attribut einen Namen als Wert übergeben, dann geben Sie dem Browser vor, unter welchem Namen die Datei gespeichert werden soll. Das ist vor allem dann von Vorteil, wenn der Dateiname eher kryptisch ist.\\

\verb|<a href=DC9287349723.jpg download=FenderAmericanVintage.jpg>Foto meiner Gitarre</a>|

\subsection{Hausaufgabe:}

\begin{itemize}
	\item Erstellen Sie einige Bilddateien, damit Sie diese nutzen können, um explizite Links in Ihrem HTML-Dokumenten einprogrammieren können.\\
	
	\textbf{Wichtig}:\\
	
	Sie müssen diese Bilddateien selbst erstellt haben und dürfen keine rechtlich geschützten Gegenstände aufnehmen. Sie sollten ebenfalls darauf achten, dass keine Personen auf den Bildern zu sehen sind, außer wenn Sie das schriftliche Einverständnis dieser Personen haben. Denn auch wenn die Bilddateien vorerst nicht veröffentlicht werden sollen, könnten Sie ansonsten rechtliche Probleme bekommen, bei denen durchaus Bußgelder im vierstelligen Bereich drohen.\\
	
	Diesen Hinweis können Sie auf alle Daten und Dateien beziehen, mit denen Sie arbeiten. Es spielt hier zunächst keine Rolle, ob sie Dateien selbst erstellen oder \glqq{}nur\grqq{} weiterverwenden. Auch wenn Sie sie gar nicht veröffentlichen kann es teuer werden: Die Rechtslage ist hier sehr schnell gegen sie. Mehr darüber lernen Sie in der Veranstaltung Medienrecht. Wenn Sie hier wie unsere Studierenden in Media Systems von einem Anwalt unterrichtet werden, dann können Sie sich freuen, denn der kann Ihnen nicht nur erklären, wie die Rechtslage auf dem Papier ist, sondern auch wie tatsächlich im Gerichtssaal entschieden wird und was all diese Gesetzestexte bedeuten.
\end{itemize}

\subsection{URLs – absolute und relative Adressen}

Wenn Sie im WWW unterwegs sind, rufen Sie Seiten wie \verb|www.haw-hamburg.de| auf. Eine solche Adresse wird als Uniform Resource Locator (kurz \textbf{URL}\index{URL}) bezeichnet. Aber auch wenn das Protokoll angegeben wird (wie bei \verb|http://www.haw-hamburg.de|) und in einer Reihe weitere Fälle spricht man von einer URL. Auch \glqq{}Adressangaben\grqq{}, die sich auf Dateien auf Ihrem Computer beziehen werden als URL bezeichnet. Kurz gesagt ist eine URL eine standardisierte Angabe darüber, wo eine Datei zu finden ist.\\

Eine \textbf{absolute URL}\index{URL!absolut} ist nun eine URL, die den vollständigen Pfad zu einer Datei angibt. (Zur Erinnerung: Das WWW ist nichts als eine Ansammlung von Dateien auf Rechnern, die weltweit vernetzt sind.)\\

Im Gegensatz dazu ist eine \textbf{relative URL}\index{URL!relativ} eine Adressangabe, die den Pfad von dem Standort aus beschreibt, an dem die Datei gespeichert ist, in der die URL einprogrammiert wurde. Meist werden Sie aus einem einfachen Grund mit relativen URLs programmieren: Da die Dateien bei der Programmierung nicht an derselben Stelle gespeichert sind wie später, wenn sie online abrufbar sind, müssten Sie bei absoluten URLs später alle Adressen einmal ändern und würden mit Sicherheit Fehler erzeugen.\\

Ein \glqq{}exzellentes\grqq{} Beispiel finden Sie unter den Tutorials zu Java. Dort wurden viele Links absolut programmiert. Als dann Java von Sun Systems an Oracle verkauft wurde, wurden nicht alle Links überarbeitet. Heute können Sie einzig aus diesem Grund viele Tutorials nicht aufrufen: Da der Link mit \verb|www.sun.com| beginnt, die entsprechende Seite aber bei \verb|www.oracle.com| liegt, führt der Link ins Leere. Häufig wurden die entsprechenden Tutorials dann auf \verb|www.oracle.com| in einem anderen Verzeichnis als bei \verb|www.sun.com| gespeichert, sodass auch eine manuelle Änderung von \verb|sun| in \verb|oracle| nicht hilft, um das Tutorial zu finden.\\

Bei URLs, die nicht auf einen Dateinamen enden, suchen Webbrowser automatisch nach einer Datei namens \verb|index.html|. Deshalb ist es wichtig, dass Sie bei einer Webanwendung immer eine Datei index.html einprogrammieren. (Ausnahmen sind Webpages und Webanwendungen, die Sie mithilfe eines Frameworks, eines CMS oder anderer \glqq{}Hilfen\grqq{} erstellen.\\

Aber wenn Sie beispielsweise eine Webanwendung programmieren, die aus mehreren HTML-Dokumenten besteht, dann brauchen Sie für einen Link von einem dieser Dokumente zum anderen nicht die absolute URL angeben. Nehmen wir an, alle HTML-Dokumente Ihrer Seite würden innerhalb eines Verzeichnisses liegen. Dann brauchen Sie nur den Dateinamen des HTML-Dokuments als URL angeben, auf das Sie verlinken wollen. Eine solche URL wird dementsprechend als \textbf{relative URL}\index{URL!relativ} bezeichnet.\\

Wichtig ist aber vor allem, dass Sie grundsätzlich verstehen, was eine URL ist und wie sie syntaktisch richtig geschrieben wird.

\section{Formulare}

\textbf{Formulare}\index{Formulare} sind Bereiche einer Webanwendung, über die Nutzer Daten per Tastatur oder Maus eingeben können. Interaktive Elemente wie Spiele können auch dazu gehören.\\

Wie Sie wissen können Sie mit HTML lediglich Container definieren, deren Darstellung über CSS festgelegt wird. Wenn Sie dann noch Nutzereingaben verarbeiten wollen, benötigen Sie zusätzlich eine Programmiersprache wie PHP oder JavaScript.\\

Da Sie also mit HTML alleine eine Nutzereingabe nicht verarbeiten können macht es scheinbar wenig Sinn, in HTML Formulare zu erstellen. Auf der anderen Seite wird ja in HTML definiert, aus welchen Elementen eine Webanwendung zusammengestellt wird. Und da auch Formulare solche Elemente sind bzw. aus solchen Elementen zusammengestellt werden, müssen wir uns bei der Programmierung in HTML mit Formularen beschäftigen.\\

Hier sind wir dann auch an einem Punkt angelangt, wo die sonst sehr klare Trennung zwischen HTML und PHP verschwimmt: Alles, was mit Formularen zu tun hat müssen sowohl PHP-EntwicklerInnen als auch HTML-EntwicklerInnen beherrschen.

\subsection{Elemente eines Formulars}

Wie alles andere in HTML programmieren wir auch Formulare  als Container. Hier ist es der \verb|<form>|-Container.\\

Wie gewohnt können Sie über das \verb|id|-Attribut ein Formular zu einem Objekt im Sinne des DOM machen, das einen Namen hat und auf das verlinkt werden kann.

\begin{verbatim}
<form id=registrierung>
<!- - Hier werden die einzelnen Eingabemöglichkeiten von Nutzern einprogrammiert. - ->
</form>
\end{verbatim}

Der \verb|<form>|-Container ist das Wurzelelement für Formulare, so wie der <html>-Container das Wurzelelement für HTML-Dokumente ist: Alles, was ein Nutzer eingeben darf oder soll wird einfach in Form verschiedener Container in den <form>-Container einprogrammiert. Sie können in jedem HTML-Dokument beliebig viele <form>-Container programmieren.\\

Die Grundidee eines Formulars ist, dass Nutzer hier verschiedene Angaben machen und Optionen anwählen können, die dann gewissermaßen als ein Paket verarbeitet werden. Deshalb brauchen Sie in jedem Formular ein Element, das einzig dafür da ist, dass der Nutzer bestätigt, dass alle Daten des jeweiligen Formulars abgeschickt werden sollen. Und auch wenn Sie beliebig viele Formulare auf einer Seite programmieren können, sollten Sie möglichst nicht zu viele individuelle Formulare programmieren, da Nutzer sonst mehr damit beschäftigt sind, die vielen Formulare einzeln abzusenden, als damit, das nötige Formular auszufüllen.\\

Später werden die Eingaben von Nutzern wie beschrieben an Programme weiter gegeben, die z.B. in PHP oder JavaScript programmiert wurden. Diese Datenübergabe steuern Sie dann durch zwei Attribute des <form>-Containers: Das action-Attribut gibt die URL des Programms an, das steuert, wie mit den Eingaben des Nutzers umgegangen werden soll. Das method-Attribut ist für die Übertragung per HTTP wichtig und steuert, wie die Daten an das Programm übertragen werden. Da wir uns momentan auf die Programmierung eines Formulars in HTML kümmern, dessen Eingaben noch nicht von einem Programm verwendet werden sollen, lassen wir diese Attribute vorerst außen vor.\\

Wie gewohnt wird die genaue Darstellung bzw. die Anordnung der Elemente im Webbrowser später über CSS programmiert.\\

Wichtig: Im Gegensatz zu HTML4.01 bietet Ihnen HTML5 eine Vielzahl an Möglichkeiten, damit Sie prüfen können, ob die Eingabe eines Nutzers valide ist. Bei einem Datum ist es dann beispielsweise unmöglich, den 32. Dezember einzugeben. Die entsprechenden Kontrollfunktionen mussten Sie vor HTML5 mithilfe einer Programmiersprache wie PHP oder JavaScript selbst programmieren.

\subsection{Das name-Attribut und das id-Attribut – Sonderfälle in Formularen}

Innerhalb eines <form>-Containers erstellen Sie für jede Eingabemöglichkeit einen weiteren Container. Damit Sie die Eingaben später weiter verwenden können, müssen Sie bei vielen Containern ein name-Attribut programmieren, dessen Wert innerhalb des <form>-Containers nur einmal vorkommen darf. \\

Wenn Sie sich jetzt wundern, warum hier nicht mehr das id- sondern das name-Attribut verwendet werden muss: Das id-Attribut wurde mit HTML5 so erweitert, dass Elemente einer Webanwendung als Objekte in einer objektorientierten Sprache verwenden werden können. Das name-Attribut wird für die Übergabe von Nutzereingaben eines Formulars an ein Programm verwendet. Deshalb gibt es bei Formularen beide Attribute (name und id).\\

Bei allen imperativen Programmiersprachen werden Werte gespeichert, indem sie jeweils einer Variablen zugeordnet werden. Eine Variable hat einen Bezeichner und einen Typ. In diesem Kurs verwenden wir den Begriff des Bezeichners auch um Verwechselungen mit dem name-Attribut zu vermeiden. Bei den meisten imperativen Programmiersprachen müssen Sie den Bezeichner festlegen, bevor Sie ihn nutzen dürfen. Danach können Sie mit dem Wert der Variable an mehreren Stellen etwas tun. Das macht vor allem dann Sinn, wenn ein Wert sich immer wieder ändern kann oder er an vielen Stellen innerhalb eines Programms geändert werden kann bzw. soll. \\

Der Typ einer Variablen sagt etwas darüber aus, um was für eine Art von Variable es sich handelt. Für Sie und mich ist es beispielsweise klar erkennbar, ob eine Zeichenfolge nun ein Text ist oder ein Datum, eine Rechenaufgabe oder etwas anderes. Für einen Computer ist das nicht klar. Ein Computer kann z.B. nicht unterscheiden, ob die Zeichenfolge 10 die Zahl 10 oder die Zahl 2 (binär 10) oder der Text 10 sein soll; all diese Interpretationen werden innerhalb des Computers unterschiedlich gespeichert und verarbeitet. Deshalb gibt es bei der Nutzung von Variablen innerhalb eines Computerprogramms immer einen Typ für jede einzelne Variable.\\

Wichtig: In Sprachen wie Java wird der Typ für jede Variable vom Programmierer festgelegt und kann sich dann nicht mehr ändern. Diese nicht-Änderbarkeit des Datentyps einer Variablen wird als statische Typisierung bezeichnet. Daneben gibt es noch die dynamische Typisierung. Hier wird der Typ einer Variablen von der Programmiersprache verwaltet und bei Bedarf geändert. Schon vorweg sei gesagt, dass PHP eine dynamisch typisierte Sprache ist. Beide Verfahren haben Vor- und Nachteile und keiner (!) ist schlecht.

\subsection{Container für Formulare}

Die restlichen Inhalte des Kapitels sollten Sie überfliegen, um sich zunächst einen groben Überblick darüber zu verschaffen, welche Arten von Eingaben HTML5 direkt unterstützt. Normalerweise hätte ich Sie hierfür auf die Webanwendung der W3Schools verwiesen, aber leider sind dort die Tags nach Ihrem Namen und nicht nach der Funktion sortiert. Das ist für Einsteiger eher verwirrend, denn so brauchen Sie eine ganze Weile, um z.B. das passende Tag zu finden, wenn Sie wollen, dass der Browser prüft, ob eine Eingabe ein reales Datum sein kann oder nicht.\\

Wichtig: Während viele dieser Eingabemöglichkeiten auf einem Rechner nur die Kontrolle durchführen, ob eine Eingabe des Nutzers zum jeweiligen Typ passt, öffnen sich bei Smartphones häufig kleine Fenster, über die Nutzer z.B. ein Datum anwählen können. Auch hier gilt wieder: Vor HTML5 hätten Sie solche Komfortfunktionen noch selbst programmieren müssen, mit HTML5 brauchen Sie sich um eine solche Programmierung nicht zu kümmern. Alles, was Sie hier tun müssen, ist das passende Tag auszuwählen.

\subsubsection{Formularfelder für Texteingaben}

In diesem Unterabschnitt finden Sie die meisten Typen, die Sie nutzen können, damit Nutzer einen Text oder eine Zahl eingeben können. Wenn ein Tag hier nicht aufgeführt ist, dann liegt das daran, dass Sie wie bei Formular-Tags in HTML 4.01 etwas programmieren müssten, damit die betroffenen Tags ihre Aufgabe erfüllen. Ein Beispiel ist das Tag, mit dem Sie es einem Nutzer ermöglichen können, nach einem Begriff zu suchen. Denn die Suche müssen Sie dann doch wieder selbst programmieren. Also taucht es hier nicht auf.\\

Das value-Attribut\\

Bei allen Formular-Containern (außer dem file-Container) können Sie mit dem value-Attribut eine Antwort eintragen, die der Nutzer aber jederzeit überschreiben kann. Wenn ein Nutzer das nicht tut, wird dieser Wert beim Absenden des Formulars so übertragen, als wenn der Nutzer ihn eingetragen hätte.\\

Das required-Attribut\\

Sie können festlegen, dass Nutzer einzelne Felder ausfüllen müssen, bevor sie ein Formular absenden können. Dazu programmieren Sie schlicht das Attribut required.
Kurzer Text\\

Quellcode: <input> (alternativ: <input type=text>)\\

Damit können Nutzer einen kurzen Text von bis zu 20 Zeichen eingeben.\\

Wenn Sie hier das Attribut type=password vergeben, dann wird die Eingabe maskiert; niemand kann also am Monitor sehen, was der Nutzer eingibt. Das ist aber nur ein Schutz gegen neugierige Kollegen, die nicht verfolgen können, welche Tasten der Nutzer drückt. Gegen die meisten Angriffsarten ist es dagegen vollkommen nutzlos. Um Nutzereingaben gegen diese zu schützen müssen Sie im Programm kryptographische Protokolle integrieren und sollten am besten keine Tastatureingaben als Passwort verwenden. Aber das ist ein Thema für Veranstaltungen in höheren Semestern.\\

Langer Text\\

Quellcode: <textarea>\\

Mit diesem Container ermöglichen Sie es Nutzern, Texte beliebiger Länge einzugeben. Im Gegensatz zu anderen Containern können Sie hier mit den Attributen rows und cols die Größe festlegen. Tipp: Auch wenn es im Moment nicht so gut aussieht, nutzen Sie dazu besser CSS.\\

URL eingeben\\

Quellcode: <input type=url>\\

Nutzer können eine URL eingeben. Einziger Vorteil gegenüber type=text ist die größere Länge.\\

Emails:\\

Quellcode: <input type=email>\\

Hier prüft der Browser, ob die Eingabe eine valide Email-Adresse sein kann: Ist das @-Symbol enthalten? Gibt es eine valide Endung? usw.\\

<form>
Nutzername: <input name=username required>
E-Mail: <input type=email name=email required>
Webanwendung: <input type=url name=webpage>
Ihr Anliegen: <textfield name=userrequest required>
</form>
Quellcode 2.8: Ein einfaches Formular mit Texteingaben

\subsubsection{Formularfelder für Zahleneingaben}

Zahlen eingeben:\\

Quellcode: <input type=number>\\

Hier können Nutzer ganze Zahlen eingeben. Eine Eingabe ist auch per Maus möglich, da zusammen mit dem Eingabefeld noch zwei kleine Schaltflächen eingeblendet werden, über die der Wert erhöht oder gesenkt werden kann.\\

Für diese diesen input-Typ können Sie die selben Attribute verwenden, die auch beim nachfolgenden input-Typ range gelten. \\

Tipp: Verwenden Sie number, wenn Nutzer eine genau Zahl eingeben sollen und Range, wenn ein grober Wert als Eingabe genügt.\\

Zahlen mit einem Schieber auswählen:\\

Quellcode: <input type=range min=... max=...>\\

Im Gegensatz zum Typ number müssen Sie hier die Attribute min und max vorgeben, weil Nutzer kein Feld für eine Eingabe erhalten, sondern einen Slider (zu Deutsch Schieberegler), mit dem sie einen Wert anwählen können. Über das Attribut step können Sie zusätzlich programmieren, wie groß der Abstand zwischen zwei wählbaren Zahlen sein darf.\\

Telefonnummern:\\

Quellcode: <input type=tel>\\

Der Name sagts schon: Damit können Nutzer eine Telefonnummer eingeben. Der Vorteil gegenüber type=number besteht darin, dass so auch eine internationale Vorwahl eingegeben werden kann. Wegen des + ist das bei number nicht möglich.\\

<form>
Nutzername: <input name=username required>
E-Mail: <input type=email name=email required>
Telefon: <input type=tel name=phonenumber>
Webanwendung: <input type=url name=webpage>
Ihr Anliegen: <textfield name=userrequest required>
Ihre Dringlichkeit: <input type=range name=importancy min=1 max=9>
</form>
Quellcode 2.9: Ein einfaches Formular

\subsubsection{Formularfelder für Datums- und Zeitangaben}

Datum\\

Quellcode: <input type=date>\\

Bei Smartphones öffnet sich in diesem Fall ein Feld, über das Nutzer ein Datum, wie z.B. ihr Geburtsdatum anwählen können.\\

Monat und Jahr\\

Quellcode: <input type=month>\\

Hier können Nutzer Monat und Jahr eingeben.\\

Monat und Jahr\\

Quellcode: <input type=week>\\

Hier können Nutzer Woche und Jahr eingeben.\\

Datum und Uhrzeit\\

Quellcode: <input type=datetime-local>\\

Zusätzlich zu type=date bietet dieser Typ noch die Angabe einer Uhrzeit an. Der Typ datetime (ohne –local) ist nicht gültig.\\

Uhrzeit\\

Quellcode: <input type=time>\\

Hier kann eine Uhrzeit eingegeben werden.\\

<form>
Nutzername: <input name=username required>
E-Mail: <input type=email name=email required>
Telefon: <input type=tel name=phonenumber>
Webanwendung: <input type=url name=webpage>
Ihr Anliegen: <textfield name=userrequest required>
Ihre Dringlichkeit: <input type=range name=importancy min=1 max=9>
Für Anrufe teilen Sie uns bitte noch mit, wann wir Sie am besten erreichen. Von <input type=time name=callNoEarlierThan> bis <input type=time name=callNoLaterThan>.
</form>
Quellcode 2.10: Ein einfaches Formular

\subsubsection{Auswahlmöglichkeiten}

Die input-Typen, die Sie bis jetzt kennen gelernt haben, lassen Nutzern eine große Freiheit bei der Eingabe. Einzig das Format (z.B. bei einer Telefonnummer) muss stimmen. Sie als Entwickler können dabei keine Antwortmöglichkeiten fest vorgeben. Für ein Bestellformular bei einem Lieferservice sind diese Formularfelder deshalb nicht geeignet. Es folgen Eingabefelder, die dafür gedacht sind, dass Sie Nutzern eine Reihe an Wahlmöglichkeiten anbieten.\\

Im Gegensatz zu den bisherigen input-Containern können Sie hier die Beschriftung einfach dadurch vornehmen, dass Sie sie innerhalb des Containers programmieren. Besser ist die Nutzung des <label>-Containers, zu dem wir im Anschluss kommen.\\

Checkboxen:\\

Quellcode: <input type=checkbox name=bezeichner value=wert>\\

Damit programmieren Sie ein Kästchen, das von Nutzern an- oder abgewählt werden kann.
Die Attribute name und value werden erst dann von Belang, wenn Sie ein Programm entwickeln, mit dem Sie die Nutzereingaben verwenden wollen (für den Moment können Sie das folgende also überspringen): Das Attribut name kennen Sie bereits. Wenn Sie bei einer Checkbox kein value-Attribut programmiert haben, dann wird dem Programm, das über das action-Attribut des <form> festgelegt wurde die Nachricht bezeichner=on übertragen. (Respektive der Name, den Sie programmiert haben.) Haben Sie dagegen ein Attribut value wie oben programmiert, dann wird dem Programm die Nachricht bezeichner=wert übermittelt.\\

Wichtig (ebenfalls erst in Bezug auf Programme, die Nutzereingaben verwalten): Wenn eine Checkbox nicht angewählt wird, dann wird keine Nachricht an das Programm weitergemeldet. Das mag jetzt überflüssig klingen, aber nehmen wir an, Sie programmieren eine Liste mit solchen Checkboxen, in denen ein Nutzer angeben soll, welche Zutaten er für ein Rezept bereits zu Hause hat. \\

Dann würden Sie wahrscheinlich das folgende im Programm festlegen: Computer, erstelle eine Einkaufsliste all der Zutaten, die der Nutzer noch nicht hat. Da das Programm aber nur diejenigen Zutaten übermittelt bekommt, die der Nutzer schon hat, funktioniert das so nicht.\\

Deshalb sollten Sie keinesfalls Checkboxen programmieren, bei denen es für die weitere Nutzung wichtig ist, dass eine Meldung an das Programm übertragen wird, wonach sie deaktiviert wurde.\\

Radio Buttons:\\

Quellcode: <input type=radio name=bezeichner value=wert>\\

Radio Buttons und Checkboxen (nachfolgender type) verwirren Einsteiger häufig, weil der Unterschied zunächst nicht klar ist. Dabei ist er recht simpel: \\

-	Checkboxen sind für die Fälle gedacht, in denen Nutzer beliebig viele Optionen anwählen dürfen. (Denken Sie an einen Bestellservice, bei dem Nutzer beliebig viele Beilagen zu einem Gericht auswählen können.)

-	Radio Buttons sind dafür gedacht, dass Nutzer sich für eine von vielen Optionen entscheiden müssen. (Denken Sie hier an ein Reisebüro, bei dem Nutzer sich zwischen erster und zweiter Klasse entscheiden müssen.)

<form>
Wählen Sie bitte Ihre Beilage:
<input type=radio name=reis value=reis>Reis
<input type=radio name=fries value=fries>Pommes Frites
<input type=radio name=potatoes value=potatoes>Kartoffeln
<form>
Quellcode 2.11: Einfache Auswahlmöglichkeiten

Drop-Down-Liste\\

Quellcode: <select><option>Beschriftung</option><option>...</select>\\

Über den <select>-Container legen Sie fest, dass eine Drop-Down-Liste angezeigt werden sollen. Für jeden Eintrag müssen Sie einen <option>-Container innerhalb des <select>-Containerns programmieren. Damit Nutzer einen Eintrag angezeigt bekommen, müssen Sie bei jedem <option>-Container einen Text als Inhalt programmieren.\\

Genau wie Radio-Buttons beschränken Drop-Down-Menüs den Nutzer darauf ein Angebot von mehreren auszuwählen. Der Unterschied besteht darin, dass die Auswahlmöglichkeiten mittels Radio-Buttons vollständig angezeigt werden, während die Optionen (deshalb der Name) einer Drop-Down-Liste nur dann angezeigt werden, wenn Nutzer die Liste angewählt haben.\\

Wenn Sie innerhalb eines Drop-Down-Menüs einzelne <option>-Container in einem unter-Drop-Down-Menü versammeln wollen, können Sie dafür den <optgroup>-Container verwenden. Einziger Unterschied gegenüber <option>-Containern ist, dass Nutzer durch das Anwählen des <optgroup>-Containers noch keine Option anwählen. Dieser hat also keinen value, sondern sein Name wird über das label-Attribut definiert.\\

Wenn Sie eine <optgroup> als nicht anwählbar markieren wollen (z.B. weil der Inhalt noch nicht programmiert ist, dann benutzen Sie dafür das Attribut disabled.\\

Datalist\\

Quellcode: <datalist><option><option>...</datalist>\\

Eine Datalist sieht zunächst wie ein Textfeld aus, aber über die option-Container erhalten Nutzer gültige Werte angezeigt. Im Gegensatz zur Drop-Down-Liste sind die <option>-Container mit dem value-Attribut vollständig, es gibt hier also keinen zusätzlichen Container-Inhalt.\\

Farbauswahl:\\

Quellcode: <input type=color>\\

Damit ermöglichen Sie es Nutzern, eine Farbe aus einer Palette auszuwählen. Das könnte beispielsweise nützlich sein, wenn Spieler eine Farbe für Ihre Spielfiguren auswählen sollen. Wie gewohnt nutzen Sie hier das name-Attribut, um die Eingabe an ein Programm zu übergeben. 

\subsubsection{Schalter}

Neben den folgenden input types gibt es noch die <button>-Container, mit denen Sie dieselben Funktionen realisieren können.\\

Schalter mit Beschriftung\\

Quellcode: <input type=button value=“Beschriftung auf der Schaltfläche“ onclick=function()>\\

Damit programmieren Sie einen Schalter, über den eine Funktion des Programms aufgerufen wird, das über das action-Attribut des <form>-Containers festgelegt wurde. Der Name dieser Funktion wird dem Attribut onclick zugeordnet. Was Funktionen sind und wie Sie sie mithilfe des onclick-Attributs nutzen können erfahren Sie im Kapitel ... .\\

Schalter mit Bild\\

Quellcode: <input type=image src=“URL eines Bildes“ onclick=...>\\

Bei dieser Variante programmieren Sie einen Button, der keinen Text, sondern ein Bild enthält.\\

Wichtig: Da ein Button wieder deaktiviert wird, wenn Nutzer den Mausbutton loslassen, macht die Programmierung eines name- und/oder eines value-Attributs hier keinen Sinn. Vielmehr dienen Buttons dazu, dass Nutzer damit bestätigen, dass sie die Eingaben tatsächlich abschicken wollen. Für Sie als EntwicklerIn bedeutet dass, dass die Variablen an das Programm übertragen werden, das über das action-Attribut des <form>-Containers festgelegt wurde.\\

Reset-Schalter\\

Quellcode: <input type=reset>\\

Mit diesem Schalter können Nutzer alle Eingaben löschen.\\

Absende-Schalter\\

Quellcoe: <input type=submit>\\

Mit diesem Schalter übergeben Nutzer die Daten zur weiteren Verwendung durch Ihre Webanwendung.\\

Im Gegensatz zu type=button und type=image rufen Sie hier also keine bestimmte Funktion des Programms auf, sondern durch einen Klick auf diesen Schalter werden sämtliche Eingaben des Nutzers an das Programm übergeben.\\

Wichtig: Sie brauchen keine zusätzliche Beschriftung zu programmieren. Das übernimmt der Browser für Sie.\\

Beachten Sie dabei bitte, dass es hier nur um die Eingaben innerhalb eines Formulars geht. Wenn Sie innerhalb eines HTML-Dokuments mehrere <form>-Container programmiert haben, benötigen Sie für jeden dieser Container einen type=submit. Denn durch diesen werden ausschließlich diejenigen Daten weitergeleitet, die sich im selben <form>-Container befinden.

\subsubsection{Dateien hochladen}

Quellcode: <input type=file>\\

Dadurch wird ein Bereich auf der Webanwendung angezeigt, über den ein Nutzer die Möglichkeit bekommen kann, Dateien auf seinem Rechner auszuwählen, die dann auf den Server hochgeladen werden können.\\

Wichtig: Bitte vergessen Sie nicht, dass Sie in HTML nur programmieren, dass Nutzer eine solche Eingabemöglichkeit erhalten sollen. Die tatsächliche Übertragung der Datei auf den Server wird hier noch nicht realisiert. Das ist Teil der Programmierung eines Servers. Und um die kümmern wir uns im Rahmen der HTML-Programmierung wie besprochen nicht.

\subsection{Container für die Gruppierung und Zuordnung von Eingabeelementen}

Um mehrere Elemente zu gruppieren müssen diese lediglich in einem <fieldset>-Container zusammengefasst und durch einen <br />Container getrennt werden. <br>-Container sind Container ohne Inhalt, die einen Zeilenumbruch bewirken. In HTML5 kann der / deshalb auch weggelassen werden. Vor HTML5 waren sie generell sehr wichtig, da sie für die Gestaltung von Belang waren. Aber da das jetzt in CSS geregelt wird, brauchen Sie sie nur noch in seltenen Fällen wie eben der Zeilentrennung innerhalb eines <fieldset>.\\

Eine Überschrift für ein <fieldset> programmieren Sie nicht mit einem <h...>-Container, sondern mit einem <legend>-Container, der im Gegensatz zu den <h...>-Containern keine Ziffer enthält: Er „heißt“ immer <legend>. Ansonsten gibt es keine Unterschiede zwischen den beiden.\\

Ein weiterer Container, den Sie benötigen, um Formulare zu programmieren ist der <label>-Container. Dieser erzeugt keinen sichtbaren Unterschied, aber er ist wichtig, damit ein Webbrowser z.B. erkennen kann, dass ein Text, der neben einem input-Container steht als Beschriftung für dieses Eingabefeld gedacht ist. Das wirkt sich ggf. auf die Anzeige aus.\\

An dieser Stelle ist das id-Attribut des Containers wichtig, auf den das Label sich beziehen soll. Denn der Label-Container hat an sich noch keine Bindung zu einem anderen Container. Er bekommt die erst, indem das for-Attribut genutzt wird: Dieses bekommt als Wert den Wert des id-Attributs desjenigen Containers, auf den das Label sich beziehen soll.\\

Hier wäre dann ein typisches Formular, wie Sie es für Nutzerregistrierungen verwenden können:\\

<form>
<fieldset>
<legend>Bitte geben Sie Ihre persönlichen Daten ein:</legend>
<label for=surname>Nachname:</label>
<input id=surname name=surname required >
<br>
<label for=email>E-Mail:</label>
<input type=email id=email name=email required >
<br>
<label for=age>Alter:</label>
<input type=number id=age min=0 max=140 name=age required >
<br>
<label for=birthdate>Geburtsdatum:</label>
<input type=date id=birthdate name=birthdate required >
</fieldset>
<input type=submit>
</form>          
Quellcode 2.12: Registrierungsformular

\subsection{Zusammenfassung}

Es gibt in HTML5 deutlich mehr Formularfelder als bei 4.01. Der Grund ist recht einfach: Die neuen Felder prüfen (bis auf type=text, type=button und ähnliche), ob die Nutzereingabe valide ist. Die Eingabe einer Telefonnummer im Feld für die Eingabe der Mailadresse ist damit ausgeschlossen. Auch unsinnige Eingaben wie der 99. März sind damit unmöglich. Bei HTML4.01 hätten Sie dazu noch umfangreichen Code in PHP bzw. JavaScript programmieren müssen.

\section{Multimediale Inhalte einfügen}

Auch dieser Abschnitt hat sich gegenüber HTML4.01 deutlich geändert. Es gibt vier neue Container, die speziell für die Einbindung von Audio- und Videodateien sowie für ein gutes Layout aller multimedialen Dateien gedacht sind. Der große Unterschied gegenüber HTML4.01 besteht darin, dass Sie jetzt alle Arten von multimedialen Inhalten im Browser abspielen können, ohne dafür ein PHP- oder JavaScript-Programm zu benötigen. Das galt früher nur für Bilder.\\

Wichtig: Wir reden hier momentan ausschließlich über anzeigbare oder abspielbare Inhalte. Interaktive Formate wie Flash aber auch die Programmierung interaktiver Inhalte mit Canvas lassen wir momentan außen vor. Canvas ist ebenfalls eine Neuerung in HTML5, die die Gestaltung von Bildern und Animationen ermöglicht. Sie brauchen hierzu also kein zusätzliches Programm. Canvas geht aber noch weiter, denn mithilfe von JavaScript können Sie darin vollständige interaktive Anwendungen (also auch Spiele) programmieren.

\subsection{Bilder}

Der <img>-Container ist der einzige Container für multimediale Inhalte, den es so bereits vor HTML5 gab. Allerdings gilt hier wie überall, dass Attribute, die bei HTML4.01 fürs Layout genutzt wurden nicht mehr unterstützt werden. (Viele Browser unterstützen sie zwar immer noch, aber wenn Sie wollen, dass Ihre Webanwendung dauerhaft nutzbar ist, dann sollten sie diese Regelung für HTML5 beachten.)\\

Um eine Bilddatei einzufügen, nutzen Sie den <img>-Container. <img>-Container haben keinen Inhalt, denn die Bilddatei, die Sie anzeigen lassen wollen wird als Attribut des Containers programmiert.\\

-	Das Attribut src erhält als Wert die URL an, unter der die Bilddatei zu finden ist.
Bsp.: src=bild.jpg

-	Das Attribut alt gibt einen Alternativtitel an, der so lange angezeigt wird, wie das Bild noch nicht geladen ist. Es ist vor allem für die Barrierefreiheit wichtig.
Bsp.: alt=“schönes Bild“

-	Die Attribute width und height geben an, wie breit bzw. hoch das Bild angezeigt werden soll. Sie ordnen hier jedem der beiden eine Zahl zu, die für die jeweilige Größe in Pixeln, also Bildpunkten steht. 

Wichtig: Sie überschreiben damit das Seitenverhältnis des Bildes. Wenn Sie Bilder also für bestimmte Displaygrößen ändern wollen, dann sollten Sie zunächst über ein PHP- oder JavaScript-Programm das Seitenverhältnis berechnen und dann anhand dieser Berechnung dort (also im Programm) die Änderung von Höhe und Breite durchführen.

\subsection{<figure> und <figcaption>}

Wenn Sie ein wissenschaftliches Buch aufschlagen, sehen Sie zu jeder ergänzenden Darstellung einen Untertitel. Mit dem <figcaption>-Container gibt es in HTML5 die Möglichkeit genau dasselbe in standardisierter Form auf einer Webanwendung zu tun. Dieser Container darf allerdings nur innerhalb eines <figure>-Containers verwendet werden.\\

Nun fragen Sie sich vielleicht, was dieser <figure>-Container denn soll, wo es doch bereits den <img>-Container gibt. Die Antwort ist recht einfach: Dieser Container ist dafür gedacht jede Art von multimedialen Inhalten standardisiert bereitzustellen. Es ist also egal, ob Sie nun ein Bild, ein Video, eine Audiodatei anzeigen bzw. abspielen wollen; immer nutzen Sie die gleiche Kombination aus <figure> und <figcaption>, deren Aussehen Sie über ein CSS-Skript definieren.\\

Und nicht nur dass: Sie können mit dem <figure>-Container auch gleich Kombinationen verschiedener multimedialer Dateien erstellen, die im Sinne des semantic web als solche erkannt werden können. Nehmen wir an, Sie haben im Urlaub mehrere Bilder vom Strand geschossen und zusätzlich Aufnahmen vom Meeresrauschen, aus dem Restaurant und von anderen Stellen aufgenommen. Nun wollen Sie als diese Dateien als eine Diashow mit Sound auf Ihrer Webanwendung platzieren. Dann können Sie genau das über einen <figure>-Container erledigen. Und im Gegensatz zu HTML4.01 erkennt jeder HTML5-kompatible Browser, dass es sich bei all diesen einzelnen Dateien um eine logische Einheit (eben Ihre Diashow mit Sound) handelt, obwohl es auf dem Rechner mehrere Dateien sind.\\

Hier ein einfacher <figure>-Container:\\

<figure>
<img src=hotel01.jpg alt=“Ein Bild des Hotels, in dem wir die furchtbarsten zwei Wochen unseres Lebens hatten.“>
<figcaption>Bates Motel</figcaption>
</figure>
Quellcode 2.13: Bild mit Untertitel innerhalb eines <figure>-Containers

\subsection{<figcaption> für Fortgeschrittene}

Eine <figcaption> kann aber nicht nur einen Untertitel enthalten, sondern zusätzlich bzw. unabhängig von einem Text die URL einer Audiodatei. Dann wird das Bild mit dieser Audiodatei akustisch untermalt. Dazu wird ein <audio>-Container verwendet. Wie das geht (und das es sehr leicht zu realisieren ist) sehen Sie in Kürze.

\section{Weitere multimediale Formate}

In diesem Abschnitt erfahren Sie, wie Sie die verschiedensten multimedialen Inhalte in Ihre HTML5-Webanwendung einbinden können. Ein Hinweis vorweg: Zwar bringt HTML5 nur für einige Formate eine Unterstützung mit, aber Sie erfahren gleich, wie Sie auch andere Formate einbinden können.

\subsection{Einbindung eigener und frei verfügbarer Videodateien}

Die Einbindung von Videodateien ist fast genauso einfach wie die Einbindung von Bildern: Nutzen Sie dazu den <video>-Container innerhalb eines <figure>-Containers. Der Unterschied besteht nun darin, dass Sie mehrere Videodateien einstellen können, von denen der Webbrowser sich eines aussuchen kann. Das ist deshalb sinnvoll, weil Sie sich so nicht darum kümmern müssen, zu prüfen, welches Videoformat vom jeweiligen Webbrowser abgespielt werden kann. Das bedeutet also, dass Sie sich nicht für ein Format wie .mp4, .mov, .wmv usw. entscheiden müssen, sondern Sie können Sie alle nutzen und es ist sogar ideal, wenn Sie ein Video in möglichst vielen Formaten bereitstellen können.\\

In einem <video>-Container können Sie die folgenden Attribute nutzen:\\

-	controls regelt, welche Bedienelemente angezeigt werden. Die Standardeinstellungen bewirken, dass ein Play/Pause-Button, eine Zeitleiste und die aktuelle Zeit im Video angezeigt werden. Wenn Ihnen das nicht genügt oder Sie eine Vielzahl an Formaten abspielen wollen (z.B. Flash), dann können Sie im Netz eine Vielzahl an Playern finden, die Sie direkt in den Quellcode Ihrer Webanwendung einbinden können. Suchen Sie dazu schlicht nach „HTML5 Videoplayer“.\\

Wichtig: Auch wenn Sie lediglich die Standard-Bedienelemente anzeigen lassen wollen, müssen Sie controls als Attribut in den <figure>-Container einfügen. Sie brauchen dann aber keinen weiteren Wert zuordnen.\\

-	height und width funktionieren wie bei <img>-Containern. Es gelten die selben Hinweise wie dort.

Innerhalb des <video>-Containers erstellen Sie für jede Videodatei einen <source>-Container. In diesem geben Sie zum einen die URL der jeweiligen Videodatei über das src-Attribut und zum anderen das Kompressionsverfahren über das type-Attribut an.\\

Kompressionsverfahren dienen dazu, um aus einer großen Audio- oder Video-Datei eine kleinere Datei zu erstellen. Ob das zu sichtbaren Qualitätsverlusten führt, hängt vom Verfahren ab. Das bekannteste Kompressionsverfahren für Audio-Dateien ist unter der Abkürzung mp3 bekannt.\\

Wichtig: Nichts ist für einen Nutzer ärgerlicher als wenn er nicht weiß, warum etwas nicht funktioniert. Deshalb können Sie als letzten Eintrag im <video>-Container einen Text eintragen, der ausgegeben wird, wenn der Webbrowser keines der Formate abspielen kann.\\


<figure controls>

<video alt=“Video über die Herstellung von Büchern“>

<source src=movie.mp4 type=video/mp4>
<source src=movie.ogg type=video/ogg>
Leider kann Ihr Browser keines der Formate abspielen, in dem die Videodatei vorliegt. Bitte prüfen Sie, ob Sie eine Erweiterung installieren können, mit dem Sie eines der folgenden Formate abspielen können: MP4, Vorbis OGG

</video>

<figcaption>
Quelle: www.irgendeineseite.de
</figcaption>

</figure>

Quellcode 2.14: Einbindung eines Videos mit Caption. (Die Leerzeilen dienen genau wie die Einzüge lediglich der besseren Lesbarkeit.)

\subsection{Anpassungsmöglichkeiten für den Video-Player}

Sie wissen, dass Webanwendung von den verschiedensten Endgeräten aus aufgerufen werden: Manche User nutzen ein Smartphone mit einer langsamen Internetverbindung, andere nutzen einen Rechner, der per Kabelanschluss Daten mit bis zum 15 MB/s (entspricht ungefähr einem 100 mbps-Anschluss) herunterladen kann. Es wäre also ungeschickt, wenn Sie auf einer Webanwendung ein Dutzend Videos platzieren und den Webbrowser anweisen, alle vollständig herunterzuladen, egal ob der Nutzer sie nun sehen will oder nicht. Deshalb können Sie das Verhalten des Videoplayers anpassen, indem Sie die folgenden Attribute bzw. Attributbelegungen in den <video>-Container einprogrammieren:\\

-	preload=none

Diese Attributbelegung bewirkt, dass der Nutzer einen kleinen Platzhalter sieht, der lediglich anzeigt, dass eine Video-Datei zur Verfügung steht. Die Datei selbst wird nicht heruntergeladen, bis der Nutzer sie anfordert. Diese Option ist gut geeignet, wenn Sie eine Webanwendung mit vielen Videos programmieren wollen, die auch mit einem Smartphone noch übersichtlich sein soll.\\

-	preload=metadata

Diese Attributbelegung bewirkt, dass der Nutzer einen Platzhalter sieht, der im Gegensatz zu preload=none auf der Webanwendung so groß angezeigt wird, wie das Video selbst. Das Video wird auch in diesem Fall nicht heruntergeladen, bis der Nutzer es anfordert. Diese Option ist vor allem für Rechner gut geeignet, wenn auf einer Seite viele Videos platziert werden. Denn selbst bei durchschnittlichen DSL-Anschlüssen kann es sonst mehrere Minuten dauern, bis alle Inhalte einer einzelnen Seite herunter geladen sind. Der Vorteil dieser preload-Belegung besteht darin, dass sich das Layout der Seite nicht ändert, wenn Nutzer Videos starten.\\

-	autoplay

dürfte selbsterklärend sein: Das Video beginnt automatisch. Dieses Attribut macht in Verbindung mit den beiden eben vorgestellten preload-Varianten natürlich keinen Sinn. Dieses Attribut sollten Sie keinesfalls bei multimedialen Dateien verwenden, die eine Audio-Komponente haben, denn im schlimmsten Fall schließen Nutzer Ihre Webanwendung schlicht deshalb, weil sie sich von der Tonspur gestört fühlen.

-	loop
Wurde das Video einmal gestartet, bewirkt dieses Attribut, dass es immer wieder von vorne beginnt. Das ist vor allem dann sinnvoll, wenn Sie ein Video anstelle eines Bildes in den Hintergrund einer Webanwendung einblenden wollen. 

-	poster
Wird dieses Attribut ohne weitere Zuordnung verwendet, dann wird der erste Frame (quasi das erste Bild) des Videos als Stellvertreter angezeigt. Das macht in Verbindung mit preload=none natürlich keinen Sinn.

Als Wert kann diesem Attribut die URL einer Bilddatei zugeordnet werden. Dann wird dieses Bild als Stellvertreter des Videos angezeigt, bis es gestartet wird.

\subsection{Einbindung von geschützten Inhalten (Stichwort: DRM)}

Mit den beschriebenen Möglichkeiten können Sie Videodateien einbinden, auf die Sie freien Zugriff haben. Aber wie Sie wissen ist das beispielsweise bei Videos auf YouTube nicht der Fall. Wenn Sie sicher sind, dass Sie das Recht dazu haben, dann dürfen Sie solche Videos mit einem <iframe>-Container einbinden. Da hier bis auf allowfullscreen keine neuen Attribute vorkommen, sollten Sie das folgende Codefragment ohne weitere Erklärungen einbinden können:\\

<iframe src=http://www.youtube.de/...     allowfullscreen />

Quellcode 2.15: Einbindung von Videos, auf die nicht direkt verlinkt werden kann.

\subsection{Der <audio>-Container}

Alles, was Sie beim <video>-Container nutzen können und das bei einer Audio-Datei Sinn macht, können Sie genau so bei einem <audio>-Container nutzen. Kommen wir also zu den Unterschieden gegenüber einem <video>-Container:\\

-	Ein <audio>-Container hat in aller Regel kein Bild, also gibt es für den Nutzer keinen sichtbaren Unterschied zwischen den Attributbelegungen prelaod=none und preload=meta.

-	Wenn Sie ein Bild zu einer Audiodatei anzeigen wollen (oder eine Diashow), dann nutzen Sie dazu das Verfahren, auf das bei der Erklärung zur <figcaption> hingewiesen wurde.

-	Dem type-Attribut müssen Sie natürlich audio-Typen zuordnen.
Bsp.: type=audio/mp3

Wichtig: Das gilt auch dann, wenn ein Type sowohl für Audio- als auch für Video-Dateien existiert. Bsp.: Das Kompressionsverfahren OGG ist für Audio- und Videoverfahren definiert. Also müssen Sie hier je nach Medienformat type=video/ogg oder type=audio/ogg angeben.

\subsection{Close Captions, Untertitel, Einbindung von Webcams usw.}

Neben den genannten Möglichkeiten gibt es aber auch noch Dinge wie Untertitel oder Texteinblendungen für Menschen mit beschränktem Hörvermögen. Wir werden diese Möglichkeiten nicht im Rahmen der Veranstaltung behandeln. Wenn Sie hieran interessiert sind, möchte ich Sie auf das Format WebVTT hinweisen, das Ihnen in diesen Fällen eine Vielzahl praktischer Erweiterungen für Ihre Webanwendung anbietet.\\

Ähnliches gilt für die Nutzung von Webcams, Mikrophonen und anderen Eingabemöglichkeiten für den Nutzer: Alles, was über die Nutzung von Tastatur und Maus hinausgeht ist nicht Teil dieser Veranstaltung. Hier sollten Sie bei Interesse nach dem Begriff getUserMedia suchen.\\

Grundsätzlich sollten Sie jedoch zunächst HTML, CSS und JavaScript beherrschen, bevor Sie sich in diese Bereiche einarbeiten.

\subsection{Hinweis bezüglich Flash und ähnlichen Formaten}

Adobes Flash bzw. Shockwave war über Jahre hinweg der Standard, wenn es um das Entwickeln von interaktiven Elementen bzw. Spielen auf einer Webanwendung ging. JavaScript bot hier zu wenige Möglichkeiten und einzig Java wurde so entwickelt, dass es im Browser nutzbar war. Warum die meisten Entwickler Flash nutzten soll uns an dieser Stelle nicht interessieren; Tatsache ist, dass es den de-facto-Standard für Webanwendungen darstellte. Wie schon oben angesprochen ändert sich das gerade, was nicht zuletzt daran liegen dürfte, dass JavaScript als Standardsprache für HTML5 festgelegt wurde. \\

Wenn Sie nun Flash-Anwendungen auf Ihrer Seite anbieten wollen, müssen Sie aus diesem Grund einen HTML5-Flash-Player integrieren. Danach suchen Sie genau wie nach HTML5-Video-Playern. Auch die Einbindung funktioniert wieder so ähnlich wie dort. Aber auch hier gilt wieder, dass das kein Thema dieser Veranstaltung ist, sondern lediglich eine Zusatzinformation für interessierte Leser.\\

Ein weiterer Nachteil von Flash besteht darin, dass Flash im Gegensatz zu den multimedialen Containern von HTML5 nicht per CSS angepasst werden kann. Neben dem Nachteil, dass Flash-Inhalte somit schwieriger ans Layout der Seite anzupassen sind, folgt daraus, dass sie im Regelfall nicht barrierefrei sind.

\subsection{Hausaufgabe}

Kommen wir jetzt zur Datei index.html, die Sie zwar in Ihrem Projektordner haben, die aber bislang leer ist. Damit dies eine Startseite für Ihre Webanwendung wird und Sie zwischen den einzelnen Seiten hin- und herwechseln können, programmieren Sie bitte folgendes: (Wenn nicht anders geschrieben programmieren Sie es bitte in der index.html.)\\

-	Fügen Sie die Strukturelemente hinzu, die Sie für HTML5-Seiten kennen gelernt haben.
-	Internationalisieren und Lokalisieren Sie die Seite.
-	Erstellen Sie ein Log-In-Formular im aside-Container.
-	Erstellen Sie eine neue Seite mit einem Registrieren-Formular, sodass Sie Nutzern später die Möglichkeit geben, sich zu registrieren. 
-	Erstellen Sie eine Zusammenfassung der geplanten und vorhandenen Inhalte Ihrer Webanwendung im main-Container.
-	Verlinken Sie an den passenden Stellen auf die entsprechenden Unterseiten.
-	Programmieren Sie umgekehrt auf allen bisherigen Seiten Links. Nutzer müssen mindestens die Möglichkeit haben, über einen Link wieder auf die Startseite zurück zu kommen.
-	Nehmen Sie Bilder, Videos und Audio-Dateien auf, die zu einzelnen Passagen auf Ihrer Webanwendung passen und stellen Sie diese auf Ihrer Webanwendung ein.
-	Achten Sie darauf, dass bezüglich der Barrierefreiheit zumindest die drei Punkte beachtet werden, die Sie oben kennen gelernt haben.
-	Erstellen Sie zu wenigstens einer Ihrer Seiten eine Umfrage, bei der Sie auch verschiedene Auswahlmöglichkeiten programmieren.
-	Nicht vergessen: Prüfen Sie, ob Sie für Firefox, Safari, IE oder Edge Polyfills nutzen müssen.

\section{Weitere Formatierungen und Möglichkeiten in HTML}

Wie es die Überschrift schon sagt finden Sie hier eine Reihe weiterer Möglichkeiten, um Inhalte auf Ihrer Webanwendung zu programmieren bzw. um die Inhalte genauer zu definieren. Die folgenden Abschnitte haben keine feste Reihenfolge, sondern sollen Ihnen lediglich einen Einblick geben, was Sie in HTML5 (zum Teil aber auch schon in HTML4.01) noch an Möglichkeiten haben:

\subsection{Spoiler und andere ausklappbare Texte}

Kennen Sie das? Sie sitzen mit Freunden zusammen und einer davon erzählt das Ende eines Filmes, den Sie noch sehen wollten. So etwas wird mit dem englischen Begriff Spoiler bezeichnet und um jemanden zu warnen, dass gleich ein Spoiler kommt, gibt es den Begriff Spoiler Alarm.\\

Nun nehmen wir an, Sie wollen Filmrezensionen auf Ihrer Webanwendung veröffentlichen, wollen aber dass Ihre Leser selbst entscheiden können, ob Sie die Spoiler mitlesen wollen oder nicht. In HTML5 ist das kein Problem. Nun gibt es aber keinen spoiler-Container, sondern Sie benutzen für solche Fälle zwei Container:\\

-	Der <summary>-Container enthält eine kurze Beschreibung dessen, worum es geht. 
Bsp.: Sie erstellen eine Webanwendung über die Geschichte Kroatiens. Im laufenden Text wollen Sie etwas über den Geburtsort des amtierenden Präsidenten schreiben. Andererseits sind Sie nicht sicher, ob das die meisten Leser interessiert. Also erstellen Sie den folgenden Container:

<summary>Über den Geburtsort des kroatischen Präsidenten</summary>

-	Anschließend ergänzen Sie nach dem Wort Präsidenten (oder an anderer Stelle innerhalb des <summary>-Containers noch einen <details>-Container, in dem Sie all das über den besagten Geburtsort schreiben, was Sie für interessant halten.
Hier ein Beispiel für den oben genannten Spoiler-Fall:

<summary>Das Ende von Hamlet
<details>Alle sind tot.</details>
</summary>

Quellcode 2.16: Ein einfacher <summary>-Container

Wenn Sie innerhalb des „Spoilers“ noch weitere Spoiler unterbringen wollen ist das kein Problem; ähnlich wie bei <article> und <section> können Sie <summary> und <details> beliebig komplex verschachteln. Dabei müssen Sie lediglich darauf achten, dass der äußerste Container ein <summary> ist, und dass Sie die Container jeweils im Wechsel nutzen müssen.\\

Sprich: Sie dürfen zwar innerhalb eines <summary> mehrere <details>-Container programmieren, aber keinen weiteren <summary>. Den müssten Sie dann wieder innerhalb eines der <details> programmieren. Umgekehrt gilt das selbe.\\

Natürlich müssen Sie auch bei diesen Containern prüfen, ob Sie inzwischen in allen Webbrowsern unterstützt werden und ggf. ein passendes Polyfill einbinden.

\subsection{Zeitangaben}

Ein weiterer praktischer Container ist der <time>-Container. Damit können Sie Zeitangaben und Zeiträume im Sinne des semantic Web definieren. Leider ist das Element noch nicht so ausgereift, dass damit alle möglichen Zeitangaben darstellbar wären, denn im Kern wird hierdurch ein Element definiert, dass einen Zeitpunkt oder einen in Sekunden messbaren Zeitraum festlegt.\\

Und da nunmal eine Zeitangabe wie vom 20. Februar bis zum 3. März nicht eindeutig ist (denken Sie an Schaltjahre, bei denen es einen 29. Februar gibt), gibt es auch keinen einzelnen <time>-Container, mit dem Sie diesen Zeitraum zusammen fassen können. Aus dem gleichen Grund können Sie Zeiträume nicht in Monaten oder Jahren festlegen; in solchen Fällen müssen Sie zwei <time>-Container programmieren, den einen für den Anfang, den anderen für das Ende des Zeitraums.\\

So viel zum Negativen, kommen wir jetzt zur praktischen Anwendung von <time>:\\

-	<time>-Container können beliebige Inhalte haben, können aber auch ohne Inhalt als <time /> beendet werden.

-	Um Zeitpunkte oder Zeiträume zu definieren wird unabhängig vom Inhalte des Containers das datetime-Attribut verwendet. Für dieses gibt es eine Vielzahl ein Möglichkeiten, Werte zuzuordnen:

-	Hier die Varianten für Zeitpunkte:

o	Für Jahre: Eine vierstellige Zahl
Bsp.: 1905

o	Für Jahr und Monat: Eine vierstellige Zahl, ein Bindestrich, eine zweistellige Zahl
Bsp.: 2107-03 für März 2107

o	Für ein Datum ohne Uhrzeit: Zusätzlich eine weitere zweistellige Zahl, verbunden per Bindestrich
Bsp.: 2015-09-15 für den 15. September 2015

o	Für Monat und Tag: Zwei zweistellige Zahlen, verbunden durch einen Bindestrich.
Bsp.: 12-08 für den 8. Dezember

o	Für die Uhrzeit gibt es mehrere Varianten:

	Zeitpunkt ohne Angabe der Zeitzone:
17:22

	Zeitpunkt für GMT:
22:13Z

	Zeitpunkt für eine andere Zeitzone:
02:18-05	(Für GMT - 5)
14:47+5:30	(für GMT + 5½)

o	Datum und Uhrzeit können verbunden werden, indem zunächst das Datum, dann nach einer Leerstelle die Uhrzeit aufgeführt wird:

Bsp.: 2017-07-21 20:15-7 (für 20 Uhr 15 in der Zeitzone GMT-7 am 21. Juli 2017)

-	Und jetzt die Varianten für Zeiträume. Diese werden jeweils durch ein großes P für Period bzw. Zeitraum begonnen:

•	P30M entspricht einem Zeitraum von 30 Minuten

•	P5D 20M 7S entspricht einem Zeitraum von 5 Tagen, 20 Minuten und 7 Sekunden.

•	Wie oben aufgeführt können keine Zeiträume in Monaten oder Jahren definiert werden.
Zwar programmieren Sie die <time>-Container wie alle anderen Container, aber zur Sicherheit hier ein paar Code-Beispiele:
...
Die Veranstaltung dauert voraussichtlich<time datetime=“P5D 20M 7s“> mehr als 5 Stunden 20 Minuten</time>.
...
Der erste Termin findet am <time datetime=2015-09-14 13:00Z>14. September 2015 um 13 Uhr</time> statt.
...
Veranstaltungen finden vom <time datetime=2015-09-15>15. September</time> bis zum <time datetime=2016-02-03> 3. Februar</time> statt.
...
Quellcode 2.17: Zeitangaben mit dem <time>-Container

\subsection{Hervorhebung von Texten}

Bei HTML4.01 wurden Textpassagen meist mit Fettdruck, Unterstreichungen oder Kursivschrift hervorgehoben. Die entsprechenden nicht-semantischen Container lauten schlicht <b> oder <strong> für Fettdruck, <u> für unterstrichen und <i> (Englisch für italic bzw. kursiv). \\

Diese können weiterhin verwendet werden. Alle drei haben jedoch Nachteile, wenn die Seite von Personen mit eingeschränktem Sehvermögen genutzt werden oder das Display veraltet ist. Außerdem werden Hyperlinks in Webbrowsern in aller Regel als unterstrichener Text präsentiert.\\

Deshalb bietet HTML5 den <mark>-Container, dessen „Attribute“ background-color und color per CSS angepasst werden können. \\

Bei allen vier Containern müssen Sie also lediglich eine Textpassage, die Ihnen wichtig erscheint mit dem öffnenden und schließenden Tag des jeweiligen Conatainers umschließen.

\subsection{Unterdrückung von Übersetzungen für Textpassagen}

Aktuelle Browser bieten häufig die Übersetzung von Webanwendung aus anderen Sprachen an. Dazu ist das lang-Attribut des <html>-Containers ein wichtiger Hinweis. Doch wenn Sie wollen, dass bestimmte Stellen nicht übersetzt werden sollen, dann können Sie diese durch das Attribut translate=no von der Übersetzung ausschließen.\\

Wenn das nur für einzelne Wörter gilt (z.B. bei Eigennamen wie Müller oder Excel), dann können Sie den <span>-Container verwenden: Dieser ändert die Formatierung des Inhalts zunächst nicht und kann genutzt werden, um beliebig wenige Zeichen innerhalb anderer Container abzugrenzen:\\

<p> 
... 
<span translate=no>Tony Marshall</span> sang während 
<span translate=no>Andy Müller</span> ein Tor schoss. 
...
</p>
Quellcode 2.18: Verhinderung von automatischen Übersetzungen durch Browser.

\subsubsection{Vererbung von Attributen}

Wichtig: Attribute gelten für den gesamten Bereich eines Containers, also auch für alle Container, die sich darin befinden. Dieses Konzept werden Sie in umfangreicherer Form kennen lernen, wenn Sie einen Kurs zur objektorientierten Programmierung belegen.\\

Im folgenden Quellcode haben wir solch einen Fall: Das Attribut translate=no wird für den äußeren <article>-Container deklariert. Damit gilt das Übersetzungsverbot auch in allen Containern, die sich innerhalb dieses <article>-Containers gelten:\\

<article translate=no>
(Einleitender Text) ...
<section> ... Sein Vater war von Beruf Müller.
</section>
(Noch mehr Text)
</article>
Quellcode 2.19: Vererbung von Attributen

In diesem Fall würde der gesamte Satz „Sein Vater war von Beruf Müller.“ nicht übersetzt werden.\\

Aber Sie können innerhalb von Containern weitere Container programmieren, in denen Attribute überschrieben werden. Stellen wir uns dazu vor, Sie wollen auf Ihrer Webanwendung einen latainischen Text im Original veröffentlichen und einige Kommentare dazu posten. Dann möchten Sie natürlich nicht, dass die lateinischen Passagen übersetzt werden, während das bei den Kommentaren sinnvoll wäre. Hier ein entsprechender Quellcode:\\

<article translate=no id=“Carmina Burana mit Kommentar“>
<p> (Originaltext) <span translate=yes>An dieser Stelle scheint im Text von ... ein Übersetzungsfehler vorzukommen, denn ... </span> ... (Fortsetzung des Originaltexts) ... </p>
</article>
Quellcode 2.20: Überschreiben eines vererbten Attributs

Der „Originaltext“ und die „Fortsetzung des Originaltexts“ werden beiden nicht übersetzt, weil Sie Teil des <article>-Containers sind, für den die Übersetzung mittels des translate=no Attributs unterdrückt wird.\\

Dagegen wird der Text „An dieser Stelle ...“ übersetzt, weil er Teil des <span>-Containers ist, für den die Übersetzung mittels des translate=yes Attributs explizit erlaubt wird.\\

Hier sei nochmal darauf hingewiesen: Das translate=yes Attribut gilt nur innerhalb des <span>-Containers. Anschließend gilt es dann nicht mehr. 

\subsection{Aufzählungen (Ordered und Unordered Lists)}

Wenn Sie eine Aufzählung von Elementen erstellen wollen, wie Einkaufslisten, To Do Listen oder ähnliches dann wird das in HTML als unordered List <ul> bezeichnet. \\

Wollen Sie dagegen eine Liste erstellen, bei der die einzelnen Einträge z.B. nummeriert sind, um die Reihenfolge anzugeben, dann wird das in HTML als ordered list <ol> bezeichnet. Wenn Sie die Art der Nummerierung ändern wollen (z.B. in Großbuchstaben statt Zahlen), dann können Sie dazu das type-Attribut nutzen.\\

Die einzelnen Einträge werden dann als <li>-Container innerhalb eines <ul>- oder <ol>-Containers programmiert. Wenn Sie also zwischen einer ordered list und einer unordered list wechseln wollen, müssen Sie nur einen Buchstaben im öffnenden und im schließenden Tag ändern, der Rest bleibt gleich:\\

<ol>
<li>Michael Schumacher</li>
<li>Damon Hill</li>
<li>Jacques Villeneuve</li>
</ol>

<ul>
<li>5g Hefe</li>
<li>3 Eier</li>
<100ml Wasser</li>
</ul>
Quellcode 2.21: Geordnete und Ungeordnete Liste

\subsection{Glossare (Description Lists)}

Manchmal benötigen Sie dagegen eine Listenform, bei der Sie Begriffe und Ihre Bedeutung aufzählen wollen. In dem Fall sind <ul> und <ol> nicht geeignet. Hier greifen Sie am besten auf eine description list <dl> zurück.\\

Innerhalb des <dl>-Containers verwenden Sie dann einen description title <dt>-Container, um den Namen des Eintrags festzulegen und anschließend einen description description <dd>-Container, um die Erklärung einzuprogrammieren:\\

<dl>
<dt>Kaffee, schwarz</dt>
<dd>Heißes Getränk, koffeinhaltig, häufig im Becher von Herrn Alpers anzutreffen.</dd>
<dt>Bohnesuppe</dt>
<dd>Kohlenhydrathaltiges Gericht, häufig in Italo-Western von Bud Spencer konsumiert.</dd>
</dl>
Quellcode 2.22: Glossare / descriptive list

\subsection{Tabellen (table)}

Tabellen sind einer der wenigen Container, mit denen Sie auch unter HTML5 ohne CSS das Layout einer Seite festlegen können. Diese sollten Sie aber aufgrund der vielen verschiedenen Displaygrößen von Endgeräten nur dann einsetzen, wenn es sich nicht vermeiden lässt.\\

Eine Tabelle definieren Sie durch einen <table>-Container. Für jede Zeile definieren Sie darin einen table row <tr>-Container, in dem Sie für jede Spalte einen <td>-Container programmieren.\\

Wenn Sie eine Spaltenüberschrift vergeben wollen, verwenden Sie anstelle des <td>-Containers einen table header <th>-Container.\\

Der folgende Quellcode soll Ihnen verdeutlichen, warum es wichtig ist, Quellcode übersichtlich zu programmieren, so wie Sie das bei den bisherigen Codebeispielen gesehen haben, denn hier ist das eindeutig nicht der Fall: Benutzen Sie dazu Zeileneinzüge (Tabulatoren) und Zeilenumrüche (Enter Taste).\\

<table><tr><th>Uhrzeit</th><th>Montag</th><th>Dienstag</th><th>Mittwoch</th><th>Donnerstag</th><th>Freitag</th></tr><tr><td>8.30-10.00</td><td>PRG</td>...</tr>...</table>
Quellcode 2.23: Ausgesprochen unübersichtliches Beispiel für eine Tabelle

\subsection{Microdata}

Microdata sind zwar eine zentrale Säule des semantic Web, aber aufgrund des Umfangs dieser Veranstaltung können wir sie uns nur sehr kurz ansehen.\\

Zur Erinnerung: Microdata haben nichts mit der Darstellung oder der Struktur einer Webanwendung zu tun, sondern Sie dienen dazu, dem Browser anzuzeigen, welche Bedeutung einzelne Elemente der Seite haben. Hier ein paar Anwendungsfälle:\\

-	Der Browser soll eine Adresse ohne weitere Programmierung an eine Anwendung wie google maps weiterleiten können.

-	Sie wollen, dass der Browser einen Termin in den Kalender Ihres Mail-Programms übertragen kann.

-	Sie veranstalten ein Event, erstellen eine Webanwendung dazu und wollen, dass eine Suchmaschine erkennt, dass es sich um ein Event handelt.

All diese Fälle und noch wesentlich werden bislang unter Begriffen wie search engine optimization (kurz SEO) und machine-readable content zusammen gefasst. Microdata sind ein Mittel, das in HTML5 unterstützt wird, um diese Fälle zu lösen.

\subsubsection{Programmierung von Microdata}

Dazu müssen Sie als erstes einen Container mit dem Attribut itemscope deklarieren. Dieses Attribut ändert wie geschrieben nichts an einer Webanwendung, sondern er teilt dem Webbrowser mit, dass es in diesem Container Elemente im Sinne des semantic web geben kann.\\

Für diejenigen, die bereits ein wenig Erfahrung mit der objektorientierten Programmierung haben: Damit deklarieren Sie diesen Container explizit zu einem Objekt, das Sie in einer Sprache wie JavaScript wie ein vollwertiges Objekt verwenden können.\\

Für alle anderen: Ein Objekt im Sinne der objektorientierten Programmierung besteht im Grunde aus folgenden abstrakten Bestandteilen:\\

-	Einem Namen oder Bezeichner, der für jedes Objekt individuell sein muss.
Die Lösung in HTML kennen Sie bereits; es ist das id-Attribut. Sie brauchen aber vorerst keine id-Attribute zu vergeben, weil wir an dieser Stelle nur Microdata programmieren. Die Änderung von Microdata durch eine Programmiersprache folgt in der Einführung in JavaScript (also nur für MediaSystems Studierende) später.

-	Einer beliebigen Anzahl an Eigenschaften, die bei unterschiedlichen Objekten des gleichen Typs unterschiedlich ausgeprägt sein können.
Damit beschäftigen wir uns gleich. Hier ein Beispiel: Eigenschaften können z.B. Telefonnummern sein. Und dass unterschiedliche Elemente unterschiedlich ausgeprägt sein können, bedeutet bei Elementen vom Datentyp Telefonnummer nicht anderes, als dass unterschiedliche Telefonnummern schlicht unterschiedliche Zahlenkombinationen sind.

-	Einer Datenstruktur oder einem Datentyp, die für jede Eigenschaft individuell beschreibt, um was für eine Eigenschaft es sich handelt.
Das klingt schwieriger, als es ist; in HTML bedeutet es nur, dass wir damit sagen, dass ein bestimmter Text der Name eines Ansprechpartners ist, dass ein anderer Text seine Anschrift ist, usw.

-	Einer beliebigen Anzahl an Methoden (alte Bezeichnung: Funktionen), mit denen die Eigenschaften geändert werden können.

In HTML haben Sie keinen Zugriff auf Methoden. Diese sind Teil von Programmiersprachen wie JavaScript. Dementsprechend beschäftigen wir uns damit vorerst nicht.

\subsubsection{Festlegung des Objekttyps}

Gerade haben Sie gelernt, dass Sie das Attribut itemscope verwenden müssen, um festzulegen, dass ein Container Microdata enthalten soll.\\

Dann müssen Sie festlegen, welche Art von Microdata das sein soll. Zwar könnten Sie hier auch willkürlich eigene Typen festlegen, aber damit hätten Sie die Idee der Microdata ad absurdum geführt, denn was Sie sich bei eigenen Typen denken, kann kein Webbrowser wissen, also wäre es weitestgehend zwecklos, so vorzugehen.\\

Auf der Seite
http://schema.org/docs/schemas.html 
können Sie eine Vielzahl an sogenannten Schemata (das sind unsere Objekttypen) nachschlagen.\\

Wenn Sie nun eine Schema gefunden haben, dass Ihnen gefällt, dann programmieren Sie es mit dem Attribut itemtype in den entsprechenden Container. Im folgenden Beispiel haben wir das Schema für eine Person verwendet, die u.a. Möglichkeiten anbietet, um eine Anschrift als Microdata zu programmieren:\\

<p itemscope itemtype=http://schema.org/Person>
</p>
Quellcode 2.24: Definition eines Absatzes als Microdata für die Anschrift einer Person beinhalten soll.\\

Wie Sie sehen haben wir hier noch keinerlei Angaben zur Person selbst einprogrammiert. Dieser Container würde auf einer Webanwendung also nicht sichtbar sein und er würde vorerst auch noch keinen Zweck erfüllen. Aber dieser Schritt ist wichtig, weil der Webbrowser sonst nicht wissen kann, was er mit den Microdata anfangen soll, die in diesem Container auftauchen.

\subsubsection{Eigenschaften von Objekten}

Nehmen wir an, der Name der Person lautet „Martin Schinken“. Für unsere Microdata benötigen wir jetzt also eine Möglichkeit, um eine Eigenschaft „Name“ zu programmieren und diese als Teil des Objekts einzuprogrammieren.\\

In HTML5 wird das wieder über ein Attribut eines Containers erledigt. Das Attribut lautet itemprop (kurz für the items property). Nun gibt es wieder eine Vielzahl an möglichen Arten von Eigenschaften, also müssen wir diese gleich festlegen.\\

Hinweis: Wenn Sie wie in diesem Fall  erfahren, wofür ein „Befehl“ einer Programmsprache steht (hier itemprop für the items property), dann setzen Sie in einen Programm bitte nicht diese Langform (hier „the items property“) ein, denn nur der „Befehl“ (hier itemprop) ist ein gültiger Teil der Programmiersprache. Die Langform soll Ihnen lediglich als eine Eselsbrücke dienen.\\

Aufgabe:\\

-	Schlagen Sie nach, wie die itemprop heißt, die für den Namen einer Person verwendet wird.\\

Jetzt müssen wir nur noch einen Container innerhalb des <p>-Containers programmieren, der das Attribut itemprop mit dem Wert beinhaltet, den Sie gerade nachgeschlagen haben. \\

Anmerkung: Diejenigen von Ihnen, die diese Veranstaltung durch eine Klausur abschließen, sollten das selbst gemacht haben, weil eine Aufgabe in Ihrer Klausur darin besteht, eine passende itemprop für einen itemtype aus einer Liste auszuwählen.\\

Der Quellcode sollte jetzt also so aussehen: (Für die drei Punkte setzen Sie bitte die von Ihnen gerade recherchierte itemprop ein.)\\

<p itemscope itemtype=http://schema.org/Person>
<span itemprop=...>Martin Schinken</span>
</p>
Quellcode 2.25: Microdata, die den Namen einer Person enthält.\\

Aufgabe:\\

Bei diesem Quellcode sind zwei Fehler enthalten. Der eine ist die fehlende itemprop, die Sie nachtragen sollten. Den anderen kennen Sie schon etwas länger. Genauer gesagt sind in diesem Quellcode also nicht zwei Fehler enthalten, sondern vielmehr fehlen hier zwei Einträge. (Tipp: Wäre der Name Martin Holz oder Martin Hammer, dann wären es ebenfalls zwei Fehler. Beim Namen Martin Borowski dagegen würde der zweite Fehler nicht auftreten.)\\

-	Ergänzen Sie den Code so, dass die beiden Fehler bereinigt werden.

\subsubsection{Umfangreiche Microdata am Beispiel einer Person mit Adressangabe}

Aufgabe:\\

Auch das nachfolgende Codefragment ist lückenhaft. Recherchieren Sie, welche Attributbelegungen jeweils Sinn machen:\\

<section itemscope itemtype=http://schema.org/Person>
<h1>Kontakt</h1>
<dl>
<dt>Name</dt>
<dd itemprop= ... >Ihr Name</dd>
<dt>Position</dt>
<dd>
<span itemprop= ... >Student/in</span> an der
<span itemprop= ... > HAW Hamburg</span>
</dd>
</dl>
<div itemprop= ... itemscope itemtype= ... >
<span itemprop= ... >Hamburger Str. 231</span>
<span itemprop= ... >Hamburg</span>,
<span itemprop= ... >22081</span>
</div>
<h1>Online bin ich aktiv bei:</h1>
<ul>
<li><a href=http://www.twitter.com/ihrTwitterAccount itemprop= ... >Twitter</a></li>
<li><a href=http://www.blogger.com/ihrBlogAccount itemprop= ... >Webblog</a></li>
</ul>
</section>
Quellcode 2.26: Umfangreiche Microdata als Aufgabe zur Recherche

\subsection{Validator für Microdata}

Um zu prüfen, ob Ihre Microdata eindeutig programmiert sind, können Sie auf der folgenden Webanwendung einen Testbereich finden:\\

http://developers.google.com/structured-data  \\

Aufgabe:\\

Prüfen Sie dort, ob Ihre Lösungen zu den beiden letzten Aufgaben valide sind.

\section{Zusammenfassung}

Sie wissen jetzt:\\

-	wie die Grundstruktur jeder Webanwendung aussieht,
-	verstehen was Internationalisierung und Lokalisierung ist,
-	wissen um die Bedeutung von meta-Containern,
-	kennen die neuen Container header, footer, main, aside, usw.
-	und wissen wie Sie Polyfills finden und nutzen können.
Sie wissen außerdem:
-	wie Sie Verbindungen zwischen Webanwendung programmieren können,
-	wie Sie Nutzereingaben in HTML ermöglichen können
-	und wie Sie multimediale Inhalte in die Webanwendung einbinden können.\\

Außerdem verstehen Sie, was das semantische Web ist und wie Sie mittels Microdata eine semantische Webanwendung programmieren können.\\

Was jetzt noch fehlt und leider nicht Teil dieses Kurses ist, ist die Entwicklung von interaktiven Oberflächen in HTML5 mit Hilfe von Canvas.\\

Im nächsten Kapitel folgt die zweite Hälfte des Kursteils, in dem sie die Entwicklung statischer Webanwendung kennen lernen: CSS, die Programmiersprache, mit der Sie die Gestaltung einer Webanwendung programmieren.

\subsection{Mehr Text braucht die Welt … und Zeilenumbrüche}

\textbf{Ab hier finden Sie die Abschnitte des ersten HTML-Skripts, das ich erstellt habe. Dieser Teil ist weitgehend veraltet, da er das Vorgehen bei der Programmierung in HTML 4.01 beinhaltet. Er ist zurzeit noch enthalten, da ich prüfe, welche Abschnitte ggf. noch aktuell sind.}\\

Nun gut, die Überschrift mag nicht ganz der Wahrheit entsprechen, aber tun wir einmal so, als wenn sie zuträfe. Erweitern Sie also einfach den body-Container um einige Sätze. Es ist egal, ob diese Sinn machen oder reines Buchstabenchaos sind, da es hier darum geht, dass Sie sehen, welche Anzeige aus Ihrem Quellcode erzeugt wird. Speichern Sie das Ergebnis dann als seite02.html im bekannten Verzeichnis ab, wechseln Sie im Browser auf localhost bzw. aktualisieren Sie die Seite und öffnen sie dann die neue Webanwendung.\\

Einschub: .htm und .html\\

Wenn Sie sich etwas intensiver mit der Entwicklung von Webanwendung beschäftigen, dann werden Sie sehen, dass manche HTML-Skripte unter Dateibezeichnungen gespeichert werden, die nicht auf .html enden, sondern auf .htm. Streckenweise werden Sie auch Kursunterlagen finden, in denen gesagt wird, dass Sie die Skripte in Dateien speichern sollen, deren Namen mit .htm enden. Neugierige Naturen fragen nun: Wo ist der Unterschied? Kurzantwort: Es gibt keinen. Lange Antwort: Bis in die 90er Jahre war es wichtig, so effizient wie möglich zu programmieren. Deshalb gab es z.B. bei damaligen Betriebssystemen die Einschränkung, dass Dateitypen durch eine Endung festgelegt wurden, die nur drei Buchstaben haben durfte. Als immer mehr Hauptspeicher in Rechnern zur Verfügung stand war diese Beschränkung nicht mehr nötig. Wo also früher .htm stand, wurde irgendwann .html eingegeben. Das ist allerdings nicht bei allen Dateitypen so simpel, Sie dürfen es also nur bei HTML-Skripten voraussetzen.\\

Wichtig: Auch wenn Sie beide Endungen nutzen dürfen gilt: seite.htm ist eine andere Datei als seite.html\\

Wieder zurück zu unserer ersten Webanwendung:\\

Und was sehen Sie? Etwas sehr unschönes: Der gesamte Text zieht sich in der ersten Zeile der Webanwendung hin und wird dann auf der nächsten Zeile von links nach rechts fortgesetzt, ohne dass Zeilenumbrüche eingefügt werden.\\

Was ein Zeilenumbruch ist? Immer wenn Sie bei einer Textverarbeitung die Enter- bzw. Return-Taste drücken, wird die aktuelle Zeile beendet und der Text in der nächsten Zeile fortgesetzt. Bei Microsoft Word erhalten Sie dagegen einen Zeilenumbruch durch die Tastenkombination Shift \& Return.\\

Aber Sie können einen Zeilenumbruch in HTML nicht generieren, indem Sie wie bei einer Textverarbeitung einfach die Enter-Taste drücken, sondern Sie müssen wieder ein Tag setzen:\\

<br />\\

Dieses Tag fügt einen Zeilenumbruch in die Webanwendung ein. Nun fragen Sie sich vielleicht, was denn der Slash / in diesem Tag soll. Das ist ganz einfach: Da jedes Tag einen Container öffnet oder schließt, ein Zeilenumbruch aber gleich abgeschlossen ist, würde es wenig Sinn machen, ein öffnendes und ein schließendes br-Tag zu programmieren. Als Kurzschreibweise können Sie deshalb diese Form nutzen. Und damit handelt es sich hier also nicht einfach nur um einen Tag, sondern um einen vollständigen Container.\\

Das selbe gilt für alle Tags, die keinen eigenen Container brauchen. Hier wäre ein Beispiel: Wenn Sie eine waagerechte Linie in Ihre Seite einfügen wollen, nutzen Sie einfach das folgende Tag:\\

<hr />\\

Aufgaben: \\

(1)	Probieren Sie das gleich mal aus: Fügen Sie wenigstens einen br-Container und einen hr-Container in Ihr Dokument ein. Speichern Sie die Datei und lassen Sie sie sich im Browser anzeigen.

(2)	Und probieren Sie jetzt einmal aus, was mit einem Text passiert, den Sie in einen hr-Container einfügen. (Sprich: Sie programmieren eine öffnendes hr-Tag vor diesem Text und ein schließendes hr-Tag dahinter.)

\subsection{Hyperlinks – Verbindungen zwischen Elementen}

Mittlerweile haben Sie mehrere Seiten gespeichert. Und da ist es unschön, dass wir noch keine Links haben, mit denen wir zwischen den Seiten wechseln können. Sonst hat ja jede Webanwendung Hyperlinks, über die wir genau das tun können, also sollten wir als nächstes Hyperlinks in unsere Seite einfügen.\\

Das Skript für einen einfachen Link sieht so aus:\\

<a href=URL>Ein wenig Text</a>\\

Bei einem konkreten Link müssen wir jetzt also noch die URL des Ziels einfügen.
Beispiel: Wir wollen in unser Skript für die erste Seite einen Link auf die zweite Seite einfügen. Das sähe dann so aus:\\

<a href=seite02.html>Hier geht’s zur zweiten Seite</a>

\subsection{Entitys - Umlaute und andere Sonderzeichen}

All das erklärt aber immer noch nicht, wie Sie Umlaute oder andere Sonderzeichen (die im englischen Alphabet nicht vorkommen) in Ihre Webanwendung einpflegen. Leider müssen wir hier auf eine etwas umständliche Programmierung zurückgreifen.\\

So müssen Sie beispielsweise anstelle des Buchstaben ü das Skript \&uuml; in den fließenden Text eintragen. Aber das sieht nur auf den ersten Blick unübersichtlich aus: Jeder dieser Skriptbefehle beginnt mit dem kaufmännischen Und-Zeichen und endet mit einem Semikolon.\\

Die Zeichenkette dazwischen ist für alle Umlaute simpel: Zunächst der Buchstabe (a, o, u bei kleinen Umlauten und A, O, U bei großen Umlauten) und dann die Zeichenfolge uml für Umlaut.\\

Ähnliches gilt für die übrigen Sonderzeichen, die bei HTML als Entitys bezeichnet werden. Warum sie nicht entsprechend der englischen Grammatik als Entities bezeichnet werden, kann ich Ihnen nicht sagen. Zu Fragen und Nebenwirkungen wenden Sie sich bitte den Anglisten Ihres Vertrauens.\\

Hier eine Tabelle der für den Einstieg wichtigsten Entities :\\

Ä	\&Auml;	
ä	\&auml;	
Ö	\&Ouml;	
ö	\&ouml;	
Ü	\&Uuml;	
ü	\&uuml;	
ß	\&szlig;	(ß mit Ligatur)
€	\&euro;	

\&	\&amp;	(ampers and)
>	\&gt;	(greater than)
<	\&lt;	(less than)

Anführungszeichen unten	\&bdquo;	
Anführungszeichen oben	\&rdquo;	

Sollten Sie einmal ein Zeichen nutzen wollen, dass Sie in keiner HTML-Tabelle finden, dann gibt es noch eine Lösung: Zeichen, die in der Unicode-Tabelle bzw. der ISO 10646  aufgeführt werden, können wie folgt eingefügt werden. Wählen Sie dazu die Nummer aus der Codetabelle, die Ihrem Zeichen entspricht und fügen es zwischen \&\# sowie dem Semikolon ein. Sollte der Wert ein Hexadezimalwert sein, müssen Sie noch ein x einfügen.\\

Beispiel: Nehmen wir an, Sie wollen ein bestimmtes chinesisches Element in Ihrem Text darstellen, dass im Unicode die Codenummer 9FB9 hat. Es ist leicht zu erkennen, dass es sich hier um eine hexadezimale Zahl handelt. Also müssen wir noch ein x in \&\#    ; einfügen. Es ergibt sich also folgendes Skript, mit dem wir das gewünschte Zeichen einfügen können:\\

\&\#x9fb9;\\

Kontrolle\\

Machen Sie das einmal selbst: Laden Sie sich die Unicode-Tabellen herunter (Umfang knapp 130 MB). Fügen Sie dann Ihre Email-Adresse in das HTML-Skript, wobei Sie den Wert für das @-Symbol anhand der Codetabelle in Ihren Text einfügen. Sie werden hier nicht lange suchen müssen: Der Eintrag befindet sich gleich auf der ersten Seite in einem der Dokumente, die Sie heruntergeladen haben.\\

Hinweis: Beachten Sie bitte, dass es sich hier um eigenständige Zeichen handelt, die mit dieser Zeichenkette erzeugt werden und nicht etwa um Container oder Tags. Spitze Klammern haben an dieser Stelle also nichts verloren.

\subsection{Deutsche Umlaute ohne Entities}

Kommt es Ihnen zu kompliziert vor, wenn Sie jedes Sonderzeichen als eine Entity eingeben müssen? Da sind Sie nicht alleine. Deshalb kommen wir jetzt zu einer Zeile, die uns die ganze Arbeit abnimmt. Mit dieser Zeile teilen wir dem Webbrowser mit, dass wir unsere Zeichen mit UTF-8 codieren. Und dann versteht braucht er keine Entities mehr, sondern wir können direkt Umlaute (oder auch Schriftzeichen anderer Sprachen) direkt im Skript unterbringen.\\

Für Fortgeschrittene: Unter Umständen kommen Sie in die Situation, dass Sie UTF-16 oder UTF-32 nutzen wollen. Selbst bei aktuellen chinesischen Texten wird das nicht passieren, da auch diese mit dem Zeichenvorrat von UTF-8 abgedeckt sind. Hiervon rät das W3C bei Webanwendung ab. Prüfen Sie in einer solchen Situation, welche alternativen Möglichkeiten Sie haben. Ggf. gibt es ein anderes Dokumentformat, das Ihnen in dieser Situation weiter hilft.\\

Hier das Skript, das Sie bitte als erste Zeile in Ihren head-Container einpflegen:\\

<meta charset=utf-8>\\

Um Ihre Webanwendung zu internationalisieren sollten Sie außerdem das öffnende html-Tag wie folgt erweitern:\\

<html lang=de>\\

Wichtig: Dennoch müssen Sie die Entities beherrschen, da Programmiersprachen wie JavaScript hiervon nicht betroffen sind. Dort müssen Sie selbst Anführungszeichen als \&quot; eintragen, was nicht nur die Fehleranfälligkeit erhöht, sondern auch die Programmierung deutlich erschwert.\\

\subsection{Hervorhebungen}

Bislang haben Sie zwei Möglichkeiten kennen gelernt, um Texte innerhalb eines HTML-Skripts hervorheben zu lassen: Zeilenumbrüche, mittels derer Sie Absätze erzeugen können und waagerechte Linien.\\

Die meisten Nutzer möchten aber auch kursiv, fett oder unterstrichen nutzen, um Textstellen hervorzuheben. Fortgeschrittene Nutzer verwenden außerdem standardisierte Einstellungen, mit denen Sie mehrere Überschriftentypen haben. Dadurch kann ein Leser am Format eines Textes erkennen, ob es sich um eine Kapitelüberschrift, eine Unterkapitelüberschrift usw. handelt.\\

Für die Programmierung in HTML handelt es sich jeweils wieder um Container. Sie programmieren also wieder ein öffnendes und schließendes Tag für jede Art der Hervorhebung:\\

Fettdruck	<b> … </b>	(bold)
Kursive Schrift	<i> … </i>	(italic)
Unterstrichener Text	<u> … </u>	(underline)

Kapitelüberschrift	<h1> … </h1>	(Header)
Unterkapitel	<h2> … </h2>	
…	<h3> … </h3>	

Aufgabe: \\

Warum ist das folgende HTML-Skript falsch? <b><i>Hallo</b></i> Welt\\

Kontrolle\\

Sie können bis jetzt Texte auf eine Webanwendung hochladen, in denen jedes Zeichen vorkommen kann, dass in irgend einer Sprache der Welt vorkommt.\\

Sie können außerdem einzelne Bereiche eines Textes hervorheben und eine erste Strukturierung mit Hilfe von Überschriften erzeugen.\\

Sie wollen jetzt endlich etwas darüber erfahren, wie Sie einzelne Elemente einer Webanwendung anordnen können? Nicht so schnell mit den jungen Pferden. Schauen wir uns doch zunächst an, wie Sie in HTML Eingaben durch den Nutzer ermöglicht werden können. Und damit sind wir im Bereich von Formularen.

\subsection{Formulare}

Der Begriff Formular umfasst alles, was bei einem HTML-Skript Eingaben durch den Nutzer ermöglicht. Grundsätzlich gilt: Für Formulare verwenden wir form-Container:
<form> … </form>\\

Wichtig: Ohne Unterstützung durch Sprachen wie PHP, JavaScript oder ähnliche können wir zwar Nutzereingaben in einer Webanwendung einprogrammieren, aber wir können diese Eingaben noch nicht weiter nutzen. Wir behandeln Sie aber schon jetzt, weil Sie sich zunächst überlegen sollen, welche Interaktionsmöglichkeiten ein Nutzer haben sollte, um eine bestimmte Funktionalität der Webanwendung zu realisieren. Wie die Webanwendung auf diese Eingaben reagiert werden wir dann später programmieren.\\

Für Fortgeschrittene: Aber auch ohne eine dieser Sprachen können wir seit HTML5 Einschränkungen Programmieren, die z.B. dafür sorgen, dass ein Nutzer bei einem Eingabefeld nur ein echtes Datum (also z.B. keinen 33. 7. 1922) eingeben kann. Wenn Sie daran interessiert sind, solche Einschränkungen gleich in HTML5 zu programmieren, nutzen Sie dazu bitte Selfhtml oder eine andere Quelle.

\subsubsection{Nutzereingaben in HTML programmieren}

Wenn Sie eine Nutzereingabe annehmen wollen, dann müssen Sie innerhalb eines form-Containers einen input-Container für jede Eingabe programmieren:\\

<form>
<input>Erste Eingabe:</input>
<input>Zweite Eingaben:</input>
… 
</form>\\

Kontrolle:\\

(1)	Prüfen Sie, wie dieser Quellcode auf Ihrer Seite angezeigt wird und was Sie wohl ändern müssen, damit die Abfrage sinnvoll dargestellt wird.

(2)	Oh, und natürlich sollen Sie die HTML-Skripte Ihres Projekts jetzt so erweitern, dass Nutzereingaben tatsächlich möglich sind.

(3)	Prüfen Sie außerdem, was mit den Eingaben passiert, wenn Sie den Quellcode ändern, die geänderte Fassung speichern und die Seite dann nochmal laden.

\subsubsection{Überblick über Formularelemente}

Mit dem input-Container können wir bislang nur Eingaben verarbeiten, die z.B. über eine Tastatur eingegeben werden. Ein Datum kann also bei dieser einfachen Form von input-Containern nicht über ein Fenster angewählt werden, obwohl das in HTML mit einem input-Container programmiert werden kann. \\

Wenn wir solche Eingabemöglichkeiten in HTML realisieren wollen, müssen wir den Typ eines input-Containers explizit definieren. Dazu ergänzen Sie schlicht das input-Tag wie folgt, wobei Sie anstelle der drei Punkte den Bezeichner angeben müssen, der dem jeweiligen Typ entspricht:\\

<input type=“ … “>\\

Wenn Sie einige dieser Typen kennen, sollte das ausreichen. Gehen Sie auch hier schlicht danach vor, welche Typen für die Funktionalitäten Ihrer Webanwendung von Belang sind. Leider wird nicht jeder Typ von jedem Browser unterstützt. Entwickeln Sie Ihre Seite zunächst so, als wenn diese Einschränkung nicht gültig wäre. Die Anpassung einer Webanwendung für unterschiedliche Browser ist ein fortgeschrittenes Thema, das den professionellen Webdeveloper auszeichnet.\\

Für jeden Typen gibt es noch eine Reihe sogenannter Attribute, mit denen Sie weitere Einstellungen für die Nutzereingabe vornehmen können. Dazu müssen Sie Ihr input-Tag wie folgt erweitern:\\

<input type=`` … `` attribut1=`` … `` attribut2=`` … ``>\\

Wie sie sehen gibt es hier im Gegensatz zu vielen imperativen Sprachen kein Zeichen, das Sie zwischen die verschiedenen Einträge setzen müssen. In C und den nachfolgenden Sprachen mussten Sie beispielsweise ein Komma setzen. So etwas gibt es in HTML nicht.\\

Genau wie den Typ legen Sie also ein Attribut für eine Eingabe fest, indem Sie das Attribut in das öffnende input-Tag eintragen, dazu ein Gleichzeichen und anschließend den Wert den Sie dem Attribut geben bzw. zuordnen wollen. Diesen Wert müssen Sie immer in Anführungszeichen setzen.\\

Zunächst einige Typen, mit denen Nutzer verschiedene Arten von Texten eingeben können:\\

•	``text``: Der Name ist selbstredend: Hier kann der Nutzer einen Text eingeben.
•	``search``: Dieser Typ ermöglicht es dem Nutzer nach einem Begriff zu suchen. Als Entwickler können Sie optional Begriffe festlegen, die gefunden werden können.
•	``number``: Sie können bei diesem Typen einen Zahlenbereich vorgeben.
•	``email`` und ``url`` sind wenig sinnvolle Typen, da Sie mit ihnen zwar prüfen können, ob eine Eingabe eine gültige Email-Adresse bzw. eine gültige URL sein könnte, aber die Prüfung, ob sie tatsächlich existiert bzw. ob sich hinter der Angabe ein Skriptbefehl versteckt, über den ein Angriff auf Ihre Seite erfolgen soll, das kann dieser Typ nicht feststellen.\\

Jetzt einige Typen, die es dem Nutzer ermöglichen, Angaben aus einer Tabelle bzw. einem neuen Fenster auszuwählen:\\

•	``tel``: Damit kann der Nutzer eine Telefonnummer eingeben.
•	``date``: Hier kann der Nutzer ein Datum anwählen.
•	``time``: Hier kann er einen Zeitpunkt anwählen.
•	``datetime``: Zusätzlich zum Datum (wie bei date) kann hier noch eine Uhrzeit angewählt werden.
•	``datetime-local``: Im Gegensatz zu datetime wird hier die Zeitzone ignoriert.
•	``month``: Hier kann der Nutzer das Jahr und den Monat als vier- bzw. zweistellige Zahl eingeben.
•	``week```: Dreimal dürfen Sie raten…
•	``color``: Hier kann der Nutzer eine Farbe aus einer Farbtafel auswählen.
Abschließend noch eine Möglichkeit, um Schieberegler einzufügen:
•	``range``: Damit können Sie Schiebregler und Schalter programmieren, sodass der Nutzer keinen festen Wert eingeben muss bzw. kann. Ein solches Element kann sowohl horizontal wie vertikal programmiert werden.\\

Aufgabe:\\

Prüfen Sie, welche Typen für die Nutzereingaben in all Ihren HTML-Skripten sinnvoll sind. 30 Nutzereingaben sollten es wenigsten sein. Und nein, Sie sollen nicht auf jeder Webanwendung die gleichen Nutzereingaben programmieren; stellen Sie sich vor, Sie wären auf einer Seite unterwegs, die Sie pausenlos dazu auffordert, die gleichen Eingaben zu machen.

\subsubsection{Der Typ ``text``}

Diesen Typen kennen Sie bereits: Jeder input-Container, der keinen expliziten Typen hat wird in HTML als input vom Typ Text interpretiert.\\

Sie können hier die folgenden Attribute verwenden. Wie immer gilt: Prüfen Sie, welche dieser Attribute bei Ihren Eingaben Sinn machen und programmieren Sie dann entsprechend. Sie brauchen wirklich nicht jedes Attribut im Detail zu wissen; wichtig ist, dass Sie im Stande sind, Attribute zu programmieren. Professionelle Entwickler zeichnen sich nicht dadurch aus, dass Sie alles wissen, sondern dadurch, dass sie solche Details schnell nachschlagen.\\

Das führt auch direkt zum nächsten Hinweis: Es gehört zu den essentiellen Kenntnissen, die Sie bei der Programmierung in vielen Sprachen benötigen, dass Sie lernen, Referenzen zu nutzen. Ihre nächste Aufgabe wird darin bestehen, genau das zu üben. Und keine Sorge: Das fällt jedem anfangs schwer, weil Referenzen nicht dafür verfasst sind, lesbar zu sein. Dafür sind nämlich Lehrbücher da. Referenzen sind Nachschlagewerke für Programmierer.\\

•	Das Universalattribut id. Dazu folgt etwas später eine Erklärung. Für den Moment ignorieren Sie es bitte.
•	``name`` kann von JavaScript genutzt werden, um die Eingabe zu nutzen.
•	``size`` gibt an, wie viele Zeichen sichtbar sein sollen. Wenn Sie hier einen zu großen Wert angeben, dann führt das zu Problemen mit kleinen Displays wie bei Smartphones. Ist der Wert zu klein, dann stört das den Nutzer, weil er zu wenig von seiner Eingabe sieht.
•	``maxlength`` gibt an, wie lang die Eingabe des Nutzers maximal sein darf. Im Gegensatz zu size, das nur die Anzeige beeinflusst, schränkt maxlength also tatsächlich die Eingabemöglichkeiten des Nutzers ein.
•	``value`` legt einen Standardinhalt fest. So könnten Sie in ein Eingabefeld für den Vor- und Nachnamen so etwas wie Mäxchen Müller einblenden lassen, das durch die Nutzereingabe überschrieben wird.
•	``placeholder`` hat eine ähnliche Funktion wie value. Probieren Sie es doch einfach mal aus, um den Unterschied zu sehen.
•	``readonly`` kann dazu genutzt werden, damit ein text-Feld nicht geändert werden kann. So kann zum Beispiel verhindert werden, dass ein Nutzer eine Angabe einträgt, bevor er den Nutzungsbedingungen Ihrer Seite zugestimmt hat. Schlagen Sie einmal nach, welche Werte dieses Attribut haben kann.
•	``required`` dagegen verhindert, dass die Eingaben eines Formulars abgeschickt werden können, wenn eine entsprechende Angabe vom Nutzer noch nicht durchgeführt wurde.
•	``pattern`` ist ein sehr mächtiges Attribut, weil Sie hierüber genau programmieren können, was für Zeichen bei einer Texteingabe erlaubt sind und in welcher Reihenfolge diese auftreten dürfen. Um das richtig nutzen zu können, sollten Sie ein bereits die Veranstaltung Theoretische Informatik besucht haben.\\

Aufgabe: \\

Programmieren Sie jetzt für jede Nutzereingabe nötige und sinnvolle Attribute mit Werten, die Sinn machen.

\subsubsection{Das id-Attribut – Variablen in HTML}

Sie haben gelernt, dass Sie in HTML keine Nutzerangaben verarbeiten können. Aber wie Sie jetzt wissen können Sie Eingabefelder programmieren, mit denen ein Nutzer Eingaben erstellen kann. Wie Sie vielleicht wissen werden solche Eingaben in Programmiersprachen mit Hilfe sogenannter Variablen gespeichert und verarbeitet. Damit wir also später die Nutzereingaben in PHP verarbeiten können, müssen wir nun jeden input-Container einer Variablen zuordnen.\\

Und das erledigen wir über das id-Attribut. Wir programmieren es genauso, wie wir andere Attribute eines input-Containers programmieren. Der Unterschied besteht allerdings in zwei Dingen:\\

(1.)	Das id-Attribut ändert nichts an der Darstellung oder Verarbeitung der Nutzereingabe; es ermöglicht lediglich, dass wir diese Eingabe mit einer Sprache wie PHP oder JavaScript verwenden können.

(2.)	Der Wert des Attributs darf (erst ab HTML5) alle Zeichen außer einem Leerzeichen enthalten.\\

Wichtig: Ein id-Attribut darf auf jeder Webanwendung nur einmal verwendet werden, da sonst der Browser nicht weiß, welches Element Sie mit diesem id-Attribut referenzieren. Sie sollten deshalb über Kommentare jedes vergebende id-Attribut und seinen Zweck am Anfang Ihrer HTML-Skripte festhalten.\\

Praktisch: Das id-Attribut kann außerdem dazu genutzt werden, um später auf eine Stelle in einem HTML-Skript zu verlinken. Wenn Sie später wissen, wie Sie ein Dokument mit anderen Methoden als dem br-Container strukturieren können, dann können Sie es dort einsetzen.\\

Kontrolle:\\

Haben Sie alle input-Container mit einem id-Attribut bezeichnet? Nein, na dann wissen Sie ja, was Sie jetzt tun sollten.

\subsection{Universalattribute}

Sie haben gerade das id-Attribut kennen gelernt. Dieses Attribut unterscheidet sich von allen anderen Attributen, die Sie bislang kennen gelernt haben in einem entscheidenden Punkt: An jeder Stelle eines HTML-Skriptes, wo Sie es nutzen können hat es die selbe Bedeutung.\\

Alle Attribute, die in HTML diese Bedingung erfüllen, also alle Attribute, deren Bedeutung immer gleich bleibt, werden als Universalattribute bezeichnet. Leider werden bei selfhtml in den Beispielen zu Formularen id-Attribute aufgeführt, aber es wird dort nicht explizit angegeben, welche Universalattribute noch verwendet werden dürfen.

\subsection{Labels – Beschriftungen für input-Container}

Bislang haben Sie für jede Abfrage manuell eine Beschriftung hinzugefügt, die aber im Grunde gänzlich unabhängig von diesem input-Container eingefügt wird. Zur Erinnerung: Vorhin haben wir eine Eingabe wie folgt abgefragt: Erste Eingabe:<input \textbackslash>\\

Nun wäre es aber sinnvoll, wenn bereits im Quellcode und damit für den Browser erkennbar der Text Erste Eingabe zum input-Container zugeordnet wäre. Ursprünglich hatten wir erfolglos versucht, das zu erreichen, indem wir die Zeile wie folgt programmiert haben: <input>Erste Eingabe:</input>\\

HTML bietet uns für diese Fälle den label-Container an. Das sähe dann so aus:
<label>Erste Eingabe:<input /></label>\\

Nun möchten wir ja solche Eingaben immer mit einem id-Attribut versehen, damit wir die Eingabe weiter verarbeiten können. Aber dann haben wir ein Problem, denn wir dürfen die Zuordnung des id-Attributs ja nur einmal vornehmen, weil der Browser sonst nicht weiß, auf welches Element wir uns beziehen.\\

Wenn wir jetzt aber das id-Attribut im input-Container belassen, dann ist das label immer noch unabhängig vom input und wir bekommen bei einem Hyperlink auf die id ggf. nicht die Ansicht, die wir uns wünschen.\\

Deshalb gibt es das for-Attribut für label-Container. Wenn ein label sich auf einen input beziehen soll, dann verwenden wir das for-Attribut und ordnen diesem den selben Wert bzw. den selben Eintrag zu, den wir dem entsprechenden it-Attribut des input-Containers zugeordnet haben. Das sieht dann so aus:\\

<label for=``ersteEingabe``>Erste Eingabe:<input id=``ersteEingabe`` /></label>\\

Aufgabe:\\

Na kommen Sie schon. Sie wissen, welche Aufgabe jetzt ansteht.\verb|^^|\\

Kontrolle:\\

Sie können jetzt Texte auf Webanwendung programmieren und unterschiedliche Arten von Eingaben durch Nutzer vornehmen lassen.\\

Bevor wir uns der Frage zuwenden, wie solche Eingaben verarbeitet werden können, kommen wir zunächst zur fortgeschrittenen Programmierung von statischen Webanwendung: Der Programmierung mit CSS.

\subsection{CSS – Cascading Style Sheets}

In der Anfangszeit von HTML wurden Elemente einer Seite ausschließlich durch Container kombiniert, wie sie Sie bis jetzt kennen gelernt haben. Diese Container bewirkten direkt, dass ein Element an einer bestimmten Stelle oder in einer bestimmten Formatierung auf der Webanwendung platziert wird.\\

Dabei waren viele Einstellungen durch den Browser oder durch HTML vordefiniert, sodass es Designentscheidungen gab, die nicht in HTML umgesetzt werden konnten. Außerdem war eine Planung des Seitenlayouts nur für jede Webanwendung einzeln möglich. CSS stellt deshalb Möglichkeiten zum Entwurf von Strukturen und Formaten bereit, die dann durch einen kurzen Befehl auf beliebig vielen Einzelseiten umgesetzt werden kann, sodass ein einheitliches Layout aller Webanwendung mit geringem Aufwand möglich wird.\\

Wichtig: Wenn etwas mit CSS möglich ist, erledigen Sie es mit CSS!\\

Die Möglichkeiten gehen dabei so weit, dass Sie für jedes Element auf den Bildpunkt genau definieren können, wo ein Element platziert werden soll. Außerdem bietet CSS Ihnen die Möglichkeit beispielsweise zwischen einer Ausgabe auf dem Bildschirm und auf einem Drucker zu unterscheiden. Selbst die Ausgabe über Lautsprecher ist hierdurch möglich.\\

Das bedeutet auch, dass Sie teilweise durch CSS Vorgaben aus HTML-Skripten überschreiben können.  Die entsprechenden Passagen des HTML-Skriptes bleiben dann aber im Quellcode unverändert erhalten. Gerade als Einsteiger können Sie hier Schwierigkeiten bekommen, denn das bedeutet, dass es mitunter keine Auswirkungen hat, wenn Sie im HTML-Skript etwas ändern. Hier hilft einzig und alleine die Erfahrung weiter, denn weder der Webbrowser noch das HTML-Skript wird Ihnen mitteilen, dass eine Passage Ihres HTML-Skriptes durch ein CSS-Skript überschrieben wurde.\\

CSS-Bereiche eines HTML-Skripts sind leicht erkennbar, da hier häufig geschweifte Klammern zum Einsatz kommen. Das bedeutet aber auch, dass Sie nicht mehr ausschließlich über Tags und Container Format und Position von Elementen auf Ihrer Seite programmieren, sondern eben zusätzlich über die Skripte, die für CSS verwendet werden.\\

Einen doctype wie in HTML gibt es nicht.\\

Auch bei CSS geht die Entwicklung kontinuierlich weiter. Und so wie die aktuelle Version von HTML die Nummer 5 trägt, ist bei CSS die Version 2.1 aktueller Standard. Aber im Gegensatz zu HTML dauert es bei CSS deutlich länger, bis neue Standards veröffentlicht und umgesetzt werden.\\

Leider können Sie bei einem CSS-Skript keine Angabe zur Version machen. Im Gegensatz zu HTML sind Sie hier also darauf angewiesen, dass der Webbrowser schon weiß, welche Version Sie verwenden. Erkennt er das nicht, dann passiert es im schlimmsten Fall, dass er Ihre Formatierungen kommentarlos ignoriert.

\subsubsection{Für Fortgeschrittene – Präprozessoren: ACSS und SASS}

Um die Arbeit weiter zu reduzieren, wurden für CSS Präprozessoren entwickelt. Ein Präprozessor ist ein Programm, das in einer bestimmten Programmiersprache häufig wiederkehrende Aufgaben vorbereitet, sodass ein Entwickler diese nicht in voller Länge selbst programmieren muss. Für die beiden Präprozessoren ACSS und SASS gilt allerdings, dass Sie zunächst CSS beherrschen müssen, um sie sinnvoll nutzen zu können. SASS bietet dabei Möglichkeiten an, die einfache Kontrollstrukturen und andere Elemente imperativer Programmiersprachen einfügt und auf der dynamisch typisierten Sprache Ruby aufsetzt. Nicht zuletzt deshalb werden wir uns in diesem Kurs nicht mit CSS-Präprozessoren beschäftigen: Diese Möglichkeiten bietet uns bereits PHP an.

\subsection{CSS und unterschiedliche Displayformate}

Sie wissen, dass mittlerweile Geräte auf dem Markt sind, die jeweils die unterschiedlichsten Formate aufweisen. Vorhin haben Sie etwas über responsive web design gehört und erfahren, dass es hier nicht um ein gestalterisches Element geht, sondern darum, u.a. solche unterschiedlichen Formate in der Programmierung zu beachten. Denn dann werden die von Ihnen eingestellten Inhalte so präsentiert, dass jeder, der Ihre Webanwendung aufruft sie in einer ansprechenden Form präsentiert bekommt.\\

Bei CSS 2.1 (und damit unter HTML4) müssen Sie deshalb für jedes Ausgabeformat eine CSS-Deklaration programmieren. Hier drei Beispiele, von denen das letzte explizit die Breite des Gerätes beachtet:\\

<link rel="stylesheet" type="text/css" href="standard.css" media="screen" />\\

<link rel="stylesheet" type="text/css" href="drucker.css" media="print" />\\

<link rel="stylesheet" type="text/css" media="screen and (max-device-width: 480px)" href="smartphoneMittelgross.css" />

\subsection{Schriftsätze und –farben – CSS/font}

Sie können auch Container skripten, die einen eigenen Schriftsatz mit individueller Schriftgröße und Schriftfarbe verwenden. Allerdings ist es wichtig, dass der Webserver und die üblichen Browser sowohl die Schriftart kennen als dass bei dieser Schriftart auch die Größe definiert ist, da sonst die Standardeinstellungen greifen und Ihr Layout somit über den Haufen geworfen wird.\\

Der Bezeichner für diesen Container lautet font. Im Gegensatz zu den Containern, die Sie bislang kennen gelernt haben, ist ein font-Container ohne zusätzliche Angaben im Tag recht sinnlos.\\

(( Recherche: http://wiki.selfhtml.org/wiki/CSS/Eigenschaften/Schriftformatierung/font ))


%\chapter{ML, Teil 2 - Textverarbeitung mit LaTeX}

LaTeX ist aus Sicht der Programmierung eine Markup Language. Im Gegensatz zu HTML handelt es sich hier aber um eine Sprache, die dafür gedacht ist, verschiedenste (vorrangig gedruckte) Dokumente zu erzeugen. Entwickelt wurde sie von Leslie Lamport, einem der Träger des \textbf{Turing Award}. (Wem das kein Begriff ist: Der Turing Award wird als Nobelpreis der Informatik bezeichnet.) Im Gegensatz zu TeX, von dem LaTeX eine Erweiterung ist müssen wir hier nur all das Programmieren, was anders als beim Standard-Dokument ist.\\

Dieser Kurs ist eine knappe Einführung. Umfangreichere Kurs werden an den meisten Hochschulen aber auch direkt von lokalen LaTeX-Gruppen angeboten. Im Department Medientechnik bietet beispielsweise Prof. Görne alle zwei Semester eine Schulung an, die für die Mitglieder des Departments kostenlos ist. Hier können Sie natürlich auch individuelle Fragen stellen.\\

LaTeX hat einige essentielle Vorteile gegenüber den meisten Textverarbeitungen, die Sie aus dem Alltag kennen. Das hier ist nur eine Auswahl:

\begin{itemize}
	\item Genau wie in HTML kümmern Sie sich um den Inhalt und nicht um die Darstellung.
	\item Sie können aber jedes Detail anpassen.
	\item Da Sie direkt im Code arbeiten und auch Einrückungen manuell programmieren kann LaTeX Ihre Arbeit nicht torpedieren.
	\item Es gibt eine große Community, die im Bedarfsfall helfen kann.
	\item Es ist vollständig kostenlos.
	\item Formeln lassen sich sehr einfach eintragen und ändern.
	\item Überschriften werden wissenschaftliche nummeriert.
	\item Inhaltsverzeichnisse, Glossare, nummerierte Abschnitte und ähnliches passen sich automatisch an Änderungen an.
\end{itemize}

Wie bei allen Markup Languages gibt es allerdings einen Nachteil:

\begin{itemize}
	\item Sie müssen sich in die Programmierung einarbeiten. Aber wenn Sie HTML beherrschen, dann wird das für Sie keinen allzu großen Anspruch darstellen. Streckenweise ist es einfacher als die Einarbeitung in HTML5, da wir hier (genauer gesagt in LaTeX 2) kein semantisches Web haben und uns auch nicht um Aspekte des Responsive Design kümmern müssen.
\end{itemize}

Zur Erinnerung: Installieren Sie bitte \verb|TeXStudio|, um mit der Programmierung zu beginnen.

\section{Grundlagen}

Das meiste, was Sie an Grundlagen wissen müssen kennen Sie jetzt schon: Sie wissen was eine Markup Language ist. Bei LaTeX gibt es eine andere Syntax und andere Bezeichnungen für Elemente und Container, die müssen Sie also lernen. Außerdem gibt es einige zusätzliche Container, mit denen Sie z.B. ein Inhaltsverzeichnis generieren können. Ähnlich wie in HTML5 gibt es aber nur im Bedarfsfall ein schließendes Tag.\\

\subsection{Struktur eines LaTeX-Dokuments}

In HTML kennen sie die drei zentralen Container html, head und body.\\

In Latex haben wir keinen html-Container bzw. keinen latex-Container und auch keine Doctype Declaration.\\

Stattdessen sprechen wir von einer \textbf{Präambel}\index{Präambel}\index{LaTeX!Präambel}. Diese enthält alles, was wir bei HTML als Attribut im \verb|html|-Tag und im \verb|<head>| untergebracht haben. Danach folgt der eigentliche Inhalt des Dokuments, für den wir keine spezielle Bezeichnung haben.\\

In LaTeX gibt es leider keinen Bezeichner der dem \verb|Tag| in HTML entspricht. Um es Ihnen beim Einstieg einfacher zu machen werde ich den Begriff in diesem Kapitel dennoch verwenden. Allerdings gibt es hier den Begriff des \verb|Elements|, der für eine kleinste Einheit eines Dokuments steht.

\subsection{Einfache Präambel}

Die folgende Präambel beinhaltet alles, was Sie für die meisten Dokumenten benötigen dürften:

\begin{verbatim}
\documentclass[11pt, a4paper, oneside, draft]{book}

\usepackage{palatino, url}
\usepackage[ngermanb]{babel}
\usepackage[utf8]{inputenc}

\setlength{\parindent}{0cm}
\end{verbatim}

Wie Sie sehen nutzen wir in LaTex keine spitzen Klammern, um Tags anzuzeigen, sondern wir beginnen jedes \glqq{}Tag\grqq{} mit einem Backslash\verb|\|. Bitte beachten Sie den Unterschied: Es handelt sich nicht um den Slash, den Sie mit der Tastenkombination \verb|Shift 7| erhalten, sondern mit der Tastenkombination \verb|Alt gr ß|. Sie müssen dazu also die rechte Alt-Taste (die mit \verb|Alt gr| beschriftet sein sollte) drücken, gedrückt halten und zusätzlich das ß. Das ist anfangs etwas ungewohnt, gibt sich aber mit der Zeit.\\

\subsubsection{Struktur von Umgebungen}

\begin{itemize}
	\item Jeder LaTeX-Container beginnt mit einem Backslash \verb|\| .
	\item Danach folgt der Bezeichner.
	\item Es kann ein paar eckiger Klammern mit einem oder mehreren Einträgen folgen. \verb|[]|
	\item Es können ein oder mehrere Paare geschweifter Klammern mit jeweils einem oder mehreren Einträgen folgen. \verb|{}|
\end{itemize}

Dabei gibt es keine allzu genaue Systematik dafür, was in den geschweiften oder eckigen Klammern steht. Die geschweiften Klammern entsprechen aber meist dem Inhalt eines Containers.\\

\subsubsection{documentclass}

Die documentclass gibt in geschweiften Klammern an, um welche Art von Dokument es sich handelt. Daraus folgen einige Standardeinstellungen, die Sie aber auch ändern können. Hier eine Auswahl:

\begin{itemize}
	\item \verb|book| ist dafür gedacht, um die typischen Elemente eines Buches bereit zu stellen: Teile, Kapitel, usw.
	\item \verb|report| ist für umfangreiche Reportagen gedacht, also auch beispielsweise für Forschungsberichte, die eine Zusammenfassung und mehrer Kapitel umfassen können.
	\item \verb|article| ist für Artikel gedacht, die z.B. in Zeitschiften erscheinen sollen. Auch hier ist z.B. eine Zusammenfassung vorgesehen.
	\item \verb|letter| ist für Anschreiben gedacht.
\end{itemize}

Oben habe ich mich für die documentclass \verb|book| entschieden und die folgenden Optionen festgelegt:

\begin{itemize}
	\item \verb|11pt| legt als Schriftgrad 11 Points fest. Der Standardwert liegt bei 10 pt. Wenn der Ihnen genügt, brauchen Sie also gar keine Angabe zu machen.
	\item \verb|a4paper| legt fest, dass das Seitenformat Din A4 ist. Außer bei der Ausgabe als pdf-Dokument ist ein amerikanisches Seitenformat hier der Standard.
	\item \verb|oneside| legt fest, dass alle Seiten an der gleichen Stelle nummeriert werden. Alternativ dazu können Sie \verb|twoside| wählen. Dann werden Seiten abwechselnd links und rechts mit einer Nummer versehen.
	\item \verb|draft| ist eine praktische Option, da sie dafür sorgt, dass Zeilen, an denen Ein Wort in den Seitenrahmen hineinragt mit einem schwarzen Quadrat gekennzeichnet werden. Das ist in sofern praktisch, als Sie schneller sehen, wo Sie ein wenig Feintuning beim Zeilenumbruch machen müssen.
\end{itemize}

\subsubsection{usepackage}

Dieser LaTeX-Container ermöglicht es Ihnen Packages zu nutzen, mit denen Sie bestimmte Formatierungen anpassen können. Zum Teil erhalten Sie dadurch neue Container, die im \glqq{}Standard-\grqq{}LaTeX nicht enthalten sind.\\

Wenn Sie mehrere Packages nutzen wollen, ohne nähere Optionen auszuwählen, dann können Sie diese durch Kommata getrennt gemeinsam als Argument eines usepackage-Containers verwenden. Bsp.: \verb|\usepackage{palatino, url}| fügt die Packages palatino und url hinzu. palatino ist ein Schriftsatz und url ist ein Container, mit dem Sie URLs im fließenden Text hervorheben können.\\

Wenn Sie dagegen bei einem Package Optionen festlegen wollen, dann müssen Sie für jedes Package einen eigenen Container programmieren. Bsp.: \verb|\usepackage[ngermanb]{babel}| fügt das babel-Package hinzu, das die Syntaxprüfung für verschiedene Sprachen ermöglicht. Hier müssen wir eine Option wählen, da wir uns für eine Sprache entscheiden müssen. Die Option german entspricht allerdings der alten Rechtschreibung, weshalb mit \verb|ngerman| eine eigene Option für die seit Ende der 90er Jahre geltenden Rechtschreibregeln entwickelt wurde.\\

Das Package \verb|inputenc| kann die verwendete Codierung festlegen. Darüber hatten wir ja bereits bei HTML gesprochen, der entsprechende usepackage-Container sollte damit klar sein.

\subsubsection{setlength}

Der Container \verb|\setlength{\parindent}{0cm}| ist dann ein Beispiel dafür, dass wir die Vorstellung eines Containers aus HTML nicht direkt in LaTeX übertragen können: Hier haben wir ein \glqq{}Tag\grqq{}, das zwei Inhalte hat, zum einen \verb|\parindent|, was für den Einzug der ersten Zeile eines Absatzes steht, zum anderen \verb|0cm| was naheliegender Weise für 0 Zentimeter steht.\\

Absätze beginnen bei LaTeX mit einem kleinen Einzug in der ersten Zeile und können nur anhand dieses Einzugs erkannt werden. Wenn Sie dagegen Absätze dadurch trennen wollen, dass Sie eine leere Zeile einfügen wollen und keinen Einzug haben wollen, dann müssen Sie in der Präambel diese Zeile einfügen.\\

\section{Text, Sonderzeichen und Formeln}

Fast alles, was wir von jetzt an kennen lernen wird im body des Dokuments programmiert. Dieser wird ähnlich einem HTML-Container als \verb|\begin{document}| begonnen und mit \verb|\end{document}| beendet.

Wenn Sie dort einen einfachen Text verfassen wollen, dann können Sie direkt damit beginnen: Anders als in HTML gibt es keine expliziten Absatz-Container wie \verb|<p></p>| in LaTeX.\\

Das Ende eines Absatzes \glqq{}programmieren\grqq{} Sie dadurch, dass Sie einfach die Enter-Taste drücken. Wollen Sie zusätzliche Leerzeilen einfügen, dann müssen Sie \verb|\\| eingeben.

\subsection{Sonderzeichen}

Im wissenschaftlichen Bereich nutzen wir immer wieder Sonderzeichen, die dann bei einer Textverarbeitung wie LaTeX nicht direkt eingegeben werden kann, weil dieses Zeichen dort eine besondere Bedeutung hat. Den Backslash und einige mehr haben Sie schon kennen gelernt. Wenn Sie ein solches Zeichen verwenden wollen oder Teile von Programmtexten einfügen wollen, gibt es unter anderem zwei einfache Möglichkeiten. Die beiden folgenden Varianten funktionieren für alle Sonderzeichen identisch:\\

\begin{itemize}
	\item Wenn Sie innerhalb eines Absatzes einzelne Sonderzeichen oder kurze Textpassagen mit Sonderzeichen einbinden wollen, dann nutzen Sie dazu \verb~\verb|Text mit Sonderzeichen|~. Der \verb|\verb|-Container hat eine Besonderheit: Sie können hier jedes Zeichen (also nicht nur geschweifte Klammern) nutzen, um ihn abzugrenzen.\\
	
	Wenn Sie also das Sonderzeichen \verb+|+ verwenden wollen, dann nehmen Sie einfach ein anderes, um den Rahmen des \verb|\verb|-Containers festzulegen. Bsp.: \verb|\verb ~ Text mit Sonderzeichen ~| (Hier wurde \verb|~| als Zeichen verwendet, um den Inhalt des Containers abzugrenzen.)
	
	\item Wenn Sie dagegen mehrere Zeilen mit Sonderzeichen anzeigen lassen, dann nutzen Sie die sogenante verbatim-Umgebung:
	
	\begin{verbatim}
		\begin{verbatim}
		Zeile mit Sonderzeichen
		Noch eine Zeile mit Sonderzeichen
		Noch viele Zeilen mit Sonderzeichen
		...
		\end{ verbatim}
	\end{verbatim}
	
	Anmerkung: Sollten Sie den seltenen Fall haben, dass Sie innerhalb einer verbatim-Umgebung \verb|\end{verbatim}| eintragen wollen, dann lassen Sie einfach eine Leerstelle zwischen der geschweiften Klammer \verb|{| und \verb|verbatim}| stehen.
\end{itemize}

\subsection{Formeln}

In vielen Texten wird erklärt, dass Sie Formeln in LaTeX mit dem Dollar-Symbol \verb|$| abgrenzen. Das funktioniert zwar, allerdings handelt es sich dabei um eine TeX-Anweisung. In LaTeX gibt es für Formeln im Fließtext folgende Zeichensequenz:\\

\verb| \( ... Formel ... \)|\\

Dabei gilt, das alles, was zwischen \verb|\(| und \verb|\)| steht im Sinne des mathematischen Modus interpretiert wird. Der mathematische Modus bietet außerordentlich vielfältige Möglichkeiten, um Formeln einzutragen. Wenn Sie in irgend einem mathematischen Buch ein Symbol sehen, dass in einer Formel verwendet wird, dann gibt es ein LaTeX-Tag, mit dem Sie dieses Symbol im mathematischen Modus erzeugen können.\\

Wie Sie sich vorstellen können, würde eine Einführung in den mathematischen Modus alleine schon ein Buch füllen. An dieser Stelle werde ich nur auf einige wenige Möglichkeiten eingehen, die Ihnen bei den ersten Schritten helfen werden. Alles weitere können Sie in aller Regel durch eine Recherche im Netz sehr schnell in Erfahrung bringen.

\begin{itemize}
	\item Für Multiplikationen sollten Sie \verb|\cdot| nutzen. Dadurch wird ein Multiplikationspunkt eingefügt.
	\item Einen Bruch können Sie mit \verb|\frac{Divident}{Divisor}| darstellen. Divident und Divisor können dabei beliebig komplexe Formeln sein.
	\item Um eine Potenz darzustellen benutzen Sie den Hochpfeil \verb|^|. Wie immer gilt: Wenn der Exponent ein Ausdruck ist, dann nutzen Sie die geschweiften Klammern. Bsp.: \(x^{2 + 3}\) programmieren Sie als \verb|x ^{2 + 3}|.
	\item Indizes wie \(X_{i,j}\) stellen Sie durch \verb|X_{i,j}| dar.
\end{itemize}

Wenn Sie dagegen mehrere Formeln als eigenständige Absätze anzeigen lassen wollen, dann nutzen Sie dafür die equation-Umgebung:

\begin{verbatim}
\begin{equation}
Eine Zeile für jede Zeile der Formel.
In dieser Umgebung gilt wieder der mathematische Modus, wie Sie ihn gerade kennen gelernt haben.
\end{equation}
\end{verbatim}

\section{Umgebungen}

Gerade haben Sie mit \verb|\begin{verbatim}| und \verb|\end{verbatim}| etwas kennen gelernt, das konzeptionell den Containern in HTML entspricht. Bei LaTeX werden diese Container aber als \textbf{Umgebung}en\index{Umgebung}\index{LaTeX!Umgebung} bezeichnet.\\

Wenn Sie beim Anfangs-\glqq{}Tag\grqq{} einer Umgebung Optionen programmieren, dann gilt hier dasselbe, was Sie bei den Attributen bzw. Attributen mit Wertzuweisung in HTML kennen gelernt haben: Diese gelten für alles, was sich innerhalb der Umgebung befindet.

\section{Inhaltsverzeichnis, Kapitel und Abschnitte}

Wenn Sie längere Texte als einen Brief schreiben, dann benötigen Sie noch Möglichkeiten, um z.B. Kapitelüberschriften einzufügen. Aus diesen Überschriften wird später übrigens das \textbf{Inhaltsverzeichnis}\index{LaTeX!Inhaltsverzeichnis} an genau der Stelle erzeugt und ins Dokument eingefügt, an der Sie \verb|\tableofcontents| ins Dokument eintragen.

\subsection{Kapitel, Abschnitte usw.}

Bezüglich der Größe und Strukturierung anhand von Überschriften ist LaTeX komfortabler als HTML: Während dort \verb|<h1>| und \verb|<h2>| nur optisch unterschiedlich sind, bewirken \verb|\chapter{Titel}| und \verb|\section{Titel}|, dass der eine von LaTeX als Teil des anderen interpretiert wird.\\

\textbf{Wichtig}:\\

Diese Überschriften sind keine Umgebungen, sondern werden wie abgeschlossene Container in HTML programmiert und behandelt. Das bedeutet, dass der Text und die Unterüberschriften z.B. innerhalb eines Kapitels für LaTeX nicht Teil des Kapitels sind. Auch wenn das in aller Regel kein Problem ist müssen wir deshalb selbst darauf achten, welche Teile unserer Texte zu welchem Kapitel gehören.\\

Hier nun die Hauptstrukturüberschriften in LaTeX:

\begin{itemize}
	\item \verb|\part[Kurztitel]{Titel}| wird in aller Regel nur bei Büchern genutzt. Es handelt sich hier um eine Überschrift, die den Inhalt mehrerer Kapitel zusammenfasst. (Denken Sie an so etwas wie die Unterteilung eines Mathebuchs in \verb|Teil 1 - Algebra|, \verb|Teil 2 - Geometrie| usw.)
	\item \verb|\chapter| entspricht einem Kapitel.
	\item \verb|\section| entspricht einem Abschnitt innerhalb eines Kapitels.
	\begin{itemize}
		\item Um Unterabschnitte zu beginnen, nutzen Sie \verb|\subsection|.
		\item Dann gibt es noch die \verb|\subsubsection| usw.
	\end{itemize}
	\item \verb|\paragraph| ist ein Absatz innerhalb eines Abschnitts, der eine eigene Überschrift erhalten soll. Auch hier können sie mit dem Präfix \verb|sub| wie bei sections eine Priorisierung erstellen. Allerdings werden paragraphs nicht ins Inhaltsverzeichnis mit aufgenommen und auch nicht nummeriert.
\end{itemize}

Übrigens können Sie bei all diesen Typen jeweils in den geschweiften Klammern die Überschrift festlegen, die im Text angezeigt wird.\\

Und in eckigen Klammern können Sie eine Kurzfassung der Überschrift einfügen, die z.B. im Inhaltsverzeichnis angezeigt wird. Wenn Sie dort nichts angeben, wird auch im Inhaltsverzeichnis die vollständige Überschrift übernommen.

\section{Auslagern von Kapiteln}

Je länger Ihr Dokument wird, desto stärker wird der Wunsch werden, einzelne Teile auszulagern, damit Sie etwas übersichtlicher arbeiten können. In HTML war das nur über den Umweg von PHP möglich, in LaTeX ist es einfacher:\\

Hier verwenden Sie \verb|\include{Dateiname}|, wobei der Dateiname auf \verb|.tex| enden muss. Diese Endung wird aber in der include-Anweisung nicht aufgeführt, sondern nur der Dateiname vor dem .tex .

\section{Titelblatt und Glossar}

Bei einigen Dokumentklassen ist ein Titelblatt vorgesehen, bei anderen müssen sie über den Eintrag der Option \verb|titlepage| zur \verb|\documentclass| angeben, dass ein Titelblatt hinzugefügt werden soll.\\

Dass alleine genügt aber noch nicht. Was wir noch brauchen sind die Angaben für das Titelblatt und die Angabe, wo genau das Titelblatt eigefügt werden soll:\\

Für die nötigen Angaben fügen Sie am Anfang des body (also direkt nach \verb|\begin{document}|) die folgenden LaTeX-Tags ein:

\begin{itemize}
	\item \verb|\title{}| enthält den Titel, der auf dem Dokument eingeblendet wird.
	\item \verb|\author{}| enthält den Namen des/der Authoren.
	\item \verb|\date{}| enthält ein Datum.
	\begin{itemize}
		\item Dabei können Sie mit \verb|\date{\today}| automatisch das aktuelle Datum eintragen.
	\end{itemize}
\end{itemize}

Wenn Sie diese Angaben eingetragen haben, können Sie per \verb|\maketitle| an beliebigen Stellen im Dokument ein Titelblatt erzeugen und einfügen lassen.

\section{Glossar}

Um ein Glossar bzw. Stichwortverzeichnis zu generieren müssen Sie leider deutlich mehr Arbeit aufwenden, aber wenn Sie das getan haben, wird genau wie beim Inhaltsverzeichnis ein Verzeichnis für Schlagwörter automatisch generiert und aktualisert.\\

\begin{enumerate}
	\item \verb|\makeindex| muss zur Präambel hinzugefügt werden.
	\item \verb|\usepackage{makeidx}| kann zusätzlich in der Präambel aufgenommen werden, um die Darstellung des Stichwortverzeichnisses anders zu gestalten.
\end{enumerate}

Wenn Sie das erledigt haben, müssen Sie an der Stelle des Dokuments, wo das Stichwortverzeichnis eingefügt werden soll, die folgenden Zeilen einfügen:

\begin{itemize}
	\item \verb|\renewcommand{\indexname}{Stichwortverzeichnis}| legt den Titel des Stichwortverzeichnisses fest. (Es gibt eine Standardbezeichnung.)
	\item \verb|\addcontentsline{toc}{chapter}{Stichwortverzeichnis}| stellt sicher, dass das Stichwortverzeichnis ins Inhaltsverzeichnis aufgenommen wird.
	\item \verb|\printindex| fügt an der Stelle, an der es steht das Stichwortverzeichnis in das Dokument ein.
\end{itemize}

\subsection{Stichwörter und Bezüge festlegen}

Jetzt kommt der Teil, der die eigentliche Arbeit für das Glossar ausmacht: Jeder Begriff, der im Glossar aufgenommen werden soll muss mit \verb|\index{Bezeichnung im Glossar}| aufgenommen werden.\\

Bsp.: Sie wollen einen Verweis auf eine Textpassage festlegen, in der es um die Nutzung von Dampfmaschinen im Allgemeinen geht. Das sähe dann so aus:

\begin{verbatim}
	Bis zu einem dem Autor nicht bekannten und 
	aus Faulheit nicht recherchierten Zeitpunkt 
	waren Dampfmaschinen\index{Dampfmaschine} 
	eine recht weitverbreitete Antriebsart. ...
\end{verbatim}

Wenn Sie dabei Haupt- und Unterstichworter nutzen wollen, dann sieht das so aus: \verb|\index{Hauptstichwort!Unterstichwort}|.\\

Bsp.: Sie schreiben über die Programmierung in C und wollen einen entsprechenden Verweis im Stichwortverzeichnis erzeugen. Das sähe so aus:

\begin{verbatim}
	C\index{Programmiersprache!C} ist eine 
	kompilierte Sprache ...
\end{verbatim}

Doch während das Inhaltsverzeichnis von LaTeX automatisch generiert wird, müssen wir die Generierung des Glossars manuell starten. TeXStudio hat dafür einen Assistenten, den sie per \verb|F12| oder über den entsprechenden Eintrag im Menü \verb|Assistenten| starten können.

\subsection{Weitere Verzeichnisse}

Später werden Sie erfahren, wie Sie Abbildungen und die sogenannten Figures in LaTeX programmieren können. Für diese können Sie ähnlich wie beim Inhaltsverzeichnis Verzeichnisse automatisch generieren lassen. Dazu nutzen Sie \verb|\listoffigures| und \verb|listoftables|.

\section{Listen und Tabellen}

Immer wenn Sie mehrere Einträge gruppieren und/oder sortieren wollen, nutzen Sie sogenannten Listen bzw. Tabellen.\\

Aus HTML kennen Sie die unordered und ordered lists sowie die description lists. Und alle drei werden (wenn auch mit einer etwas anderen Syntax) nahezu identisch in LaTeX umgesetzt:\\

Einzelne Einträge werden durch ein vorangestelltes \verb|\item| gekennzeichnet. Als Option können Sie hier noch Labels vergeben.\\

Alle Listen werden als Umgebung (also mit \verb|\begin{}| und \verb|\end{}|) programmiert. Die Argumente lauten dabei:

\begin{itemize}
	\item itemize (für Listen mit Punkten)
	\item enumerate (für nummerierte Listen)
	\item description (für Glossare)
\end{itemize}

\subsection{Tabellen}

Hier haben wir einen Unterschied gegenüber HTML: Dort ist eine Tabelle ein Rahmen, anhand dessen wir Elemente unsers Dokuments an einer festen Position im Dokument anordnen können. \\

In LaTeX dagegen wird zwischen einem table und einem tabular matter unterschieden. Das was wir umgangssprachlich als Tabelle bezeichnen ist in LaTeX die \textbf{tabular matter}\index{tabular matter}\index{LaTeX!tabular matter}.\\

Die unterschiedlichen Bezeichnungen basieren darauf, dass LaTeX hier den Begriff Tabelle so nutzt wie das bei Schriftsetzern üblich ist. Die differenzieren hier genauer als wir das umgangssprachlich tun.\\

Um klar zwischen umgangssprachlicher Tabelle und den genannten formalen Tabellen bzw. tabellarischen Aufstellungen zu unterscheiden wird hier bei letzteren die englische Bezeichnung gewählt.\\

Hinweis, wenn Sie mehr dazu im Netz recherchieren wollen: Leider ist der Begriff des formular table im Englischen doppelt belegt: Dort wird er häufig für eine formale Eindeckung eines Tisches verwendet. Wenn Sie also nach \verb|formal table| suchen, werden Sie fast ausschließlich Anleitungen für Diener finden. Bei den verbleibenden Einträgen steht in aller Regel nur, dass es sich dabei um etwas anderes handelt, als das, was wir umgangssprachlich als Tabelle bezeichnen. Eine Aussage, die uns im Regelfall gar nicht weiterhilft.

\subsubsection{formal table}

Eine Tabelle (\textbf{formal table}\index{formal table}\index{LaTeX!formal table}) in LaTeX ist dagegen eine Umgebung, die den Titel einer Tabelle und ein Label enthalten muss, auf das wir von anderen Stellen des Dokuments aus verweisen können.\\

Es handelt sich also um etwas, das weitgehend dem \verb|<figcaption>|-Container in HTML entspricht. Es unterscheidet sich allerdings davon, weil ein formal table (\verb|table|-Umgebung) in LaTeX eine Umgebung für beliebige Inhalte außer für Bilder, Videos und Audiodateien wie in HTML ist.\\

Hier ein Beispiel:

\begin{verbatim}
	\begin[wie immer optional: Buchstabe, der die 
	Position des formal table festlegt]{table}
	\caption[Kurztitel]{Titel des formal table}
	\label{Referenz, entspricht einem Anker in HTML}
	...
	(Hier kann z.B. ein tabular matter aber auch 
	beliebiger anderer Inhalt eingefügt werden.)
	...
	\end{table}
\end{verbatim}

Die Position (Optional der table-Umgebung) kann durch einen von vier Buchstaben festgelegt werden:

\begin{enumerate}
	\item \verb|t| (top), also am oberen Rand der aktuellen Seite
	\item \verb|b| (bottom), also am unteren Rand der aktuellen Seite
	\item \verb|h| (here), also an genau der Stelle, wo die Umgebung im Dokument eingefügt wurde.
	\item \verb|p| (page): Bei dieser Option wird die Tabelle auf einer eigenen Seite angezeigt.
\end{enumerate}

\subsubsection{tabular matter}

Dieser Bereich wird ähnlich wie der mathematische Modus nur kurz angeschnitten. Wenn Sie mehr über Tabellen in LaTeX wissen wollen, dann recherchieren Sie dazu bitte im Netz.\\

\textbf{Wichtig}:\\

Eine tabular-Umgebung ist für Texte gedacht. Wenn Sie Formeln in einer Tabelle gruppieren wollen, nutzen Sie bitte die array-Umgebung (siehe nächster Abschnitt).

\begin{verbatim}
\begin{tabular}{ausrichtung1, ausrichtung2, ...}
erste Zeile, erste Spalte & erste Zeile, zweite Spalte & ... \\
zweite Zeile, erste Spalte & zweite Zeile, zweite Spalte & ... \\
...
\end{tabular}
\end{verbatim}

Eine solche Tabelle wird also durch eine \verb|tabular|-Umgebung definiert. Nach dem ersten Paar geschweifter Klammern folgt ein zweites Paar, in dem für jede Spalte die Ausrichtung definiert wird:

\begin{itemize}
	\item \verb|l| linksbündig
	\item \verb|c| zentriert
	\item \verb|r| rechtsbündig
	\item \verb|p{breite}| definiert eine maximale Breite einer Spalte
\end{itemize}

\textbf{Wichtig}:\\

Es gibt im Gegensatz zu HTML keine automatische Anpassung der Breite von Spalten.

\subsubsection{array}

Neben dem tabular matter, der für Texte gedacht ist, gibt es noch die array-Umgebung, die für mathematische Formeln gedacht ist. Diese müssen Sie allerdings zusätzlich per \verb|\usepackage{array}| in der Präambel importieren.\\

Um Missverständnisse zu vermeiden: In einer array-Umgebung brauchen Sie nicht mehr den mathematischen Modus zu aktivieren, denn er ist dort automatisch aktiviert.

\section{figures}

Jetzt kommen wir zu dem, was der \verb|<figcaption>| in HTML entspricht: Eine Umgebung für Bilder in LaTeX. Sie nutzen die \verb|figures|-Umgebung also genauso, wie Sie die \verb|table|-Umgebung für Tabellen und Texte nutzen konnten.\\

Es stellen sich also zwei Fragen:

\begin{itemize}
	\item Warum gibt es die \verb|figures|- und die \verb|table|-Umgebungen?
	\item Wie können wir Bilddateien in LaTeX-Dokumenten einbinden?
\end{itemize}

Die Antwort auf die erste Antwort ist simpel: Es gibt getrennte Verzeichnisse für formal tables und figures. Und diese Verzeichnisse werden aus den jeweiligen Umgebungen automatisch generiert, wenn Sie \verb|\listoffigures| bzw. \verb|\listoftables| verwenden, um an einer Stelle im Dokument das entsprechende Verzeichnis erzeugen zu lassen.

\subsection{Bilddateien in LaTeX}

Bevor wir uns mit der Einbindung von Bilddateien beschäftigen können, müssen wir uns mit dem Thema \verb|pdfLaTeX| und \verb|LaTeX| beschäftigen. Ersteres ist eine Erweiterung, mit der wir pdf-Dokumente aus LaTeX-Dokumenten erzeugen können. Dennoch gibt es bei beiden einen entscheidenden Unterschied:

\begin{itemize}
	\item Wenn wir \verb|pdfLaTeX| verwenden, können wir PNG-, JPG- und PDF-Dateien als Bilddateien einbinden.
	\item Wenn wir dagegen \glqq{}nur\grqq{} \verb|LaTeX| verwenden, können wir ausschließlich EPS-Dateien verwenden.
\end{itemize}

Um überhaupt Bilddateien einbinden zu können, müssen wir die Präambel erweitern: \verb|\usepackage{graphix}|\\

Um eine Bilddatei in unserem Dokument anzeigen zu lassen verwenden wir \verb|\includegraphics{Dateiname ohne Endung}| . Das bedeutet, dass der Compiler automatisch nach einer Datei sucht. Wenn wir also sicherstellen wollen, dass wir ein Dokument sowohl mit \verb|pdfLaTeX| als auch mit \verb|LaTeX| konvertieren können, dann müssen wir die Bilddatei mit gleichem Namen einmal als PNG-, JPG- und PDF-Datei und einmal als EPS-Datei im gleichen Verzeichnis speichern wie das .tex-Dokument.\\

Sie können noch die Breite und Höhe als optionale Argumente \verb|width = ...cm| bzw. \verb|height = ...cm| festlegen. Dabei können Sie auch andere Maße wie z.B. pt verwenden, so lange diese in LaTeX definiert sind.\\

\textbf{Wichtig}:\\

Auch bei gleicher Bezeichnung sind diese Maße nicht mit denen in Adobe-Produkten identisch; leider beharrt besagter Konzern darauf eigene Definitionen einzelner Maßstäbe zu verwenden.\\

Außerdem können Sie noch über Werte wie \verb|.75/coumnwidth| die Größe proportional anpassen. Allerdings bedeutet das nicht, dass (z.B. bei einer JPG-Datei) das Bild gestochen scharf ist. Das ist nur bei einer Vektorgrafik sichergestellt.

\section{Referenzen und Labels}

An ein oder zwei Stellen haben Sie bereits \verb|\label{Text}| gesehen. Das entspricht einem Anker in HTML. Also brauchen wir jetzt noch den \glqq{}Link\grqq{} auf einen solchen Anker. In LaTeX heißen die aber nicht Link, sondern \textbf{Referenz}\index{Referenz}\index{LaTeX!Referenz}. Eine Referenz programmieren Sie mit \verb|ref{Text}|.\\

U.a. wegen Labels empfehle ich bei der Erstellung von LaTeX-Dokumenten von Anfang an die Arbeit mit einem erweiterten Editor wie TeXStudio: Dieser zeigt Ihnen alle im Dokument verwendeten Labels an. Und das ist deshalb wichtig, weil es zu Inkonsistenzen kommen wird, wenn Sie zwei Labels gleich bezeichnen.

\section{Boxen}

Wenn Sie sich eine LaTeX-Referenz ansehen, werden Sie immer wieder über Container bzw. Elemente mit \verb|box| im Namen stolpern. Eine Box bezeichnet so etwas wie einen Bereich, der abgeschlossen ist und Inhalte einer (z.B. gedruckten) Seite enthalten kann. Die größte Box entspricht dabei dem Bereich, der auf einer Seite insgesamt bedruckt werden kann. Generell wird aber das, was wir als einzelne Einheit betrachten als Box bezeichnet.\\

Es gibt beispielsweise die Möglichkeit durch \verb|\fbox{text}| eine Textpassage im laufenden Text mit einem Rahmen zu umgeben.\\

Weitere Boxen, mit denen Rahmen erzeugt werden können sind \verb|\shadowbox|, \verb|\doublebox|, \verb|\ovalbox| und \verb|Ovalbox|.\\

Dann gibt es die \verb|\parbox[pos]{width}{text}|, mit der ein Text am oberen \verb|t| oder unteren \verb|b| Rand einer Seite angezeigt werden kann, die die Breite \verb|width| hat und \verb|text| beinhaltet.\\

Aber auch das waren wieder nur einige ausgewählte Möglichkeiten, um Teile Ihres Dokuments hervorzuheben.

\section{Abschluss}

Damit haben Sie jetzt neben HTML eine weitere Markup Language kennen gelernt. Doch während Sie HTML kaum im Studium nutzen werden, sollten Sie LaTeX so schnell wie möglich verinnerlichen. Es ist für wissenschaftlichen Arbeiten ein international anerkannter Standard.
%\chapter{ML, Teil 2 - Textverarbeitung mit LaTeX}
%\chapter{6.	Programmierung der Gestaltung einer Webpage mittels CSS}
\section{6.1.	Einbindung von CSS in HTML}
\section{6.2.	Programmierung einer CSS-Datei}
\section{6.3.	Programmierung von Formatierungen}
\subsection{6.3.1.	Mehr über Farben}
\subsection{6.3.2.	Zwei weitere Properties in Bezug auf Farben}
\section{6.4.	Ein wenig Form}
\section{6.5.	Definition von CSS für einzelne Container eines Typs}
\section{6.6.	Mehr zu CSS}
Leider beginnt dieses Kapitel mit einer Enttäuschung für all diejenigen, die dachten, jetzt käme der spannende Teil. Denn da dieses Skript für Studierenden mit den Schwerpunkten Informatik bzw. Elektrotechnik im Medienumfeld entworfen wurde, erhalten Sie hier lediglich einen grundlegenden Einblick darin, wie Sie die Elemente in HTML5 gruppieren können. Der Grund ist einfach und wurde auch schon mehrfach aufgeführt: Gestaltung ist die Aufgabe von Designern. Wir gehen deshalb auch nicht auf die Besonderheiten von CSS (bzw. der aktuellen Version CSS3) ein.
Die in diesem Kapitel besprochenen Aspekte werden im Sinne des MVC als View bezeichnet.
Wichtig: Bevor Sie mit diesem Kapitel beginnen sollten Sie eine Webpage programmiert haben, in der Sie eine Vielzahl der HTML-Container einprogrammiert haben, die Sie im letzten Kapitel kennen gelernt haben. Denn die Aufgaben in diesem Kapitel setzen eine solche Seite voraus.
6.1.	Einbindung von CSS in HTML
Sie haben zwei Möglichkeiten, um CSS-Code in einem HTML-Dokument zu verwenden. Zum einen können Sie CSS-Code in Anführungszeichen setzen und ihn dann als Wert dem style-Attribut eines beliebigen Containers übergeben. Dieses Verfahren hat allerdings einen massiven Nachteil: Wenn Sie bei einer umfangreichen Webpage so vorgehen, wird es selbst bei kleinen Änderungen sehr mühsam, diese überall zu programmieren. Außerdem wird Ihr Layout so kaum einheitlich.
Deshalb sollten Sie besser die zweite Variante wählen. Hier programmieren Sie den CSS-Code in einer oder mehreren eigenen Datei/en und binden diese mit folgender Zeile in Ihr HTML-Dokument ein, die Sie in den <head>-Container der HTML-Dokumente programmieren:
<link rel=stylesheet href=style.css>
Die Datei kann natürlich auch anders als style.css heißen, wichtig ist nur, dass die Endung .css ist. Außerdem gilt in Bezug auf die URL dasselbe, was Sie schon im letzten Kapitel über relative und absolute Adressierung gelernt haben.
6.2.	Programmierung einer CSS-Datei
Im Gegensatz zu einem HTML-Dokument besteht eine CSS-Datei lediglich aus den Namen von Containern eines HTML-Dokuments und jeweils einem Rumpf, in dem die Formatierungsvorgaben einprogrammiert werden.
Wenn Sie CSS wie oben beschrieben innerhalb eines HTML-Dokuments über das style-Attribut programmieren, dann stehen in den Anführungszeichen genau die Formatierungsvorgaben, die im Rumpf bei der CSS-Datei stehen.
Nehmen wir an, Sie wollen eine Formatierungsvorschrift für alle <h1>-Container einzelner HTML-Dokumente festlegen, haben sich aber noch auf keine Formatierungsvorschrift festgelegt. Dann sieht Ihre CSS-Datei so aus:
h1 {
	
}
Quellcode 3.1: CSS-Rumpf für den <h1>-Container
In Dokumentationen über CSS werden Sie des Öfteren auf den Begriff des Selektors stoßen. Der Selektor ist in CSS schlicht der Bezeichner des Containers, den Sie formatieren wollen.
Damit wissen Sie jetzt ungefähr 50% dessen, was Sie im Rahmen dieses Kurses für die Programmierung von CSS wissen müssen. Alles was Sie noch lernen müssen ist wie Sie den sogenannten Properties Werte zuordnen müssen, um die Gestaltung der Container festzulegen und was es mit dem . (Punkt) und # vor Selektoren in CSS auf sich hat.
6.3.	Programmierung von Formatierungen
Schauen wir uns an, was wir nun in den Rumpf programmieren. Sie kennen ja die Attribute aus HTML. Bei denen konnte ein Wert mit einem Gleichzeichen zugeordnet werden. Bei CSS gibt es dazu zwei Unterschiede: Hier reden wir nicht von Attributen sondern von Properties und die Zuordnung eines Wertes wird durch einen Doppelpunkts und nicht durch ein Gleichzeichen durchgeführt. Die Fortgeschrittenen unter Ihnen haben den Begriff Property schon im Zusammenhang mit Objekten im Sinne der objektorientierten Programmierung gehört. CSS hat nichts mit objektorientierter Programmierung zu tun und deshalb reden wir hier nicht von Objekten, sondern von Elementen, deren Eigenschaften (also Properties) wir ändern.
Und welche Eigenschaft können wir fast überall ändern? Genau: Die Schriftfarbe. Und wie machen wir das? Wir nehmen die Eigenschaft Schriftfarbe, die naheliegender Weise color heißt und ordnen Ihr einen Wert zu. Fangen wir mit so etwas profanem wie rot an. Wenn Sie also wollen, dass alle Überschriften in einem schreienden Rot angezeigt werden, müssen Sie nur das folgende in Ihre CSS-Datei programmieren: 
h1 {
	color : red;
}
Quellcode 3.2: Rote Überschrift
Wenn Sie die CSS-Datei wie oben beschrieben in Ihr HTML-Dokument eingebunden haben und Sie im gleichen Verzeichnis gespeichert haben wie das HTML-Dokument, dann müssen sie nur noch den Browser aktualisieren (z.B. per F5 beim Firefox) und schon werden alle Überschriften in rot angezeigt. Und wenn Sie sonst nichts in der CSS-Datei geändert haben, dann ändert sich auch sonst nichts.
6.3.1.	Mehr über Farben
Anstelle von vielen englischen Wörtern für Farben, können Sie auch noch präzise Farbangaben zum Beispiel in der RGB-Notation programmieren.
An dieser Stelle werden wir Themen wie RGB u.a. nicht ausführliche behandeln. Hier werden Sie nur erfahren, wie Sie diese Farbangaben in einem CSS-Dokument einprogrammieren können:
-	RGB-Farben: Diese Farbangaben bestehen aus drei Zahlen, jeweils von 0 bis 255 (oder von 0 bis FF als Hexadezimalzahl). Um sie in CSS zu programmieren, können sie auch als Hexadezimalwerte vorliegen.

•	Variante a: Die Zahlen liegen als Hexadezimalzahlen vor. Nehmen wir an, es sind die Zahlen 1F, 9, 27. Dann lautet die Formatierungsvorschrift in CSS: 
color : \#1f0927;
Sie schreiben also zunächst ein \#-Symbol, um zu zeigen, dass ein RGB-Wert in Hexadezimalzahlen folgt. Danach folgen die drei Zahlen ohne Leerstelle, wobei Sie für jede einstellige Zahl (wie hier die 9) noch eine führende Null ergänzen müssen.

•	Variante b: Die Zahlen liegen als Dezimalzahlen vor. Nehmen wir an, es sind die Zahlen 128, 92, 7. Dann lautet die Anweisung:
color: rgb(128, 92, 7);

-	Es gibt außerdem Fälle, in denen der Alpha-Wert (Opacity) der Farbe angegben ist. Das ist ein Wert von 0 bis 1, der die Transparenz definiert. Nehmen wir an, Alpha wäre bei 0.3 und die Farbe ist die selbe wie bei Variante b. Dann lautet die Anweisung:
color: rgba(128, 92, 7, 0.3);

Das ist eine Kurzfassung der folgenden Anweisung:
color : rgb(128, 92, 7); opacity(0.3); 
Ist es Ihnen aufgefallen? Bei der letzten Anweisung wurde zweimal ein Semikolon gesetzt, um jeweils eine Formatierung zu beenden, obwohl beide für die Property color gelten. Das ist sehr wichtig: Sie können einer Property mehrere Formatierungen zuordnen, aber jede einzelne muss durch ein Semikolon beendet werden.
Außerdem wird hier die englische Schreibweise für die Dezimaltrennung verwendet: Dort verwenden Sie kein Komma, sondern einen Punkt, um zwischen ganzen Zahlen und „Nachkommastellen“ zu trennen.
Für den Fall, dass Sie unsicher sind, folgt hier ein aktualisierter Quellcode:
h1 {
	color : rgb(128, 92, 7); opacity(0.3);
}
Quellcode 3.3: Eine Überschrift mit einem abgedunkelten Farbton.
Alternativ zu RGB können Sie auch HSL-Farben programmieren. Aber das lassen wir an dieser Stelle außen vor, weil Designer im Regelfall Farbwerte in mehreren Standards angeben.
6.3.2.	Zwei weitere Properties in Bezug auf Farben
Hier folgen noch ein paar Properties, mit denen Sie verschiedene Bereiche Ihrer Webpage farblich anpassen können:
-	background: Damit färben Sie den Hintergrund eines Elements. Wenn Sie hier anstelle einer Farbe ein Bild einstellen wollen, z.B. image.jpg, dann können Sie es mittels 
background : url(image.jpg); 
als Hintergrund des Elements festlegen.

-	mark ist keine Property, sondern ein HTML-Container, mit dem Sie Textpassagen hervorheben können. Schlagen Sie im Zweifelsfall nochmal im HTML-Kapitel nach.
Aufgaben:
-	Öffnen Sie die Webpage http://www.colorpicker.com/ und wählen Sie dort eine Farbe aus, die Sie für Ihre Überschriften verwenden wollen. Am oberen Rand finden Sie den hexadezimalen Wert, den Sie einfach kopieren können, um ihn im CSS-Code zu verwenden.

-	Vergeben Sie jetzt für die verschiedenen Überschriften Ihrer Webpage einen jeweils etwas helleren Farbton.

-	Wählen Sie nun einen anderen Farbton aus, in dem Sie Absätze anzeigen lassen wollen.

-	Stellen Sie zusätzlich einen Farbton für den Hintergrund jeder Überschrift und jedes Absatzes (das sind die <p>-Container) ein.
Vermutlich sieht Ihre Webpage jetzt alles andere als augenfreundlich aus, aber das macht nichts. Einzig wenn Sie merken, dass Texte gar nicht mehr zu lesen sind, sollten Sie die Einstellungen anpassen.
6.4.	Ein wenig Form
Wie Sie sehen sind die Hintergründe der Absätze Ihrer Texte und der Überschriften im Moment rechteckig. Das sieht nicht wirklich schön aus, da wären abgerundete Ecken schöner. Also kümmern wir uns darum.
Das Prinzip fürs Abrunden von Ecken funktioniert wie folgt: Sie legen einen Radius fest, der in Bildpunkten (Pixel, kurz px) gemessen wird und ordnen diesen Wert der Property border-radius zu. Meist ist ein Wert zwischen 5 und 25 gut, aber Sie haben hier die freie Auswahl:
border-radius : 20px;
Manchmal finden Sie auch Quellcode, in dem border-radius mit bis zu vier Werten verwendet wird. Schlagen Sie ggf. bei w3schools nach, wie diese Angaben zu verstehen sind.
Unter Umständen überschneidet Ihr Text jetzt die abgerundeten Ecken. Da können Sie sich mit der Property padding behelfen. Padding legt einen Abstand vom Rand fest, den der Text haben muss.
padding : 5 px;
Sie können auch die Größe festlegen, die die farbige Fläche im Hintergrund haben soll. Dazu verwenden Sie die beiden Properties width und height, die trivialerweise festlegen, welche Breite und Höhe der Hintergrund haben soll. Allerdings könnten Sie hier eine Festlegung treffen, die bei kleinen Displays zu Problemen in der Darstellung oder der Bedienung führt.
6.5.	Definition von CSS für einzelne Container eines Typs
Stellen Sie sich vor, Sie wollen jetzt noch die Formatierung von einigen <p>-Containern anpassen, die aber z.B. nur dann angewendet werden soll, wenn in dem Container Quellcode angezeigt werden soll. Da es in HTML (auch in Version 5) keine solchen Spezialcontainer gibt, hätten Sie mit Ihrem bisherigen Wissen jetzt ein Problem.
Für die Lösung dieses Problems gibt es zwei Möglichkeiten:
-	Zum einen können Sie über das class-Attribut einzelne Container in HTML so hervorheben, dass Sie diesen durch ein entsprechendes CSS-Skript formatieren lassen.

Bsp.: In HTML haben Sie mehrere Absätze mit dem Attribut class=PHP programmiert: 

<p class=PHP>

Jetzt können Sie in CSS einen entsprechenden Teil programmieren:

.PHP {
	color : blue;
	font-family : monospace;
}

Damit werden zusätzlich zu allen CSS-Anweisungen, die für <p>-Container gelten bei den <p class=PHP>-Containern die Schrift in blau und einem etwas veränderten Textformat angezeigt.

-	Zum anderen können Sie aber auch anhand der Bezeichner von id-Containern CSS-Skripte definieren. Anstelle des Punktes (wie bei class) wird hier ein \# verwendet.

Bsp.: In HTML haben Sie einen Container mit dem Attribut id=nutzer programmiert.
In CSS können Sie diesen wie folgt formatieren:

\#nutzer \{ ... \}
6.6.	Mehr zu CSS
Dieses Kapitel ist sehr kurz ausgefallen, weil Sie sich im Rahmen dieser Veranstaltung auf gute Programmiertechniken konzentrieren sollen und CSS eine Sprache für Mediendesigner ist, um das Design in den Dokumenten einer Webanwendung festzulegen: Sie entscheiden sich dafür, wie Sie einen Container gestalten wollen und programmieren das schlicht Zeile für Zeile in einer CSS-Datei. Das ist Programmieren für Sechstklässler und hat nichts aber auch nicht das Geringste mit Medieninformatik zu tun.
Deshalb hier Ihre Hausaufgaben:
-	Sehen Sie sich auf http://www.w3schools.com/css/css3\_intro.asp um, was es noch für Gestaltungsmöglichkeiten in CSS3 gibt und setzen Sie auf Ihrer Webpage alles um, was Sie gestalterisch spannend finden.

-	Programmieren Sie einige HTML-Container über die entsprechenden Attribute, sodass diese für Administratoren bzw. eingeloggte Nutzer eine andere Darstellung erhalten können. Momentan können die entsprechenden CSS-Container noch leer bleiben.


\chapter{Gestaltung mit CSS}
%\chapter[Dynamische Webentwicklung]{Dynamische Web(anwendungs)entwicklung mit PHP}

Ab diesem Kapitel setzen wir uns mit der Programmierung des Controllers im Sinne des MVC-Patterns auseinander, also des Teils eines Projekts, der steuert, welche Inhalte wann angezeigt werden.\\


An dieser Stelle muss der Unterschied zwischen serverseitig und clientseitig besprochen werden. Wie Sie wissen, beschäftigen wir uns in diesem Kurs mit den Grundlagen der Entwicklung von Webseiten bzw. Webanwendungen. Bei allen Anwendungen, die ein Netzwerk nutzen wird von Server und Client gesprochen: Der Server bietet einen Dienst an, der Client fordert die Ausführung dieses Dienstes an. Somit kann jedes Programm sowohl Client als auch Server sein.\\


Bislang haben wir konzeptionell im Grunde nur über das MVC-Pattern gesprochen, also die Unterteilung der Programmierung eines Projektes in die Teile Modell, Ansicht und Controller. Das ist ein Ansatz, der aus der Erfahrung bei objektorientierten Projekten entstanden ist. Bei netzbasierten Projekten wird unabhängig davon von der Unterteilung in die serverseitige und die clientseitige Programmierung gesprochen.\\


Im einfachsten Fall geht es also um die Unterscheidung der Programmteile, die die angebotenen Dienste darstellen von den Programmteilen, die diese Dienste anfordern. Hier kann aber auch danach unterteilt werden, welche Daten vom Server und welche vom Client erzeugt werden. Leider gibt es hier keine einfache entweder/oder-Entscheidung. Im Grunde können die meisten Komponenten sowohl auf dem Server als auch auf dem Client betrieben werden. Wie Sie hier erkennen können, werden die Begriffe Server und Client also sowohl für Programme als auch für vernetzte Computer benutzt.\\


Nehmen wir dafür CSS als Beispiel: Wir haben hier mehrere HTML-\\Dokumenten und vielleicht auch mehrere CSS-Dateien. Ein Server könnte nun an allen nötigen Stellen den CSS-Code mittels des style-Attributs in die HTML-Dokumente integrieren und dann dieses Ergebnis an einen Client übertragen. Er könnte aber auch alle Dateien so wie sie sind an den Client übertragen. Für den Nutzer, der am Client sitzt macht das keinen Unterschied, denn er bekommt in beiden Fällen das gleiche angezeigt. Für den Entwickler der Webpage kann es dagegen einen Unterschied machen. Und für den Serveradministrator macht es in jedem Fall einen Unterschied. Keine Sorge, denn wie das im Detail abläuft ist für uns gänzlich irrelevant, damit dürfen sich Informatik-Studierende im Master-Studium beschäftigen.\\


Es soll Ihnen lediglich deutlich machen, dass die Ausführung des programmierten Codes nicht immer auf dieselbe Art erfolgen muss. Das ist immer nur dann der Fall, wenn wir mit prozeduralen Programmiersprachen arbeiten. Wichtig für Sie ist, dass den Begriff der serverseitigen Programmierung kennen lernen, dann bei der Programmierung in PHP tun wir genau das: Wir programmieren ein Programm, dass auf dem Server laufen wird und das steuern wird, wie die Inhalte unserer HTML-Seiten geändert werden: Wenn ein Nutzer z.B. Eingaben in Formularen einträgt und dann über eine Schaltfläche, die Sie mit <input type=submit> programmiert haben, die Übertragung an das Programm startet, dann wird PHP (oder eine andere Programmiersprache) interessant.\\


Im Rahmen dieses Kurses brauchen Sie sich allerdings über die Installation und Wartung eines Servers keine Gedanken zu machen. Alles, womit wir uns hier beschäftigen können Sie auf Ihrem Rechner betreiben. Ggf. nötige Programme, die einen Server auf Ihrem Rechner betreiben bekommen Sie kostenlos im Netz.\\


Umgekehrt möchte ich Sie allerdings davor warnen, den Aufwand beim Betrieb eines Servers zu unterschätzen. Denn hier geht es nicht allein um das Anmieten bei einem Betreiber und um eine grundlegende Konfiguration. Vielmehr müssen Sie bei einem solchen Server sicherstellen, dass dieser so gut wie möglich gegen Angriffe geschützt ist. Denn wenn Sie hier nicht ausreichende Vorkehrungen treffen, können Sie sicher sein, dass Ihr Server von Angreifern missbraucht wird, um Angriffe gegen andere Nutzer zu starten. Im schlimmsten Fall kann das für Sie bedeuten, wegen Beihilfe zu einer Straftat verurteilt zu werden.\\

\textbf{Wichtig:} Dieses Kapitel ist eine einfache Einführung in die imperative Programmierung mit PHP. Zum Teil werden Sie hier Anmerkungen finden, die klar aufzeigen, dass vieles hier vereinfacht dargestellt ist. Insbesonder werden hier viele Dinge, die in PHP nicht möglich sind außen vor gelassen, obwohl es sich dabei um Möglichkeiten aus dem Bereich der imperativen Programmierung handelt. Die Struktur dieses Kapitels ist aber ansonsten gleich derjenigen, die im zweiten Teil dieses Buches eingeführt wird. Wenn Sie also die Grundlagen der imperativen Programmierung erlernen wollen, dann blättern Sie gleich weiter in den zweiten Teil dieses Buches. Und immer, wenn Sie dort einen Abschnitt beendet haben, schlagen Sie in diesem Kapitel nach, wie der entsprechende Abschnitt in PHP umgesetzt wird. So haben Sie die Grundlagen der imperativen Programmierung und erlernen, wie diese in PHP umgesetzt werden, bzw. welche dieser Grundlagen in PHP nicht enthalten sind.

\section{PHP 5}

Bis auf die Zahl der Version haben PHP und HTML nichts gemein. Während HTML5 nach 14 Jahren eine vollständig überarbeitete Version der Markup Language ist, wurde PHP in der Version 5.0 vor 11 Jahren veröffentlicht und seitdem immer wieder erweitert. Die letzte Version trug die Nummer 5.6. Die aktuellste Version ist nicht PHP 6 sondern 7 und im November 2015 erschienen.\\

Wie an der Überschrift zu erkennen ist, wird in diesem Kapitel aber noch die Version 5, besser gesagt die Version 5.6 behandelt.\\

Doch auch wenn es eine neue Version geben wird, werden wir bei diesem Kurs für die Medientechnik bis auf weiteres bei PHP 5.6 bleiben. Wobei alles, was Sie hier kennen lernen auch in den PHP-Versionen 5.0 – 5.5 nutzbar sein sollte.\\


Doch was ist PHP eigentlich? Schauen wir uns dazu den Namen an: PHP steht für Hypertext Preprocessor. Was Hypertext ist wissen Sie, aber über Präprozessoren haben wir noch nicht gesprochen. Das sind Programme, deren Zweck darin besteht, Teile eines Programms vorzubereiten, die in einer (anderen) Programmiersprache entwickelt werden. Wenn Sie das zusammennehmen ist PHP schlicht eine Sprache, die HTML-Code erzeugen kann. Und genau das wollen wir: Eine Sprache, mit der wir (neue) Inhalte für unsere Webpage erzeugen können.\\


Vielleicht fragen Sie jetzt, was denn mit CSS-Dateien ist. Und sie haben es richtig erkannt: Mit PHP können wir zwar CSS-Code über das style-Attribut in einen Container integrieren, aber an unseren CSS-Dateien können wir nichts direkt ändern. Andererseits ist das kein Problem, wenn Sie sauber zwischen Controller und View trennen: In PHP programmieren wir, welche Inhalte wann dargestellt werden. In CSS (und nur dort) programmieren wir, wie diese Inhalte dargestellt werden.\\


Sehen wir uns nun an, was Thomas Theis in seinem Band \glqq{}Einstieg in PHP 5.6 und MySQL 5.6\grqq{} über die Vorteile von PHP schreibt:

\begin{itemize}
	
	\item Demnach ist PHP für Einsteiger leicht erlernbar, weil es nur all das beinhalten würde, was für die serverseitige Programmierung einer Webanwendung nötig ist.
	
	Die Vorstellung, dass es Einsteigern leichter fällt, eine Sprache zu erlernen, die weniger Möglichkeiten bietet, als das bei anderen Sprachen der Fall ist, ist absurd. Es mag sein, dass die vollständige Dokumentation einer solchen Sprache übersichtlicher ist, aber letztlich hängt die Leichtigkeit des Einstiegs davon ab, wie gut die Dokumentation zur Sprache auf Einsteiger ausgerichtet ist. Und in der professionellen Softwareentwicklung spielt ein leichter Einstieg in eine Sprache eher eine untergeordnete Rolle. Hier ist es vor allem wichtig, welche Möglichkeiten genutzt werden können.
	
	\item PHP sei nach Theis eine gute Wahl, weil es von einer Vielzahl von Servern unterstützt wird.
	
	Doch auch das ist eher ein schwaches Argument; letztlich ist der Preis relevant, den wir für den Betrieb aufwenden müssen. Hier ist beispielsweise JavaScript mit node.js aus Gründen im Nachteil, die daraus resultieren, dass es mehr Zugriffsrechte benötigt, was die Betreiber von Servern sich eben entsprechend bezahlen lassen.
	
	\item Weiter sei die Tatsache wichtig, dass PHP kostenlos ist.
	
	Das ist zwar eine sympathische Tatsache, aber kostenlos nutzbare Sprachen und sogar Frameworks sind heute keine Ausnahme mehr. Egal ob Java EE, Ruby on Rails, HTML5 mit CSS3 und JavaScript oder woran Sie noch denken: Alle können kostenlos genutzt werden und bieten einen vergleichbaren, wenn nicht sogar größeren Umfang.
	
	\item Dann sei es ein Vorteil, dass PHP ausschließlich serverseitig eingesetzt werden kann.
	
	Worin der Nachteil besteht, eine Sprache sowohl server- als auch clientseitig ausführen zu können, erschließt sich dem Autor dieses Buches nicht. So bedeutet die Tatsache, dass eine Sprache auch clientseitig ausgeführt werden kann, eben nicht, dass sie vollständig dort ausgeführt wird. Und damit kann ein Nutzer eben auch nicht wie von Theis behauptet den gesamten Code einer Webanwendung einsehen. Im Gegenteil ist diese Beschränkung von PHP von Nachteil: Wenn bei der Entwicklung festgestellt wird, dass ein Programmteil effizienter vom Client ausgeführt werden kann, ohne dabei sicherheitsrelevante Daten preiszugeben, dann sollte ein Entwickler das tun. Genau das verhindert aber PHP mit seiner Fixierung auf die Serverseitigkeit.
\end{itemize}

Aus meiner Sicht gibt es nur einen Grund, sich mit PHP zu beschäftigen: Es ist schlicht eine der häufigsten Sprachen, die bei der Entwicklung dynamischer Webanwendungen zum Einsatz kommen. Wenn Sie also in die Webentwicklung gehen wollen, haben Sie mit Kenntnissen in diesem Bereich mehr Auswahl unter potentiellen Arbeitgebern.\\

Andererseits ist das der aktuelle Stand. Und der ändert sich in kaum einem Bereich so schnell wie in der Informatik. JavaScript hat aus Sicht dieses Autors zurzeit wesentlich mehr Potential, um zur meistverwendeten Sprache für die Entwicklung von Webanwendungen zu werden. PHP ist mit seinen Beschränkungen dagegen einfach nicht universell genug einsetzbar. Das könnte sich mit der Version 7 natürlich geändert haben.

\subsection{Vorbereitung des Computers, um in PHP programmieren zu können.}

Während Sie HTML- und CSS-Dateien einfach mit Hilfe eines Browsers ausführen konnten, ist das bei PHP leider nicht der Fall. Wie gerade beschrieben müssen Sie einen Server betreiben, der eine Laufzeitumgebung anbietet, die PHP ausführen kann.\\


Hier gibt es zwei Standardlösungen, die Sie kostenlos im Netz herunterladen können und die Ihnen die gesamte Konfiguration abnehmen: XAMPP und EasyPHP. Letzteres ist aus meiner Sicht deutlich komfortabler, aber im Grunde ist nur wichtig, dass Sie eines der beiden installieren. Nach der Installation müssen Sie dann Ihre Webanwendungsdateien (also die HTML-, CSS- und PHP-Dateien) in dem Verzeichnis speichern, auf das der Server zugreift. Welches das ist, müssen Sie anhand der Dokumentation des Pakets prüfen, das Sie nutzen.

\section{PHP-Programme in HTML-Dokumenten integrieren}

Genau wie bei CSS und JavaScript können wir innerhalb eines HTML-Dokuments PHP-Code ausführen lassen. Allerdings gibt es hier einen wichtigen Unterschied:

\begin{itemize}
	\item Der Dateiname eines HTML-Dokuments, das das tut, sollte in \verb|.php| umbenannt werden. Richtig! Die Datei sollte dann nicht mehr auf  \verb|.hmtl| enden. Der Unterschied ist in sofern wichtig, als Sie bei der Bearbeitung eines Projekts so schneller die relevanten Dateien finden können.
	
	Um PHP-Code direkt in HTML-Code zu programmieren, muss ein besonderer Container verwendet werden. Er zeichnet sich dadurch aus, dass der gesamte PHP-Code im öffnenden Tag steht und es kein schließendes Tag gibt. (Zur Erinnerung: Das war bei JavaScript anders. Dort wurde ein \verb|<script>|-Container verwendet, dessen Inhalt der JavaScript-Code ist.)

	Der PHP-Container sieht so aus:
	\begin{verbatim}
	<?php
	
	?>	
	\end{verbatim}
	Quellcode 4.1: Ein HTML-Container, in dem PHP-Code programmiert werden kann.
\end{itemize}

Wenn wir JavaScript- oder CSS-Dateien haben, dann können wir diese auch über den \verb|<head>|-Container einbinden. Das ist bei PHP leider nicht der Fall. Dort gibt es zwei andere Möglichkeiten:

\begin{itemize}
	\item Zum einen können wir PHP-Code, der im laufenden HTML-Code eingefügt werden soll in eine Datei speichern. Diese können wir dann innerhalb des \verb|<?php ... ?>|-Containers durch den Befehl\\ \verb|include()| aufrufen:
	
	\begin{verbatim}
	<?php
	include(phpdatei.php);
	?>	
	\end{verbatim}
	Quellcode 4.2.: Einfügen von PHP-Code in ein HTML-Dokument

	\item Zum anderen müssen wir bei Formularen (wenn wir sie per PHP auswerten wollen) über das action-Attribut des \verb|<form>|-Containers explizit den Dateinamen zuordnen. Das bewirkt, dass beim Anwählen des submit-Buttons durch einen Nutzer die Eingaben an das genannte PHP-Programm übergeben werden.

	In diesem Fall müssen wir noch die Übertragungsmethode festlegen. Dazu programmieren wir das \verb|method|-Attribut des \verb|<form>|-\\Containers. Es gibt zwar zwei Möglichkeiten, welchen Wert Sie method zuordnen, Sie sollten hier aber immer den Wert \verb|POST| wählen. Die Alternative zu POST lautet GET.
\end{itemize}

\paragraph{Aufgabe:}

\begin{itemize}
	\item Erstellen Sie jetzt für jedes Formular, das Sie in Ihrer Webpage programmiert haben eine Datei mit einem Namen, der zu dem Formular passt. Die Endung muss zwar nicht auf \verb|.php| lauten, allerdings lässt sich so bei größeren Projekten leichter der Überblick bewahren.

	\item Ergänzen Sie Ihre Formulare so, dass bei jedem Formular die Nutzereingaben an das passende PHP-Programm übergeben werden.
\end{itemize}

\textbf{Anm.:} Wenn Sie danach die Webpage neu starten (vorausgesetzt, Sie haben die Dateien zuvor in das Verzeichnis verschoben, in dem XAMPP bzw. \\EasyPHP darauf zugreifen können und jeweils den Server gestartet), und bei einem Formular den Submit-Button anwählen, werden Sie wahrscheinlich eine Fehlermeldung erhalten, denn bislang haben wir ja noch kein PHP-Programm erstellt. Aber das ist vollkommen in Ordnung.\\

\textbf{Wichtig:} Dieser Abschnitts könnte von Ihnen in einer Hinsicht missverstanden werden: Streng genommen wird eben kein PHP-Programm in ein HTML-Dokument integriert, sondern es passiert folgendes: Wann immer der Server in einem HTML-Dokument auf ein PHP-Programm hingewiesen wird startet er einen Prozess, in dem dieses Programm laufen wird. Wann immer nun das PHP-Programm etwas tun soll, dann wird es auf dem Server ausgeführt. Wenn dann HTML-Code durch das PHP-Programm ausgegeben wird, wird dieser Code an der entsprechenden Stelle des HTML-Dokuments eingefügt. Danach wird das vollständige HTML-Dokument\\ vom Server zum Client übertragen. Der Nutzer wird also nie erkennen können, dass bzw. wo die Webpage mit PHP erzeugt wurde. Im nächsten Abschnitt gibts dazu gleich ein Beispiel.

\section{PHP-Code innerhalb von HTML – Erstes Beispiel}

Rufen Sie eine Ihrer Seiten mit einem Webbrowser auf. Es sollte eine Seite mit mehreren Absätzen sein. Öffnen Sie nun das HTML-Dokument mit einem Editor. Gehen Sie hier zu einer beliebigen Zeile, in der sich lediglich Text befindet. Ändern Sie das Dokument dann so ab, wie im folgenden Beispiel und speichern Sie die Datei anschließend wieder. Vergessen Sie dabei bitte nicht die Änderung des Dateinamens. \\

Achten Sie außerdem bei den entsprechenden Seiten darauf, Links in den statischen Webpages zu ändern, die auf Dateien mit der Endung \verb|.htm| bzw. \verb|.html| verweisen.

\begin{verbatim}
<p>
.... Ihr Text ....
</p>
\end{verbatim}

Ändern Sie diesen Teil so ab:

\begin{verbatim}
<p>
<?php echo ".... Ihr Text ...."; ?>
</p>
\end{verbatim}
Quellcode 4.3: Textausgabe mit PHP\\

Aktualisieren Sie jetzt die Seite im Browser, achten Sie dabei aber darauf, dass in der Adresszeile die Endung der Datei nicht mehr \verb|.htm| bzw. \verb|.html| heißt, sondern \verb|.php|. Sehen Sie den Unterschied? Nein? Natürlich nicht, denn der PHP-Teil  tut ja nichts anderes, als genau den HTML-Code ins HTML-Dokument einzufügen, der da vorher schon stand. Das mag relativ sinnfrei wirken, aber Sie werden in den nächsten Abschnitten lernen, dass es Ihnen mit PHP möglich ist, eine von mehreren HTML-Code-Passagen auszuwählen und diese in ein HTML-Dokument einfügen zu lassen. So können Sie später entscheiden, ob ein bestimmter Nutzer Zugriff auf Inhalte bekommt, ob eine bestimmte Nutzerin bestimmte Auswahloptionen anwählen darf usw.\\

Wenn Sie Sonderzeichen per \verb|echo()| ausgeben wollen, also Zeichen, die eigentlich eine Funktion übernehmen, dann schreiben Sie schlich ein \verb|\| davor. (Nicht zu verwechseln mit \verb|/|.) Wenn Sie also Anführungszeichen ausgeben lassen wollen, die ja sonst in \verb|echo()| bewirken, dass die Ausgabe beendet wird, dann schreiben Sie anstelle von \grqq{} einfach \verb|\|\grqq{} und schon wird \grqq{} im HTML-Dokument eingeblendet.

\paragraph{Aufgaben:}

Probieren Sie folgendes selbst aus:

\begin{itemize}
	\item Wie siehts mit Umlauten aus, die Sie mit \verb|echo()| durch PHP erzeugen lassen? Internationalisierung und Lokalisierung betreffen ja nur das HTML-Dokument und nicht das PHP-Programm.
	
	\item Und was ist mit HTML-Tags? Können Sie vollständige HTML-Container per \verb|echo()| erzeugen und in ein HTML-Dokument einfügen?	
\end{itemize}

\subsection{Kommentare}

Bislang haben Sie ausschließlich Zeilen programmiert (egal ob in HTML, CSS oder PHP), die dann vom Computer in irgend einer Form ausgeführt wurden und die für die Anzeige oder den Aufbau der Webanwendung wichtig waren. Kommentare sind etwas anderes: Sie dienen dazu, dass andere Entwickler erkennen können, welche Aufgabe ein Programmteil übernimmt.\\

Deshalb werden Kommentare auch häufig entweder schlecht oder gar nicht verwendet: Während Sie etwas programmieren wissen Sie im Regelfall genau, was Sie tun, also brauchen Sie in diesem Moment keine Kommentare. Aber Sie können sich darauf verlassen, dass Sie früher oder später einzelne Passagen Ihres Projektes nochmal ändern müssen und dann werden Sie die Kommentare brauchen.\\

Vielleicht fragen Sie sich, was das denn soll, da man doch bloß den Code lesen muss, um zu erkennen, was er tut. Wenn Ihnen diese Frage durch den Kopf geht, dann rufen Sie sich bitte in Erinnerung, was eigentlich der Sinn und Zweck des semantic web ist, bzw. was der Unterschied zwischen Syntax und Semantik ist und warum imperative Programme sehr leicht so zu programmieren sind, dass sie etwas anderes tun als Sie wollen.\\

Zur Erinnerung: Eine Programmzeile in einer imperativen Sprache weist den Rechner an, bestimmte Anweisungen auszuführen. Sie sagt dagegen nichts darüber aus, wofür diese Anweisungen dienen, bzw. in welchem Kontext jede einzelne Zeile steht. Wenn Sie also keine Kommentare verfassen oder in Ihren Kommentaren nur steht, was der Quellcode tut, dann werden auch Sie bereits wenige Tage später nicht mehr wissen, welche Aufgabe damit erfüllt wird. Und wenn Sie dann eine Änderung durchführen, ist die Wahrscheinlichkeit sehr groß, dass Sie zwar eine neue Funktionalität realisieren, aber gleichzeitig eine alte Funktionalität entfernen.\\

Hier ein einfacher Algorithmus, um das Problem zu veranschaulichen:
\begin{itemize}
	\item Erzeuge eine Variable mit dem Namen a.
	\item Weise a die Adresse zu, an der der Anfang einer Liste mit elf Einträgen steht.
	\item Erzeuge eine Variable mit dem Namen b.
	\item Erzeuge eine Variable mit dem Namen c.
	\item Erzeuge eine Variable mit dem Namen d.
	\item Erzeuge eine Variable mit dem Namen e.
	\item Erzeuge eine Variable mit dem Namen m.
	\item Erzeuge eine Variable mit dem Namen n.
	\item Erzeuge eine Variable mit dem Namen q.
	\item Tu das folgende zehnmal, wobei du den Durchlauf in der Variable m speicherst.
	\item n bekommt jetzt den Wert 10-m.
	\item Tu das folgende n-mal, wobei du die Nummer des Durchlaufs in der Variable q speicherst:
	\begin{itemize}
		\item Lies den Wert aus der Speicherstelle mit der Nummer von a + q und speichere ihn in c.
		\item Lies den Wert aus der Speicherstelle mit der Nummer von a + q + 1 und speichere ihn in d.
		\item Wenn c größer als d ist, tue folgendes:
		\begin{itemize}
			\item Speichere c in e.
			\item Speichere d in c.
			\item Speichere e in d.
			\item Speichere den Wert von c in der Speicherstelle mit der Nummer a + q.
			\item Speichere den Wert von d in der Speicherstelle mit der Nummer a + q + 1.
		\end{itemize}
		sonst mache einfach weiter. 
	\end{itemize}
\end{itemize}
Beispiel für schlechte Kommentare.\\

Und? Was macht dieses Programm? Wenn Sie jetzt sagen, dass Sie das nicht erkennen können, dann ist das kein Wunder, denn Sie können zwar im Detail nachlesen, was der Computer in jeder Zeile tun soll, aber nirgends steht, welchen Zweck das ganze erfüllen soll. Dieser Zweck dagegen ist recht simpel:\\

Dieses Programm sortiert die Zahlen einer Liste mit elf Einträgen nach der Größe der Zahlen.\\

Beispiel für einen guten Kommentar.\\

Das Beispiel oben zeigt, wie schlechte Kommentare aussehen: Es sind Unmengen an Zeilen, die nur in anderen Worten wiedergeben, was im Quellcode steht. Ein guter Kommentar fasst dagegen zusammen, warum das passiert und ggf. warum das so passiert. Letzteres ist dann wichtig, wenn Ihr Programmteil nicht nur einen einzelnen Zweck erfüllt, sondern zusätzlich noch bestimmte Sonderfälle abfangen muss.\\

Jetzt wissen Sie auch, warum wir bei HTML nicht über Kommentare gesprochen haben: Wenn sie guten HTML5-Code entwickeln, ist für jeden HTML-Entwickler sofort erkennbar, warum Sie die einzelnen Container programmiert haben. Gerade mit den Möglichkeiten des semantic web bleiben hier kaum Fragen offen.\\

Bei den meisten Programmiersprachen ist das anders: Dort müssen Sie Kommentare verwenden, um zu erklären, welchem Zweck einzelne Programmteile dienen. In der Praxis fehlen Kommentare aber häufig. Gerade wenn Sie ein Projekt in einem Unternehmen übernehmen sollen, sind Sie deshalb in einer üblichen aber dennoch unschönen Situation.\\

Es gibt zwei Möglichkeiten, um Kommentare in PHP zu programmieren: Entweder als Ergänzung zu einer Programmzeile oder als einen mehrzeiligen Absatz.\\

Alles, was in einer Programmzeile nach \verb|//| steht, ist ein Kommentar. Die nächste Zeile ist dann wieder normaler Programmcode.\\

Einen mehrzeiligen Kommentar beginnen Sie mit \verb|/*| und beenden ihn nach beliebig vielen Zeilen und Zeichen mit \verb|*/|.\\

Wichtig: Alles, was als Kommentar gilt wird nicht ausgeführt. Wenn Sie also einen mehrzeiligen Kommentar erstellen, achten Sie bitte darauf, keine auszuführenden Programmzeilen in diesen Kommentar einzufügen. Andererseits können Sie \verb|/* ... */| nutzen, um Teile des Quellcodes auszukommentieren. Sie können dadurch prüfen, wie der Code ausgeführt wird, wenn diese Zeilen entfallen.

\section{Überarbeitung: Grundlagen der Programmierung in PHP}

In den Kapiteln zu HTML und CSS haben Sie gelernt, wie Sie Bereiche einer Webanwendung programmieren können und wie Sie diese gestalten können. Jetzt kommen wir dazu, was wir mittels Programmierung in PHP tun müssen, um die Inhalte einer Webpage dynamisch (also zu beliebigen Zeitpunkten) ändern zu können.\\

Die folgenden Grundlagen finden Sie in ähnlicher Form in einer Vielzahl von Programmiersprachen, die zum imperativen Paradigma gehören. Wie schon in den ersten Kapiteln erklärt denken deshalb viele Programmierer irrtümlicherweise, dass sie wissen, was Programmieren ist, und dass es ausreichen würde, immer nur die Syntax einer Sprache zu erlernen, um in dieser (neuen) Sprache zu programmieren. Aber wie Sie dort lesen konnten ist das schlicht und ergreifend falsch und wird auch nicht dadurch richtig, dass es immer wieder falsch wiederholt wird.\\

Stellvertretend möchte ich hier die Publikationen des heise-Verlags nennen: Dort wird nahezu durchgehend das Konzept der imperativen Programmierung als das einzige Konzept der Programmierung verklärt. Daraus folgen das Aussagen wie diejenige, dass es im Grunde genügen würde eine der dort vorgestellten Sprachen zu lernen, um alle Programmiersprachen nutzen zu können. (Sonderheft \glqq{}Programmieren\grqq{} 2012) Wie schon erklärt ist das so abwegig, als würden Sie denken, dass ein Führerschein für PkWs genügen würde, um eine Planierraupe sicher zu fahren. Was dabei herauskommt können Sie in Serien wie \glqq{}Hör mal wer da hämmert\grqq{} sehen.\\

Um die Grundlagen der imperativen Programmierung und damit z.B. die Programmierung in PHP zu erlernen, stelle ich Ihnen sechs Themen vor: Variablen, Operationen, Kontrollstrukturen, Funktionen, Datenstrukturen und abschließend Schleifen bzw. Rekursionen:

\begin{itemize}
	\item \textbf{Variablen}\index{Variable} sind die Teile eines imperativen Programms, in denen Daten gespeichert werden.
	
	\textbf{Wichtig:} Im Gegensatz zur Mathematik und z.B. zu Programmiersprachen, die das funktionale Paradigma umsetzen, darf der Wert einer Variablen bei prozeduralen Programmiersprachen wie PHP \\während des Ablaufs eines Programms geändert werden.
	
	\item \textbf{Operationen}\index{Operation} sind einfache Anweisungen, mit denen die Inhalte von Variablen geändert werden können oder durch die neue Variablen erzeugt werden können. 
\end{itemize}
	
Im Grunde könnten wir also mit Variablen und Operationen bereits Programme erstellen. Besonders spannend wären diese Programme aber nicht.
	
\begin{itemize}	
	\item \textbf{Kontrollstrukturen}\index{Kontrollstruktur} ermöglichen es Ihnen, zu kontrollieren, unter welchen Bedingungen welcher Teil eines Programms ausgeführt wird.
	
	\item \textbf{Funktionen}\index{Funktion}\index{PHP!Funktion} fassen mehrere Operationen zusammen, um komplexe Abläufe als eine Einheit nutzen zu können. Da Funktionen einen Bezeichner haben, über den Sie aufgerufen (\glqq{}gestartet\grqq{}) werden, können Sie sie an verschiedenen Stellen eines Programms ausführen lassen, ohne sie jedes Mal komplett neu programmieren zu müssen.
	
	\item Funktionen ähneln sehr den sogenannten \textbf{Prozeduren}\index{Prozedur} und werden in vielen Sprachen auch nicht von diesen unterschieden. Der Unterschied ist der, dass eine Funktion einen Rückgabewert hat, dass also durch den Aufruf einer Funktion (evtl. mit einem oder mehreren Werten) ein Wert erzeugt wird. Im Gegensatz dazu ist eine Prozedur schlicht eine Abfolge von Operationen, die nicht dazu dienen, einen Wert zu berechnen, der als Ergebnis der Prozedur zurück gegeben wird. In PHP unterscheiden wir nicht zwischen Funktionen und Prozeduren.
	
	\item \textbf{Datenstrukturen}\index{Datenstruktur} fassen mehrere Variablen zusammen, die gleichartig sind. Und so, wie wir mit einer Funktion mehr Übersicht in unseren Programmablauf bekommen, helfen uns Datenstrukturen, wenn wir mit vielen Variablen arbeiten wollen oder müssen.
	
	\item \textbf{Schleifen}\index{Schleife} und \textbf{Rekursionen}\index{Rekursion} ähneln Prozeduren und Funktionen. Der Unterschied besteht darin, dass eine Funktion bzw. eine Prozedur genau einmal ausgeführt wird, wenn Sie genutzt wird. Bei Schleifen und Rekursionen führen wir den Inhalten mehrfach aus, indem wir an einer Stelle angeben, wie oft sie ausgeführt werden sollen.
	
	\item Im nächsten Kapitel führen wir die sogenannten Datenbanken ein. \textbf{Datenbanken}\index{Datenbank} sind dann sinnvoll, wenn wir nicht mehr nur Dutzende von Variablen haben, die wir in einer Datenstruktur gruppieren können, sondern wir über Tausende von Variablen in Dutzenden Datenstrukturen sprechen. Ein Beispiel wäre die Kundendatenbank eines Versandhändlers. Aber keine Sorge: Dafür sind Datenbanken zwar gedacht, aber wir können sie auch bei viel weniger Variablen nutzen, um uns die Nutzung anzusehen.
	
	Datenbanken werden anhand von eigenen Programmiersprachen programmiert. Eine dieser Sprachen heißt MySQL. Und der Grund aus dem viele Bücher eine Einführung in PHP und MySQL anbieten ist der, dass wir bei der Programmierung in PHP (wie in vielen anderen Programmiersprachen) häufig Datenbanken benötigen, um effizient große Datenmengen zu verwalten.
\end{itemize}

Jetzt also zu den Variablen.

\subsection{Variablen}

Wie eingangs beschrieben sind Variablen die Teile von prozeduralen Programmen, in denen wir auf Daten zugreifen können. Sie kennen Variablen bislang aus dem Mathematikunterricht der Oberstufe, wo jede Variable im Regelfall für eine Zahl steht. Dabei haben sie gelernt, dass der Wert einer Variablen sich nicht ändern darf: Wenn x an mehreren Stellen einer Aufgabe auftaucht, dann hat es immer denselben Wert, auch wenn wir ihn noch nicht kennen.\\

\textbf{Wichtig:} \\

Variablen und insbesondere Variablen in der Form, wie Sie sie jetzt kennenlernen sind nur eine Spezialform von Daten mit denen Sie in bestimmten Programmiersprachen wie C, C++, Java, Ruby, Python und vielen anderen mehr arbeiten. Insbesondere, wenn Sie einen Informatikstudiengang besuchen ist es außerordentlich wichtig, dass Sie so bald als möglich das Studium von Teil 2 dieses Buches fortsetzen. Dort werden Sie eine wesentlich allgemeinere Form von Daten in Computern kennen lernen und nur wenn Sie diese wesentlich allgemeinere Form von Daten wirklich verstanden haben, haben Sie die Grundlage, um \glqq{}echte/r\grqq{} InformatikerIn zu werden. So lange Sie dagegen auf die Programmierung fixiert bleiben, die Sie in diesem Teil des Buches kennen lernen (eben die imperative Programmierung), werden Sie zwar einen Computer dazu bringen können, Aufgaben für Sie zu erfüllen, aber Sie werden keinesfalls ernsthafte Informatik betreiben oder gar ein umfangreiches Verständnis dessen haben, was Programmierung alles ist.\\

Beachten Sie jedoch auch, dass Sie nach Ende des gesamten Buches lediglich die Programmierung mit Sprachen erlernt haben werden, die die diskrete Mathematik zur Grundlage haben. Die Programmierung kontinuierlicher Systeme wie Steuer- und Regelsysteme sind eine Kernkompetenz von Elektro- und Nachrichtentechnikern, auch wenn InformatikerInnen mit einer entsprechenden Vertiefung in diesem Bereich gleichwertige Fachkräfte werden können.

\subsubsection{Erzeugung einer Variablen}

Wie alles andere fallen auch Variablen nicht vom Himmel, sondern wir müssen dem Rechner mitteilen, dass wir eine Variable benötigen. Wir können dazu an einer beliebigen Stelle in unserem Programm einen Namen festlegen und diesem einen Wert zuordnen. Damit das Programm erkennt, dass es sich um eine Variable handelt, müssen wir lediglich ein \$-Zeichen an den Anfang des Variablennamens setzen. \\

Das ist alles.\\

Wenn wir außerdem einen Startwert vorgeben wollen, dann können wir das tun. Hier wie an jeder Stelle, an der wir den Wert einer Variablen festlegen bzw. ändern wollen, nutzen wir dazu ein einfaches Gleichzeichen.\\

Wenn Sie eine Variable für einen Text (das kann auch HTML-Code sein) erzeugen wollen, dann schreiben Sie diesen \glqq{}Wert\grqq{} der Variablen in Anführungs-zeichen.\\

\begin{verbatim}
// Erzeugung einer Variablen mit der Bezeichnung name.
$name;                         

// Erzeugung einer Variablen, der der Wert 3.14 
zugeordnet wird.
$pi = 3.14;                    

$quellcode = "<p>Wir brauchen die Vorrats- 
datenspeicherung, weil wir die Vorratsdaten-
speicherung brauchen.</p>";
\end{verbatim}Quellcode 4.4: Erzeugung von Variablen in PHP\\

Wenn Sie den Namen einer Variablen speichern und ihn dann später nutzen wollen, geht das auch:

\begin{verbatim}
$name = "var";
$var = 12;
echo($$name);
\end{verbatim}

Die letzten Zeile bewirkt, dass nicht etwa var sondern 12 ins HTML-Dokument eingefügt wird.

\subsubsection{Statische und dynamische Variablen}

Wir unterscheiden zunächst zwischen konstanten bzw. statischen Variablen, die Ihren Wert nicht ändern können und dynamischen Variablen, deren Wert sich jederzeit ändern kann. Konstante Variablen verwenden wir meist nur dann, wenn es wichtig ist, dass sich der Wert dieser Variablen nicht ändern darf oder es einfacher ist, einen Wert durch eine kurze Bezeichnung darzustellen. Letzteres reduziert die Wahrscheinlichkeit für Fehler im Programm.\\

Deshalb gibt es auch keinen besonderen \glqq{}Befehl\grqq{}, mit dem wir dem Programm mitteilen, dass eine Variable dynamisch ist. Bei statischen Variablen, die in PHP als Konstanten bezeichnet werden, müssen wir dagegen das Schlüsselwort const verwenden, dafür lassen wir das \$-Zeichen am Anfang des Variablennamens weg:\\

\verb|const pi=3.14159265358979323846264338327950;|\\
Quellcode 4.5: Erzeugung einer Konstanten mit Zuordnung eines Wertes.

\subsection{Datentypen}

Im Gegensatz zu Sprachen wie C oder Java lässt PHP zu, dass sich der Datentyp einer Variablen jederzeit ändern kann. Jetzt fragen Sie sich vielleicht, was denn der Datentyp ist.\\

Alle Daten haben einen Typ (eben den Datentyp). Denken Sie hier an Dinge wie Zahlen, Buchstaben, Symbole und ähnliche Arten von Zeichen, die Sie aus dem Alltag kennen. Es geht also um Kategorien von Dingen, die der Computer speichert. Im Gegensatz Dingen Ihres Alltags richten sich Datentypen aber danach, wie der Computer Daten intern speichert und wie er sie interpretieren soll.\\


Im Kern speichern Computer alles in Form von Zahlen zwischen 0 und 255. Reicht dieser Zahlenbereich nicht, dann nimmt ein Computer eben mehrere dieser Zahlen und behandelt sie wie eine gemeinsame Zahl. Nichts anderes tun wir, wenn wir rechnen: Wir nehmen die Zahlen von 0 bis 9 und wenn die nicht reichen, nehmen wir eine zweite Zahl von 0 bis 9 usw. Nur haben wir für andere Zwecke noch andere Bereiche, mit denen wir arbeiten: Die Buchstaben von a bis z. Ein Computer kennt dagegen nur seine Zahlen von 0 bis 255. (Genaueres dazu lernen Sie in den Veranstaltungen zur technischen Informatik und der Kommunikations- bzw. Nachrichtentechnik.)\\

Der Computer kann also im Grunde einen Buchstaben auch nur als Zahl speichern. Er kann auch keine Nachkommstellen als Nachkommastellen speichern. Nochmal: Alles was er hat sind Speichereinheiten, die jeweils eine Zahl von 0 bis 255 speichern können.\\


Wenn wir also so etwas wie einen Buchstaben eingeben, dann speichert der Computer ihn als eine Zahl in einer Speicherstelle. Und damit er weiß, dass er diese Zahl in diesem Fall als Buchstaben interpretieren muss, braucht er noch einen weiteren Speicher, in dem er genau das speichert. Wir als Softwareentwickler bezeichnen das als Datentyp. Für uns ist es egal, wie der Computer den Datentyp realisiert, wichtig ist nur, dass er nicht plötzlich anfängt, unsere Buchstaben als Zahlen auszugeben und umgekehrt. In diesem Kurs werden wir uns mit dem Mechanismus, der hinter Datentypen steht nicht weiter auseinander setzen. Wichtig ist nur, dass Sie grundsätzlich verstehen, warum jede Variable einen Datentyp hat.

\subsubsection{Statische und dynamische Datentypen: Typecasting}

Sprachen wie PHP bieten nicht nur dynamische Variablen an, also Variablen deren Wert sich ändern kann. Zusätzlich kann sich auch der Datentyp einer Variablen ändern. Nehmen wir an, Sie dividieren eine ganze Zahl in einem PHP-Programm und als Ergebnis erhalten Sie eine Fließkommazahl. Während dieser Unterschied für Sie kein merklicher Unterschied ist, ändert sich in diesem Moment der Datentyp des Wertes, denn der Computer muss eine Fließkommazahl anders speichern, da er ja wie besprochen nur ganze Zahlen von 0 bis 255 nutzen kann, um alles zu speichern, dass er speichern soll.\\

Sprachen, bei denen wie in PHP der Datentyp geändert werden kann werden auch als dynamisch typisierte Programmiersprachen bezeichnet.\\

Bei statisch typisierten Sprachen wie C und Java müssten Sie nun eine neue Variable erzeugen, der Sie den neuen Wert zuordnen. Denn dort kann sich der Datentyp einer Variablen nicht ändern. Bei dynamisch typisierten Sprachen ist das anders: Dort ändert sich der Datentyp je nachdem, wie sich der Wert einer Variablen ändert.\\

Das Ändern des Datentyps wird auch als Typecasting bezeichnet.\\

Erfahrene Entwickler in statisch typisierten Sprachen sagen deshalb häufig, dass dynamisch typisierte Sprachen unsicher seien, weil man je nie wissen könne, welchen Datentyp eine Variable hat. Aber das ist Unsinn, jeder Softwareentwickler sollte wissen, wie in einer dynamisch und wie in einer statisch typisierten Sprache programmiert werden muss. Außer natürlich es ist jemand, der in nicht-imperative Programmiersprachen Software entwickelt oder die nur in einer Sprache programmiert. Denn dann ist dieser Unterschied irrelevant.\\

In kurzen Worten: Wenn Sie es mit Programmierern zu tun haben, die darauf beharren, dass dynamisch typisierte Sprachen unsicher sind, weil diese keine Typsicherheit bieten, dann behandeln Sie diese Menschen am besten wie Geisteskranke: Immer nett zu ihnen sein und halten Sie sie von allem fern, was sie nicht verstehen. Genau wie manche Authisten können einige dieser Menschen in einem kleinen Spezialgebiet wirklich großartige Dinge leisten. Aber der Rest stellt für sie eine vollständige Überforderung dar. Das Problem besteht allerdings darin, dass es nur wenige Authisten gibt, während Menschen mit der beschriebenen beschränkten Auffassung von Programmierung die Norm darstellen. Wenn Sie nichts über Authismus wissen, sehen Sie den Oscar-prämierten Film ,,Rainman``. Dieser zeigt zwar ein sehr einseitiges Bild der Krankheit, aber für nicht-Psychologen und zum Verständnis des Vergleichs sollte das genügen.

\subsection{Operationen}

Operationen entsprechen weitestgehend dem, was Sie in der Oberstufenmathematik kennen gelernt haben, gehen aber darüber hinaus: Sie nehmen zwei Operanden, verknüpfen diese mithilfe eines Operatoren und bekommen ein Ergebnis, indem Sie den gebildeten Ausdruck auswerten. Komplexe Operationen (also Operationen mit mehreren Operanden) bilden Sie, indem Sie Klammern verwenden.

\subsubsection{Arithmetische Operatoren}

Die vier Operatoren +, -, * und / entsprechen der Addition, Subtraktion, Multiplikation und Division. Der Doppelstern ** entspricht der Potenzierung.\\

Zusätzlich gibt es noch die Modulo-Operation, die in PHP durch das \%-Zeichen ausgedrückt wird. Wenn Sie den entsprechenden Teil des Mathematikunterrichts inzwischen erfolgreich verdrängt haben, sollten Sie die alten Kenntnisse wieder aufwärmen; insbesondere im Bereich der IT-Sicherheit ist die modulare Arithmetik essentiell.

\subsection{Anonyme Variablen}

Oben haben Sie erfahren, wie Sie Variablen erzeugen. Dort haben Sie gelesen, dass jede Variable einen Namen hat. Aber was ist dann eine anonyme Variable?\\

Stellen Sie sich vor, Sie haben folgenden PHP-Code erstellt:\\

\begin{verbatim}
$a = 5;
$b = 3;
echo($a + $b);
\end{verbatim}
Quellcode 4.6: Anonyme Variable\\

Hier haben Sie zwei Variablen addiert und das Ergebnis ausgegeben. Das klingt logisch, aber hier fehlt noch etwas: Wenn ein Computer eine Operation ausführt, dann speichert er das Ergebnis irgendwo ab. Im Regelfall geben sie dieses \glqq{}irgendwo\grqq{} vor, indem Sie dafür eine Variable erzeugen oder den Computer anweisen, den Wert einer anderen Variablen zu überschreiben. Das ist bei diesem Code-Beispiel nicht der Fall. Da wir also das Ergebnis der Operation nicht über einen Namen ansprechen können, reden wir hier von einer anonymen Variable.Quellcode 4.6: Anonyme Variable

Wenn wir keine anonymen Variablen verwenden wollen oder können, dann müssten wir das obige Programm so programmieren:Quellcode 4.6: Anonyme Variable

\begin{verbatim}
$a = 5;
$b = 3;
$c = $a + $b;
echo($c);
\end{verbatim}
Quellcode 4.7: Nutzung einer Hilfsvariable, um keine anonyme Variable zu verwenden.\\

Wie Sie sehen hat dieses Programm keinen Mehrwert gegenüber dem mit der anonymen Variable, sondern ist nur länger und damit fehleranfälliger. Also nutzen sie anonyme Variablen, wenn es Sinn macht.

\subsection{Operationen für Text-Variablen}

Wenn Sie sich wundern, warum es Operatoren für Text-Variablen gibt, dann sehen Sie Variablen und Operationen noch so beschränkt, wie Sie sie intuitiv im Mathematikunterricht kennen gelernt haben. Deshalb hier nochmal eine allgemeinere Beschreibung, die für die Programmierung in prozeduralen Sprachen gilt: Variablen können beliebige Daten speichern und mit Operationen ändern Sie diese Daten oder erzeugen daraus neue Daten.\\

Eine einfache Operation für Texte ist die Konkatenation, also die Verknüpfung von zwei Texten. Für diese wird der . (Punkt) verwendet.\\

Wichtig: Bis auf Variablen, die durch Konkatenation entstehen sind bei echo() keine anonymen Variablen erlaubt. Eine Zeile wie echo(``Das Ergebnis ist: ``.\$a + \$b); wäre also nicht erlaubt.\\

\begin{verbatim}
\$name = ``Horst``;
\$zustand = ``krank``;
echo(\$name.``ist``.\$zustand);
\end{verbatim}
Quellcode 4.8: Konkatentation von Texten.

\subsection{Nachteile der dynamischen Typisierung}

Wie beschrieben wird der Datentyp bei dynamisch typisierten Sprachen von der Programmiersprache immer dann geändert, wenn das sinnvoll erscheint. Dafür gibt es natürlich festgelegte Regeln, die sich je nach Sprache unterscheiden. Überlegen Sie, was alles beim folgenden Code passieren könnte und probieren Sie es dann aus, um zu erfahren, wie PHP programmiert ist.\\

\begin{verbatim}
\$a = 2;
\$b = 3;
echo(\$a.\$b);
\$c = \$a.\$b;
\$d = \$c*2;
echo(\$d);
\end{verbatim}
Quellcode 4.9: Beispiel für dynamische Typisierung\\

Codebeispiele wie dieses werden von vielen Programmierern als Beweis genutzt, dass die dynamische Typisierung unsicher ist. Alles was eine Beweisführung in dieser Art tatsächlich belegt, ist die Unfähigkeit der Betreffenden, über den eigenen Tellerrand zu schauen.

\subsubsection{Kurzschreibweise für Zuordnung und Operation}

Gerade bei der Konkatenation werden Sie des Öfteren eine Variable mit dem neuen \glqq{}Wert\grqq{} ergänzen wollen. Im folgenden Codebeispiel sehen Sie eine Kurzform, die Sie in diesen Fällen nutzen können. 

\begin{verbatim}
\$return = ``Siggi``:
\$taetigkeit = ``geht shoppen.``;
\$return = \$return.\$taetigkeit;

// Die letzte Zeile können Sie verkürzen:

\$return .= \$taetigkeit;
\end{verbatim}
Quellcode 4.10: Verkürzte Form für Zuordnung und Operation.

\subsubsection{Aufgabe}

-	Probieren Sie aus, für welche Operationen das funktioniert.

\section{Überarbeitung: Kontrollstrukturen und boolesche Ausdrücke}

Wie eingangs erläutert kommen wir jetzt zum ersten Teil der Programmierung, mit dem Sie tatsächlich eine Logik in Ihr Programm bekommen können. Während Sie also mit den vorigen Elementen von PHP kaum einen Mehrwert gegenüber der Programmierung in HTML erreichen konnten kommen wir jetzt zu den eigentlich spannenden Aspekten bei der Webanwendungsentwicklung mit PHP.

\subsection{Grundidee}

Nachdem Sie jetzt wissen, wie Sie Werte in PHP (und anderen prozedurale Sprachen) über Variablen speichern und ändern können, kommen wir jetzt dazu, wie Sie abhängig vom Ergebnis einer Operation unterschiedliche Programmteile ausführen lassen. Ob diese dann eine Ausgabe erzeugen, die als Quellcode in die Webpage eingefügt werden oder ob dadurch die Werte von Variablen geändert werden ist für diese Grundidee belanglos.

\subsection{Vergleichsoperationen}

Bei den Variablen haben Sie Operatoren kennen gelernt, die die Werte von Variablen nutzen, um neue Variablen zu erzeugen oder den Wert von bestehenden Variablen zu ändern. Wenn wir dagegen nur wissen wollen, ob z.B. \$a größer als \$b ist oder ob \$name der Name des Administrators ist, dann wollen wir ja keine Variablen ändern. Vielmehr möchten wir hier Operatoren nutzen, um zu entscheiden, wie das Programm weiter ausgeführt wird. Aber auch dafür nutzen wir Operationen, die wie alle Operationen eine anonyme Variable erzeugen. Diese Operationen heißen Vergleichsoperationen.\\


Auch hier begegnet Ihnen im Grunde nichts, was Sie nicht schon aus der Schulzeit kennen würden, nur die Schreibweise müssen Sie sich einprägen. Das einzige, was Sie anfangs verwirren wird ist die Tatsache, dass ein Vergleich auf Gleichheit durch ein doppeltes Gleichheitszeichen ausgedrückt wird, denn das einfache Gleichheitszeichen wird ja bereits verwendet, um einer Variablen einen Wert zuzuordnen.\\


Das Ergebnis von Vergleichsoperationen wird in aller Regel als anonyme Variable verwendet. Dennoch kann es vorkommen, dass Sie das Ergebnis speichern wollen, dass Sie also eine boolesche Variable erzeugen wollen, eine Variable, die wahr oder falsch sein kann. In diesen Fall gehen Sie wie bei anderen Zuordnungen von Variablen vor.
Hier die Liste der Vergleichsoperatoren:\\


-	= =	prüft auf die Gleichheit der Werte zweier Operanden.
-	= = =	prüft außerdem, ob Sie den gleichen Datentyp haben.
-	! =	prüft, ob zwei Operanden unterschiedliche Werte haben.
-	! = = 	prüft außerdem, ob auch der Datentyp unterschiedlich ist.
-	>	prüft, ob der erste Operand größer ist, als der zweite.
-	> =	prüft, ob der erste Operand nicht kleiner ist, als der zweite.
-	<	prüft, ob der zweite Operand größer ist, als der erste.
-	< =	prüft, ob der zweite Operand nicht größer ist, als der zweite.

Sie finden, dass das zu viele Operatoren sind? Vollkommen richtig: Fürs erste brauchen Sie nur = =, != und < oder >. Wie Sie in der technischen Informatik lernen (also in der Veranstaltung Informatik bzw. Informatik 1) brauchen Sie dazu noch Konjunktionen und Disjunktionen. Wie Sie die in PHP programmieren lernen Sie jetzt. 

\subsection{Conditionals 1 – if-then-else}

Es gibt in PHP zwei Möglichkeiten, um Kontrollstrukturen zu programmieren. Beide werden im Englischen als Conditionals bezeichnet.\\


Um eine Kontrolle durchzuführen, überlegen Sie sich alle möglichen Fälle und erstellen für jeden Fall einen sogenannten if-Zweig. Wenn Sie mehrere Fälle nicht mit einem eigenen if-Zweig behandeln, dann haben Sie zwei Möglichkeiten: Entweder Sie ignorieren diese Fälle ganz, lassen also das Programm einfach weiterlaufen oder Sie programmieren einen else-Zweig, in dem all das passiert, was eben in den verbliebenen Fällen passieren soll.\\


Die Bedingung, die geprüft werden soll steht in runden Klammern hinter dem \glqq{}Befehl\grqq{} if. Das Schlüsselwort \glqq{}then\grqq{} gibt es in PHP nicht. Wenn ein Programmteil bei einem if-Zweig ausgeführt werden soll, dann steht er direkt hinter dem Conditional des if.

\subsubsection{Rümpfe}

Bislang haben Sie immer nur Fälle kennen gelernt, in denen Ihr Programm Zeile für Zeile abgearbeitet wird. Jetzt kommen wir dagegen zu Fällen, in denen mehrere Zeilen nur in bestimmten Fällen ausgeführt werden sollen. In anderen Worten: Wenn diese Fälle nicht eintreten, dann sollen die entsprechenden Zeilen übersprungen werden. Also brauchen wir eine Möglichkeit, um im Quellcode anzuzeigen, welche Teile ggf. übersprungen werden sollen. Solche zusammengehörenden Mehrzeiler werden als Rumpf (z.B. Rumpf einer Funktion) bezeichnet. In PHP werden Rümpfe durch geschweifte Klammern markiert: \\

\begin{verbatim}
\{ ... \}
Anstelle der Punkte wird schlicht der Programmcode eingetragen.
Der folgende PHP-Code ist ein einfacher if-then-else-Fall:
if (\$login==``admin``)
\{
	echo(``<!- - Hier den Code einfügen, damit der Nutzer auf die Administrationsoberfläche gelangt. - -> ``);
\}
else 
\{
	echo(``Die angeforderte Seite steht nur Administratoren zur Verfügung.``);
\}
\end{verbatim}
Quellcode 4.11: Einfaches if Conditional

\subsection{Conditionals 2 – Switch-Cases}

Wenn-Dann-Kontrollstrukturen sind leicht zu verstehen, aber gerade wenn Sie den weiteren Programmablauf von der Belegung einer Variablen steuern wollen, sind die Switch-Cases häufig mit weniger Programmieraufwand verbunden.\\

Dabei tragen Sie in den Klammern hinter dem \glqq{}Befehl\grqq{} switch den Namen der Variablen ein und bei den einzelnen Cases den Wert der Variablen. Hier ein einfaches Beispiel, das genau dasselbe bewirkt wie der if-then-else-Fall oben:\\

\begin{verbatim}
switch(\$login)
\{
	case(``admin``):
	echo(``<!- - Hier den Code einfügen, damit der Nutzer auf die Administrationsoberfläche gelangt. - -> ``);
	break;
	default:
	echo(``Die angeforderte Seite steht nur Administratoren zur Verfügung.``);
\}
\end{verbatim}
Quellcode 4.12: Einfaches Switch-Case-Condititonal\\

Im Gegensatz zum if-then-else-Conditional wird bei Switch-Cases der default-Fall immer ausgeführt. Für jeden Fall, bei dem Sie das vermeiden wollen müssen Sie noch die Anweisung break ans Ende des jeweiligen Cases programmieren. Das ist auch der Grund, warum Sie bei Switch-Cases keine geschweiften Klammern benötigen.\\

Letztlich bleibt es Ihnen überlassen, ob Sie nun if-then-else oder Switch-Case verwenden. Aus Gründen der besseren Lesbarkeit sollten Sie allerdings immer dann Switch-Cases verwenden, wenn das möglich ist, da hier der Zusammenhang zwischen den Fällen leicht erkennbar ist.\\

Noch ein abschließender Hinweis: Ob Sie nun (wie bei den obigen Beispielen) für die geschweiften Klammern eine eigene Zeile reservieren oder nicht, ist Geschmacksfrage. Die folgenden Beispiele sind genauso gültig wie die beiden obigen. Beachten Sie bitte noch, dass das Weglassen der geschweiften Klammern bei if-then-else nur dann erlaubt ist, wenn der Fall mit einer Programmzeile erfüllt wird.\\

\begin{verbatim}
if (\$login==``admin``) \{
	echo(``<!- - Hier den Code einfügen, damit der Nutzer auf die Administrationsoberfläche gelangt. - -> ``); 
\} else \{
echo(``Die angeforderte Seite steht nur Administratoren zur Verfügung.``);
\} 

if (\$login==``admin``) echo(``<!- - Hier den Code einfügen, damit der Nutzer auf die Administrationsoberfläche gelangt. - -> ``); 
else echo(``Die angeforderte Seite steht nur Administratoren zur Verfügung.``); 

switch(\$login) \{
	case(``admin``): echo(``<!- - Hier den Code einfügen, damit der Nutzer auf die Administrationsoberfläche gelangt. - -> ``);
	break;
	default: echo(``Die angeforderte Seite steht nur Administratoren zur Verfügung.``);
\}
\end{verbatim}
Quellcode 4.13: Eine Frage des persönlichen Geschmacks: Position der geschweiften Klammern.\\

Neben den hier vorgestellten Möglichkeiten if-then-else oder Switch-Cases zu programmieren, gibt es noch eine weitere Möglichkeit, die Sie dann nutzen können, wenn Sie sehr umfangreichen HTML-Code generieren wollen. Hier unser obiges Beispiel mit den nötigen Änderungen innerhalb eines HTML-Dokuments:\\

\begin{verbatim}
<?php
...
if (\$login==``admin``) \{ ?>
	<!- - Hier den Code einfügen, damit der Nutzer auf die Administrationsoberfläche gelangt. - ->
	<?php \} else \{ ?>
	<p>Die angeforderte Seite steht nur Administratoren zur Verfügung.</p>
	<?php \} ?>
\end{verbatim}
Quellcode 4.14: Einfaches if Conditional, wenn umfangreicher HTML-Code zu übergeben ist

\subsubsection{Prüfung mehrerer Bedingungen}

Wenn Sie nun einen bestimmten Fall nur dann ausführen wollen, wenn mehrere Bedingungen oder eine von mehreren Bedingungen erfüllt sind, gibt es zwei Möglichkeiten: (Um den Code übersichtlich zu halten, werden jetzt keine konkreten Fälle genannt, sondern es wird für jede Bedingung einfach der Buchstabe B und eine Zahl verwendet. B5 ist also im Folgenden die fünfte Bedingung, wobei hier nicht von Belang ist, was diese Bedingung ist.

\subsubsection{Mehrere Bedingungen müssen gelten}

Nehmen wir an, drei Bedingungen (B1, B2 und B3) sollen gelten, damit ein Programmteil ausgeführt wird. Dann können wir verschachtelte if-then-else-Fälle programmieren, um das umzusetzen:\\
\begin{verbatim}
if(B1)\{ if(B2) \{if(B3) \{ /* Hier steht der auszuführende Code */ \}\}\}
\end{verbatim}
Quellcode 4.14: Verschachteltes if\\

Leider ist diese Form aber eher unübersichtlich. Deshalb gibt es den Konjunktionsoperator: \&\& (Doppeltes Und-Symbol) Wenn wir das einsetzen, wird unser Code besser lesbar:\\

\begin{verbatim}
if(B1 \&\& B2 \&\& B3)\{ /* Hier steht der auszuführende Code. */ \}
\end{verbatim}
Quellcode 4.15: Konjunktion bei if-Cases

\subsubsection{Eine von mehreren Bedingungen muss gelten}

Wenn dagegen eine der drei Bedingungen gelten soll, aber in jedem Fall der gleiche Code ausgeführt werden soll, erhöht die erste Variante die Fehleranfälligkeit unseres Codes massiv:\\

\begin{verbatim}
if(B1)\{ /*Hier steht der auszuführende Code. */ \}
if(B2)\{ /* Hier steht er nochmal. */ \}
if(B3)\{ /* Und hier schon wieder. */ \}
\end{verbatim}
Quellcode 4.16: Disjunktive if-Cases\\

In diesen Fällen ist es ein must-have, Disjunktionen anzuwenden. Die werden als | | programmiert:\\

\begin{verbatim}
if(B1 || B2 || B3)\{ /* Hier steht der auszuführende Code. */ \}
\end{verbatim}
Quellcode 4.17: Disjunktion von Bedingungen\\

Alles weitere in Bezug auf die Kombinationsmöglichkeiten von Konjunktionen und Disjunktionen lernen Sie in den Veranstaltungen zur technischen Informatik und den einführenden mathematischen Veranstaltungen Ihres Studiums.\\

Aufgabe:\\

-	Erweitern Sie Ihre Seite wie folgt: (Lesen Sie zunächst alle Aufgabenteile einmal durch. Es ist nicht gesagt, dass Sie die einzelnen Teile in genau der Reihenfolge umsetzen müssen, die hier vorgegeben ist.)

•	Programmieren Sie eine Variable \$login, die standardmäßig den Wert guest hat.

•	Programmieren Sie außerdem  Teile Ihrer Webpage, sodass diese nur angezeigt werden, wenn \$login den Wert admin hat. Dazu benötigen Sie die CSS-Property visibility. Es sollte sich zwar von selbst verstehen, aber zur Sicherheit sei auf folgendes hingewiesen: Recherchieren Sie dazu, welche Belegungen visibility haben kann. Denn nur mit diesem Wissen können Sie die Aufgabe erfüllen.

•	Programmieren Sie Teile Ihrer Webpage so um, dass sie nur dann angezeigt werden, wenn \$login den Wert user oder admin hat. Damit ist gemeint, dass diese Teile sowohl für user als auch für admin angezeigt werden sollen.

Hinweis: Da Sie noch keinen Log-in entwickelt haben, können Sie zurzeit noch nicht prüfen, ob Sie alles richtig gemacht haben. Diese Aufgabe dient vorrangig dazu, dass Sie selbst prüfen können, ob Sie die Inhalte der vorigen Kapitel soweit verinnerlicht haben, dass Sie sinnvoll damit arbeiten können:\\

-	Wenn Sie Probleme damit haben, zu verstehen, was Sie tun sollen, sollten Sie dringend die bisher behandelten Themen wiederholen. Versuchen Sie vor allem zu verstehen, was der Sinn der einzelnen Abschnitte bzw. der darin behandelten Themen ist. Wenn Sie die bislang behandelten Themen nur überflogen haben, dann sollten Sie das dringend abstellen oder sich nach einem Ausbildungsplatz umsehen: Studieren bedeutet, sich intensiv mit den Inhalten von Veranstaltungen auseinander zu setzen... und zwar vor und nach den Veranstaltungen... mehrere Stunden pro Woche.

-	Wenn Sie dagegen ein Problem damit haben, das Verstandene in ein Programm umzusetzen, dann bedeutet das lediglich, dass Sie mehr Übung brauchen. Andererseits ist das das Hauptproblem der meisten Studienanfänger: Sie wenden das Gelernte zu selten an und verinnerlichen es deshalb nicht.

-	Wenn Sie sich dagegen wundern, warum Sie solche einfachen Schritte programmieren sollen, bzw. warum das hier so detailliert steht, dann sind Sie entweder schon deutlich weiter oder Sie sollten sich einen systematischeren Arbeitsstil angewöhnen. Denn auch dafür ist diese Übung gedacht.

Grundsätzlich gilt aber im Studium immer: Es ist gerade zu Beginn vollkommen normal, sich von den Inhalten und Übungen überfordert zu fühlen. Wichtig ist es dann, konzentriert weiter zu arbeiten. Außerdem ist es vollkommen normal, einzelne Veranstaltungen erst im dritten Anlauf zu bestehen. Manche Inhalte brauchen viel Zeit, damit Sie sie wirklich verinnerlicht haben.

\section{Überarbeitung: Funktionen}

Wenn die Seiten Ihrer Webpage immer umfangreicher werden, werden Sie früher oder später merken, dass Teile Ihrer Seite sich wiederholen. So kann es sein, dass Sie mehrere <summary>-Container programmiert haben, die nur von bestimmten Nutzern geöffnet werden dürfen. Hier wäre es unschön, wenn Sie jedes Mal die entsprechende Kontrollstruktur in PHP programmieren würden. Für solche Fälle gibt es die sogenannten Funktionen, die bei der objektorientierten Programmierung als Methoden bezeichnet werden.\\

Funktionen kennen Sie aus der Oberstufen-Mathematik. Dort tauchten Sie in Formen wie f(x) oder ähnlichem auf. Bei der Programmierung in PHP können Sie längere Bezeichnungen als ein einfaches f verwenden. Dadurch ist es möglich, dass Sie eine Funktion so benennen, dass bereits aus dem Namen hervorgeht, was diese Funktion tun soll.\\

Im Gegensatz zu dem, was Sie in der Mathematik kennen gelernt haben steht es Ihnen hier aber frei, ob und wenn ja, wie viele Parameter einer Funktion übergeben werden sollen. Denn da Ihr Programm ja bereits Variablen hat, können Sie aus einer Funktion auf diese Variablen zugreifen.\\

Jede Funktion besteht aus einem Namen und einem Klammernpaar, das direkt neben dem Namen steht. In dieses Klammernpaar programmieren Sie dann all die Parameter, die von der Funktion verarbeitet werden sollen.\\

Mindestens eine solche Funktion haben Sie schon benutzt.\\

Aufgabe:\\

-	Wie heißt diese Funktion?

(Wenn Sie sich mit der Antwort schwer tun, sehen Sie nochmal in Ihren bisherigen Programmen nach, wo etwas auftaucht, das aus einem Wort und einer anschließenden Klammer besteht.)

-	Was tut diese Funktion?

\subsection{Was ist ein Parameter?}

Der Parameter der Funktion echo() ist immer ein String. Wenn Sie hier eine Zahlenvariable (nennen wir sie einfach \$zahl) einsetzen, dann wird der Wert von \$zahl in einer neuen Variable gespeichert, deren Bezeichner Sie nicht kennen und auch nicht kennen brauchen, weil für Sie nur wichtig ist, dass echo() diesen Wert ausgibt. Diese (für uns anonyme) Variable wird nun zu einem String gecastet. Das ändert jedoch nichts am Datentyp oder Wert der Variablen \$zahl selbst.\\

Beim Aufruf einer Funktion brauchen wir also nichts über die Bezeichnungen der Variablen zu wissen, die innerhalb der Funktion verwendet werden. Diese Tatsache wird später bei Rekursionen wichtig: Bislang haben Sie gelernt, dass Sie den Bezeichner einer Variablen im gesamten Programm nur einmal verwenden dürfen, weil Sie sie an jeder Stelle des Programms nutzen können. Jetzt kommt eine Einschränkung für diese Aussage: Wenn eine Variable nur innerhalb eines Funktionsrumpfes vorkommt, dann kann Sie außerhalb des Funktionsrumpfes nicht verwendet werden, weil sie dort unbekannt ist. Das bedeutet auch, dass Sie eine Funktion mehrfach aufrufen können, denn die Variablen, die innerhalb eines Funktionsaufrufes verwendet werden, werden in einem eigenen Speicherbereich gespeichert. Wenn Sie das momentan noch nicht so recht einordnen oder verstehen können, lassen Sie sich davon nicht irritieren; wir kommen beim Thema Rekursion darauf zurück.\\

Wenn Sie dagegen eine Funktion selbst programmieren wollen, dann müssen Sie darin Parameter selbst programmieren.\\

Als Parameter werden Variablen bezeichnet, die der Zwischenspeicherung von Werten dienen, die bei einer Funktion verwendet werden. Im folgenden Funktionskopf wären das die Variablen \$param1, \$param2 und \$param3:\\

\begin{verbatim}
function meineFunktion(\$param1, \$param2, \$param3) \{ ... \}
\end{verbatim}

Die Entwickler von PHP haben z.B. die Funktion echo() in etwa so programmiert:

\begin{verbatim}
function echo(\$param) \{ 
	/* Quelltext, der dafür sorgt, dass \$param in ein HTML-Dokument eingefügt wird. */ 
\}
\end{verbatim}

Bei der Programmierung des Funktionsrumpfes verwenden Sie dann die Parameter, als wären es ganz normale Variablen, die Sie an anderer Stelle des Programms programmiert haben. \\

Wichtig: Beachten Sie bitte, dass diese Parameter außerhalb des Rumpfes nicht genutzt werden können. Um die Fehlerwahrscheinlichkeit zu reduzieren sollten Sie Variablen, die Sie außerhalb eines Funktionsrumpfes nutzen nicht als Parameter verwenden.\\

Für Fortgeschrittene: Es gibt auch die Möglichkeit, die Variable innerhalb einer Funktion zu überschreiben, von der der Parameter an die Funktion übergeben wurde. In diesem Fall wird bei PHP von einer Referenz auf die Variable gesprochen. Dabei kommt in PHP das \&-Zeichen zum Einsatz. Da wichtig ist, dass Sie einen sauberen Programmierstil lernen, lassen wir diesen Fall hier außen vor. Wenn wir von der Nutzung von Parametern in Funktionen sprechen, so wie Sie es hier kennen gelernt haben, dann wird bei PHP von der Kopie einer Variablen gesprochen. Die Bezeichnung Parameter, wie in diesem Kurs verwendet ist dagegen in anderen Sprachen üblich. Deshalb werden wir in diesem Kurs dabei bleiben. Schließlich sollen Sie hier Dinge lernen, die in möglichst vielen Sprachen gelten.

\subsubsection{Rückgabewert einer Funktion}

Allerdings gibt es etwas, was Sie bei echo() nicht kennen gelernt haben, das Sie aber aus der Oberstufenmathematik kennen: Funktionen haben einen Rückgabewert. Das bedeutet, dass eine Funktion im Regelfall nicht nur etwas tut, bzw. etwas mit den übergebenen Werten tut, sondern dass sie auch ein Ergebnis zurückgibt. Dieses Ergebnis kann der Wert einer beliebigen Variablen sein. Es kann auch einfach eine boolesche Variable sein, die nur anzeigt, ob die Funktion wie gewünscht ausgeführt wurde und sonst nichts weiter mit dem genauen Ablauf der Funktion zu tun hat.\\

In diesen Fällen können Sie also bei einem Programm eine Funktion als Wert einer Variablen programmieren: \$a = eineFunktion(); ist somit eine legale Anweisung, die der Variablen \$a den Wert zuordnet, der nach dem Aufruf von eineFunktion() zurückgegeben wird.\\

Aufgabe:\\

-	Angenommen es gibt eine Funktion namens derNutzerIstAdmin(), die eine anonyme boolesche Variable zurückgibt, die dann wahr ist, wenn der angemeldete Nutzer ein Administrator ist. Wie müssten Sie den folgenden Code erweitern, damit der <summary>-Container nur dann angezeigt wird, wenn der Nutzer ein Administrator ist?

\begin{verbatim}
<?php
...
if( ...... ) \{ ?>
	<summary> ....
	<?php \} else ... \{ >
	...
	<?php \} ?>
\end{verbatim}
Quellcode 4.18: Aufruf einer Funktion bei Conditionals\\

Übrigens müssen Sie natürlich für diesen Fall nicht unbedingt eine Funktion verwenden. Sie könnten auch nach dem Log-in des Nutzers einen Programmteil durchlaufen lassen, in dem geprüft wird, ob der Nutzer ein Administrator ist oder nicht. Und das Ergebnis dieser Prüfung könnten Sie dann einer Variablen zuordnen, die Sie anschließend weiter verwenden würden.\\

Das ist ein Beispiel für etwas, das Sie bei der Programmierung in jeder prozeduralen Programmiersprache (aber auch bei anderen Arten von Sprachen) beachten müssen: Vieles lässt sich auf unterschiedlichste Art und Weise realisieren. Und meist gibt es keine beste Lösung, sondern viele Lösungsmöglichkeiten, von denen einige besser, andere schlechter und viele aus den unterschiedlichsten Gründen sehr schlecht sind. 

\subsection{Eigene Funktionen programmieren}

Wenn Sie einen Programmteil zu einer Funktion machen wollen, dann setzen Sie den gesamten Programmteil in geschweifte Klammern und schreiben in der Zeile davor das Schlüsselwort function und anschließend einen Funktionsnamen, der mit runden Klammern abgeschlossen wird.

\begin{verbatim}
function meineFunktion(\$p1, \$p2)\{
	... // Hier programmieren Sie, was innerhalb der Funktion passieren soll, als wäre es ein
	// beliebiger Teil Ihres Programms.
	// Allerdings können Sie dafür die Parameter \$p1 und \$p2 verwenden.
	// Die Werte dieser Parameter werden beim Funktionsaufruf vergeben.
\}
\end{verbatim}
Quellcode 4.19: Definition einer Funktion ohne Parameter\\

Wann immer Sie in Zukunft wollen, dass der Rumpf der Funktion aufgerufen wird, müssen sie nur noch den Funktionsnamen an der entsprechenden Stelle einprogrammieren.

\section{Überarbeitung: Datenstrukturen}

Dieser Abschnitt ist etwas problematisch, weil wir bislang vorrangig über die Möglichkeiten gesprochen haben, die die Sprache PHP Ihnen bietet. Datenstrukturen gehören genau wie Rekursionen in einen Bereich, bei dem viele Programmierer nur das nutzen, was die Programmiersprache Ihnen anbietet. Und genau dadurch können Sie einfache Programmierer von Menschen mit einem Verständnis für Informatik unterscheiden.\\

Zunächst einmal ist eine Datenstruktur etwas, womit Sie eine Vielzahl gleichartiger Variablen strukturiert als eine Einheit abspeichern können. Nehmen wir an, Sie haben eine Webpage mit einem Spiel entwickelt und wollen eine High-Score-Liste generieren.\\

Das würde natürlich voraussetzen, dass jeder Spieler einen Punktestand für sein Spiel bekommt. Soweit ist es auch kein Problem, aber alles, was Sie mit dem bisherigen Wissen tun können, ist den aktuellen Punktestand mit einem Namen jeweils in einer Variablen zu speichern. Sie könnten weiterhin für die zehn besten Spieler jeweils ein solches Variablenpaar erstellen. Aber Sie hätten keine Möglichkeit, um alle Ergebnisse von Spielern in dieser Weise zu speichern. Denn auch wenn Sie mit PHP eine dynamische Webanwendung entwickeln können, können Sie eben nicht im Vorfeld programmieren, dass Ihr Programm je nach Bedarf zusätzliche Variablen generieren soll.\\

Die Lösung für dieses Dilemma heißt Datenstrukturen. In aller Regel bieten prozedurale Sprachen Ihnen zumindest ein Array als Datenstruktur an, dass Sie ohne weitere Kenntnisse nutzen können. In Veranstaltungen wie Algorithmen und Datenstrukturen lernen Sie dann, wie Sie andere Datenstrukturen selbst programmieren können und wann welche Datenstruktur Sinn macht. Prozedurale Sprachen wie Pascal ermöglichen es Ihnen außerdem beliebige Datenstrukturen selbst zu programmieren.\\

Leider ist PHP derart beschränkt, dass Sie keine eigenen Datenstrukturen programmieren können. Erinnern Sie sich noch an meine Kritik an den \glqq{}Vorteilen\grqq{} die der Autor Thomas Theis bezüglich PHP formulierte? Genau hier haben wir einen Fall, in dem es eben ein massiver Nachteil ist, dass PHP Sie derart in Ihren Möglichkeiten einschränkt. Seit Version 5.3 gibt es allerdings eine Reihe an Klassen in der PHP Standard Library, die weitere Datenstrukturen realisieren. Bei O’Reilly wird betont, dass es ein großer Vorteil ist, dass diese in C programmiert ist, was in einer höheren Geschwindigkeit resultiert. Nur bedeutet das für Sie, dass Sie hier keine individuelle Anpassung durchführen können.\\

In PHP können Sie deshalb leider nur zwei Arten von Datenstrukturen direkt verwenden: Felder und Assoziative Felder. Gleich vorweg: Fehlgeleitete Naturen bezeichnen Assoziative Felder als Hash-Tables. Ein Hash-Table ist aber etwas ganz anderes. Was genau, das lernen Sie z.B. in Veranstaltungen wie Algorithmen und Datenstrukturen. Und nein, das was bei Java als Hash-Table bezeichnet wird ist ebenfalls kein Hash-Table...\\

Nach dieser ausführlichen Einleitung, die vorrangig dazu dient, dass Sie sich im Laufe Ihres Studiums intensiv mit Algorithmen und Datenstrukturen beschäftigen, kommen wir nun zu den beiden Feldertypen, die PHP anbietet:

\subsection{Felder bzw. Arrays}

Ein Array ist eine Art Liste, bei der jeder Eintrag über eine Zahl angesprochen werden kann. Die Nummerierung beginnt jeweils bei 0, sodass Sie den fünfzigsten Wert eines Arrays über die Nummer 49 abrufen.\\

Leider wissen die Entwickler von PHP nicht, was ein Array ist, denn das, was sie als Array implementiert haben ist eine verkettete Liste mit Indizierung jedes Elements. Damit haben die Entwickler von PHP eine Datenstruktur entwickelt, die weder den Geschwindigkeitsvorteil eines Array bietet, noch die Flexibilität beim Einfügen und Entfernen von Elementen einer verketteten Liste.\\

Deshalb hier zunächst die Erklärung, was ein Array ist:\\

Ein Array ist ein Teil des Speichers eines Computers, der in gleichgroße Einheiten unterteilt werden, die genau so groß sind wie es für eine bestimmte Variable nötig ist. Denn bei der Programmierung eines Arrays wird am Anfang festgelegt, welchen Datentyp alle Variablen haben sollen. \\

Der Vorteil eines Array besteht darin, dass das Auslesen oder Überschreiben einer Stelle des Arrays sehr schnell zu erledigen ist. Dafür ist diese Datenstruktur sehr ineffizient, wenn wir ein Element löschen wollen oder ein neues Element hinzufügen wollen, denn beides erfordert bei einem Array einen sehr großen Kopieraufwand. Eine Vergrößerung eines Arrays (z.B. indem ans Ende noch zehn neue Elemente angehängt werden) ist dagegen unmöglich, weil ja in der Zwischenzeit der Speicher \glqq{}hinter\grqq{} dem Array in aller Regel mit anderen Daten beschrieben wurde. \\

Auch hier gilt wieder: Wenn Sie das im Detail verstehen wollen, belegen Sie bitte einen Kurs in Algorithmen und Datenstrukturen.\\

Und jetzt zu dem, was Felder in PHP können, die dort als Arrays bezeichnet werden:\\

Genau wie bei echten Arrays können Sie dort einen Eintrag des Array mithilfe eines Indexes überschreiben oder auslesen. Sie können aber zusätzlich jederzeit neue Elemente ans Ende des Feldes anfügen.\\

Verstehen Sie die hier geäußerte Kritik bitte nicht falsch: Die Datenstrukturen, die PHP anbietet sind sinnvoll. Allerdings gibt es hier ein Problem: Sie lernen jetzt etwas unter dem Namen Array kennen, das kein Array ist. Wenn Sie dann eine andere Programmiersprache kennen lernen, die Ihnen ein echtes Array unter dem Namen Array anbietet werden Sie im schlimmsten Fall nicht verstehen, warum Sie es nicht so programmieren können, wie Sie das bei PHP gelernt haben.\\

Zum anderen müssen Sie bei komplexen Aufgabenstellungen mit hohen Datenmengen bestimmte Datenstrukturen verwenden, damit das Programm nicht zu langsam wird. Und genau das ist in PHP nicht möglich. Viele Webanwendung (oder auch Browsergames) sind also nicht deshalb langsam, weil die Internetanbindung zu langsam ist, sondern weil PHP mit der fehlenden Möglichkeit zur Implementierung von Datenstrukturen einen Flaschenhals darstellt.

\subsubsection{Programmierung eines Arrays}

Um ein PHP-Array zu erzeugen können Sie einerseits die Zuordnung \$name = array(ersterWert, zweiterWert, ...); verwenden.\\

Sie können aber auch direkt über \$name[1] = zweiterWert; ein Array erzeugen, indem Sie einem Eintrag des zu erzeugenden Arrays einen Wert zuordnen.\\

Um einen Eintrag auszulesen, verwenden Sie dann beispielsweise \$name[27], um das 28-igste Element des Array auszugeben. Sie erhalten durch diesen Zugriff wie gewohnt eine Variable. Aber diese ist nicht anonym, denn \$name[27] ist genau wie \$a ein Name für eine Variable. Dass es außerdem einen Zugriff auf das Element eines Arrays darstellt ist davon unabhängig.

\subsubsection{Assoziative Felder (eigentlich Dictionary)}
Der Unterschied zwischen einem Feld und einem assoziativen Feld besteht darin, dass die Indizes eines assoziativen Feldes Texte sind, also etwas, das zwischen ``...`` steht. Hier wird dann nicht mehr vom Index, sondern vom Schlüssel gesprochen. Und weil der Rechner nicht ahnen kann, welchen Schlüssel Sie verwenden wollen, müssen Sie die Schlüssel bei der Erzeugung explizit angeben.\\

Zum Vergleich: Um Elemente in ein Array einzutragen, konnten Sie bei der Erzeugung des Arrays \$array = array(``Kuchen``, ``Brötchen``, ...); eingeben. Dadurch wurde festgelegt: \\

\$array[0]=``Kuchen``; \$array[1]=``Brötchen``;\\

Um ein assoziatives Feld zu erzeugen, können sie auch wieder auf zwei Arten vorgehen: Entweder Sie legen mehrere Einträge fest: \$name = array(``Erster Eintrag`` => 22, ``Zweiter Eintrag`` => 234234, ...); Wenn Sie ein Array so programmieren könnten, dann würde das so aussehen: \$array = array(0 => ``Kuchen``, 1=>``Brötchen``, ...); Aber so programmieren Sie kein Array.\\

Einzelnen Eintrag für ein assoziatives Feld programmieren Sie so: \$name[``Noch ein Eintrag`` => 232];\\

Der Folgepfeil => dient hier schlicht dazu, um aufzuzeigen, dass Sie einen Schlüssel und einen Wert hinzufügen, während Sie ja beim Feld einfach einen Wert hinzufügen, der dann in die nächste Zeile des Arrays eingefügt wird.\\

Wichtig: Wie beim \glqq{}Array\grqq{} haben die Entwickler von PHP auch hier eine falsche Bezeichnung verwendet: Eine solche Datenstruktur wird als Dictionary bezeichnet (engl. für Wörterbuch, bzw. Nachschlagewerk). \\

Bei einem Dictionary handelt es sich um eine Ansammlung von sogenannten key-value-Paaren. Der key (dt. Schlüssel) ist ein beliebiger Wert, unter dem die Datenstruktur Werte oder Datenstrukturen als value (dt. Wert) speichert. Ein echtes Wörterbuch könnten sie beispielsweise so speichern: Sie erzeugen einen Dictionary, der die Buchstaben von a bis z als keys speichert. Unter jedem key könnten Sie nun einen weiteren Dictionary speichern, der ebenfalls die Werte von a bis z als keys hat, wenn es Wörter gibt, die mit den beiden keys der beiden Dictionaries anfangen. Und auch die values dieser Dictionaries hätten wieder jeweils einen Dictionary mit den keys von a bis z als Wert, usw. usf. Das wäre zwar nicht unbedingt effizient, aber wie gesagt könnten Sie so ein Wörterbuch als Datenstruktur speichern.\\

Aufgabe:\\

-	Wie lautete der Befehl, um Wert zum Schlüssel ``Horst`` aus dem assoziativen Feld \$Kneipengaeste auszulesen?

\section{Überarbeitung: Wiederholungen von Programmteilen (Schleifen und Rekursionen)}

Im letzten Teil der Einführung in die prozedurale Programmierung mit PHP sprechen wir darüber, wie wir Teile unseres Programms so lange wiederholen können, bis eine bestimmte Bedingung erfüllt ist. In den meisten Einführungen zu diesem Thema lernen Sie lediglich Schleifen kennen. Leider haben Schleifen aber eine ganz böse Einschränkung, die bei den besagten Einführungen ignoriert wird: Sie ermöglichen es nicht, mehrere Ausführungen parallel zu erlauben. Hier sprechen wir allerdings nicht über \glqq{}echte\grqq{} Parallelprogrammierung; die würde für einen Kurs im ersten Studienjahr etwas zu weit führen.\\

Damit Sie verstehen, wovon hier die Rede ist, folgt ein Beispiel, bei dem Wiederholungen eine Rolle spielen, die pseudo-parallel ausgeführt werden. Pseudo-parallele Abläufe sind Dinge, die unabhängig voneinander passieren können und scheinbar gleichzeitig ausgeführt werden. Nehmen wir an, Sie erstellen ein Programm für Vertreter eines Unternehmens. Diese Vertreter benötigen eine Reiseplanung, bei der sie jeweils Kunden besuchen, die in der Nähe voneinander wohnen. Nun sind die Daten der Kunden aber nach dem Namen sortiert. Was Sie also programmieren müssen ist eine Sortierung nach der Postleitzahl.\\

Und hier sind wir wieder bei einem Problem, das Informatikstudierende in einer Veranstaltung mit dem Namen Algorithmen und Datenstrukturen besprechen. Denn die Frage ist jetzt: Wie können wir dieses Umsortieren so durchführen, dass es auch bei Zehntausenden von Kunden noch schnell funktioniert.\\

Hier eine Übung, die Ihnen veranschaulicht, dass die zeitlichen Unterschiede beim Sortieren störend sein können: \\

-	Ein naiver Ansatz zum Sortieren sähe so aus: Sie vergleichen die Postleitzahl des ersten Kunden mit der des zweiten Kunden und tauschen ggf. beide aus. Dann vergleichen Sie die Postleitzahl des zweiten Kunden (der evtl. ursprünglich der erste Kunde war und jetzt durch den Tausch zum zweiten Kunden wurde) mit der des dritten Kunden und machen so weiter, bis Sie einmal zum letzten Kunden gekommen sind. Das wiederholen Sie so oft, bis Sie keine Kunden mehr vertauschen. Nehmen wir an, Sie hätten eine Anzahl Kunden, die wir mit n bezeichnen. Dann müssen Sie im Extremfall (auch bekannt als worst case) n-mal (n-1) Vergleiche (also n²-n Vergleiche) durchführen und jedes Mal einen Austausch durchführen. Bei zehntausend Kunden dauert das selbst auf einem schnellen Rechner knapp eine Minute. Es sind rund 108 Operationen.\\

So etwas können Sie mit den sogenannten Schleifen programmieren.\\


-	Nehmen wir dagegen an, Sie teilen Ihre Kunden in zwei Gruppen, ohne dabei zu sortieren. Und jede dieser beiden Gruppen teilen Sie dann wieder in zwei Gruppen. Und das machen Sie so lange, bis Sie nur noch Gruppen mit einem oder zwei Kunden haben. Jetzt kommt der pseudo-parallele Teil: Nun sortieren Sie die Kunden jeder Gruppe. Das macht also in jeder Gruppe höchstens eine Sortier- und Austauschoperation. Und da wir durch die Teilungen genau halb so viele Gruppen wie Kunden haben, macht das maximal n/2 Vergleichsoperationen. \\


Im nächsten Schritt nehmen wir jeweils zwei unserer kleinen Gruppen. Hier vergleichen wir das erste Element der ersten Liste mit dem ersten Element der zweiten Liste. Das größere speichern wir in einer neuen Liste. Danach haben wir maximal noch zwei Vergleiche und schon ist aus unseren zwei kleinen Listen eine größere geworden. Da wir jetzt n/4 Listen haben, die jeweils maximal dreimal verglichen werden mussten, haben wir maximal n/2 + n/4 Operationen für diesen Schritt unserer Sortierung gebraucht. (Wenn Sie nachgerechnet haben, fragen Sie sich vielleicht, warum es nicht 3 * n/4 sind. In dem Fall fragen Sie am besten Ihren Mathedozenten, damit er Ihnen nochmal das Bruchrechnen erklärt.) Bis jetzt haben wir also 2 * n/2 + 1 * n/4 Operationen.\\


Von jetzt an führen wir den letzten Schritt immer wieder auf die neuen Listen aus, bis wir eine sortierte Liste erhalten. Wenn Sie dabei jeden Schritt genau notieren und die Anzahl Operationen berechnen, alles summieren und Ihre mathematischen Kenntnisse anwenden, erhalten Sie am Ende eine Anzahl, die sich in einen Ausdruck umformen lässt, der n * log2 n ähnelt. Wenn wir nun wieder unsere 10.000 Kunden für n einsetzen, dann landen wir bei ca. 1,4 * 105. Diese Sortierung ist also fast 1.000-mal so schnell wie die oben genannte. Während wir bei der Sortierung mit einer Schleife knapp eine Minute warten müssen, ist bei dieser Art der Sortierung das Ergebnis scheinbar sofort da. Was meinen Sie, welche Variante Ihr Auftraggeber haben möchte.\\


Obwohl auch diese Variante Wiederholungen beinhaltet, können Sie sie NICHT mit einer Schleife programmieren, sondern ausschließlich mit einer sogenannten Rekursion. Denn wie Sie sehen werden hier einige Aufgaben pseudo-parallel erledigt und das ist mit einer Schleife unmöglich.\\

Das was Sie gerade gesehen haben ist übrigens Informatik für Erstsemester, es ist also nicht allzu anspruchsvoll. Es wird als divide and conquer-Algorithmus bezeichnet. Wie Sie sehen hat das nichts mit der Beherrschung einer Programmiersprache zu tun, sondern mit Logik und Mathematik. Wenn Sie Informatik an einer Universität studieren würden, müssten Sie die mathematischen Teile selbst berechnen und anhand der Ergebnisse beweisen, wie viel effizienter die eine Variante gegenüber der anderen ist. Informatik-Studierende an einer FH müssen dagegen lediglich wissen, welche Effizienz welches Verfahren bietet. Denn Informatiker von Universitäten werden darauf vorbereitet, neue Verfahren zu entwickeln und müssen deshalb im Stande sein, formal zu beweisen, welchen Mehrwert ihr neues Verfahren hat. FH-Informatiker werden dagegen darauf vorbereitet, neue Verfahren sinnvoll in zum Teil neue Anwendungen zu integrieren. Also brauchen sie die Beweisführung nicht zu kennen, müssen aber die Effizienz eines Algorithmus kennen und in Abhängigkeit davon Algorithmen einsetzen. Wenn Sie dagegen einfach \glqq{}nur\grqq{} programmieren wollen, dann suchen Sie eigentlich nach einer Ausbildung zum/zur Fachinformatiker/in.

\subsection{Rekursionen}

Da also Rekursionen wesentlich mehr bieten als Schleifen und meist auch wesentlich schneller arbeiten, kommen wir zuerst zu dieser Art der Wiederholung in Programmen. Rekursionen sind nur eine Sonderform von Funktionen und deshalb gibt es keine neuen Elemente einer Programmiersprache, die Sie jetzt lernen müssen. Das Besondere dabei ist, dass eine Rekursion eine Funktion ist, die nicht sofort einen Rückgabewert zurückgibt, sondern die so lange mit einem jeweils neuen Zwischenergebnis erneut aufgerufen wird, bis eine bestimmte Bedingung erfüllt ist. Das kann sie fast beliebig oft tun. Die maximale Häufigkeit folgt aus der Größe des Computerspeichers. Bei diesen Bedingungen spricht man von Abbruchbedingungen, denn nur wenn sie erfüllt werden hört die Rekursion bzw. der wiederholte Aufruf der Funktion auf.\\

Es gibt übrigens nichts, was Sie mit einer Schleife programmieren können, das nicht auch als Rekursion programmierbar wäre. Deshalb ist es eigentlich unsinnig, sich überhaupt Schleifen anzusehen, aber wir machen es natürlich trotzdem, weil viele Programmierer nur mit Schleifen programmieren können. Und natürlich würde es blöd aussehen, wenn Sie nicht wissen, was eine Schleife ist.\\

Die folgenden Beispiele zeigen, wie Sie einfache Rekursionen programmieren können:
// Rekursion, die den Abstand zwischen zwei ganzen Zahlen berechnet.\\

\begin{verbatim}
function abstand(\$zahl, \$andereZahl){
	if (\$zahl > \$andereZahl) \{
		\$andereZahl += 1;
	\} else if (\$zahl < \$andereZahl) \{
	\$zahl += 1;
\} else \{
return 0;
\}
return 1 + abstand(\$zahl, \$andreZahl);
\}
\$a = 1;
\$b = 10;
\$c = abstand(\$a, \$b);
echo(``Der Abstand zwischen \$a und \$b betägt: \$c.``);
\end{verbatim}
Quellcode 4.20: Rekursion um der Rekursion willen\\


Der Anfang der Funktion sollte Ihnen klar sein: \\

-	Wir haben hier drei Fälle: 

•	\$zahl > \$andereZahl
•	\$andereZahl > \$zahl 
•	\$zahl = = \$andereZahl (das ist der else-Zweig).

-	In jedem der beiden ersten Zweige wird die Zahl um eins erhöht, die kleiner als die andere Zahl ist. 

-	Wenn der Abstand gleich Null ist, ist die Abbruchbedingung erreicht, also muss hier ein Wert zurückgegeben werden und die Funktion darf nicht nochmal aufgerufen werden.

-	Die letzte Zeile der Funktion ist dann das, was die Funktion zu einer Rekursion macht: Hier wird die Funktion mit neuen Werten erneut aufgerufen.
Aber warum steht hier 1 + vor dem Rekursionsaufruf? Ganz einfach: Da der Abstand zwischen den beiden Zahlen um 1 geringer geworden ist, müssen wir irgendwie speichern, dass er ursprünglich um ein 1 größer war. Und das passiert durch dieses 1 +.

-	Eine andere häufige Fragen an dieser Stelle lautet: Wie soll denn das Programm weiterlaufen, wenn es hinter dem return einen Funktionsaufruf hat? Das ist im Grunde ganz simpel: Das Programm gibt hier erstmal nichts zurück, sondern es führt quasi eine Zwischenspeicherung des Zustands durch, ruft dann die Funktion mit den neuen Parametern aus und erst, wenn von dieser Funktion ein Rückgabewert kommt, berechnet es den Rückgabewert dieser Funktion aus. Und das geht relativ oft (einige hundert- bis tausendmal).

\subsubsection{Aufgabe}

-	Anhänger statisch typisierter Programmiersprachen würden diese Funktion zum Beweis nutzen, dass dynamisch typisierte Sprachen unsicher sind. Was genau führt dazu, dass diese Funktion endlos laufen kann, bzw. mit einer Fehlermeldung abbricht?

-	In anderen Worten: Was fehlt hier, das aber längst nicht bei jeder Rekursion nötig ist?\\

Nehmen wir jetzt unsere Sortieraufgabe. Da dieser Kurs sich an Einsteiger richtet, wird die folgende Rekursion in Pseudo-Code notiert; in PHP wäre sie umfangreicher und damit nicht mehr übersichtlich. Eine detaillierte Erklärung zum Ablauf folgt nach dem Pseudo-Code.\\

// Rekursion, die eine Liste von Zahlen nach dem devide \& conquer-Ansatz sortiert.\\

\begin{verbatim}
function sortList(Liste){
	if(Liste enthält mehr als eine Zahl){
		\$geordneteListe = teileListe(Liste)
	} else return Liste;
	return \$geordneteListe;
}

/* Diese Hilfsfunktion teilt die Liste in zwei Teillisten auf und ruft außerdem die Sortierung für jede der Teillisten auf. */

function teileListe(Liste){
	\$n = Anzahl Zahlen in der Liste;
	return sortListen(neue Liste mit den ersten n/2 Zahlen der Liste, neue Liste mit den restlichen Zahlen der Liste);
}

/* Zusammen mit teileListe() teilt diese Funktion die Liste so lange auf, bis sie nur noch aus Listen mit einem Element besteht. Danach wird aus den Teillisten rekursiv eine Gesamtliste erstellt. */

function sortListen(Liste1, Liste2)
if(Liste1 enthält mehr als eine Zahl){
	\$geordneteListe1 = teileListen(Liste1)
}
if(Liste2 enthält mehr als eine Zahl){
	\$geordneteListe2 = teileListen(Liste2)
} 
return mergeListen(\$geordneteListe1, \$geordneteListe2);
}  

/* In der folgenden Funktion wird öfter += verwendet. Das soll in diesem Fall bedeuten, dass das Programm etwas an eine Liste anfügen soll. */

function mergeListen(Liste1, Liste2){
	if(Erste Zahl von Liste1 > Erste Zahl von Liste2) {
		\$geordneteListe += Erste Zahl von Liste1;
		Lösche die erste Zahl aus Liste1;
		if (Liste1 ist nicht leer){
			return \$geordneteListe += mergeListen(Liste1, Liste2);
		} else return \$geordneteListe += Liste2;
	}
	else {
		\$geordneteListe += Erste Zahl von Liste2;
		Lösche die erste Zahl aus Liste2;
		if (Liste2 ist nicht leer){
			return \$geordneteListe += mergeListen(Liste1, Liste2);
		} else return \$geordneteListe += Liste1;
	}
}
\end{verbatim}
Pseudo-Code 4.21: Beispiel für eine Endrekursion.\\

Was passiert hier?\\

Anfangs wird die Funktion sortListe mit einer Liste aufgerufen. Da wir aber in Zukunft immer zwei Listen haben werden (schließlich bewältigen wir das Problem, indem wir die Liste immer wieder teilen) brauchen wir eine zweite Funktion, die bei jedem Aufruf zwei Listen annehmen kann. Bei solchen Fällen, in denen wir eine Hilfsfunktion vor der eigentlichen Rekursion haben reden wir von einer Endrekursion.\\

Die erste Funktion prüft zunächst, ob unsere Liste schon sortiert ist, was dann der Fall ist, wenn sie entweder leer ist oder nur ein Element enthalt. Ist das der Fall, dann wird die Liste so zurückgegeben, wie sie ist. Das passiert oben im else-Zweig. Sonst wird die Liste in zwei Hälften zerlegt, die gleichgroß sind, bzw. deren Größe sich maximal um 1 unterscheidet. Diese beiden Listen werden nun an die eigentliche Rekursion übergeben, die die Listen immer weiter zerlegt. Diese Zerlegung passiert in der zweiten Funktion.\\

Die beiden ersten if-Cases der zweiten Funktion sorgen lediglich dafür, die Listen so lange zu teilen, dass am Ende nur mehr Listen mit einem Element existieren.\\

Damit kommen wir zur dritten Funktion mergeListen(), das den zweiten Teil unserer Rekursion löst. mergeListen() vergleicht nur die erste Zahl der beiden übergebenen Listen und fügt das größere der beiden in eine andere Liste an, die \$geordneteListe heißt und die später zurückgegeben wird. Danach wird diese größte Zahl aus der ursprünglichen Liste gelöscht. Wenn in beiden Listen noch wenigstens eine Zahl ist, werden die beiden Listen mit mergeSort() aufgerufen und das Ergebnis dieses rekursiven Aufrufs wird an \$geordneteListe angehängt.\\

Ist dagegen eine der beiden Listen leer, dann wird einfach der Rest der verbliebenen Liste an \$geordneteListe angehängt, bevor \$geordneteListe zurückgegeben wird.
An dieser Stelle fragen Sie sich vielleicht, wo denn die rekursiven Schritte nach dem Zusammenfügen (mergen) der Listen auftauchen, die aus mehr als einem Element bestehen. Schauen wir uns dazu den Ablauf an, nachdem aus jeweils zwei Listen mit einem Element eine sortierte Liste mit zwei Elementen geworden ist:\\

Der Rückgabewert (also unsere sortierte Liste) landet in der Funktion sortListen(), die ihn wiederum an die Funktion teileListen() zurückgibt. Und dort wird dieser Rückgabewert entweder als \$geordneteListe1 oder als \$geordneteListe2 gespeichert. Damit wird diese kurze geordnete Liste zusammen mit einer anderen geordneten kurzen Liste an die Funktion mergeListen() übergeben, die dann diese beiden kurzen zu einer längeren geordneten Liste zusammenfügt.\\

Ganz zum Schluss (wenn die ganzen Teillisten endlich wieder zu einer zusammen gefügt wurden), landet der Rückgabewert von mergeListen()wieder im if-Case unserer anfänglichen Hilfsfunktion sortListe() und wird von dieser schließlich als Rückgabewert ausgegeben.

\subsubsection{Aufgabe}

-	Programmieren Sie die oben in Pseudo-Code gegebene Rekursion mit einem Array in PHP.

-	Begründen Sie, warum diese Rekursion im Gegensatz zum vorigen Beispiel nicht endlos weiterlaufen kann.

\subsection{Schleifen}

Bei Schleifen führen Sie im Gegensatz zu Rekursionen einen Programmteil immer auf die gleiche Weise mehrfach aus. Sie können hier zwar auch Kontrollstrukturen programmieren, um ggf. genauer zu steuern, wie die einzelnen Schleifendurchläufe programmiert werden sollen, aber eine pseudo-parallele Ausführung wie bei Rekursionen ist hier nicht möglich.

\subsubsection{foreach-Schleifen, for- und while-Schleifen}

Wenn Sie mit jedem Eintrag eines Feldes etwas tun wollen, dann brauchen Sie dafür keine Rekursion zu programmieren. Sie können stattdessen eine foreach-Schleife nutzen. Dazu brauchen Sie den Bezeichner des Arrays und Sie vergeben zusätzlich eine Variablenbezeichnung (im Beispiel \$element). Bei jedem Durchlauf der foreach-Schleife ist der Wert der von \$element der Wert eines Indexes des Arrays. \\

Wenn am Anfang nicht feststeht, wie oft die Schleife durchlaufen werden soll, müssen Sie eine while-Schleife nutzen, sonst können Sie auch eine for-Schleife nutzen.\\

While-Schleifen gibt es in zwei Varianten: Bei der while-Schleife wird er nur so lange ausgeführt, wenn eine bestimmte Bedingung erfüllt ist. Bei der do-while-Schleife wird er dagegen immer einmal ausgeführt, danach erfolgt die Ausführung wie bei einer while-Schleife.\\

Die Bedingung wird bei einer while-Schleife genau so programmiert, wie Sie das bei if-Conditionals kennen gelernt haben. Allerdings sollte sich das Ergebnis der Bedingung im Laufe der Schleifenausführung ändern, denn sonst haben Sie eine Endlosschleife. Das ist ein Programmteil, in dem das Programm stecken bleibt, bis es vom Nutzer komplett abgebrochen wird.\\

Bei for-Schleifen müssen drei Dinge festgelegt werden: \\

-	Eine Variable mit einem ganzzahligen Wert, 
-	Eine Abbruchbedingung, bei der diese Variable genutzt wird.
-	Ein Befehl, der steuert, wie sich der Wert nach je einer Ausführung der Schleife ändert.\\

Hier für jede der Schleifen ein Beispiel:\\

\begin{verbatim}
\$a = 2;
\$b = 10;
\$array ist ein Array, das Sie mit Zahlen gefüllt haben.


// foreach-Schleife: Das Schlüsselwort as müssen Sie hier immer in dieser Art programmieren:
foreach(\$array as \$element) {
	/* Im foreach-Schleifenrumpf programmieren Sie schlicht, was mit dem aktuellen Element des Arrays passieren soll. */
}

// Einfache while-Schleife
while(\$a < \$b){
	\$a += 1;
}

// Einfache do-while-Schleife
do {
	\$a +=1;
} while (\$a < \$b);

// Einfache for-Schleife
for (\$i = 0; \$i < 5; \$i++){
	if (\$array[i] < i) { ... }
	else ...
}
\end{verbatim}
Quellcode 4.22: Schleifen

\section{Überarbeitung: Hinweis bezüglich objektorientierter Softwareentwicklung in PHP}

PHP ermöglicht es Ihnen auch, klassenbasierte objektorientierte Programme zu entwickeln. Objektorientierte Softwareentwicklung geht aber weit über die Nutzung zusätzlicher Programmiermöglichkeiten in prozeduralen Sprachen hinaus. Deshalb müssten wir hier im Grunde zunächst eine umfangreiche Einführung in das Konzept durchführen, bevor wir uns ansehen könnten, wie es in PHP umgesetzt wird. Und dafür bleibt in diesem Kurs leider keine Zeit.\\

Diejenigen von Ihnen, die die Entwicklung von Webanwendungen vertiefen wollen, sollten diesem Thema aber bearbeiten. An dieser Stelle sei noch folgendes gesagt: Werfen Sie jeden Kurs in die Mülltonne, der Ihnen Objektorientierung damit erklärt, dass Sie dort Objekte der realen Welt modellieren. Das ist zwar mit objektorientierter Programmierung möglich, aber wer ernsthaft glaubt, dass es bei objektorientierter Softwareentwicklung darum geht, hat ungefähr so viel Ahnung vom Thema wie ein Maulwurf vom Fliegen. Dem fliegt ja gelegentlich mal was aufs Dach...

\section{Überarbeitung: Auswertung von Nutzereingaben in HTML durch PHP}

Alles, was Sie bislang über PHP gelernt haben erlaubt Ihnen zwar, dynamisch HTML-Code in HTML-Dokumente zu integrieren, aber bislang können Sie noch keine Nutzereingaben verarbeiten. Damit können Sie also bislang weder einen Log-In, noch eine Registrierung realisieren. Ein Spiel könnten Sie hier realisieren, indem Sie eine Funktion über das onclick-Attribut eines HTML-Containers aufrufen lassen.\\

Im Folgenden werden wir grundsätzlich mit Daten arbeiten, die Nutzer in Formulare eintragen oder über das Aktivieren von Schaltflächen auslösen. Bei der Einführung in Formulare in HTML haben Sie all die Attribute kurz angesprochen, die dafür nötig sind. Deshalb schauen wir uns zunächst an, welche das sind.\\

Wichtig: Sie können zwar an beliebigen Stellen eines HTML-Dokuments HTML-Code durch PHP generieren, aber Eingaben von Nutzern können ausschließlich über HTML-Container durchgeführt werden, die entweder in Formularen eingesetzt werden sollen oder in denen das Attribut onclick programmiert werden kann.\\

Zunächst müssen Sie jeden <form>-Container um zwei Attribute erweitern, wie schon zu Beginn dieses Kapitels erläutert: Das action-Attribut erhält als Wert die URL der PHP-Datei, die die Auswertung durchführt. Das method-Attribut erhält das Wort POST als Wert, auch wenn hier alternativ GET möglich wäre. Es steuert, wie die Daten vom Client zum Server übertragen werden. Kurz gesagt muss ein Angreifer bei der Nutzung von POST etwas mehr Arbeit investieren, um die Datenübertragung abzuhören. Das ist zwar nicht viel und ersetzt auch keine Verschlüsselung, aber es ist ein Schritt in die richtige Richtung.\\

Außerdem müssen Sie jeder Eingabemöglichkeit (außer Buttons) ein name-Attribut zuweisen. Der Wert jedes name-Attributs darf innerhalb eines Formulars nur einmal verwendet werden.\\

Hier ein HTML-Code-Fragment, wenn Sie sich unsicher sein sollten:\\

\begin{verbatim}
...
<form action=registrierung.php method=POST>
<label ...
<input id= userid name=userid>
...
\end{verbatim}
Quellcode 4.23: HTML-Formular für die Verarbeitung mit PHP\\

Im Quellcode oben wird also eine Variable mit der Namen userid erzeugt, deren Wert an die Datei registrierung.php übertragen wird, nachdem Nutzer die entsprechende Schaltfläche der Webanwendung angewählt haben. Alle Eingaben, die dabei übertragen werden werden in einem assoziativen Feld mit dem Namen \$\_POST gespeichert. Um also die Variable userid in Ihrem PHP-Programm zu verwenden, müssen Sie lediglich den Wert von \$\_POST[``userid``] einer Variablen Ihres PHP-Programms zuweisen. \\

Das ist alles. 

\subsection{Aufgabe}

-	Programmieren Sie eine Checkbox in den Registrierungs- und Log-In-Formularen Ihrer Webpage. Wenn diese Checkbox aktiviert ist, soll die Eingabe des Passworts am Monitor lesbar sein. Ist sie deaktiviert, dann soll das Passwort nicht lesbar sein.

\section{Überarbeitung: Abschluss}

Sie haben jetzt alles kennen gelernt, was Sie benötigen, um eine dynamische Webpage zu programmieren. Wenn Sie beispielsweise ein Programm entwickeln wollen, um einen Schaltplan für die Laborversuche in Elektrotechnik zu entwerfen, wissen Sie jetzt alles, was dazu nötig ist. Wenn Sie denken, dass das ein langweiliges Programm wird, kann ich Ihnen leider nur zustimmen, aber dafür ist die Fehleranfälligkeit niedrig.\\

Es gibt noch einige Themen, die wir in diesem Kapitel nicht besprochen haben, die Sie sich aber ansehen sollten, wenn Sie langfristig als Webentwickler mit PHP arbeiten wollen. Neben der klassenbasierten objektorientierten Programmierung wären da die folgenden Dinge zu nennen:\\

-	Welche Datentypen kennt PHP und wie können Sie den Datentyp einer Variablen erfahren?
-	Wie werden Exponentialzahlen in PHP programmiert?
-	Was genau ist der Unterschied zwischen POST und GET und wie müssen Sie Werte in einem PHP-Programm verarbeiten, die per GET übertragen wurden?
-	Welche Reihenfolge gilt bei Operatoren? (Bsp.: Wird = = vor < = ausgewertet oder umgekehrt?)
-	Was sind Pseudo-Zufallszahlen und wie können Sie sie für ein PHP-Programm nutzen?
-	Was bewirken break und continue bei Schleifen?
-	Was hat es mit include und require bzw. include\_once und require\_once auf sich?
-	Was ist der Unterschied zwischen Kopie und Referenz bei einem Funktionsaufruf?
-	Wie können Sie Funktionen programmieren, die eine dynamische Anzahl von Parametern annehmen können? (Auch bekannt als variable Parameterliste)
-	Welche Funktionen bietet PHP an, um z.B. mathematische Funktionen zu realisieren?
-	Mit welchen Funktionen können Sie Strings z.B. auf eine bestimmte Länge kürzen?
-	Was hat es mit dem Gültigkeitsbereich einer Variablen auf sich?
-	Was ist eine Generatorfunktion?
-	Welche Fehler können Sie durch die sogenannten Exceptions lösen? Und warum ist das wichtig?
-	Wie können Sie Verschlüsselungsverfahren einprogrammieren?
-	Wie können Sie Nutzereingaben ohne eine Datenbank oder den Einsatz von Cookies und ähnlichem so lange speichern, wie der Nutzer sie braucht? 


\chapter{Funktionalität mit PHP 5.6}
%\chapter{Einbindung einer Datenbank mit MySQL}

Im Studiengang Medientechnik wird dieser Teil ausführlich in der Veranstaltung P2 behandelt, die Media Systems Studierenden lernen das Thema intensiv in der Veranstaltung RDB kennen. Deshalb stellt dieses Kapitel nur eine oberflächliche Einführung dar. Sie lernen hier also nur das nötigste, um eine Datenbank mit MySQL an eine Webanwendung anzubinden.\\

\textbf{Wichtig}: Bitte beachten Sie, dass es sich bei diesem Kapitel um eine ganz oberflächliche Einführung in die Programmierung von Datenbanken (genauer von relationalen Datenbanken) handelt. Alles, was Sie hier lernen können sind einige ganz simple Anweisungen, um Daten in einer Datenbank zu speichern und um Daten aus einer Datenbank auszulesen. Die eigentliche Entwicklung von Datenbanken ist dagegen ein sehr umfangreiches Fachgebiet der Informatik. Und auch wenn relationale Datenbanken zunächst sehr simpel erscheinen ist das, was im Hintergrund passiert alles andere als simpel. Für eine ernsthafte Einarbeitung in die Programmierung von Datenbanken im Sinne eines Studiums ist dieses Kapitel also keinesfalls geeignet.

\subsection{Das ist eine Datenbank}

Eine Datenbank ist im Grunde nichts anderes als eine Datenstruktur mit wenigen zusätzlichen Eigenschaften. Die meisten Entwickler denken bei Datenbanken an die sogenannten relationalen Datenbanken (RDB) und in diesem Kurs werden wir uns auch keine anderen Datenbanken ansehen. Der Grund ist recht einfach: Wie immer in der Programmierung hängt die Wahl der Mittel davon ab, was wir wollen. Und für einfache Webanwendungen brauchen wir etwas, worin wir systematisch Daten speichern können, die wir relativ schnell laden können. Außerdem ist es wichtig, dass die gespeicherten Daten konsistent sind. Und dafür ist eine relationale Datenbank eine gute Lösung.\\

Konsistenz bedeutet so viel wie, dass etwas überall gleich ist. Nehmen wir als Beispiel eine Datenbank, die die Kontostände von Bankkunden beinhaltet. Sie fragen sich jetzt vielleicht, wie es denn möglich sein soll, dass der Kontostand eines Bankkunden zu einem Zeitpunkt inkonsistent sein soll, schließlich kann ein  Bankkonto doch immer nur einen Stand haben. In der Praxis gibt es aber tatsächlich Situationen, in denen ein Bankkonto unterschiedliche Stände zum gleichen Zeitpunkt haben kann, wenn wir ein entsprechend schlecht entwickeltes Programm entwickeln. Stellen Sie sich dazu vor, Sie heben von Ihrem Konto Geld ab und zur gleichen Zeit bucht jemand von Ihrem Konto Geld ab. Im Hintergrund passiert jetzt folgendes: Der Geldautomat, an dem Sie Geld abheben liest den Kontostand aus der Datenbank aus, reduziert den Wert um die Summe, die Sie abgehoben haben und speichert den neuen Kontostand in der Datenbank. Bei der Abbuchung passiert das gleiche: Der Kontostand aus der Datenbank wird ausgelesen, der Abbuchungsbetrag abgezogen und der neue Kontostand gespeichert.\\

Relationale Datenbanken werden unter anderem so entwickelt, dass der Fehler, den Sie in der folgenden Aufgabe benennen sollen nicht vorkommen kann. Datenbanken, die dagegen die sogenannten Transaktionen nicht kennen, sind gegen solche Fehler nicht abgesichert.

\subsubsection{Aufgabe:}

Warum kann (!) es wie bei diesem Beispiel dazu kommen, dass Ihr Kontostand in der Datenbank nur um die Summe reduziert wird, die Sie abgehoben haben oder nur um die Summe, die abgebucht wurde?\\

Tipp: Wenn Sie nicht auf die Lösung kommen, dann setzen Sie mit Sicherheit etwas voraus, das hier nicht gegeben ist, etwas das also erst noch programmiert werden müsste.\\

Relationale Datenbanken sind in sogenannte Relationen unterteilt. Stellen Sie sich unter einer Relation so etwas wie eine Tabelle vor, bei der jede Spalte eine Überschrift haben muss.\\

Die Spalten einer Relation nennt man Attribute. Und weil wir in den einzelnen Zellen der Relation Werte eintragen, werden die bei Relationen als Attributwerte bezeichnet. Der Begriff des Attributs sollten Ihnen bekannt vorkommen: In der Programmierung mit PHP hatten Sie es mit Eigenschaften zu tun, die Sie dort als Variablen kennen gelernt haben. In der Programmierung mit HTML haben Sie sie dagegen als Properties kennen gelernt. Es gibt hier zwar jeweils gewisse Unterschiede, aber das Grundprinzip ist jeweils gleich: Über Attribute, Variablen und Properties legen wir Eigenschaften fest, indem wir Ihnen jeweils einen Wert zuordnen.\\

Genau wie Variablen haben Attribute einen Datentyp. Bei Datenbanken gibt es aber nicht die selben Datentypen wie in PHP: Für unsere Zwecke nutzen Sie schlicht die Typen character (für Zeichen bzw. Strings), int (für ganze Zahlen) und double (für Fließkommazahlen).\\

Zusätzlich müssen Sie jedoch bei der Erzeugung einer Relation noch die maximale Länge jedes Attributs angeben.\\

Auch für die Zeilen einer Relation gibt es wieder einen Begriff: Diese werden als Datensätze bezeichnet.

\section{Transaktionen – Die „Befehle“ für relationale Datenbanken}

Auch bei der Arbeit mit Datenbanken begegnen Ihnen wieder neue Begriffe, die im Grunde für Dinge stehen, die Sie schon bei der Programmierung in HTML und PHP kennen gelernt haben. Wenn Sie sich in die Datenbankprogrammierung vertiefen wollen, wird es wichtig, dass Sie hier die genauen Unterschiede verstehen, aber für unseren Kurs sind diese Unterschiede nicht so wichtig.\\

Im Alltag arbeiten Sie mit Datenbanken ähnlich wie mit Variablen: Sie speichern in Datenbanken Werte ab oder lassen sich von Datenbanken Werte ausgeben. Aber im Gegensatz zu Variablen können Sie nicht direkt zwei Einträge einer Datenbank vergleichen bzw. andere Operationen darauf ausführen. Wenn Sie beispielsweise in einer Datenbank eine Highscoreliste für ein Spiel gespeichert haben, dann können Sie die Einträge dieser Liste nur dann umsortieren, wenn Sie zunächst die Werte der Datenbank in eine Datenstruktur in Ihrem PHP-Programm geladen haben. Und nach dem Sortieren müssten Sie die Werte aus der Datenstruktur wieder in der Datenbank speichern.\\

Jede „Operation“, bei der Sie Daten aus einer Datenbank auslesen oder in einer Datenbank abspeichern wird bei relationalen Datenbanken als Transaktion bezeichnet. Was relationale Datenbanken im Detail sind, ist für unseren Kurs irrelevant, Sie müssen sich aber merken, dass es noch andere wichtige Datenbanken gibt.\\

Eine ähnliche Bedeutung wie die Transaktion hat der Begriff der Query. Genau wie bei der Transaktion brauchen Sie für den Einstieg nur zu wissen, dass damit irgendeine Datenübertragung an eine Datenbank gemeint ist. Eine Query kann also beispielsweise eine Transaktion sein.\\

Wichtig: Um eine Transaktion auszuführen müssen sie zwischen Ihrem Programm und der Datenbank zunächst eine Verbindung aufbauen.\\

Genauso wichtig: Sobald eine Transaktion beendet ist, müssen Sie die Verbindung zwischen Datenbank und Programm wieder trennen. Wenn Sie das nicht tun, kann es passieren, dass andere Teile des Programms oder andere Nutzer nicht auf die Datenbank zugreifen können.\\

Bei den sogenannten NoSQL-Datenbanken wird davon gesprochen, dass diese keine Transaktionen kennen. Natürlich gibt es auch dort Anweisungen, die verwendet werden, um die Datenbank anzusprechen, doch 

\subsection{Eine Datenbank anlegen}

In einem PHP-Programm erzeugen und verwenden Sie Variablen je nach Bedarf und müssen dazu nichts weiter organisieren. Wenn Sie mit einer Datenbank arbeiten wollen, dann brauchen Sie dazu als erstes einen Server, auf dem die Datenbank läuft. Mit XAMPP, easyPHP und anderen Softwarepaketen haben Sie das schon erledigt, weil diese eben nicht nur dafür sorgen, dass „Ihr Rechner“ PHP „versteht“, sondern eben auch einen Datenbankserver bereit stellen.\\

Wichtig: Der Betrieb eines Servers gehört im Gegensatz zur Nutzung eines Servers zu den anspruchsvollsten Aufgaben, denen Sie sich im Bereich der Informatik widmen können. Das Problem ist dabei nicht, den Server dazu zu bringen, dass z.B. eine Datenbank darauf läuft, das Problem ist vielmehr bei jedem Server, Angriffe abzuwehren und ihn so zu konfigurieren, dass er seine Aufgabe auch bei vielen Anfragen von Nutzern zuverlässig erfüllt. Das diese Aufgabe schwer bis nicht erfüllbar ist, erleben Sie jedes Mal, wenn Sie im Netz eine Seite aufrufen, die nicht erreichbar ist. Das gleiche gilt für die Fälle, wenn sie keine E-Mails abrufen können. Dazu kommt noch, dass die Tätigkeit als Serveradministrator ausgesprochen undankbar ist, denn egal wie gut Sie Ihre Aufgabe erfüllen gilt immer: Erst wenn das System ausfällt, merkt jemand, dass es Sie gibt. Und dann soll die Lösung ganz schnell da sein.\\

Für den Fall, dass Ihnen der Unterschied zwischen dem Betrieb und der Nutzung eines Servers nicht klar ist: Denken Sie an den Unterschied dazwischen, als Kellner zu arbeiten und einem Gast, der etwas bestellt.\\

Jetzt aber wieder zur Erstellung einer Datenbank: Da Sie die im Regelfall nur einmal erstellen werden, nutzen Sie am besten das Programm MyPHPAdmin. Das ist ein Programm, das Sie auch wieder bei easyPHP, XAMPP und ähnlichen Programmpaketen finden.\\

Nachdem Sie MyPHPAdmin gestartet haben, finden Sie auf der linken Seite der Nutzeroberfläche eine Schaltfläche, um eine neue Datenbank anzulegen.\\

Nachdem Sie diese angewählt haben, müssen Sie nur noch zwei Dinge festlegen: Den Namen der Datenbank und die sogenannte Kollation. Die Kollation beinhaltet zwei Dinge: Den Character Set und eine Sortierungsreihenfolge. In HTML haben wir bei der Lokalisierung UTF-8 als Character Set ausgewählt, von daher werden Sie jetzt die vielen möglichen Einträge in der Kollation wahrscheinlich wundern. Denn dort taucht nicht einfach nur utf-8 auf, sondern eine Vielzahl möglicher Werte. Diese verschiedenen Einträge haben einen einfachen Grund: Es geht wie geschrieben bei der Kollation nicht nur um den Schriftsatz, bzw. die Codierung der Zeichen, sondern zusätzlich um die Reihenfolge.\\

Wenn Sie hier eine ungeeignete Kollation auswählen, dann werden beispielsweise Namen mit Ä am Anfang nicht nach Namen einsortiert, die mit A beginnen, sondern erst nach Namen mit Z. Wählen Sie deshalb bitte als Kollation den Eintrag utf8\_unicode\_ci. (Der Eintrag utf8\_general\_ci bewirkt für unsere Zwecke das gleiche.) Wenn Sie hier nichts auswählen, dann wird utf8\_swedish\_ci ausgewählt.\\

Wichtig: Auch wenn Sie HTML mit UTF-8 lokalisiert haben und jetzt die Datenbank auf UTF-8 eingestellt ist, bedeutet das nicht, dass PHP automatisch mit UTF-8 umgehen kann. Um sicher zu stellen, dass in Ihrer Webanwendung tatsächlich Sonderzeichen richtig angezeigt werden, müssen sie ggf. prüfen, wie Sie PHP bzw. die Datenübertragung mit der Datenbank anpassen müssen.\\

Bevor wir auf einige Transaktionen eingehen, die fürs erste reichen sollten, erfahren Sie alles, was Sie in PHP vorbereiten müssen, um von dort aus auf die Datenbank zuzugreifen.

\section{Zugriff auf eine Datenbank aus PHP heraus}

In diesem Kurs nutzen wir PHP um die Funktionalität einer Webanwendung zu programmieren. PHP wiederum wird in HTML integriert. Wie das geht, haben Sie im letzten Kapitel kennen gelernt. Die Einbindung von MySQL-Queries in PHP sieht komplizierter aus, aber das liegt schlicht daran, dass wir hier die Verbindung zu einer Datenbank mit Funktionen aufbauen und abbauen müssen. Außerdem müssen wir auch die Queries jeweils mit Funktionen durchführen.\\

Jede MySQL-Query wird in PHP mit einer Funktion aufgerufen, deren Name mit mysqli\_ beginnt. Eine Komplettübersicht mit näheren Informationen finden sie z.B. auf \url{http://www.w3schools.com/php/php\_ref\_mysqli.asp} . Ob das i nun für improved steht (wie bei den W3Schools zu lesen) oder für instruction, spielt keine Rolle. Merken Sie sich einfach, dass alle PHP-Funktionen, die sich auf MySQL beziehen mit mysqli\_ beginnen.\\

Wie alle Funktionen haben auch die mysqli\_-Funktionen einen Rückgabewert, den wir einer Variablen zuordnen können. Und wir werden häufig die gleichen Funktionsaufrufe mehr vielen Stellen im Programm nutzen. Deshalb werden wir regelmäßig nicht nur die Funktionen aufrufen, um so auf die Datenbank bzw. die Verbindung zur Datenbank zuzugreifen, sondern wir werden die Rückgabewerte in Variablen speichern, um uns zumindest einen Teil der Tipparbeit zu sparen.\\

Außerdem sollten Sie jeweils prüfen, ob Sie ggf. einzelnen Transaktionen in eigene Funktionen oder sogar eigene PHP-Dateien ausgliedern, um so Ihren Quellcode übersichtlicher zu gestalten. 

\subsection{Verbindung aufbauen}

\textbf{Wichtig}: Streng genommen ist die Überschrift sowie die folgende Erklärung falsch: Wenn wir MySQL nutzen, dann bauen wir keine Verbindung zu einer Datenbank auf, sondern wir nutzen eine bestehende Verbindung von unserem Rechner zu einem Datenbankserver, um auf die dortige Datenbank zuzugreifen. (Das gilt selbst dann, wenn der Datenbankserver auf unserem Rechner läuft.) Was das bedeutet können Sie aber erst dann nachvollziehen, wenn Sie eine Veranstaltung zum Thema Netzwerke, Kommunikationstechnik und/oder Nachrichtentechnik besucht haben. Deshalb bleiben wir hier bei der streng genommen falschen Bezeichnung des Aufbaus einer Verbindung.\\

Um die Verbindung zu einer Relation in einer Datenbank aus PHP aufzubauen, müssen wir eine oder zwei Funktionen nutzen. In die erste Funktion tragen wir u.a. die Zugangsdaten ein. In der zweiten geben wir dann an, auf welche Datenbank wir mit diesen Zugangsdaten zugreifen wollen. Das mag ein wenig umständlich erscheinen, aber wie gewohnt gilt hier: Im Hintergrund laufen viele Prozesse ab, die dafür sorgen, dass unsere Verbindung zum Datenbankserver aufgebaut wird, bzw. die dafür sorgen, dass wir tatsächlich Daten übertragen können. Dabei ist bereits die Idee einer dauerhaften Verbindung schlicht falsch, doch das ist wieder ein Thema für Ihr späteres Studium.\\

Die erste Funktion, mit der Sie u.a. die Zugangsdaten zum Server angeben, heißt mysqli\_connect(host, username, password, dbname, port, socket). Auch wenn das zunächst seltsam wirkt: Sämtliche Argumente können in der Funktion verwendet werden, keines muss verwendet werden. Sehen wir uns das im Detail an:\\

-	host ist der Name des Servers, auf dem die Datenbank liegt. Es kann auch eine IP-Adresse sein. Ggf. müssen Sie hier Anführungszeichen verwenden. Wenn Sie einen DB-Server auf Ihrem Rechner betreiben, dann lautet der Wert im Regelfall ``localhost´´. Ggf. funktioniert auch `` ´´.

-	username ist der Nutzername, mit dem Sie sich am DB-Server anmelden wollen. Bei einer Standard-Installation von XAMPP oder easyPHP ist der Wert ``root´´.

-	password sollte selbsterklärend sein. Allerdings gibt es bei der Standard-Installation von XAMPP und easyPHP kein Passwort. In dem Fall tragen Sie hier nichts ein. Wichtig: Das ist beispielsweise eine Sicherheitslücke, die Sie bei einem Server im Netz niemals zulassen dürfen. Aber hier geht es wieder um das Thema Server-Administration und das lassen wir in diesem Kurs wie gesagt außen vor.

-	dbname ist der Name der zu verwendenden Datenbank.

-	port und socket haben etwas mit der Datenübertragung über ein Netzwerk bzw. mit der Adressierung des Servers zu tun. Um zu verstehen, was es damit auf sich hat, brauchen Sie eine Einführung in Netzwerke.

Tipp: Konsistente Variablenbezeichnungen\\

Wie geschrieben sollten Sie den Rückgabewert von mysqli\_connect() in einer Variablen speichern. Um deutlich zu machen, dass es sich hier um eine Variable handelt, die etwas mit der Datenbank zu tun hat, sollten Sie diese Variable nicht einfach \$connection oder \$verbindung nennen. Besser wäre eine Bezeichnung wie \$db\_connection. So kann jedes Teammitglied erkennen.\\

Die erste Zeile des Verbindungsaufbaus könnte somit wie folgt aussehen:\\

\begin{verbatim}
\$db\_connection = mysqli\_connect(``localhost´´, ``user1723´´, ``ganzSchlechtesPasswort´´, ``kneipe´´);
\end{verbatim}

Dabei würden wir also eine Verbindung zum DB-Server auf unserem Rechner (localhost) aufbauen, bei dem wir uns mit einem Nutzeraccount (user1723) und einem Passwort (ganzSchlechtesPasswort) anmelden. Außerdem würden wir bereits hier festlegen, dass wir eine bestimmte Datenbank (kneipe) nutzen wollen.\\

Für den Einstieg sollte eine der beiden folgenden Varianten ausreichen. Wenn wir so vorgehen, können wir später zwischen verschiedenen Datenbanken auf unserem DB-Server wechseln, während wir mit eben genannten Variante auf eine DB (kneipe) festgelegt sind.\\

\begin{verbatim}
\$db\_connection = mysqli\_connect(``localhost´´, ``root´´);
\$db\_connection = mysqli\_connect(`` ´´, ``root´´);
\end{verbatim}

Wenn wir so vorgehen, dann müssen wir noch in einer zweiten Anweisung festlegen, auf welche Datenbank wir zugreifen wollen. Das machen wir über die Funktion \verb{mysqli\_select\_db(connection, db)}.\\

Hier gibt es nicht viel, was Sie verstehen müssen: connection ist der Rückgabewert von \verb{mysqli\_connection()} und db ist der Name der Datenbank. \\


Tipp: Prüfung, ob Zugriff auf DB möglich ist.\\


Um sicher zu gehen, dass der Zugriff erfolgreich war, können Sie noch eine Variable definieren, der Sie den Rückgabewert der Funktion zuorden. Hier ist das ein boolescher Wert. Ist es true, dann können Sie auf die Datenbank zugreifen, ist es false, dann ist die Datenbank gerade gesperrt. Z.B. weil gerade eine andere Transaktion durchgeführt wird oder weil jemand anders die Verbindung auf den Server noch nicht beendet hat.\\

Zusammenfassung:\\


Wenn wir also den Nutzernamen root und kein Passwort nutzen, dann haben wir die folgenden zwei Möglichkeiten, um auf unsere Datenbank kneipe zuzugreifen:\\


-	Entweder wir teilen den Zugriff auf zwei Funktionen auf:

\begin{verbatim}
\$db\_connection = mysqli\_connect(``localhost´´, ``root´´);
\$db\_transaction\_possible = mysqli\_select\_db(\$db\_connection, ``kneipe´´);
\end{verbatim}

-	Oder wir erledigen das Ganze in einem Zug:

\begin{verbatim}
\$db\_connection = mysqli\_connect(``localhost´´, ``root´´, `` ´´ , ``kneipe´´);
\end{verbatim}

Beachten Sie aber bei der zweiten Variante, dass Sie jetzt nicht prüfen können, ob der Zugriff auf die Datenbank möglich ist. Deshalb sollten Sie immer die erste Variante wählen.

\subsection{Transaktionen in PHP}

Wir haben zwar noch nicht besprochen, wie Sie Transaktionen programmieren, aber hier erhalten Sie schon einmal den Code, um eine Transaktion in PHP durchzuführen. Beachten Sie dabei aber immer, dass Sie sicher sein müssen, dass der Zugriff auf die DB überhaupt möglich ist. Sonst weisen Sie PHP an, Daten aus der DB abzurufen oder sie dort zu speichern, ohne dass das passiert!\\

Kurz zur Erinnerung: Um HTML-Code in PHP zu generieren, benutzen Sie die Funktion echo(...) wobei die drei Punkte für den HTML-Code stehen bzw. für Funktionen und Variablen, die als HTML ins HTML-Dokument integriert werden.\\

Wie gewohnt nutzen wir in PHP dagegen eine Funktion, um eine Query an die DB zu schicken und erhalten dabei einen Rückgabewert. Wenn wir also Daten aus der Datenbank abfragen, dann müssen wir den Rückgabewert der Funktion immer dann in einer Variablen speichern, wenn wir keine anonyme Variable verwenden. Wenn Sie nicht mehr wissen, was mit anonymen Variablen gemeint ist, dann lesen Sie es bitte nochmal im PHP-Kapitel nach.\\


Der Funktionsname, um eine Query in PHP aufzurufen ist denkbar naheliegend:

\begin{verbatimg}
mysqli\_query(connection, query).
\end{verbatim}

Dabei steht connection wieder für unsere oben aufgebaute Verbindung zum DB-Server und query ist eine beliebige MySQL-Zeile. Die müssen Sie zusätzlich in Anführungszeichen setzen.\\


Wdh.: Konsistente Variablenbezeichnung\\


Auch hier ist es empfehlenswert, selbstredende Variablenbezeichnungen zu verwenden. Beispielsweise könnten Sie die folgende Zeile verwenden, wenn Sie eine Transaktion durchführen, die einen Datensatz zurückgibt. (Da wir uns noch keine konkreten Transaktionen angesehen haben, stehen hier drei Punkte stellvertretend für die Transaktion.)

\begin{verbatim}
\$db\_result = mysqli\_query(\$db\_connection, ...);
\end{verbatim}

Anschließend können Sie dann die Ausgabe der Datenbank weiter verarbeiten. \\


Wenn das erledigt ist, sollten Sie die Verbindung zur DB immer beenden, selbst wenn Sie weitere Transaktionen durchführen wollen. Sie stellen so sicher, dass Sie die Datenbank nicht blockieren. Das gilt auch dann, wenn nur Sie bzw. nur Ihre Anwendung Zugriff auf die DB haben soll. Der Grund ist recht einfach: Stellen Sie sich vor, Sie entwickeln ein Spiel, das von vielen Nutzern genutzt werden soll. Dann werden durch die Interaktionen der verschiedenen Spieler unabhängig voneinander zu verschiedenen Zeiten Anfragen an die Datenbank geschickt. Wenn hier mehrere Abfragen zusammen abgearbeitet werden, die nur gemeinsam haben, dass Sie für denselben Spieler bearbeitet werden sollen, dann kann das dazu führen, dass die anderen Spieler nicht weiterspielen können, weil die Datenbank für sie nicht erreichbar ist. Aber auch wenn es nur einen einzigen Nutzer gibt, können Probleme auftreten, wenn Sie meinen Rat nicht beachten. Denn unterschiedliche Komponenten einer Anwendung können unabhängig voneinander versuchen auf die gleiche Datenbank zuzugreifen. Wenn Sie hier die Datenbank länger als für einzelne Transaktionen nötig blockieren, kann auch das das Programm blockieren oder zumindest verlangsamen.

\subsection{Beenden der Verbindung}

Auch das Beenden einer Verbindung ist relativ simpel. Mit dem Funktionsaufruf \verb{mysqli\_close(\$db\_connection);} beenden Sie die Verbindung.\\


\subsubsection{Zusammenfassung:}

Sie kennen jetzt die Vorgehensweise und die verwendeten Funktionen, um aus PHP auf eine Datenbank und einzelnen Relationen dieser Datenbank zuzugreifen. Um tatsächlich mit der Relation zu arbeiten, brauchen Sie noch das Wissen, welche Queries bzw. Transaktionen es gibt und wie die genutzt werden können, um Werte aus der Datenbank in PHP verwenden zu können, bzw. um PHP-Variablen in der Datenbank zu speichern.

\section{Queries und Transaktionen in MySQL}

Wenn Sie zum ersten Mal auf eine Relation einer Datenbank zugreifen, wissen Sie vielleicht nicht, wie die einzelnen Attribute der Relation heißen. Bei diesem Beispiel gehen wir davon aus, dass es bereits eine Relation namens kneipe in Ihrer Datenbank gibt. Wenn das nicht der Fall ist, dann arbeiten Sie bitte dennoch diesen Abschnitt durch. Im nächsten Abschnitt erfahren Sie, wie Sie einer Datenbank eine Tabelle hinzufügen.\\


Die Attributsnamen einer Relation können Sie mit der MySQL-Anweisung describe kneipe abfragen.\\


Wenn Sie nun mit \$db\_result = mysqli\_query(\$db\_connection, ``describe kneipe´´); der Variablen \$db\_result die Rückgabe der Datenbank zugeordnet haben, dann müssen Sie daran denken, dass Sie niemals einen einzelnen Wert zurückbekommen, sondern immer einen Datensatz oder mehrere Datensätze. PHP kennt aber keine Datensätze, sondern nur Variablen und Datenstrukturen. Also müssen wir den Datensatz bzw. die Datensätze, die in \$db\_result gespeichert sind noch in eine Datenstruktur umwandeln.

Hier gibt es in PHP eine Vielzahl Methoden, die alle mit mysqli\_fetch\_ beginnen.

Wenn wir wie in diesem Fall wissen, dass wir genau eine Zeile unserer Relation in \$db\_result haben, dann können wir mit \$db\_dataset\_array = msqli\_fetch\_array(\$db\_result); ein Array erzeugen, das in \$db\_dataset\_array gespeichert wird und dessen Einträge die Attributnamen sind. (Zur Erinnerung: \$db\_dataset\_array ist eine beliebige Variable, aber im Sinne konsistenter Variablenbezeichner sollten Sie diese oder eine ähnliche Variablenbezeichnung verwenden. Da Sie außerdem die Wahl haben, den Datensatz in einem Array oder einem Dictionary zu speichern, macht es ebenfalls Sinn, den Namen wie hier entsprechend zu erweitern.)
Damit wir erfahren, wie viele Einträge unser Array enthält, können wir jetzt noch folgenden Aufruf programmieren: \$dbset\_number\_fields = mysqli\_num\_rows(\$db\_result);
Mit \$db\_dataset\_array und \$dbset\_number\_fields können Sie jetzt z.B. eine Ausgabe in HTML erzeugen, in der im Browser die Namen der Attribute aufgeführt werden. Denken Sie dabei daran, dass das erste Element eines Arrays die Nummer 0 hat.

for (int i = 0; i < \$dbset\_number\_fields; i++)
\{ echo(``Attribut ´´ . i . `` hat den Namen ´´ . \$db\_dataset\_array[i]); \}

8.4.1.	Erzeugen einer Tabelle

Wie geschrieben setzt das letzte Beispiel voraus, dass es bereits eine Relation in Ihrer Datenbank gibt. Wenn Sie dagegen bislang nur die Datenbank erzeugt haben, aber noch keine Relation enthalten ist, dann können Sie eine einfache Relation mit folgender MySQL-Anweisung erzeugen:

create table kneipe(name character(30), vorname character(30), schulden int(4));

Mit create table geben Sie an, dass die Datenbank eine Relation erzeugen soll.

kneipe ist ein Name für die Relation. Hier können Sie jedes beliebige Wort bzw. jede beliebige Buchstabenkombination verwenden, allerdings sollten Sie dabei ausschließlich Buchstaben verwenden, die im Englischen verwendet werden. Ggf. führen Umlaute und Sonderzeichen zu Problemen wegen des Character Sets.

Für jedes Attribut müssen Sie nun noch einen Namen, einen (Daten-)typ und eine Länge angeben. Für den Namen gilt weitestegehend das gleiche wie bei PHP und anderen imperativen Programmiersprachen, zu den Datentypen in MySQL hatte ich am Anfang des Kapitels schon etwas geschrieben, einzig die Länge (die in Klammern steht) muss noch etwas ergänzt werden: Bei relationalen Datenbanken gilt im Regelfall, dass Sie bei der Erzeugung einer Relation festlegen müssen, welche Länge jeder Attributwert eines Attributs maximal haben darf. Die einzelnen Attribute trennen Sie bei der Erzeugung der Relation durch Kommata.

Im Gegensatz zu Datenstrukturen in PHP können Sie eine Relation also nicht gleich mit Werten füllen.

\subsection{Arten von Transaktionen}

Es gibt vier Arten von Transaktionen: select, insert, update und delete.\\


Streng genommen ist select keine Transaktion sondern „nur“ eine Query, weil select nichts an den Einträgen in der Datenbank ändert. Aber wie gesagt ist das für den Inhalt dieses Kurses nicht relevant.\\

Ausgaben der Datenbank: \\


-	Mit select lassen Sie sich einen oder mehrere Datensätze ausgeben.
Änderung der Datenbank: 
-	Mit insert fügen Sie einen neuen Datensatz in eine Datenbank.
-	Mit update ändern Sie Einträge eines Datensatzes.
-	Mit delete löschen Sie einen Datensatz.

Wann immer das Sinn macht, können Sie außerdem mit distinct und where ... Einschränkungen vornehmen. Darauf gehen wir ein, nachdem Sie die vier Transaktionen kennen gelernt haben.

\subsubseciton{select}

Mit select fordern wir alles von einzelnen Attributwerten bis zu einer sortierten Auswahl von Datensätzen mehrerer Relationen von einer Datenbank an. \\


Richtig gelesen: Wir können zwar die meisten Operationen, die wir in PHP mit Variablen durchführen können nicht direkt in einer Datenbank durchführen, aber wir können die Rückgabe filtern lassen. Datenbanken kennen also durchaus Konjunktionen und Disjunktionen (und/oder), aber während das in PHP und anderen imperativen Programmiersprachen Operationen sind, mit denen wir eine boolesche Variable erzeugen, bewirken Sie bei Datenbanken, dass Teile der Rückgabe gefiltert werden. Genauso können wir auch mit Vergleichsoperatoren wie < und > arbeiten, um nach einzelnen Attributwerten zu filtern.\\


Fangen wir mit einer einfachen Abfrage an:\\


\begin{verbatim}
select * from kneipe
\end{verbatim}

bewirkt, dass alle Datensätze der Relation kneipe ausgegeben werden.
Wenn Sie das einführende Beispiel zu describe kneipe aufmerksam gelesen haben, dann ist Ihnen jetzt klar, dass wir mit der bisherigen Vorgehenseweise in PHP ein Problem bekommen:\\

\begin{verbatim}
\$db\_result = mysqli\_query(\$db\_connection, „select * from kneipe“);
\$db\_dataset = mysqli\_fetch\_array(\$db\_result);
\end{verbatim}

Denn die Funktion \verb{mysqli\_fetch\_array()} versucht aus der Rückgabe der Datenbank ein Array zu erzeugen. Ein Array besteht aber aus einzelnen Variablen, die jeweils in einer nummerierten Liste (eben dem Array) gespeichert werden. Doch in diesem Fall enthält \verb{\$db\_result} ja nicht mehr eine Liste mit Attributnamen, sondern eine Liste von kompletten Datensätzen. Also brauchen wir eine Funktion, die uns die einzelnen Datensätze jeweils in einem eigenen Array oder einem eigenen Dictionary abspeichert.\\


Daraus folgt gleich ein zweites Problem: Da es sich um eine Anzahl Datenstrukturen handelt, deren Anzahl wir nicht kennen, dürfen wir hier nicht nur eine einzelne Variable nutzen, sondern müssen die Datensätze als Einträge eines Arrays speichern.\\


Wdh.: Arrays und Dictionaries in PHP\\


Ein Array ist eine nummerierte Liste, in der Sie einzelne Werte speichern können. Die Werte können Sie dann später über die Nummer aufrufen. Doch diese Werte können auch komplette Datenstrukturen sein. Das bedeutet, dass Sie in PHP beispielsweise ein Array programmieren können, in dem Sie unter jedem Index einen kompletten Dictionary speichern.\\


Neu: Datensatz als Dictionary\\


Angenommen, wir hätten nur einen einzelnen Datensatz als Rückgabe, dann könnten wir den im Grunde genommen 1:1 in einem Dictionary (bei PHP assoziatives Feld genannt) speichern. Die Schlüssel wären die Attributsnamen und die Werte die jeweils zugehörigen Attributswerte.\\


Wenn wir dagegen wie bei der MySQL-Instruktion „select * from kneipe“ mehrere Datensätzen als einen Rückgabewert erhalten, dann müssen wir die Dictionaries zusätzlich als Einträge eines Arrays abspeichern. Und hierfür bietet PHP uns eine Methode an:\\

\begin{verbatim}
mysqli\_fetch\_all();
\end{verbatim}

Wir müssten jetzt also lediglich die zweite der beiden Zeilen oben anpassen und anschließend beachten, dass \verb{\$db\_datasets} ein Arrray ist, in dem Dictionaries gespeichert sind. Um das im Namen deutlich zu machen, sollten Sie die Variable \$db\_datasets, also mit einem s am Ende umbennen. Außerdem können Sie dann weiterhin mit der Variable \$db\_dataset\_array bzw. \$db\_dataset\_dict so umgehen, als wenn es sich dabei um einen einzelnen Datensatz bzw. eine einzelne Datenstruktur handelt.\\

\begin{verbatim}
\$db\_result = mysqli\_query(\$db\_connection, „select * from kneipe“);
\$db\_datasets = mysqli\_fetch\_all(\$db\_result);
\end{verbatim}

Wenn Sie jetzt also mit allen Datensätzen (bzw. in PHP die Dictionaries) nutzen wollen, dann können Sie das wie folgt tun:

\begin{verbatim}
\$db\_datasets\_number = mysqli\_num\_rows(\$db\_result);
for (int i = 0; i < \$db\_datasets\_number; i++)
{ 
	\$db\_dataset = \$db\_datasets[i]; 
	// Jetzt können Sie das einzelne Dictionary über \$db\_dataset ansprechen.
	...
} 
\end{verbatim}

\subsubsection{Filtern nach einer Bedingung}

Nehmen wir unsere kleine Datenbank und nehmen wir einmal an, mehrere Gäste der Kneipe schulden dem Wirt unterschiedliche Beträge. Nun will der Wirt sich alles Gäste ausgeben lassen, die ihm mehr als 30,- € schulden. Dazu müssen wir nur die Anweisung select * from kneipe wie folgt erweitern:\\

\begin{verbatim}
select * from kneipe where schulden > 30;
\end{verbatim}

Mit dem Wort where leiten Sie also eine Einschränkung der Ausgabe ein. Hier können Sie wie eingangs beschrieben mit <, >, =, <= und >= filtern.\\


Zur Erinnerung: Während wir mit diesen Operatoren in imperativen Programmiersprachen eine boolesche Variable (true/false) erzeugen, bewirken wir in MySQL, dass die auszugebenden Datensätze gefiltert werden.

\subsubsection{Filtern von Attributen}

Angenommen, wir wollen nur einige Attribute aber diese Attribute von allen Datensätzen als Rückgabe der Datenbank erhalten, dann müssen wir schlicht anstelle des Sterns die Namen der Attribute angeben:\\

\begin{verbatim}
select name from kneipe;
\end{verbatim}

gibt eben alle (Nach-)Namen an, die in der Liste der Schuldner aufgeführt sind. Im Regelfall wollen wir dann aber nicht nur die Nachnamen, sondern Nach- und Vornamen. Das ist kein Problem: Immer wenn wir mehrere Attribute oder mehrere Bedingungen prüfen wollen, nutzen wir Kommata, um die entsprechenden Punkte zu trennen.

\begin{verbatim}
select name, vorname from kneipe where schulden > 30, schulden < 100;
\end{verbatim}

Diese Abfrage würde beispielsweis die Aufstellung aller Schuldner geben, die mehr als 30,- € aber weniger als 100,- € Schulden angehäuft haben.

\subsubsection{Filtern aus mehreren Relationen}

Wechseln wir den Bereich. Wir haben ein Unternehmen mit Lagerhaltung und haben u.a. für Schrauben und Muttern jeweils eine Relation mit dem passenden Namen. In beiden Relationen gibt es u.a. ein Attribut durchmesser und ein Attribut preis. Wir wollen nun berechnen, wie viel 100 Schrauben und Muttern mit einem Durchmesser von 10 mm kosten.\\


Die Abfrage der Preise ist kein Problem:

\begin{verbatim}
select schrauben.preis, muttern.preis 
from schrauben, muttern 
where schrauben.durchmesser = 10, muttern.durchmesser = 10;
\end{verbatim}

Wie Sie sehen, können wir also unterschiedliche Attribute aus unterschiedlichen Relationen filtern und müssen dann lediglich den Namen der zugehörigen Relation mit einem Punkt vor den Namen des Attributs schreiben.\\


Aber: Wie schon am Anfang dieses Kapitels beschrieben können wir in Datenbanken keine Operationen durchführen, wie wir sie aus der imperativen Programmierung mit PHP kennen.\\


Das bedeutet, dass wir den Gesamtpreis nicht berechnen können. Sie müssen für diese Aufgabe also jeweils den Preis für eine Mutter und den Preis für eine Schraube in PHP einer Variablen zuordnen (z.B. \verb{\$preis\_schraube, \$preis\_mutter}), dann beide addieren und mit 100 multiplizieren.

\subsubsection{Ausgabe nur einmal pro Attributwert}

Schließlich möchte ich Ihnen noch die Beschränkung distinct vorstellen. Nehmen wir an, Sie haben eine Relation mitarbeiter, u.a. mit dem Attribut aufgabe. Nun wollen Sie nur eine Aufstellung der verschiedenen Aufgaben, aber selbst wenn mehrere Mitarbeiter die gleiche Aufgabe haben, wollen Sie jede Aufgabe nur einmal angezeigt bekommen. Das ist die Lösung:

\begin{verbatim}
select distinct aufgabe from mitarbeiter;
\end{verbatim}

\subsubsection{Löschen von Datensätzen}

Stellen wir uns vor, Sie wollen aus der Relation kneipe alle Gäste löschen, die keine Schulden mehr haben. Auch das können wir recht schnell realiseren:

\begin{verbatim}
delete from kneipe where schulden = 0 OR schulden < 0;
\end{verbatim}

Wie Sie sehen können Sie in MySQL auch die Wörter and und or nutzen.

\subsubsection{Update}

Die Syntax für Updates sieht etwas anders aus, aber das ist auch alles:

\begin{verbatim}
update relation set attribut1=... , attribut2 = ... , ... where ... ;
\end{verbatim}

Hier geben Sie also nach dem Wort update die Relation(en) an, wie Sie das auch bei select schon genutzt haben. Danach wählen Sie Attribute und Attributwerte aus, die überschrieben werden sollen. Wichtig: Wenn Sie die Beschränkung mit where vergessen, werden die Änderungen der Attributwerte bei jedem Datensatz durchgeführt.


\chapter{Langfristige Datenspeicherung mit MySQL 5.6}

%\part{Fortsetzung für Studierende der Informatik}

%\chapter{Einführung in den Aufbau von ARMv6-Prozessoren am Beispiel eines Cortex-M0-Prozessors}

\textbf{Wichtig (1)}: Dieses Kapitel ist wichtig, auch wenn hier noch nicht eine Zeile Programmcode auftaucht: Erst wenn Sie es durchgearbeitet haben, haben Sie eine Chance zu verstehen, was alles im Hintergrund passieren muss, damit die Dinge in PHP so funktionieren, wie Sie das dort kennen gelernt haben. Wenn Sie diese Kenntnisse nicht haben, werden Sie bei komplexen Problemen wie der nebenläufigen Programmierung (Programmierung von Systemen mit mehreren Prozessorkernen) bekommen.\\

\textbf{Wichtig (2)}: Dieses Kapitel ist zurzeit lediglich ein Einstieg in die Grundlagen  von Mikroprozessoren, konkret die Programmierung von ARM-Prozessoren der Baureihe Cortex-M0. Die tatsächliche Programmierung ist noch nicht Teil dieses Kapitels. Ggf. wird es hier später noch eine Ergänzung geben, zurzeit ist es lediglich eine Unterstützung, wenn Sie Probleme beim Verständnis des Aufbaus dieses Prozessortyps haben.\\

In den letzten Kapiteln haben Sie eine erste Einführung in die Programmierung von Anwendungen mit Hilfe einer Kombination aus vier Programmiersprachen erlernt. Je nachdem, in welchen Bereich der Programmierung als (Medien-)InformatikIn Sie sich später vertiefen wollen benötigen Sie aber zusätzlich ein detailliertes Verständnis aus einem oder mehreren der Bereiche, die wir im letzten Kapitel kurz angeschnitten haben oder die Ihnen dort von einer der Programmiersprachen abgenommen wurden.\\

In diesem Kapitel können Sie sowohl einige Grundlagen der maschinennahen als auch die daraus resultierenden Grundlagen der imperativen Programmierung erlernen. Auch wenn Sie an der maschinenahen Programmierung nicht interessiert sind, benötigen Sie die verschiedene Informationen dieses Kapitels, da Sie nur durch diese verstehen werden, warum bestimmte Dinge in der imperativen Programmierung genauso passieren, wie das der Fall ist. \\

Bitte beachten Sie jedoch, dass dieses Kapitel nur eine Ergänzung zu einer eigentlichen Einführung in die Programmierung von Mikroprozessoren sein kann. Ursprünglich hatte ich geplant, nach der Einführung der Konzepte eine tatsächliche Einführung in die maschinennahe Programmierung zu erstellen. Interessierte Studierende hätten sich so in diesen Bereich vertiefen können. Aus zeitlichen Gründen habe ich diesen Teil vorerst gestrichen.\\

Dieses Kapitel basiert auf den folgenden Quellen: \\

•	Das Standardwerk The definitive Guide to the Cortex-M0 von Joseph Yiu, das Sie erwerben sollten, wenn Sie mit der Softwareentwicklung mit ARM-Prozessoren des Typs Cortex-M0 weiter machen wollen. Prüfen Sie jedoch, ob es inzwischen eine aktuellere Version des Buches gibt; die zurzeit vorliegende Version wurde 2011 veröffentlicht und beinhaltet dementsprechend teilweise veraltete Inhalte.
•	Das Skript zur Veranstaltung Informatik 3 – Informatik und Elektrotechnik des Studiengangs Media Systems an der HAW Hamburg von Michael Berens und Jens Ginzel. Diese erhalten Sie ggf. zusammen mit dem Zugriff auf die Veranstaltungsinhalte im Rahmen des seminaristischen Unterrichts.
•	Dem LPC-P1114 development board Users Manual von 2011. Da wir die Version des Boards von 2011 verwenden ist dieses Dokument auf aktuellem Stand.
•	Dem Cortex-M0 Devices Generic User Guide ARM DUI 0497A (ID 112 109)
•	Der UM 10398 Dokumentation. Hierin finden Sie detaillierte Informationen, welche Belegung eines Registers beim Entwicklerboard welche Aufgabe erfüllt.
•	Die Dokumentation zum CMSIS, der Sprachbibliothek zur Cortex-M0-Programmierung in C.\\

Bevor wir zur eigentlichen Programmierung kommen, müssen Sie zunächst einen Überblick darüber bekommen, was Sie da eigentlich im Detail programmieren. Das ist nirgends so wichtig wie bei der maschinennahen Programmierung, weil Sie hier durchgehend die Werte der Komponenten des Prozessors bzw. der Peripherie beeinflussen. Und wie sollen Sie das tun, wenn Sie genau diese Komponenten gar nicht kennen oder wissen, was sie tun?\\

Wenn Sie nicht an der Veranstaltung „Informatik 3“ teilnehmen, in der die Programmierung des Cortex-M0 behandelt wird, z.B. weil Sie nicht Media Systems sondern Medientechnik studieren, dann können Sie natürlich gerne im Labor nachfragen, ob Sie dort ein Entwicklungssystem nutzen können. Alternativ können Sie jedoch auch ein günstiges Entwicklerboard mit den nötigen Anschlüssen erwerben. Zum Teil gibt es beispielsweise bei IAR  Sonderangebote, bei denen Sie ein günstiges Starter-Kit erhalten. Für höchstens 40 Euro (ggf. plus Versandkosten) sollten Sie dort alles finden, was Sie für den Einstieg benötigen. Wenn Sie ein System erwerben wollen, das dem in unseren Laboren verwendeten ähnelt, dann suchen Sie nach einem NXP LPC1114. Die Modelle mit den Bezeichnungen LPC11x14 sind nahezu baugleich, also genauso geeignet; hier steht anstelle des x ein anderer Buchstabe, der für eine zusätzlich verbaute Komponente steht. Das LPC11U14 beinhaltet beispielsweise die Unterstützung von USB 2.0 Schnittstellen.\\

Wenn Sie bereit sind, wesentlich tiefer in die Tasche zu greifen, möchte ich Sie auf den Hersteller Keil hinweisen; die Produkte dieses Herstellers werden in der Wirtschaft häufig eingesetzt, sind aber nichts für den kleinen Geldbeutel. So müssen Sie für eine gute Debugging Unit (die Schnittstelle, die nötig ist, um den Computer und ein Entwicklungsboard zu verbinden) dort zum Teil deutlich mehr als 1.000 € ausgeben. Gerade wenn Sie zunächst nur in diesen Bereich hereinschnuppern wollen ist das eigentlich zu viel. Aber wenn Sie Ihre berufliche bzw. forschende Zukunft im Bereich der Mikroprozessoren sehen, sollten Sie später darüber nachdenken.\\

\section{Mikroprozessoren und eingebettete Systeme}

Die Bezeichnung Mikroprozessor ergibt sich schlicht und ergreifend daraus, dass die ersten Mikroprozessoren in einer Bauweise angefertigt wurden, die in Mikrometern gemessen wurde (heute in Nanometern), und dass Sie Prozesse abarbeiten. So weit, so simpel.\\

In der elektrotechnischen Welt ist es üblich das Symbol µ  (ausgesprochen Mü), das Sie aus dem Bereich der Maßeinheiten kennen, auch außerhalb von Maßeinheiten als Ersetzung für das Präfix Mikro- zu verwenden. Beispielsweise wird die IDE Mikrovision von Keil üblicherweise als µVision beschriftet.\\

Wann immer Sie Kürzel wie CPU (Central Processing Unit), GPU (Graphical Processing Unig), APU (Accelerated Processing Unit) usw. sehen, haben Sie es mit Mikroprozessoren zu tun. Und ob es sich dabei nun um eine CPU, eine GPU oder eine APU handelt, hat vorrangig mit dem konkreten Aufbau des jeweiligen Mikroprozessors und weniger mit dem geplanten Einsatzbereich zu tun. Beispielsweise werden GPUs auch eingesetzt, um die virtuelle Währung Bitcoin zu generieren. Mit Computergrafik hat das nichts mehr zu tun. Im Rahmen dieses Kurses brauchen Sie nicht zu wissen, was eine CPU, eine GPU usw. voneinander unterscheidet.\\

Dementsprechend macht es in diesem Bezug auch keinen Unterschied, ob wir nun über Prozessoren von Intel, AMD, ARM, Motorola oder welchem Hersteller auch immer reden: Es sind alles Mikroprozessoren und im Kern basieren sie alle auf denselben Prinzipien. Allerdings gibt es mittlerweile eine Vielzahl an zusätzlichen Komponenten, durch die sie sich dann wiederum deutlich voneinander unterscheiden. Aber auch ohne diese zusätzlichen Komponenten unterscheiden sich die Bezeichnungen von Komponenten bei verschiedenen Prozessoren desselben Herstellers häufig. Diese Komponenten eines Mikroprozessors zu kennen und ihre Programmierung zu beherrschen zeichnet professionelle EntwickerInnen aus. \\

Wenn die Rede von Mikroprozessoren ist, dann wird häufig auch von eingebetteten Systemen (embedded system) gesprochen. Im Gegensatz zu einem Rechner, den Sie als solchen leicht erkennen können, sind eingebettete Systeme stets so in andere Systeme eingebettet, dass sie als eigenständige Computer nicht erkennbar sind. Ihre Programmierung verläuft aber genauso wie die von einem handelsüblichen Computer. Es gibt eingebettete Systeme seit weit mehr als 20 Jahren.\\

Wenn wir über die Programmierung eines Mikroprozessors reden, nutzen wir gelegentlich auch die Bezeichnung Bare Metal Target. Damit ist ein Mikroprozessor gemeint, auf dem kein Betriebssystem aktiv ist. In diesem Bereich werden Betriebssysteme als embedded OS bezeichnet, um sie als Betriebssystem für eingebettete Systeme von anderen Betriebssystemen abzugrenzen.\\

Kontrolle\\

Sie wissen jetzt, was ein Mikroprozessor ist.

\section{SoC, Mikrocontroller und die Umgebung von Mikroprozessoren}

Da es wenig Sinn macht, einen Mikroprozessor an den Strom anzuschließen und ihn ansonsten nur herumstehen zu lassen, gibt es immer eine Umgebung, an die er angeschlossen wird. Um nun mit dieser Umgebung interagieren zu können, benötigt der Mikroprozessor aber noch einige zum Teil optionale Komponenten:\\

•	Ein Timer dient dazu, mit einer gewissen Genauigkeit zeitliche Abläufe zu koordinieren. Gerade die Formulierung „mit einer gewissen Genauigkeit“ ist hier wichtig: Innerhalb eines Rechners können wir mit einer Genauigkeit im Nanosekundenbereich Daten austauschen. Aber bereits wenn wir zwei Rechner auf kurzer Distanz miteinander verbinden, dann sinkt diese zeitliche Genauigkeit. Und je komplexer die Verknüpfung (das heißt je mehr Rechner verbunden sind) bzw. je größer die Distanz, desto unsicherer ist die Genauigkeit. \\

Wenn Sie sich beispielsweise mit den Grundlagen des Internet ein wenig auskennen, dann wissen Sie, dass es etwas gibt, das als Latenz bezeichnet wird. Das ist der zeitliche Versatz zwischen Versand und Empfang, der bei einer Datenübertragung auftritt. Diese Latenz ist aber kein konstanter Wert, sondern er unterliegt einer Schwankung, die durchaus mehrere Sekunden betragen kann. Und Sie kennen wenigstens einen Fall, in denen ungenaue Zeitabstimmung die Ursache für ein Problem ist: Denken Sie an das letzte Mal, als Sie einen Film im Netz gesehen haben, bei dem die Tonspur und die Bildspur nicht synchron liefen.\\

•	I/O (kurz für Input/Output, teilweise auch nur IO) bezeichnet alle Arten von Komponenten, die es einem Mikroprozessor ermöglichen, Eingaben anzunehmen und Ergebnisse auszugeben. Wir reden hier aber nicht über Displays, sondern über die Schnittstellen, über die ein Mikroprozessor Daten weiterleitet bzw. empfangen kann. Displays, Lautsprecher, Mikrofone, Tastaturen, usw. usf. haben wiederum eigenen Interfaces, die entweder direkt oder mittelbar an die I/O-Interfaces von Prozessoren angeschlossen werden können.\\

•	Interrupt-Controller sind essentiell: Sie signalisieren dem Mikroprozessor, wenn er einen bestimmte Prozess pausieren und zunächst einen anderen ausführen soll. Das kann beispielsweise dann der Fall sein, wenn ein Nutzer am Rechner eine Taste drückt. Wenn Sie diese Funktion verstanden und das ein oder andere Programm entwickelt haben, das mit Interrupts arbeitet, dann werden Sie die Gründe für eine Vielzahl von Fällen verstehen, in denen der naive User ungefähr die folgenden Worte von sich gibt: „Warum macht die sch… Kiste nicht, was ich will?!“\\

Hinweis bezüglich Exceptions: Wenn Sie schon in höheren Programmiersprachen programmiert haben, dann haben Sie die sogenannten Exceptions kennen gelernt. Für den Moment merken Sie sich bitte, dass Exceptions im Rahmen der maschinennahen Programmierung ein Spezialfall von Interrupts sind. Teilweise werden Exceptions und Interrupts hier auch als Synonyme verwendet.\\

•	ROM und RAM sind zwei Bezeichnungen, die für Speicher verwendet werden, in dem sich Programme und Daten befinden. Für den Augenblick genügt es, wenn Sie sich merken, dass der Inhalt vom RAM gelöscht wird, wenn der Strom ausfällt bzw. abgeschaltet wird. ROM-Speicher dagegen behält seinen Inhalt auch ohne Stromversorgung dauerhaft. Dafür können die Daten beim ROM nicht geändert werden. Es gibt außerdem noch eine dritte Art von Speicher, bei dem der Inhalt auch ohne Stromfluss erhalten bleibt, der aber jederzeit überschrieben werden kann. Diese Art von Speicher nennt man Flash oder EPROM (kurz für Erasable Programmable ROM). Je nach Bauart ist nur ein vollständiges Löschen und neu Speichern von Daten möglich.\\

Wenn nun ein Prozessor mit einzelnen dieser Komponenten zu einer Einheit verbunden wird, spricht man vom Mikrocontroller. Eine Abkürzung für Mikrocontroller lautet MCU (Micro Controller Unit).\\

Darüber hinaus gibt es noch größere Verbünde von elektrotechnischen Komponenten, die dann als SoC (kurz für System-On-Chip) bezeichnet werden. Wie so oft ist der zentrale Grund für diese immer stärkere Vereinigung von immer mehr Komponenten in immer weniger Bauteilen schlicht der Kostenfaktor. SoCs sind nicht Teil dieser Veranstaltung, da Sie zunächst die Programmierung von Mikrocontrollern und FPGAs beherrschen sollten, bevor Sie sich mit diesen Geräten auseinander setzen können.\\

\subsection{Das Olimex LPC1114 Entwicklungsboard}

Im Falle des Entwicklungsboards, das in unseren Laboren zum Einsatz kommt, haben Sie einige Komponenten, die bislang nicht aufgeführt wurden, die Sie aber im Laufe der Labore programmieren werden. Diese sind zusammen mit den zugehörigen Befehlen und möglichen Belegungen im UM 10398 User Manual aufgeführt, das Sie dementsprechend für diesen Kurs benötigen:\\

(In einigen Fällen werden Sie die Bezeichnung der MCU benötigen. Hierbei handelt es sich wie geschrieben um den LPC 1114 des Herstellers NXP.)\\

•	Der SPI-Controller (kurz für Serielles Peripherie Interface). Dieser Controller steuert die Datenübertragung vom Prozessorbus zu den Interfaces der verschiedenen externen Komponenten, die Sie an das Entwicklerboard anschließen können. Ihrer Phantasie sind dabei kaum Grenzen gesetzt, weshalb die letzten Wochen dieser Veranstaltung darin bestehen, dass Sie sich eine (oder mehrere Komponenten) auswählen, diese mit dem Entwicklerboard verbinden und dann den Prozessor entsprechend programmieren. Teilweise reden wir hier auch von der Programmierung der SPI-Schnittstelle.\\

•	Ein Oszillator, der für zeitabhängige Anwendungen eine deutlich höhere Präzision bietet als der im Cortex-M0 enthaltene Timer oder der ebenfalls auf dem Entwicklerboard enthaltene WDT (kurz für Watch Dog Timer).\\

\section{Grundlagen des Cortex-M0}

Einige Kenndaten des Cortex-M0 müssen Sie kennen, um Programmierfehler zu vermeiden. Außerdem gibt es einige Begriffe, die speziell von ARM in den offiziellen Referenzen verwendet werden. Wenn Sie sie dann nicht kennen, dürften sie Sie verwirren, weshalb Sie hier mit aufgenommen wurden:\\

•	Es handelt sich um einen 32-Bit-Prozessor. Das bedeutet, der Prozessor kann in jedem Arbeitsschritt ausschließlich Werte zwischen 0 und 232-1 verwenden. Damit ist der Hauptspeicher auf rund 4 GB begrenzt.\\

Anm.: Im Falle der Evaluationsversion des Keil MDK, die wir für die Softwareentwicklung verwenden stehen 32 KB zur Verfügung. Da der LPC1114 aber ohnehin nur über 32 KB verfügt, ist das effektiv keine relevante Einschränkung.\\

•	Sie müssen hexadezimal rechnen. 0x10 – 4 ist also nicht 0x06!\\

•	Sie müssen von Angaben wie „setze das zweite und fünfte Bit“ auf die entsprechende hexadezimale Zahl (hier: 0xA) umrechnen können.\\

•	WICHTIG! Adressen entsprechen jeweils Einheiten von 8 Bit. Da der Cortex-M0 jedoch bei jedem Schritt 32 Bit einliest oder ausgibt, müssen Sie häufig mit dem Faktor 4 multiplizieren, um beispielsweise eine „benutzbare“ Adresse zu erhalten.\\

Beispiel: Sie wollen einen Wert im Speicher des Rechners ablegen und wissen, dass bis zur Adresse 0xFD Daten im Speicher liegen. Dann dürfen Sie erst ab der Adresse 0x100 Daten ablegen, da 0xFE und 0xFF nicht durch 4 teilbar sind.\\

•	Dennoch handelt es sich bei den Befehlen, die auf dem Cortex-M0 ausgeführt werden können, fast ausschließlich um 16-Bit lange Befehle. Dafür gibt es bei ARM-Prozessoren die Bezeichnung Thumb State. Wenn Sie es mit vollwertigen „32-Bit-Befehlen“ zu tun haben, dann werden diese bei ARM Prozessoren als das ARM Instruction Set bezeichnet.\\
 
•	Der Prozessor arbeitet mit einer sogenannten Load-Store-Architektur. Das bedeutet in letzter Konsequenz, dass Sie Daten aus dem Speicher immer in die sogenannten Register laden müssen, um etwas mit Ihnen zu tun. Register sind eine spezielle Form von Speicher, die genauso viele Bit speichern können, wie es der Bittigkeit des Prozessors entspricht. Sie sind Teil des eigentlichen Prozessors.\\

Anm.: Diese Beschränkung macht den Einstieg in die Programmierung etwas leichter, denn so können Sie leichter erkennen, ob eine Zahl bei einer Operation einen Wert oder die Adresse einer Speicherstelle ist.\\

•	Am Rande sei noch bemerkt, dass ARM zwischen zwischen zwei Stati und zwei Modi unterscheidet: Im Thread Mode führt der Prozessor beliebigen Programmcode aus, im Handler Mode werden bestimmte Arten von „Unterbrechungen“ behandelt. Diese Modi spiegeln verschiedene Fälle wieder, mit denen Sie es bei der Programmierung zu tun bekommen. Ebenfalls nicht direkt sichtbar ist der Debug State. Den bekommen sie automatisch dann zu Gesicht, wenn Sie den Debugger benutzen. Genau wie der Thumb State, der die beiden oben genannten Modi (Thread/Handler) zusammenfasst, ist er wie gesagt eher eine virtuelle Abgrenzung, die Sie fürs erste kaum wahrnehmen werden.\\

\section{Die Register des Cortex-MO}

Wie eingangs erwähnt können Sie nur die Daten verändern, die Sie aus dem Speicher in die Register geladen haben. Es gibt verschiedene Arten von Registern, von denen Sie die folgenden kennen müssen, um mit der Programmierung zu beginnen:

\subsection{R0 bis R12 – Die Allzweckregister}

Zunächst wären da die 13 GPRs (kurz für General Purpose Registers), die der Einfachheit halber mit R0 bis R12 angesprochen bzw. programmiert werden. Diese Register können Sie als ProgrammiererIn so mit Inhalten füllen, so wie es Ihnen passt. Wichtig: In aller Regel werden Sie im Thumb-State arbeiten. In diesem können sie von den Registern 0 bis 12 ausschließlich die Register R0 bis R7 nutzen. Das sollte aber in aller Regel genügen.\\

Mehr gibt es zu den Allzweckregistern nicht zu sagen.\\

Hier ein Beispiel in Pseudocode, um die Nutzung der Register zu verdeutlichen:\\

•	Lade den Wert von Speicheradresse 0x1FC in R0.
•	Lade den Wert von Speicheradresse 0x200 in R1.
•	Vergleiche den Wert in R0 mit dem in R1.
•	Ist R1 größer oder gleich R0, dann speichere den Wert von R1 in 0x1FC.
•	Vergleiche den Wert in R0 mit dem in R1.
•	Ist R0 kleiner als R1, dann speichere den Wert von R0 in 0x200.\\

Kontrolle\\

Dieses kleine Programm in Pseudocode ist der Kern eines größeren Algorithmus, in dem es vielfach wiederholt wird. Was ist die Aufgabe, die dieser Algorithmus löst?

\subsection{Einschub: LSB = 1 für Thumb State}

Wie Sie wissen, können (Speicher-)adressen, auf die beim Cortex-M0-Prozessor verwiesen wird nur gerade Zahlen (da durch 4 teilbar) sein. Also darf das LSB (kurz für Least Significant Bit) einer verwendbaren Adresse nur eine Null sein.\\

Es gibt jedoch eine Besonderheit: Wenn Sie im Thumb State programmieren, dann muss (!) bei einigen Befehlen das LSB einer Adresse auf 1 gesetzt werden. Das entspricht natürlich einer Erhöhung des Wertes um 1. Bei der Ausführung der Operation prüft der Prozessor also zunächst, ob das LSB gleich 1 ist. Ist das der Fall, zieht er 1 von der übergebenen Adresse ab und führt dann die Operation als Thumb State Operation also 16-bittig mit der jetzt richtigen Adresse aus. Bei welchen Befehlen das der Fall ist, können Sie sich nicht durch logisches Denken erschließen, sondern ausschließlich durchs Lesen der entsprechenden Dokumentation erfahren.

\subsection{SP – Der Stack Pointer (alternativ R13)}

Es kommt im Laufe eines Programms immer wieder vor, dass Daten vorübergehend zwischengespeichert werden müssen, die in einer bestimmten Reihenfolge gespeichert werden sollen. Das ist beispielsweise dann der Fall, wenn ein Prozess unterbrochen aber später fortgesetzt werden soll.\\

In diesem Fall wird dann die Adresse, an der der Prozess unterbrochen wurde in einer Datenstruktur abgelegt, die als Stack bezeichnet wird.\\

Für den Fall, dass Sie noch keine Veranstaltung zu Algorithmen und Datenstrukturen gehört haben, hier eine kurze Einführung in das Thema: Informatiker sind Meister der Abstraktion und Strukturierung. Nicht missverstehen: Abstraktion und Strukturierung sind zentrale Werkzeuge aller naturwissenschaftlichen Disziplinen und Ingenieurswissenschaften, aber im Gegensatz zu Elektrotechnikern beginnen Informatiker und Mathematiker ab dem ersten Semester mit Abstraktionen und Strukturen. Und so dreht sich dann gleich zu Beginn die Frage darum, wie man denn in bestimmten Situationen Daten am sinnvollsten aufbewahren kann und wie man am effizientesten auf sie zugreifen kann. Lösungen für die sinnvolle Aufbewahrung nennt man Datenstrukturen, Lösungen für effiziente Zugriffsweisen u.a. nennt man Algorithmen. Und ein Stack ist eine Datenstruktur. Wobei beide Begriffe auch dann verwendet werden, wenn sie keine effizienten Ansätze umsetzen. Aber mit solchen Fällen sollten Sie sich nur auseinandersetzen, wenn Sie wissen wollen, warum ein anderer Ansatz sinnvoller ist.\\

Einen Stack kann man sich als simplen Stapel vorstellen, bei dem Elemente bzw. Daten nur in der umgekehrten Reihenfolgen abgenommen werden können, in der sie aufgelegt wurden. Aber wohlgemerkt: Wir reden hier über Informatik, die Vorstellung eines Stapels ist also nur ein Modell, das der Anschaulichkeit dient.\\

Konkret wird ein Stack wie folgt umgesetzt: Nachdem man den Speicher so behandelt, als wenn er (bei einem 32-Bit-Prozessor) aus Einheiten mit 32 Bit Länge bestehen würde und man diesen Einheiten fortlaufenden Nummern gegeben hat, legt man willkürlich eine dieser Adressen als Startadresse eines neuen Stack fest. Um zu speichern, wo das aktuellste Element des Stack sich befindet, benötigt man nun einen Stack Pointer. Das ist schlicht ein Speicherbereich bzw. ein Register, dessen einzige Aufgabe darin besteht, zu speichern, wo sich das zuletzt gestackte Element befindet. Beim Cortex-M0 gibt es dafür das Register SP. Sobald ein Wert im Stack gespeichert oder von dort entfernt wird, muss der Wert des Stack Pointer angepasst werden. Das passierte beim Cortex-M0 bei den Branch-Befehlen im Hintergrund, sodass Sie es dann und nur dann nicht selbst machen müssen.\\

Jetzt kommt die erste Hürde für Einsteiger: Es gibt keinerlei generelle Vorschrift dafür, wie der Stack Pointer initialisiert oder geändert werden darf. Was nun beschrieben wird, dürfen Sie also ausschließlich bei Cortex-M0-Prozessoren voraussetzen; bei allen anderen Prozessoren müssen Sie prüfen, wie der Ablauf funktioniert, bzw. ob es dort überhaupt eine entsprechende Konvention gibt. Denn selbst das dürfen Sie nicht voraussetzen.\\

Greifen Sie jetzt am besten zu Stift und Papier und versuchen Sie sich das folgende Beispiel mittels einer Skizze zu veranschaulichen.\\

•	Die erste Zeile Ihres Programms weist dem Speicher 0x0 bis 0x3 des Cortex-M0 die Adresse zu, auf die der SP zu Beginn verweisen soll.\\

Beispiel: Angenommen, die Adresse, auf die der Stack Pointer zu Beginn des Programms verweisen soll lautet 0 x 2EF5 2C04. Dann weist Ihr Programm \\


der Adresse 0x0 den Wert 0x2C zu, 
der Adresse 0x1 den Wert 0x4, 
der Adresse 0x2 den Wert 0x2E und 
der Adresse 0x3 den Wert 0xF5.\\


Im Debugger würde das so aussehen:\\
 

0x00000000	2C04		DCW		0x2C04
0x00000002	2EF5		DCW		0x2EF5\\


DCW ist der Befehl, der einem Speicherbereich einen Wert zuordnet. Und auf der linken Seite steht lediglich vor den Speicheradressen ein 0x, weil Werte im Speicher grundsätzlich als Hexadezimalwerte interpretiert werden.\\

Wichtig: Sehen Sie bitte genau hin: Sie programmieren die Adresse 0x2EF52C04 in zwei Schritten, bei denen sie zuerst die hinteren vier Ziffern eingeben und erst danach die vier vorderen Zahlen! Das gilt für alle Fälle, in denen Sie eine 32-bittige Zahl zuweisen.\\

•	Im Gegensatz zum anschaulichen Beispiel wächst der Stack beim Cortex-M0 nicht von unten nach oben, sondern von größerer Adresse zu kleinerer Adresse. In anderen Worten: Die Adressen des Stack werden mit jedem gespeicherten Wert um 0x4 kleiner!\\

•	Von der Logik her verweist der SP jeweils auf die Adresse, an der das zuletzt gespeicherte Element liegt. Sie weisen hier also beim Programmstart nicht die erste Adresse des Stack zu, sondern die um 0x4 größere Adresse. \\

Zur Erinnerung: Der Cortex-M0 liest bei jedem Arbeitsschritt 32 Bit bzw. 4 Byte und damit 4 Adressen am Stück ein. Und deshalb haben wir es gerade mit dem Faktor 4 bei der Adressierung zu tun.\\

Beispiel: Sie wollen den ersten Wert an der ersten möglichen Adresse unterhalb 0x200 speichern. Daraus ergibt sich die Adresse 0x1FC. Dennoch legen Sie als Initialwert für den SP 0x200 fest, da der Wert des SP auf die Adresse verweist, an der zuletzt ein Element gespeichert wurde. Und das gilt eben auch, wenn im Stack noch kein Element gespeichert wurde. \\

In anderen Worten: Der SP wird nicht (!) mit der Adresse initialisiert, in der das erste Element gespeichert werden soll, denn bei der Speicherung eines Elements im Stack wird zuerst der SP auf die nächste mögliche Adresse gesetzt und dann wird an der dortigen (!) Adresse das Element gespeichert wird. Diese ist die nächste um 4 bzw. 0x4 kleinere Adresse als diejenige, auf die der SP zuletzt verwies.\\

Wenn wir den Initialwert des Beispiels, also die Adresse 0x200 verwenden, dann liegt das erste Element, das wir auf dem Stack ablegen unter den Adressen 0x1FC, 0x1FD, 0x1FE und 0x1FF. Der SP verweist anschließend auf die Adresse 0x1FC. Legen wir ein weiteres Element auf dem Stack ab, so liegt dieses (selbst wenn es nur eine Länge von einem Bit hat) unter den Adressen 0x1FB, 0x1FA, 0x1F9, 0x1F8 und der SP beinhaltet den Wert 0x1F8.\\

Für die Zugriffe auf den Stack gibt es wieder festgelegte Bezeichnungen: Fügen wir ein Element zum Stack hinzu, so nennen wir das PUSH, entfernen wir das zuletzt hinzugefügte, bezeichnen wir das als PULL. Außerdem gibt es noch den Befehl PEEK. Damit wird ein Zugriff auf das oberste Element eines Stack bezeichnet, das dabei aber nicht entfernt wird. In aller Regel kommen aber nur Push und Pull zum Einsatz.\\

Ob der Prozessor bei einem Pull tatsächlich den Wert an der Adresse löscht (wie auch immer ein Löschen aussehen soll), von der gepullt wurde, das steht nicht fest. Somit könnte ein Pull im Grunde wie ein Peek sein, bei dem lediglich der SP verschoben wurde.\\

Aufgabe:\\

Warum gibt es kein Löschen bei der maschinennahen Programmierung?\\

\subsubsection{Kontrolle}

Führen sie die nachfolgenden Schritte mit Stift und Papier durch und notieren Sie in Pseudocode: Sie wollen den Stack initialisieren und ihn für eigene Operationen verwenden.\\

(i) Initialisieren Sie den SP mit einer Startadresse, die beim Cortex-M0 existiert und größer als 
0xFFFF +2 ist. 
(ii) Pushen Sie den Wert 17 auf den Stack.
(iii) Unter welcher Adresse wurde der Wert gespeichert? (Hier gibt es mehrere mögliche Antworten, die alle vier richtig sind.)
(iv) Welcher Wert steht jetzt an dieser Adresse?
(v) Pushen Sie den Wert 32 auf den Stack.
(vi) Auf welche Adresse verweist der Stack Pointer jetzt?
(vii) Welcher Wert steht dort?
(viii) An welcher Adresse würde ein Element abgelegt, wenn Sie es jetzt pushen würden?\\

Hinweis: Lassen Sie sich nicht irritieren: Wenn Sie die Funktionsweise des Stack, des Stack Pointers und die Nutzung des Speichers beim Cortex-M0 verstanden haben, ist das eine ganz simple Übung. Teilweise sind die Antworten auf Fragen hier tatsächlich identisch. Es handelt sich aber nicht um einen „Idiotentest“.\\

\subsection{LR – Das Link Register (alternativ R14)}
Wenn ein Assemblerprogramm nach Abarbeitung einer Programmzeile nicht in die nächste Zeile, sondern in eine beliebige andere Zeile des Programms springt, sprechen wir von einem Branch, bzw. vom branchen. Alternativ ist die Rede vom Aufruf einer Subroutine. Eine solche Subroutine kann beispielsweise ein Programmteil sein, der an verschiedenen Stellen im Programm benötigt wird. Um ihn dann nicht mehrfach programmieren zu müssen, wird er in einem anderen Bereich des Speichers abgelegt und mit einer Zeile beendet, die sinngemäß bedeutet: Branche (zurück) auf die Adresse, die im LR gespeichert ist. Umgangssprachlich würde man sagen: Setze das Programm an der Stelle fort, von der aus du in diese Subroutine gesprungen bist.\\

Hier der Assembler-Befehl, um am Ende einer Subroutine zurück zu der Stelle im Programm zu springen, an der die Ausführung zuletzt war:\\

BX LR\\

Damit wären wir auch beim Sinn dieses Registers: Das Link Register speichert beim branchen die Adresse, an der das Programm sich gerade befindet, bzw. an der es nach Ende der Subroutine fortgesetzt werden soll. \\

Kontrolle:\\

Warum müssen Sie nach bisherigem Kenntnisstand davon ausgehen, dass ein Branch aus einer Subroutine zu einem schweren Fehler im Programmablauf führt bzw. wodurch würde der verursacht?   
  
\subsection{PC – Der Program Counter (alternativ R15)}

In diesem Register speichert der Prozessor diejenige Adresse, an der sich die Programmzeile befindet, die als nächstes ausgeführt wird. Um also die aktuelle Programmzeile zu finden, müssen Sie von dieser Adresse 4 abziehen.

\subsection{xPSR – Das kombinierte Statusregister}

Das xPSR nutzen Sie, wenn Sie den Debugger nutzen, um bestimmte Informationen über die Zustände zu erhalten, die durch eine Operation erreicht wurden. Das geht lediglich beim Debuggen, weil die Werte sich bei jeder ausgeführten Operation ändern. Das xPSR ist kein eigenständiges Register, lässt sich aber bei der Programmierung wie ein Register verwenden und stellt die für uns relevanten Inhalte der drei folgenden „echten“ Register gemeinsam dar: \\

APSR (kurz für Application Program Status Register), für das auch die Bezeichnung ALU Register üblich ist, 
IPSR (kurz für Interrupt Program Status Register) und 
EPSR (kurz für Execution Program Status Register). 

Die drei PSR zeigen über die folgenden Bits die in der nachfolgenden Tabelle eingetragenen Hinweise dar. Bitte beachten Sie, dass hier die tatsächliche Nummer jedes Bits aufgeführt wird (beginnend bei 0) und nicht die Anzahl (beginnend bei 1).\\

Hinweis: Die PSR beinhalten diese Werte nach der Ausführung einer Operation; sobald eine weitere Operation ausgeführt wurde, werden die PSR mit neuen Daten überschrieben. Um also zu sehen, welche Bits im (x)PSR gesetzt werden, wenn eine bestimmte Operation ausgeführt wurde, müssen Sie im Debugger eine Markierung an der entsprechenden Stelle des Quellcode setzen.\\

Ein typischer Anfängerfehler besteht darin, eine Markierung an der falschen Stelle im Quellcode zu setzen bzw. bei einem „Programmfehler“ die PSR zur falschen Programmzeile zu untersuchen.\\

Bit Nr.	Abkürzung	Bedeutung	Originäres Register
31	N	Negative	APSR
30	Z	Zero	
29	C	Carry	
28	V	Overflow	
24	Keine	0 = ARM, 1 = Thumb	EPSR
5…0	Keine	Exception Type	IPSR\\


Für alle folgenden Angaben gilt wie immer bei der maschinennahen Programmierung: Ob etwas Sinn macht, ist für den Prozessor irrelevant; er führt seine Aufgaben nach festen Vorgaben aus und Sie als Entwickler müssen prüfen, was das für Ihr Programm bedeutet. Hier gibt es drei Optionen: \\

(i) Es macht Sinn / erfüllt eine gewünschte Aufgaben.
(ii) Es ist irrelevant.
(iii) Sie haben einen Fehler einprogrammiert.\\

•	Bit 31 (N / Negative) wird bei einer Operation gesetzt, wenn Sie eine signed operation ausgeführt haben und das Ergebnis ein negativer Wert ist.
•	Bit 30 (Z / Zero) wird gesetzt, wenn das Ergebnis einer Operation die Zahl 0 ist. Häufigster Anwendungsfall: Vergleich zweier Zahlen.
•	Bit 29 (C / Carry): Hier wird zwischen zwei Fällen unterschieden:
o	Bei einer unsigned addition wird das Bit gesetzt, wenn ein unsigned overflow aufgetreten ist.
o	Bei einer unsigned subtraction wird das Bit genau dann gesetzt, wenn Bit 28 nicht gesetzt wird.
•	Bit 28 (V / Overflow): wird gesetzt, wenn eine signed operation durchgeführt wurde und außerdem ein signed overflow aufgetreten ist.\\

Für Bit 24 steht die Bedeutung ja bereits in der Tabelle und die Exception Types folgen im Abschnitt zu Interrupts und Exceptions.

\subsection{PRIMASK – Das Interrupt-Maskierungs-Register}

Hier handelt es sich zwar um ein vollwertiges 32-Bit-Register, aber für uns ist lediglich das LSB von Belang: Wenn wir es auf 1 setzen, dann wird der priority level effektiv auf 0 gesetzt. Da wir bislang nur kurz über Interrupts gesprochen haben, ist Ihnen nicht klar, was das bedeutet, deshalb sie hier nur kurz angemerkt, dass Interrupts die einzige Möglichkeit sind, damit der Prozessor beispielsweise auf Eingaben an einer Tastatur reagiert. Je nach Prioritätslevel werden aber bestimmte Interrupts grundsätzlich ignoriert. Und wenn dieser auf 0 gehoben wird, dann werden ausschließlich die allerwichtigsten Interrupts verarbeitet. Dazu gehört beispielsweise der Interrupt, der durch das Drücken des Resetschalters ausgelöst wird.

\subsection{Weitere Register}

Neben den hier aufgeführten Registern gibt es noch das CONTROL Register, das aber nur dann von Belang ist, wenn Sie ein Betriebssystem entwickeln wollen. In dem Fall müssten Sie sich außerdem mit dem Unterschied zwischen MSP und PSP auseinander setzen; zweier Register, die jeweils einen individuellen Stack Pointer verwalten. Im Rahmen unserer Veranstaltung werden wir uns hiermit nicht beschäftigen.

\subsection{Zusammenfassung}

Sie kennen jetzt die wichtigsten Register, die Sie für die Programmierung des Cortex-M0 nutzen werden. Wenn Sie später mit C arbeiten, ist für Sie lediglich die Initialisierung des Stack Pointers sowie die Nutzung des xPSR von Belang. Die anderen Register „verwaltet“ C für Sie.

\section{Die Aufteilung des Speichers}

Bislang haben wir lediglich allgemein über Speicher gesprochen, also darüber, dass Speicher beim Cortex-M0 in 8-Bit Segmente unterteilt wird, die über Adressen angesprochen werden, und dass Sie regelmäßig nur Einheiten von 4-Byte-Blöcken gemeinsam aus dem Speicher in die Register (und zurück) übertragen können.\\

Was wir damit außen vor gelassen haben sind insbesondere zwei Aspekte: Zum einen haben wir noch nicht darüber gesprochen, dass beim Cortex-M0 bestimmte Speicherbereiche für bestimmte Aufgaben vorgesehen oder sogar reserviert sind. Zum anderen haben wir nicht über externe Speicher gesprochen und wie diese Daten in den Prozessor geladen werden können.\\

Deshalb sehen wir uns jetzt den zweiten Teil des Mikrocontrollers an, mit dem Sie arbeiten werden: Den internen Speicher. (Externen Speicher sprechen Sie über das SPI an, aber dazu kommen wir später.)\\

\subsection{Die Aufteilung des Speichers des Cortex-M0}

Sie wissen bereits, dass der Cortex-M0 bis zu 4 GB an Speicher direkt ansprechen kann, weil es sich um einen 32-Bit-Prozessor handelt. Im Falle des von uns verwendeten Entwicklerboards von Olimex, auf dem ein LPC1114 von NXP zum Einsatz kommt, sind jedoch lediglich 32 KB programmierbar. Allerdings könnten wir mit der Evaluationsversion des MDK von Keil gar nicht mehr als 32 KB internen Speicher nutzen. Es folgt eine Aufstellung, aus der hervorgeht, wie beim Cortex-M0 4 GB an Speicher aufgeteilt werden:\\

Speicherbereich	Größe	Inhalt
0x 0 … 0x 1F FF FF FF	0,5 GB	Programm Code und Exception Vector Tabelle
0x 20 00 00 00 … 0x 3F FF FF FF	0,5 GB	Datenspeicher (z.B. static RAM)
0x 40 00 00 00 … 0x 5F FF FF FF	0,5 GB	Peripherie
0x 60 00 00 00 … 0x 9F FF FF FF	1 GB	Externer Speicher
0x A0 00 00 00 … 0x DF FF FF FF	1 GB	Externe Peripherie
0x E0 00 00 00 … 0x FF FF FF FF	0,5 GB	Systeminterna und private Peripherie\\


Beachten Sie bitte, dass es sich bei dieser Aufstellung nur um einen möglichen Fall handelt. Sie müssen für jeden Mikrocontroller prüfen, wie dort die Speicherverteilung organisiert ist. Dazu greifen Sie auf die jeweilige Dokumentation des Herstellers zurück.\\

Nachtrag zum SP: Bei ausreichend Speicher wird der Stack Pointer mit der Adresse 0x40000000 initialisiert, da er so bei der höchsten Adresse des Datenspeichers beginnt und damit so spät wie möglich mit Programmteilen kollidieren dürfte. Letzteres ist auch der Grund, warum er in umgekehrter Reihenfolge addressiert.

\subsection{Weitere Bezeichnungen für Speichereinteilungen}

Im vorigen Abschnitt haben Sie die Aufteilung des Speichers nach den enthaltenen Daten kennen gelernt. Gelegentlich wird bei der Einteilung des Speichers auch eine Dreiteilung vorgenommen. Dann bezeichnet man Speicherbereiche mit Data, Heap und Stack.\\

Es gibt auch Fälle, in denen es mehrere Data/Heap/Stack-Bereiche gibt. Damit wären wir aber bei der Programmierung von embedded OS und/oder einer Form der Verarbeitung paralleler Prozesse. Und das geht über den Inhalt dieses Kurses hinaus. Also werden wir hier davon ausgehen, dass es jeweils nur einen Stack, einen Heap und einen Data-Bereich gibt.\\

Den Stack haben Sie bereits bei der Einführung in die Arbeitsweise des Stack Pointers im Bereich der Register des Cortex-M0 kennen gelernt. Deshalb an dieser Stelle keine weiteren Erläuterungen dazu.\\

Mit Data meint man bei dieser Aufteilung den Speicherbereich, der mit den Programmdaten belegt ist, die vor dem Start des Rechners belegt sind.\\

Mit Heap ist dagegen der Speicherbereich gemeint, der während der Laufzeit des Programms belegt wird. Dieser ist also nicht fest abgegrenzt, sondern ändert sich ähnlich wie der Stack kontinuierlich während der gesamten Laufzeit des Programms. Im Gegensatz zum Stack beginnt der Heap direkt nach dem Data-Bereich und wächst in naiver Reihenfolge. Damit ist gemeint, dass neue Daten im Heap an Adressen abgelegt werden, die höher als die bisherigen sind. Der Stack dagegen wächst ja von hoher zu niedriger Adresse.\\

Somit kann es bei größeren Programmen passieren, dass Stack und Heap kollidieren, dass also ein neuer Eintrag im Heap einen bestehenden Eintrag im Stack überschreibt oder umgekehrt. Diese Fälle werden im Rahmen dieses Kurses im Regelfall nicht auftreten. Sie müssen sich aber bewusst sein, dass auch diese Fälle von Ihnen als Programmierer später berücksichtigt werden müssen. Denn auch hier gilt: Der Mikroprozessor ist gar nicht in der Lage, solche Fälle zu erkennen, also müssen Sie sich darum kümmern.

\subsection{Anbindung von externem Speicher über den SPI Controller}

Sie wissen bereits, dass das Entwicklerboard eine Schnittstelle zur Nutzung externer Peripherie beinhaltet. Über diese können Sie unter anderem einen Speicherbaustein ansprechen, um so die ersten Schritte beim Arbeiten mit externem Speicher durchzuführen. Wie das genau funktioniert folgt später.\\

Kontrolle\\

Sie wissen jetzt, das es zwar ein Standardmodell für die Verteilung der verschiedensten Daten im Speicher eines Cortex-M0 gibt, dass Sie aber für jedes Modell in der Dokumentation des Herstellers prüfen müssen, wie diese Verteilung im Einzelfall realisiert wurde. Das müssen Sie wissen, da Sie bei der maschinennahen Programmierung selbst die Nutzung des Speichers programmieren müssen. Wenn Sie hier Fehler machen, kann es passieren, dass Sie Daten überschreiben, die für die Systemsteuerung von Belang sind.\\

Sie wissen weiterhin, dass die Bezeichnungen Stack, Heap und Data verwendet werden, um die Speicherbelegung während der Laufzeit von Programmen abzugrenzen.\\

Außerdem wissen Sie, über welches Interface Sie üblicherweise externen Speicher anschließen können.

\section{Exceptions und Interrupts}

Diejenigen von Ihnen, die bereits eine objektorientierte Sprache erlernt haben kennen deshalb die sogenannten Exceptions. Dort haben Sie Exceptions vorrangig als eine Spezialform von Fehlern kennen gelernt, die während er Laufzeit eines Programms bereinigt werden können. Außerdem sind Exceptions dort auch eine Möglichkeit, um unterschiedliche Fälle im Programmablauf effizient zu steuern.\\

Bei der maschinennahen Programmierung ist ebenfalls die Rede von Exceptions, hier stellen Sie aber einen Spezialfall der sogenannten Interrupts dar. Wie Sie schon mehrfach gelesen haben sind Interrupts (dt.: Unterbrechungen) so etwas wie Änderungen im Programmablauf. Im Gegensatz zu einem Branch geht es hier aber nicht darum, häufig auftretende Programmabläufe in Form von Subroutinen auszulagern. Vielmehr ist ein Interrupt oder besser gesagt ein IRQ (kurz für external Interrupt ReQuest) die Anfrage einer Peripherie, mit der diese den Prozessor auffordert, seinen aktuellen Programmablauf zu unterbrechen, und dafür eine externe Eingabe zu verarbeiten. (Das passiert z.B. wenn der Nutzer eines Rechners eine Taste drückt.)\\

Sie müssen sich zum Verständnis klar machen, dass ein Prozessor im Grunde nur Zeile für Zeile ein Programm abarbeiten und ansonsten keine weiteren eigenständigen Entscheidungen treffen kann. Bei Mehrkernprozessoren trifft das zwar auf jeden Kern individuell zu, aber am Grundprinzip ändert sich nichts. Es muss also eine Möglichkeit geben, damit Komponenten eines Rechners, die nicht Teil des Prozessors sind, den Prozessor dazu bewegen können seine Arbeit zu unterbrechen, um andere Prozesse auszuführen. (Im Falle eines Tastendrucks durch den Nutzer könnte das z.B. die Speicherung des Buchstaben sein, dessen Taste der Nutzer gedrückt hat.) Danach soll er wieder seine unterbrochene Arbeit fortsetzen.\\

Wenn Sie das verstanden haben, dann haben Sie auch begriffen, warum ein Mehrkernprozessor u.U. genauso langsam ist wie ein ansonsten identisch gebauter Einzelkernprozessor: Die einzelnen Kerne haben keine Möglichkeit ohne externe Kontrolle als Verbund zu agieren.\\

Das mag Ihrer bisherigen Vorstellung von Computern wiedersprechen, weil Sie es i.d.R. gewohnt sind, dass Sie per USB eine neue Komponente an den Rechner anschließen können und Sie dann quasi als Teil des Computers nutzen können. Doch das ist eben nur die oberflächliche Perspektive, die einem naiven Nutzer vorgaukelt, so ein Computer sei ein ganz simples Gerät, das trotzdem unglaubliche Möglichkeiten bietet. Tatsache ist: Ein Computer ist wesentlich komplexer als es für einen Nutzer den Anschein hat. Interrupts sind das einzige Mittel, damit externe Komponenten überhaupt irgendwie genutzt werden können. Und wir reden hier von wesentlich mehr Dingen als nur von Maus, Tastatur, Drucker, usw.\\

Sie können beim Cortex-M0 übrigens bis zu 32 Arten von IRQs definieren. Es ist zusätzlich möglich einen IRQ von mehreren Komponenten gemeinsam nutzen zu lassen.\\

Kontrolle:\\

Ein Interrupt ist ein so etwas wie ein Signal, mit dem eine beliebige Komponente den Prozessor auffordert, den aktuellen Programmablauf zu unterbrechen und eine andere Aufgabe zu erfüllen. Im Gegensatz zu höheren Programmiersprachen ist hier bereits die Verarbeitung eines Tastendrucks eine eigenständig zu verarbeitende Aufgabe.\\

Aufgabe:\\

Was ist der Unterschied zwischen einem IRQ und einem Branch?

\subsection{Exception Types}

Sie erinnern sich noch? Die letzten fünf Bit des xPSR zeigen den Exception Type an. Jetzt kommen wir dazu, wie Sie Exception Types programmieren: An Anfang des Quelltextes legen sie für jede Exception eine Speicheradresse fest, an der der Code steht, der auszuführen ist, wenn die Exception auftritt.\\

Es gibt 35 Exception Types, die Sie nutzen können: Die Exceptions 1 bis 3 (Reset, NMI, Hard fault) und 11, 14, 15 (SVCall, PendSV und SystTick) haben feste Bedeutungen, aber freie Adressen. Die Exceptions 16 bis 47 sind frei belegbar. Ausschließlich über diese 33 IRQs können Sie dem Prozessor Eingaben vermitteln.\\

Hinweis: \\

Damit programmieren Sie lediglich, wo sich der Programmcode befindet, der bei einem bestimmten IRQ ausgeführt werden soll, aber nicht, wodurch er ausgelöst wird. Das tun Sie z.B. bei der Zuordnung von Eingaben des SPI.\\

Beispiel:\\

Sie wollen ein Spiel programmieren, bei dem Sie vier Tasten für die entsprechende Richtung und zwei Tasten für anwählen und abbrechen nutzen wollen. Dann könnten Sie beispielsweise den Exception Types 0x20 bis 0x23 Startadressen zuordnen, die den Code enthalten, der besagt, was passiert, wenn in eine Richtung gesteuert wird. Und den Exception Types 0x25 und 0x26 ordnen Sie die Startadressen mit dem Code für anwählen und abbrechen zu.\\

Sollte Sie diese Formulierung „Startadresse zuordnen“ verwirren, dann machen Sie sich bitte bewusst, dass bei der maschinennahen Programmierung ein Programm aus Zeilen besteht, die an aufeinander folgenden Zeilen im Speicher abgelegt sind. Die Startadresse ist dann schlicht die Adresse, an der ein Programm bzw. ein Programmteil beginnt.\\

Alle IRQs, die soeben nicht aufgeführt wurden sind für das System reserviert und müssen mit bestimmten vorgegebenen Werten initialisiert werden. (Ignorieren Sie bitte für den Augenblick alle IRQs außer den Nummern 1 und 16 bis 47.)\\

Den Reset IRQ können Sie im Bereich der maschinennahen Programmierung als eine Art Neustart verstehen. Im Gegensatz zur Digitaltechnik ist damit also nicht gemeint, dass ein Eingang auf 0 gesetzt wird. Er muss so programmiert werden, dass er auf den Anfang des eigentlichen Programms verweist. \\

! LSB + 1 ! \\

Hier (im Gegensatz zur Intialisierung des SP, aber dafür bei allen IRQ Initialisierungen) haben wir wieder einen Fall, in dem wegen des Thumb State das LSB auf 1 gesetzt werden muss. Wenn die erste auszuführende Zeile also in der Adresse 0x10C steht, so müssen Sie dem Reset IRQ die Adresse 0x1D zuordnen. \\

Wenn Sie die Formulierung „auf den Anfang des eigentlichen Programms“ verwirrt: Am Beginn Ihres Quellcodes initialisieren Sie die Adressen für den SP und die IRQs. Diese Initialisierung brauchen Sie bei einem Reset IRQ nicht erneut durchführen lassen, da Sie bis zur Überschreibung des Speichers durch eine Neuprogrammierung erhalten bleibt. Sie können also direkt zum auszuführenden Teil Ihres Programms springen. Dieser ist hier mit dem Anfang des eigentlichen Programms gemeint.

\subsection{Initialiserung von IRQs}

Eine Intialisierung ist die erstmalige Belegung eines Elements mit einem Wert. Elemente in diesem Sinne können so etwas wie die Bezeichner von IRQs oder auch andere Bezeichner in einem Programm sein. Ob dieser Wert nun als Zahl oder als Adresse zu interpretieren ist hängt vom Kontext ab, macht aber für das, was Sie als Initialisierung programmieren keinen Unterschied.\\

An dieser grundsätzlichen Bedeutung ändert sich übrigens nichts, wenn Sie z.B. über die imperative Programmierung in einer höheren Sprache reden. Dort werden dann zwar andere „Dinge“ initialisiert, aber es geht immer noch um nichts anderes, als das erstmalige Abspeichern eines Wertes zu dem jeweiligen „Ding“.\\

Die Initialisierung eines IRQ besteht also darin, dass Sie (wie zuvor beim SP) schlicht eine Adresse des Speichers „eintragen“. An dieser Adresse beginnt dann der Programmcode derjenigen „Subroutine“, die im Falle dieses IRQs ausgeführt werden soll. Diese Idee unterscheidet sich im Grunde nicht groß von einem Branch, nur dass ein Branch einen Programmteil aufruft, der grundsätzlich zu einem bestimmten Zeitpunkt bei der Ausführung eines Programmes ausgeführt werden soll. Ein Interrupt dagegen (deshalb auch die Bezeichnung als Unterbrechung) soll diesen Programmablauf unterbrechen und einen anderen Programmteil einschieben.\\

Wie mehrere IRQs behandelt werden, die parallel zueinander auftreten, werden wir hier nicht explizit ansprechen. Sie brauchen sich über solche Fälle vorerst keine Gedanken zu machen. Wichtig ist, dass Sie das Grundprinzip von IRQ und Branch verstanden haben.

\subsection{IRQs im Quellcode}

Nun aber zur eigentlichen Programmierung: Sie erinnern sich noch daran, wie Sie in einem Programm den Stack Pointer initialisieren mussten? Wesentlich einfacher aber programmiertechnisch sehr ähnlich initialisieren Sie die IRQs: Sie legen für jeden IRQ fest, an welcher Adresse der Programmcode zu finden ist, der ausgeführt werden soll, wenn dieser IRQ erfolgt. Im Gegensatz zum Stack wird der Programmcode aber in der naiven Reihenfolge abgearbeitet: Der Anfang eines Programms befindet sich bei einer kleineren Adresse als das Ende.\\

Denken Sie bitte wieder daran, dass die Startadressen jeder Subroutine durch 4 teilbar sein muss, weil der Prozessor immer 32 Bit am Stück einliest, auch wenn eine Programmzeile nur 8 oder 16 Bit lang ist.

\subsection{Codebeispiel}

Das folgende Beispiel ist nur ein Auszug, der einige IRQs außen vor lässt, die Sie ebenfalls programmieren müssen:\\

Sie wollen dem
SP den Wert bzw. die Adresse 0x1000 0268, \\
dem Reset IRQ (IRQ 1) den Wert 0x1B0 und 
dem ersten frei programmierbaren IRQ (IRQ 16) den Wert 0x1C
zuordnen, dann sieht Ihr Quellcode wie folgt aus:\\

DCW	0x0268		; Die LSB+1-Regel gilt nicht bei der SP-Initialisierung.
DCW	0x1000
DCW	0x01B1		; Thumb! Bei den IRQs muss wieder das LSB auf 1 gesetzt werden: 1 nicht 0!
DCW	0x0000
… 
Jetzt folgen 28 Zeilen, die den 14 IRQs zwischen Reset IRQ und dem ersten frei programmierbaren IRQ Adressen zuordnen.
…
DCW 	0x001D		; Thumb! Bei den IRQs muss wieder das LSB auf 1 gesetzt werden: D nicht C!
DCW	0x0000
…\\

Jetzt folgen noch 62 Zeilen, die den verbleibenden 31 IRQs jeweils einen Wert zuweisen. IRQs, die nicht verwendet werden sollen, können beispielsweise mit der Adresse initialisiert werden, mit der der Reset IRQ initialisiert wurde. Wie angekündigt haben wir hier eine Reihe an IRQ Initialisierungen ignoriert, damit Sie das Grundprinzip verstehen. So können Sie sich leichter auf diejenigen Initialisierungen konzentrieren, die für Sie am wichtigsten sind.\\

Fällt Ihnen das schwer? Kein Problem, auch diese Arbeit nimmt Ihnen bei der Arbeit mit C die Programmierumgebung ab. Dennoch sollten Sie grundsätzlich verstanden haben, was hier passiert, damit Sie nachvollziehen können, warum bestimmte Teile im C-Code Ihres Programms auftauchen, ohne dass Sie sie programmiert hätten.\\

Anmerkung: Die Reset Exception wird im Thread Mode ausgeführt, in dem ja auch die Ausführung des Programmcodes stattfindet. Bei anderen Exceptions werden Sie später den Wechsel zum Exception Handler und zurück zum Thread Handler programmieren müssen.\\

Kontrolle\\

Sie kennen den Unterschied zwischen einem Branch und einem Interrupt bzw. einer Exception. Sie wissen, wie Sie in Assembler beim Cortex-M0 Adressen zu einem bestimmten IRQ zuordnen, und dass man das Initialisierung nennt. \\

Sie denken stets daran, dass Sie bei den Initialisierungen der IRQs jeweils eine 1 addieren müssen, auch wenn das bei der Initialisierung des SP nicht der Fall ist. \\

Sie wissen weiterhin, dass Sie bei jedem Programm für den Cortex-M0 in Assembler insgesamt 47 IRQ Initialisierungen durchführen müssen, auch wenn Sie insbesondere bei den vom System reservierten Fällen noch nicht wissen, welche Adressen hier jeweils zuzuordnen sind. \\

Sie wissen, dass für die Initialisierung eines IRQ bzw. des SP jeweils zwei Programmzeilen nötig sind. \\

Sie wissen auch warum das so ist, weil Sie die Inhalte der vorigen Abschnitte dieses Kapitels nicht vergessen haben.\\

Und vor allem freuen Sie sich, weil Sie all das in C nicht zu beachten brauchen.

\section{NVIC – Der Nested Vectored Interrupt Controller}

Im vorigen Abschnitt haben Sie erfahren, wie Sie dem Prozessor Programmcode übermitteln können, den dieser im Falle eines bestimmten Interrupts ausführen soll. Der NVIC (kurz für Nested Vectored Interrupt Controller) kontrolliert schlicht, wann welcher IRQ tatsächlich an den Prozessor weitergeleitet wird. Für diese Kontrolle ist unter anderem der priority level jedes IRQs von Belang.\\

Damit er seine Aufgabe möglichst schnell ausführen kann wird ein Virtualisierungsverfahren angewendet, das als Memory Mapping bezeichnet wird. Bei der Programmierung werden Sie also nicht den NVIC selbst programmieren, sondern Sie werden Speicherbereiche ändern, was am Ende dazu führt, dass der NVIC seine Arbeit so durchführt, wie Sie das wollen. Außer bei denjenigen von Ihnen, die diese ganzen Erklärungen ignorieren und einfach den Programmcode kopieren, den sie irgendwo im Netz oder bei anderen Kommilitonen kopiert haben und die dann gnadenlos durch die Klausur fallen, weil sie nicht wissen, was all das hier soll.

\subsection{Memory Mapping}

Im Regelfall stellen die Datenübertragungswege zwischen den Komponenten eines Rechners einen Flaschenhals dar, weil sie Daten zum Teil wesentlich langsamer übertragen, als die Komponenten sie verarbeiten können. (Erinnern Sie sich bezüglich der Gründe bitte an die Hinweise zur Nachrichtentechnik aus dem ersten Kapitel.) Eine Möglichkeit, um diesen Flaschenhals zu entschärfen besteht nun darin, Daten im Speicher abzulegen und sie danach erst zu übertragen. Und das bezeichnet man als Memory Mapping. \\

Auf die Details werde ich hier nicht eingehen, da Sie für die Zwecke dieser Veranstaltung lediglich wissen müssen, dass Sie auf andere Register zugreifen, indem sie Werte ändern, die im Speicher gespeichert sind. Sie werden in dieser Veranstaltung also weder erfahren, wie Memory Mapping den genannten Flaschenhals entschärft, noch werden Sie erfahren, wie der Mikrocontroller die Datenübertragung durchführt, nachdem Sie die entsprechenden Werte geändert haben.\\

Mehr Details zu Virtualisierungstechniken wie dem Memory Mapping erfahren Sie in Veranstaltungen zu Betriebssystemen. Alternativ empfehle ich Ihnen den Band „Operating Systems: Three Easy Pieces“ von Remzi H. und Andreas C. Arpaci-Dusseau, das Sie als open book über die Webpage \url{http://www.ostep.org} beziehen können.

\section{Das SPI – Serial Peripheral Interface}

Wie am Beginn des Kapitels beschrieben handelt es sich hier um eine Schnittstelle, die auf dem Olimex Entwicklerboard enthalten ist und über die Peripherie angesprochen werden kann, die dem eigentlichen Mikrocontroller nicht bekannt ist. Das bedeutet, dass Sie hierüber Komponenten in den Programmablauf einbinden können, die Sie selbst mit dem Entwicklerboard verbunden haben.

\section{Grundlagen der eigentlichen Programmierung}

Wenn Sie dieses Buch vom Anfang an gelesen haben, dann werden Sie schon mehrfach gedacht haben: Wann fangen wir denn endlich mit der Programmierung an?\\

Vielleicht erinnern Sie sich noch an die einführenden Hinweise dazu, was an Java der Vorteil für Programmiereinsteiger ist? Genau, Sie brauchen bei Java (im Gegensatz zur maschinennahen Programmierung) nichts darüber zu wissen, wie ein Computer aufgebaut ist; Sie können nach der Installation des JDKs und eines Editors direkt loslegen. Bei der maschinennahen Programmierung müssen Sie dagegen zunächst all das wissen, was in den bisherigen Abschnitten dieses Kapitels enthalten war, um auch nur ein einfaches Programm erstellen zu können: Register und Speicheraufteilung. Von den wirklich interessanten Dingen wie der Eingabe von Daten durch einen Nutzer sind wir immer noch weit entfernt. Die fangen dann an, wenn wir den SPI-Controller mit ins Spiel bringen.

\subsection{Die Startsequenz des Cortex-M0}

Mit Startsequenz ist gemeint, in welcher Reihenfolge der Cortex-M0 welche Schritte ausführt, nachdem er gestartet wurde. Die ersten zwei Schritte kennen Sie jetzt schon: Die Initialisierung des SP und die Initialisierung der Exception Types.\\

Es gibt weiterhin die Möglichkeit, dass noch vor diesen Initialisierungen ein Bootloader zum Einsatz kommt, aber das ist etwas, womit wir es nur dann zu tun haben, wenn ein embedded OS gestartet wird. Also brauchen wir uns darum im Rahmen dieses Kurses nicht zu kümmern.\\

Nach den genannten Initialisierungen springt das Programm an die Adresse, die im Reset IRQ genannt ist. Hier kann nun die Systeminitialisierung erfolgen. Damit ist die Initialisierung verschiedener Komponenten wie beispielsweise der Clock Control Circuitry oder des PLL (Phase Locked Loop) gemeint. Keine Sorge, wenn Ihnen diese Begriffe noch nichts sagen; für den Moment müssen Sie nur wissen, dass es dabei um Einstellungen geht, die steuern, wie sich der Mikrocontroller generell verhält. So müssen Sie beispielsweise in Abhängigkeit von angeschlossener Peripherie teilweise die Taktrate senken. Aber dazu später mehr. Diese Systeminitialisierung kann auch im Rahmen der main()-Methode eines C-Programms erfolgen. Im Falle des MDK brauchen Sie sich um die Systeminitialisierung nicht selbst zu kümmern. Das MDK generiert den nötigen Code automatisch.\\

Nach der Systeminitialisierung wird bei C-Programmen noch der C startup code benötigt. Wie Sie wissen programmieren Sie den Cortex-M0, indem Sie memory gemappte Register verändern, die dann die Einstellungen der verschiedenen Komponenten des Mikrocontrollers ändern. Bei Assembler nutzen Sie hierfür entweder die tatsächlichen Speicheradressen oder Bezeichner, die vom eingesetzten Assembler abhängen. Im Gegensatz dazu verwenden Sie in Sprachen wie C Variablen, Pointer und Funktionen. Deshalb wird noch ein Programmbereich benötigt, der die Variablen und den Speicherbereich zuordnet, die von Ihrem C-Programm genutzt werden. Dieser Programmbereich wird als C Startup Code bezeichnet. Genau wie bei der Systeminitialisierung brauchen Sie sich beim MDK nicht um den C Startup Code zu kümmern; das MDK generiert ihn selbst.\\

Kontrolle\\

Das MDK nimmt Ihnen (wie verschiedene andere IDEs) verschiedene Programmieraufgaben ab, die für die Lauffähigkeit des Programms nötig sind. In diesem Fall geht es um die Systeminitialisierung und den C Startup Code.\\

Erstere nimmt grundlegende Konfigurationsschritte vor, durch die das System überhaupt erst durch ein Programm genutzt werden kann, letztere stellt gewissermaßen eine Schnittstelle zwischen Ihrem C-Programm und dem System dar.\\

Das bedeutet für Sie: Weil wir das MDK nutzen müssen Sie bei der Programmierung lediglich den SP und die IRQs initialisieren. Mehr können und brauchen Sie noch nicht.

\subsection{Prozesse – So planen Sie Ihr Programm}

Diesen Abschnitt können Sie vorerst überspringen; die enthaltenen Informationen werden dann wichtig, wenn Ihr Programm eine Vielzahl an Aufgaben erfüllt oder die Aufgabe(n) durch parallel arbeitende Prozesse realisiert wird. Kurz gesagt sind wir hier im Bereich Software Engineering.\\

Bei größeren Projekten gilt: Bevor Sie drauflos tippen, sollten Sie sich einen Überblick über die verschiedenen Prozesse verschaffen, die im Rahmen Ihres Programms aktiv sind. Dann müssen Sie planen, wie Sie diese Prozesse im Rahmen des Programmablaufs koordinieren wollen.

\subsubsection{Polling}

Mit Polling bezeichnet man die einfachste Strategie: Alle Prozesse werden nacheinander abgearbeitet. Wenn es Prozesse mit unterschiedlicher Priorität gibt, sollten Sie eine andere Strategie wählen.

\subsubsection{Interrupt Driven}

Hier lagern Sie Prozesse in sogenannte ISRs (Interrupt Service Routinen) aus und lassen Sie durch IRQs aufrufen. So können Sie Prioritäten ins Programm bringen.

\subsubsection{Concurrent Prozesse}

Diese Strategie gehört in den fortgeschrittenen Bereich: Sie lassen jeden Prozess für eine gewisse Zeit laufen, sodass der Nutzer den Eindruck erhält, alle Prozesse würden gleichzeitig laufen. Entweder Sie programmieren diese Gleichzeitigkeit selbst oder Sie nutzen ein sogenanntes RTOS (Real Time Operating System). Diese Programmierung (also ohne die Nutzung eines RTOS, das Ihnen diese Aufgabe abnimmt) ist deshalb ein fortgeschrittenes Thema, weil Sie hier z.B. sogenannte Deadlocks vermeiden müssen. Das sind Situationen, in denen ein Prozess auf die Ausgabe eines anderen Prozesses wartet, der aber wiederum auf die Ausgabe des ersten Prozesses wartet. Damit hängt das Programm fest und es passiert scheinbar nichts mehr.\\

Wichtig: Der Begriff Realtime System bzw. Echtzeitsystem bezeichnet im Gegensatz dazu komplexe IT Systeme, die wesentlich höhere Anforderungen an die Geschwindigkeit bei der Datenübertragung zwischen den einzelnen Komponenten haben, als das z.B. im Internet der Fall ist. Wenn wir über solche Echtzeitsysteme reden, meinen wir z.B. Systeme die dynamisch auf Änderungen der Temperatur eines Kernreaktors reagieren und scheinbar ohne Zeitverlust verschiedene Zu- und Abflüsse des Reaktors in Abhängigkeit von diesen Änderungen steuern.\\

Kontrolle\\

Dieser Abschnitt gehört in den Bereich nice to know.

\subsection{Ein- und Ausgaben}

Wie schon mehrfach angemerkt müssen Sie bei der Programmierung eines Mikrocontrollers auch für solche Prozesse Code implementieren, die Sie bei höheren Programmiersprachen als von der Programmiersprache „erledigt“ betrachten können. Ein Bereich, in dem Sie hierdurch einen wesentlich höheren Programmieraufwand bekommen sind Ein- und Ausgaben.\\

Um diese zu verarbeiten müssen Sie zunächst wissen, was für ein Interface genutzt wird, um die Ein- bzw. Ausgaben zu übertragen. Mit Interface können zwar auch die „Steckdosen“ gemeint sein, die Sie auf dem Entwicklerboard sehen können, aber bei der Programmierung spielt die physikalische Bauweise keine Rolle. Es folgen einige Anschlussstandards, die Sie wahrscheinlich noch nicht kennen:\\

•	Digitale I/Os
•	UART
•	I2C
•	SPI
•	Und viele andere mehr.
Die folgenden Anschlussstandards werden Sie dagegen wahrscheinlich bereits kennen:
•	USB
•	Ethernet
•	CAN
•	Graphic LCD
•	SD Card\\

Aber nochmal: Sie dürfen bei der Entwicklung von Software für Mikrocontroller nicht erwarten, einen USB-Stick in den entsprechenden Anschluss zu stecken und dann damit arbeiten zu können. Vielmehr müssen Sie einige Vorbereitungen programmieren, damit die Anschlüsse überhaupt in Ihrem Programm genutzt werden können. Die tatsächliche Datenübertragen (die ja die Nutzung eines solchen Anschlusses darstellt) ist damit aber noch nicht erfüllt, sondern muss ebenfalls eigenständig von Ihnen programmiert werden.

\subsubsection{Anbindung von Anschlüssen an den Mikrocontroller}

Sehen wir uns zunächst an, wie ein solcher Anschluss (nehmen wir hier einen SPI-Controller) mit dem Mikroprozessor verbunden ist. Sie wissen ja bereits, dass der Mikroprozessor ausschließlich über ein Interface Daten erhalten kann. Sie wissen ebenfalls, dass es nicht möglich ist, dem Mikroprozessor über dieses Interface Befehle zu erteilen, sondern dass Sie sich hier eines Interrupts bedienen müssen, der vom NVIC weitergeleitet oder verworfen wird.\\

Was Sie aber noch nicht wissen ist, dass Anschlüsse wie ein SPI-Controller mit der sogenannten Peripherie verbunden sind und somit nur dann genutzt werden können, wenn die Register dieser Peripherie entsprechend initialisiert sind.\\

Und wie konnte es anders sein: Das ist ein komplexerer Prozess als die Intialisierung des SP und der IRQs.

\subsubsection{Initialisierung der Peripherie}

In aller Regel sind vier Schritte nötig, um eine Peripherie zu initialisieren:\\

1.	Konfiguration der clock control circuitry:
Wie Sie wissen arbeiten Rechner mit einer sogenannten Taktung, sprich sie führen eine exakte Anzahl an Operationen pro Sekunde aus. Besser gesagt: Nach einem festen Zeitintervall führen Sie genau eine Operation aus. Bislang werden Sie wahrscheinlich keine Konfiguration der Taktung für einzelne Komponenten des Systems durchgeführt haben. Doch auch das müssen Sie jetzt tun.
Der Grund ist recht einfach: Jede Komponente hat eine maximale Taktung, mit der sie arbeiten kann. Aber die Tatsache, dass eine Komponente an Ihr Entwicklerboard angeschlossen werden kann (z.B. über UART) bedeutet natürlich nicht, dass sie genau dieselbe maximale Taktung hat wie Ihr Prozessor. Und da Sie eben jedes Detail selbst programmieren müssen, müssen Sie auch für jede Komponente eine passende Taktung programmieren.
Eine zu hohe Taktung bedeutet zum Glück normalerweise nur, dass eine Komponente nicht auf Datenübertragungen reagiert; im Bereich der maschinennahen Programmierung gibt es dagegen durchaus Fälle, in denen Sie mit entsprechenden Befehlen bzw. Programmen Schäden bewirken können.\\

2.	Programmierung der I/O-Konfiguration
Aktuelle Mikroprozessoren in Desktoprechnern haben mehr als 1000 Kontakte bzw. Beinchen. Und jeder dieser Kontakte entspricht einem Bit, das Sie setzen oder löschen können. Zur Erinnerung: Das bedeutet schlicht, dass Sie das Eingangssignal an- oder ausschalten, indem Sie eine 0 oder eine 1 einprogrammieren. Indem Sie also bei der maschinennahen Programmierung 32-stellige Binärwerte programmieren, setzen und löschen Sie gleichzeitig 32 Eingänge des Prozessors.
Nun werden Sie sich wundern, welchen Sinn es hat, dass z.B. ein 32-Bit-MCU mehr als 32 Kontakte hat; schließlich kann er ja nur 32 Bit pro Rechenschritt verarbeiten. Aber das ist recht simpel: Hier handelt es sich um die Ein- und Ausgänge des Mikrocontrollers, nicht um die Ein- und Ausgänge des Mikroprozessors.
Und als zweiten Schritt nach der Taktung jeder Peripherie, die Sie nutzen wollen, müssen Sie konfigurieren, über welchen I/O-Pin (also über welches Beinchen des Controllers) Daten mit der jeweiligen Peripherie ausgetauscht werden sollen. Dabei ist es möglich, mehrere periphere Komponenten über denselben I/O-Pin kommunizieren zu lassen.\\

3.	Konfiguration der Peripherie
Erst jetzt kommt die eigentliche Konfiguration der Peripherie an die Reihe. Zur Erinnerung: Sie nutzen jede Peripherie, indem Sie auf die memory gemappten Register dieser Peripherie zugreifen. Also konfigurieren Sie die Peripherie, indem Sie die entsprechenden Speicherbereiche deklarieren und initialisieren.\\

4.	Interrupt Konfiguration
Warum Sie nun Interrupts für jede Peripherie konfigurieren müssen, sollte Ihnen nach den vorigen Abschnitten klar sein. Wichtig ist lediglich, dass Sie daran denken, dass die IRQ-Konfiguration der vierte Schritt bei der Konfiguration/Initialisierung jeder Peripherie bzw. jedes Anschlusses an Ihren Mikrocontroller ist.
Kontrolle\\

Sie wissen jetzt, dass Sie für jede angeschlossene Komponente vier Bereiche programmieren müssen, egal, ob diese von Ihnen selbst am Mikrocontroller angeschlossen wurde oder ob sie Teil des Entwicklerboards ist. Erst danach können sie diese Komponente in Ihr Programm einbinden.\\

Damit haben Sie jetzt endlich einen zweiten Bereich kennen gelernt, den Sie (nach der zuvor eingeführten Initialisierung von SP und IRQs) programmieren können. Dumm nur, dass Ihr Programm damit immer noch nichts tut…

\subsection{Eingebettete Systeme und Debugging im laufenden Betrieb}

Dieser Abschnitt ist zwar im Grunde etwas zu früh, aber da wir uns soeben angesehen haben, wie man Anschlüsse in den Programmablauf einbinden kann, kommen wir jetzt dazu, wie man diese Anschlüsse nutzen kann, um im laufenden Betrieb Rückmeldungen z.B. über Fehler zu bekommen.\\

Denn wie beschrieben haben eingebettete Systeme im Regelfall kein Display und sind auch sonst quasi nicht sichtbar. Deshalb müssen Sie im Regelfall Meldungen des Systems über einen der Anschlüsse (z.B. die UART-Schnittstelle) ausgeben lassen, die Sie dann mittels eines Rechners auswerten können.\\

Da die Schnittstellen aber im Regelfall kein C beherrschen müssen Sie hier noch eine Übersetzung zwischenschalten, die die Fehlermeldungen (z.B. in Form von printf()-Ausgaben) in eine Form übersetzen, die über UART übertragen werden kann. Eine solche Übersetzung wird als retargeting bezeichnet.\\

Kontrolle\\

Sie wissen schon… Ein weiterer Abschnitt, den Sie aktuell unter der Kategorie nice to know ablegen können.

\subsection{Mangelnde Portabilität und das CMSIS}

Bislang haben Sie fast ausschließlich etwas darüber gelesen, wie Sie konzeptionell an das Programmieren herangehen müssen; konkrete Programmierzeilen waren Mangelware. Sie wissen bereits, dass Sie bei der Programmierung von Mikroprozessoren in Assembler und in C je nach Mikroprozessor zum Teil andere Befehle nutzen müssen, weil es hier keinen Standard gibt.\\

Leider gilt das nicht nur für unterschiedliche Mikroprozessoren, sondern auch für unterschiedliche IDEs und Compiler. Bei allen Abschnitten dieses Kapitel, in denen es um konkrete Programmierzeilen geht müssen Sie sich also im Klaren sein: Wenn Sie nicht die Ausstattung unseres Labors nutzen (Olimex LPC1114 Entwicklerboard, Keil MDK Evaluationsversion), dann werden Sie den Programmcode teilweise anpassen müssen, weil entweder Ihr Mikrocontroller oder Ihre IDE bzw. Ihr Compiler und Linker  andere Befehle benötigen.\\

Die Entwickler bei ARM haben aber vor Jahren begonnen, eine Lösung für dieses Problem zu entwickeln (wir sprechen also wieder einmal über Software Engineering) und das sogenannte CMSIS (kurz für Cortex Microcontroller Software Standard) entwickelt, das kontinuierlich erweitert wird. Es ist Teil vieler IDEs, kann aber auch kostenlos von der ARM Webpage heruntergeladen werden. In der Version 1.3  bot es den standardisierten Zugriff auf das NVIC, den SCB und den SysTick. Weiterhin enthielt es verschiedene Funktionen, mit denen unterschiedliche C und Assembler Compiler gleich genutzt werden konnten. In der Einführung in die Programmierung mit C erfahren Sie, wie Sie das CMSIS einsetzen können.\\

Kontrolle\\

Da für diesen Kurs keine Kenntnisse des Software Engineering vorausgesetzt werden und Software Engineering auch eigentlich nicht Teil dieses Kurses ist, können Sie die Vorteile des CMSIS und des im nächsten Abschnitt vorgestellten ABI noch nicht richtig einschätzen. Sie sollten sich allerdings die Abkürzungen merken und in den kommenden Wochen prüfen, welche Lösungen diese beiden Standards Ihnen bieten.\\

Langfristig gehört die Kenntnis dieses Bereiches zu den Kernkompetenzen, die Sie bei der Programmierung für einen Cortex-M0 benötigen.

\subsection{Das ABI – Die Klassenbibliothek für die Cortex-Programmierung}

Wenn Sie bereits mit einer höheren Programmiersprache gearbeitet haben , dann kennen Sie Bibliotheken. Das sind Sammlungen von Programmcode, den Sie für Ihre Programme nutzen können ohne sie selbst verfasst zu haben. Einen guten Programmierer zeichnet es aus, zunächst zu prüfen, ob eine Programmierlösung bereits in einer Bibliothek enthalten ist.\\

Bei ARM wird die entsprechende Sammlung von Funktionen aber auch von anderen hilfreichen Komponenten als ABI (Application Binary Interface) bezeichnet. Sie finden die vollständige Dokumentation des ABI unter http://infocenter.arm.com/ im Bereich Documentation -> ARM Software Development Tools.

\section{Wie Sie die Software in den Mikrocontroller bekommen}

Zunächst müssen Sie sich voll und ganz darüber im Klaren sein, dass der Mikroprozessor bzw. die MCU, die Sie programmieren nicht Teil des Computers ist, mit dem Sie das Programm entwickeln. Das mag trivial klingen, die Konsequenzen sind es definitiv nicht!

\subsection{Die Verbindung zwischen Entwicklungsboard und Rechner herstellen}

Der erste Schritt besteht nun darin, dass Sie das Entwicklungsboard irgendwie mit Ihrem Rechner verbinden müssen. Und auch wenn auf dem Entwicklungsboard ein USB-Port vorhanden ist, können Sie (mit wenigen Ausnahmen) das Board damit nicht (!) mit dem Computer verbinden. Na gut, Sie können hier ein USB-Kabel anschließen, aber Sie werden Ihr Programm so nicht auf den Mikrocontroller übertragen können; der USB-Port dient in aller Regel ausschließlich dazu, das Entwicklerboard mit Strom zu versorgen.\\

Um ein Programm auf das Entwicklerboard zu übertragen (Fachbegriff: Flash programmieren), müssen Sie eine Schnittstelle nutzen, die beim Cortex-M0 JTAG heißt. Physikalisch handelt es sich dabei um einen Anschluss mit 14 Pins, der auf dem Entwicklerboard leicht erkennbar ist.\\

Aber Sie werden kein einfaches Kabel finden, mit dem Sie z.B. den USB-Port Ihres Rechners mit der JTAG-Schnittstelle verbinden können. Sie benötigen hierzu ein zusätzliches physikalisches Interface, für das es verschiedenen Bezeichnungen gibt:\\

•	ICE (In-Circuit Emulator)
•	In-Circuit Debugger
•	Debug probe
•	USB-JTAG-Adapter\\

Wenn Sie (wie in unseren Laboren) die Debugging Einheit ULink2 von Keil verwenden wollen, dann verbinden Sie sie zunächst mit dem USB-Anschluss Ihres Rechners und lassen den entsprechenden Treiber vom Betriebssystem Ihres Rechners installieren. Danach verbinden Sie den  entsprechenden Anschluss des ULink2 mit dem JTAG-Port auf dem Entwicklerboard und verbinden abschließend das Entwicklerboard mit einem USB-Port Ihres Rechners, um die Stromversorgung sicher zu stellen.\\

Kontrolle\\

Sie wissen jetzt, was Sie benötigen, um ein Entwicklerboard mit einem Rechner zu verbinden, auf dem Sie Software für den Mikrocontroller entwickeln wollen.\\

Wie Sie den Quellcode in ein Programm umwandeln und dieses zum Mikrocontroller übertragen ist Inhalt der nächsten Abschnitte.

\subsection{Vom Quellcode zum Programm}

Wenn Sie Quellcode mit einer IDE wie dem MDK entwickelt haben, dann können Sie diesen mit dem Debugger auf Fehler kontrollieren und ihn recht komfortabel in ein sogenanntes Program Image umwandeln. Es ist dieses Program Image, das Sie anschließend in den Mikrocontroller übertragen bzw. in dessen Speicher Sie es flashen.\\

Ein solches Program Image beinhaltet die folgenden Abschnitte:\\

•	Die Vektortabelle (So wird der Programmbereich genannt, der die Adressen des SP und des Exception Types enthält)
•	C-Startup Routine
•	Programm Code
•	C Library Code\\

Sie werden jetzt die Schritte kennen lernen, mittels derer aus Ihrem Quellcode ein solches Program Image erzeugt und auf den Mikroprozessor übertragen wird.\\

Bei der Entwicklung ist noch wichtig, dass Sie sich an bestimmte Namenskonventionen halten, denn nur durch diese zeigen Sie dem Computer, was der Quellcode ist: Dateinamen, die auf .c enden enthalten Quellcode in C, Dateien, die auf .cpp enden, enthalten Quellcode in C++ und Dateien, die auf .s enden enthalten Quellcode in Assembler. \\

Für diejenigen unter Ihnen, die nicht viel über Dateiendungen wissen hier noch ein Hinweis: Sie können an einen Dateinamen eine beliebige Endung anhängen. Das führt aber nicht dazu, dass Sie den Inhalt der Datei ändern, sondern sorgt lediglich dafür, dass der Rechner davon ausgeht, dass es sich um eine entsprechende Datei handelt, so wie Sie einen Schatz an der Stelle erwarten, an der sich auf einer handgezeichneten Karte ein großes X befindet.\\

Beispiel: Wenn Sie an den Dateinamen die Endung .pdf anhängen und nun dem Betriebssystem z.B. durch einen Doppelklick angeben, dass es die Datei öffnen soll, dann wird es versuchen, die Datei mit dem Programm zu öffnen, mit dem Sie üblicherweise .pdf-Dateien öffnen. Nun wird dieses Programm allerdings mit dem Dateiinhalt nichts anfangen können, da Sie ja nur den Dateinamen geändert haben.\\

Sie wissen aus dem ersten Kapitel dieses Buches, dass ein Compiler Quellcode in ausführbaren Code umwandelt. Im Falle von C und Assembler passiert ein weiterer Schritt: Hier werden durch den Compiler aus dem Quellcode zunächst die sogenannten Object Files erzeugt. (Dateiendung .o) Diese sind zwar im Grunde bereits ausführbar, allerdings benötigen sie in aller Regel noch zusätzlichen Programmcode, der in Form sogenannter Linker Scripts (Dateiendung .ld) oder Scatter-Loading Files (Dateiendung .scat) eingebunden wird. Zusammen mit diesen Dateien wird nun aus den object files das Program Image erzeugt. Und diese Aufgabe übernimmt nicht mehr der Compiler, sondern der sogenannte Linker.\\

Wie Sie wissen kann ein Cortex-M0 bis zu 4 GB an Speicher verwalten, aber wie viel Speicher tatsächlich angebunden ist und wie dieser aufgeteilt ist, das muss im scatter-loading file angegeben werden. Beim Arbeiten mit dem MDK brauchen Sie sich darum nicht zu kümmern, wenn Sie im MDK eingetragen haben, welchen Mikrocontroller Sie verwenden. Denn dann nimmt das MDK die entsprechenden Einstellungen selbst vor und führt das Verlinken gemeinsam mit dem Kompilieren durch.

\subsection{Program Image flashen}

Nachdem Sie ein Program Image erzeugt haben, müssen Sie es noch in den Speicher des Mikrocontrollers übertragen. Da es sich bei diesem Speicher um Flash-Speicher handelt, redet man hier auch vom flashen. Hier gilt wieder: Ein einfaches Kopieren, wie Sie es z.B. bei USB-Sticks sonst gewöhnt sind, funktioniert nicht; Sie benötigen ein spezielles Programm, den Program Flasher. Wenn Sie das MDK nutzen (oder auch einige andere IDEs), brauchen Sie jedoch kein eigenständiges Programm zu installieren, da der Flasher dort bereits integriert ist.\\

Kontrolle\\

Sie wissen jetzt, wie Sie den Rechner und das Entwicklungsboard miteinander verbinden können und warum Sie später ein knappes Dutzend Dateien in dem Verzeichnis vorfinden werden, in dem Sie Ihren Quellcode sowie das Program Image speichern.\\

Sie kennen darüber hinaus die Dateiendungen, mit denen Sie es bei der Programmierung eines Mikrocontrollers zu tun haben und was diese Dateien beinhalten.\\

Außerdem wissen Sie, dass Sie das Program Image nicht einfach kopieren können, sondern ein Flash Programm benötigen, um es in den Speicher des Mikrocontrollers zu übertragen.\\

Zusammen mit den übrigen Inhalten dieses Kapitels kennen Sie damit alle Grundlagen, die Sie benötigen, um ein umfangreiches Programm für einen Cortex-M0 zu entwickeln. Damit können wir uns jetzt endlich der eigentlichen Programmierung zuwenden.

\section{Zugriff auf die Peripherie bei einem Cortex-M0}

Wie Sie bereits wissen erfolgt der Zugriff auf die Peripherie (also auch auf alle von Ihnen an den Cortex-M0 angeschlossenen Geräte), indem Sie die Memory-gemappten Register im Speicher des Cortex-M0 überschreiben.\\

Aber natürlich müssen Sie hierfür in Ihrem C-Programm noch einige Codezeilen einfügen, um diese Prozess zu vereinheitlichen. Im Buch von Yiu finden Sie dazu drei Codebeispiele für den Fall, dass Sie eine UART-Schnittstelle nutzen wollen:\\

Auf S. 62 finden Sie den Code, um ein Peripherieregister als einen Pointer zu definieren.\\

Auf S. 63 finden Sie den Code, um eine Datenstruktur für einen Satz von Peripherien zu definieren, in denen jede Peripherie als ein Pointer definiert ist.\\

Auf S. 64 finden Sie dann den Code, um gleich mehreren Instanzen Ihres Programms unabhängig voneinander den Zugriff auf eine solche Datenstruktur zu ermöglichen.

\section{Der CMSIS – Cortex Microcontroller Software Interface Standard}

Sie kennen bislang die einfachen Grundlagen von C, mittels derer Sie einfache Programme entwickeln können. Wie Sie bereits bei der Erklärung zum ABI erfahren haben, benötigen Sie darüber hinaus noch weitergehende Programmiermöglichkeiten, um aus einem C-Programm heraus einen Cortex-M0 ausnutzen zu können.\\

Das CMSIS ist ein Framework, das es Ihnen ermöglicht, Cortex-Programme zu entwickeln, die Sie in unterschiedlichen IDEs weiterentwickeln können. Es wurde von ARM entwickelt und wird u.a. von den folgenden IDEs unterstützt: Keil MDK, ARM DS-5, IAR embedded Workbench, TASKING Compiler und verschiedenen GNU-GCCs.\\

Leider ist die Darstellung in Yius Buch veraltet: Er behandelt hier den CMSIS in der Version 1.3, während mittlerweile die Version 4.3 veröffentlicht wurde. Eine aktuelle Darstellung wird hier zu einem späteren Zeitpunkt erfolgen.



%Der Abschnitt zu den Grundlagen der imperativen Programmierung muss noch an die Latex-Syntax angepasst werden.
% Nachfolgender Abschnitt muss noch an Latex angepasst werden.
%\chapter{Grundlagen der imperativen Programmierung}

Die Programmiermethoden, die Sie bis jetzt kennen gelernt haben entsprechend weitgehend dem, was Informatikstudierende bereits vor dem Studium beherrschen sollten. Denn genau wie ein Studium der Anglistik nicht dazu da ist, um Englisch zu lernen, sondern um sich intensiv mit den Feinheiten der englischen Sprache zu beschäftigen, geht es im Studium der (Medien-)Informatik nicht darum, das Programmieren zu lernen. Vielmehr geht es hier darum, zu lernen, welche fortgeschrittenen Konzepte es gibt, um Rechnersysteme möglichst elegant und effizient dazu zu bringen, komplexe Probleme zu lösen. In der Medieninformatik betrachten wir dabei vorrangig Probleme, die bei der Verarbeitung von audio-visuellen Daten auftreten.\\

\textbf{Datenverarbeitung}\index{Datenverarbeitung} schließt die folgenden Dinge ein: 

\begin{itemize}
	\item \textbf{Erzeugung neuer Daten aus vorhandenen Daten}, ist die offensichtlichste Beschäftigung von (Medien-)InformatikerInnen.
	\item \textbf{Speicherung}, also die Datenübertragung über einen Zeitraum.
	\item \textbf{Datenübertragung}, also die Datenübertragung von einem Ort zum anderen führt zu einem Zeitaufwand, der für Software relevant ist.\\
	Beachten Sie dabei, dass bei jeder Operation, Funktion usw. Datenmengen vom Speicher in den Prozessor, innerhalb des Prozessors und vom Prozessor zum Speicher übertragen werden. Denn resultierenden Zeitaufwand haben wir bislang bei der Programmierung vollständig ignoriert.\\
	
	\textbf{Wichtig}:\\
	
	Auch die Anzeige eines einzelnen Zeichens auf dem Bildschirm setzt voraus, dass wir als ProgrammiererInnen dieses Zeichen in einer vereinbarten Art und Weise (also nach einer für den Einzelfall festgelegten Codierung) vom Prozessor an einen Grafikprozessor übertragen. Wenn Sie also bislang dachten, dass eine Ausgabe quasi automatisch auf dem Bildschirm angezeigt wird, dann haben Sie wie die meisten ProgrammiererInnen falsch gedacht. Wir werden uns in C bei den sogenannten Header-Dateien und in Java bei der sogenannten Klassenbibliothek einmal ansehen, wie die Laufzeitumgebung bzw. \glqq{}die Programmiersprache\grqq{} uns diese Arbeit abnimmt.\\

	\textbf{Kontrolle}\\

	\item Sie finden diese drei Punkte langweilig?\\
	
	Dann finden Sie die Medieninformatik langweilig und sollten sich umgehend einen anderen Studienplatz suchen, denn im Kern läuft alles was Sie als MedieninformatikerIn tun werden auf die Beschäftigung mit diesen drei Aspekten hinaus.
\end{itemize}

Denn die Methodik in der Medieninformatik ist weitgehend die gleiche wie in der Informatik. Und beim Programmieren gibt es gar keine Unterschiede. Da Sie durch den ersten Teil des Buches und die Übung zu Hause jetzt ein wenig Verständnis fürs Programmieren aufgebaut haben, könnten Sie auf die Idee kommen, dass Sie eigentlich nur mehr Übung und Kenntnisse einzelner Programmiersprachen brauchen, um ein umfassendes Verständnis von Programmierung zu erlangen. Und genau das ist falsch. Alles, was Sie damit erreichen ist die Fähigkeit, ein bestimmte Gruppe von Programmiersprachen zu beherrschen. Damit sind Sie aber bei vielen Problemen nicht im Stande, eine Programmiersprache auszuwählen, mit der Sie das Problem effizient lösen können. Streckenweise können Sie es damit überhaupt nicht lösen.\\

Damit Sie ein umfassendes Verständnis von Programmierung erhalten, müssen wir zunächst an den Anfang zurückkehren und einige Begriffe definieren, mit denen wir dann die Eigenschaften und Möglichkeiten verschiedenster zustandsbasierter Programmiersprachen beschreiben können. Denn nur so können Sie ohne allzu große Anstrengungen von einer imperativ streng typisierten Sprache wie C zu einer logischen Programmiersprache wie PROLOG und dann zu einer dynamisch typisierten funktionalen Programmiersprache wie JavaScript wechseln.\\

Das bringt uns gleich zu einer essentiellen Unterscheidung von Programmiersprachen:\\

\begin{itemize}
	\item \textbf{Zustandsbasierte Programmierung}\index{zustandsbasiert}\index{Programmierung!zustandsbasiert} ist die Programmierung von Computern, die wie Windows-, MacOS- und Linux-Rechner zu jedem Zeitpunkt einen festen Zustand kennen. In der Medieninformatik kommen wir meist nur mit dieser Art von Rechnern in Kontakt, allerdings liegt das vorrangig daran, dass die Entwicklung nicht-zustandsbasierte Systeme ein fundiertes Verständnis der Elektro- und Nachrichtentechnik benötigen.
	\item \textbf{Nicht-zustandsbasierte Programmierung}\index{nicht-zustandsbasiert}\index{Programmierung!nicht-zustandsbasiert} ist die Programmierung von Computern, die z.B. als Steuer- und Regelsysteme bezeichnet werden. Hier erfolgt die Programmierung z.B. in Form von Gleichungen höheren Grades, die kontinuierlich ausgewertet werden. Diese Systeme kommen z.B. in Autos zum Einsatz, wo sie dann steuern, ob der Airbag ausgelöst wird. Ein anderer Einsatzbereich wäre die Steuerung des Kühlwasserkreislaufs in einem Kraftwerk.
\end{itemize}

Es ist aber wichtig an dieser Stelle zu betonen: Als angehende (Medien-) InformatikerInnen müssen Sie mehr beherrschen als nur die Programmierung in C-artigen Sprachen. Als reine/r ProgrammiererIn z.B. für Computerspiele ist es in aller Regel überflüssig. Auch für MT-Studierende, die z.B. die Programmierung in C++ für die Entwicklung von Musiksoftware erlernen wollen führt der Inhalt dieses Teils des Buches weit über das nötige Maß hinaus. Für die ist dagegen der dritte Teil dieses Buches sehr interessant: Darin geht es gerade um die Aspekte der technischen Informatik, die wir bei der Programmierung in Assemblersprachen sowie C und C++ beherrschen müssen. Dagegen ist die Entwicklung von Software für kontinuierliche (also nicht-zustandsbasierte) Systeme nicht Teil dieses Buches.

\section{Virtuelle Objekte}

Alle höheren Programmiersprachen (das sind alle Sprachen, die nicht maschinennah sind) verwenden virtuelle Objekte. Wichtig: Damit ist etwas anderes gemeint als mit den Objekten im Sinne der Objektorientierung. Um eindeutig zu differenzieren ist hier deshalb durchgehend die Rede von \glqq{}virtuellen Objekten\grqq{}. Eine Art virtueller Objekten sind die sogenannten \textbf{Variablen}\index{Variable}. Virtuellen Objekte haben bestimmte Eingenschaften, die selbst in nahe verwandten Programmiersprachen leider oft unterschiedlich bezeichnet werden. Auch um Ihnen dabei zu helfen, diese Klippe zu umschiffen, führe ich die folgende einheitliche Nomenclatur ein.\\

Es ist wichtig, dass Sie verstehen, dass ein virtuelles Objekte eine beliebige Kombination der folgenden Eigenschaften haben kann. Variablen beispielsweise sind eine Art von virtuellen Objekten, aber eben nur eine Art, die besonders bekannt ist. Andere Arten von Programmiersprachen haben andere Arten von virtuellen Objekten, die sich zum Teil deutlich von Variablen unterscheiden. So lange Sie beim Programmieren an Variablen denken und nicht verstehen, dass das nur ein Spezialfall ist werden Sie eine Vielzahl an Programmiersprachen nicht erlernen können.

\subsection{Bezeichner}

Virtuelle Objekte können einen Bezeichner haben. Das ist so etwas wie ein Name. Hier verwende ich aber bewusst den Begriff des Bezeichners, um damit von Namen und Bezeichnungen zu unterscheiden, wie wir sie im alltäglichen Sprachgebrauch nutzen.\\

Erinnern Sie sich bitte wieder an die anonymen Variablen: Das waren Variablen, die als Ergebnis einer Operation oder Funktion entstanden aber noch keinem Bezeichner zugeordnet wurden.

\subsection{Pointer}

Virtuelle Objekte können sich auf eine Speicheradresse des Computers beziehen. Diese Referenz auf eine Speicheradresse wird meist als Pointer bezeichnet.\\

In Sprachen wie Java und PHP könen wir auf den Pointer nicht zugreifen. Wenn Sie also das Programmieren mit Java und/oder PHP erlernen, dann können Sie Pointer nicht programmieren. In Hochsprachen mit maschinennahen Teilen wie C und C++ ist die Beherrschung von Pointern eine essentielle Fähigkeit, die ProgrammiererInnen beherrschen müssen.

\subsection{Wert}

Virtuelle Objekte, die sich auf eine Speicheradresse beziehen können einen Wert haben, der sich aus dem Binärwert ergibt, der unter dem Pointer gespeichert ist. Das ist aber nur ein Spezialfall, nicht die Regel. \\

In der logischen Programmiersprache PROLOG definieren wir beispielsweise virtuelle Objekte nur anhand eines Bezeichners, der gewissermaßen immer den Wert wahr hat. Da es in dieser Sprache den Wert falsch aber gar nicht gibt, haben virtuelle Objekte dort im Grunde niemals einen Wert.

\subsubsection{Änderbarkeit eines Wertes}

Hierbei handelt es sich nicht um eine zentrale Eigenschaft, die ein virtuelles Objekt haben kann, aber da aus Komfortgründen viele Programmiersprache die Möglichkeit bieten, einen Wert als konstant festzulegen, habe ich diesen Punkt in die Aufstellung aufgenommen.\\

Tatsächlich ist eine Möglichkeit für Angriffe auf IT-Systeme die Tatsache, dass der Wert eines Objekts niemals wirklich statisch sein kann: Wenn ein virtuelles Objekt einen Wert hat, dann ist das (s.o.) ein Binärwert, der in einem Speicherbereich abgelegt ist. Dieser Speicherbereich selbst ist aber immer änderbar. Also gibt es Möglichkeiten, diesen Wert durch Manipulation des Rechners oder den Missbrauch der Programmiersprache zu ändern. Die Frage ist nur, wie leicht ein solcher Angriff durchführbar ist.

\subsection{Interpretation des Wertes}

Wenn ein virtuelles Objekt einen Pointer hat, dann gibt es eine Interpretation des dort gespeicherten Wertes.\\

Das haben Sie in PHP als Datentyp einer Variablen kennen gelernt.

\subsection{Operation(en)}

Alles, was ohne weitere Programmierung mit einem Wert getan werden kann wird als Operation bezeichnet.\\

Wenn Sie das missverstehen, werden Sie vielleicht auf die Idee kommen, dass Operationen und Funktionen das gleiche sind, aber das ist falsch: Operationen und Funktionen sind etwas unterschiedliches.\\

Wenn Sie sich jetzt fragen, was denn der Unterschied zwischen einer \textbf{Operation}\index{Operation} und einer \textbf{Funktion}\index{Funktion} ist, folgt hier eine kurze Erläuterung:

\begin{itemize}
	\item Eine Operation ist etwas, das Sie mit einem Wert tun können ohne das dazu irgend eine Erweiterung im Programm erstellt werden muss. Wenn Sie beispielsweise ein virtuelles Objekt haben, das als Wert eine ganze Zahl repräsentiert, dann können Sie im Regelfall die Grundrechenarten darauf ausführen. Dagegen ist das Ziehen der Wurzel etwas, bei dem mehrfach verschiedene Operationen auf einen Wert angewendet werden.
	\item Nehmen Sie als Beispiel das Heron-Verfahren:\\
	\( x_{n+1} = \frac{1}{2} ( x_n + \frac{x_n}{a} ) \), \(a \in \mathbb{N} \).\\
	Bei jeder Iteration (wiederholte Anwendung des Verfahrens mit dem Ergebnis der vorigen Berechnung) wenden wir die Operationen Multiplikation, Division und Addition an. Die beschriebene Kombination dieser Operationen ist eine Funktion.
\end{itemize}

Bitte beachten Sie, dass diese Definition nicht allgemeingültig ist; es gibt noch andere Möglichkeiten, eine Funktion zu definieren. Für eine vollständige Aufstellung wenden Sie sich bitte an den Mathematik-Dozenten Ihres Vertrauens. Später werden Sie noch mehr über Funktionen als Teil von Computerprogrammen erfahren, diese Beschreibung ist also selbst im Bereich der Programmierung nicht vollständig.\\

\textbf{Kontrolle:}\\

Ist Ihnen klar, warum die Änderbarkeit eines Wertes keine Operation ist, die Änderung des Wertes eines Objekts dagegen schon?

\subsection{Relation}

Zwischen zwei virtuellen Objekten können feste Zusammenhänge existieren, für die es unwichtig ist, ob das virtuelle Objekt einen Wert hat. Diese Zusammenhänge werden als Relationen bezeichnet. \\

\textbf{Wichtig}: Es gibt relationale Operationen, bei denen der Wert zweier virtueller Objekte verglichen wird. Diese werden Sie als boolesche Operationen kennen lernen. Der hier eingeführte Begriff der Relation bezieht sich dagegen auf eine Relation die zwischen zwei virtuellen Objekten selbst existiert.\\

Auch hier gilt wieder, dass diese Definition des Begriffs einer Relation nicht vollständig ist; in der Mathematik werden Sie z.B. Relationen als eine mögliche Definition für Funktionen kennen lernen.

\subsection{Lebensdauer oder Gültigkeitsbereich}

Virtuelle Objekte werden erzeugt. Sie sind jeweils für eine bestimmte Dauer oder innerhalb eines festen Bereichs eines Computerprogramms gültig. Hierfür wird auch der Begriff des \textbf{Scope}\index{Scope} genutzt.\\

Bei den Programmiersprachen, mit denen wir uns beschäftigen brauchen Sie nicht zu verstehen, wie dieser Mechanismus funktioniert, weil Sie hier nicht direkt auf Speicherzellen zugreifen. Der Mechanismus, der ein virtuelles Objekt \glqq{}löscht\grqq{}, der also das Ende der Lebensdauer sicher stellt, wird als \textbf{garbage collector}\index{garbage collector} bezeichnet. Sobald Sie dagegen maschinennah programmieren, müssen Sie sich damit auseinandersetzen. Als Ausblick sei schon einmal gesagt, dass Sie Speicher nicht löschen, sondern nur überschreiben können. Sie können natürlich mit einem Schweißbrenner die Speicherbausteine zerstören und damit den Inhalt wirklich löschen, aber im Alltag erscheint das doch nicht recht praktikabel. Tatsächlich basieren auf dieser Tatsache auch einige Angriffsmethoden im Bereich der IT-Sicherheit. So lässt sich beispielsweise Kältespray einsetzen, um Speicher zu fixieren, selbst wenn der Strom anschließend abgeschaltet wird. Das ist eine Möglichkeit um unter bestimmten Bedingungen ein Passwort aus einem heruntergefahrenen Rechner mit verschlüsselter Festplatte auszulesen.

\subsection{Zugriffsrecht}

Virtuelle Objekte können verwendet werden. Ob sie von jedem anderen virtuellen Objekt bzw. von jedem beliebigen Teil eines Programms verwendet werden können wird über die Zugriffsrechte festgelegt.\\

Neben dem Zugriffsrecht auf virtuelle Objekte gibt es noch Zugriffsrechte, die durch die Konfiguration des Betriebssystems festgelegt werden. Aber das ist ein Thema, mit dem Sie sich in den Bereichen IT-Sicherheit und Betriebssysteme auseinander setzen.

\subsection{Zusammenfassung}

Obwohl wir jetzt nur acht Begriffe zur Verfügung haben können wir damit jede denkbare Programmiersprache beschreiben. Wir werden dabei zwar ggf. neue Begriffe einführen, um die Kombination aus diesen Begriffen mit einem einfachen Begriff zu bezeichnen, aber im Kern müssen Sie nur diese acht Begriffe vollständig verstehen, um zu verstehen, was Programmiersprachen unterscheidet:

\begin{itemize}
	\item Bezeichner
	\item Pointer
	\item Wert
	\item Interpretation des Wertes
	\item Operationen
	\item Lebensdauer
	\item Zugriffsrecht
\end{itemize}

\subsection{Literale}

Literale sind keine Eigenschaft von virtuellen Obejekten, aber der Begriff wird Ihnen in vielen Programmiersprachen begegnen. Ein Literal ist vereinfacht ausgedrückt ein Zeichen, das in einer Programmiersprache benutzt werden kann. Das bedeutet, dass es häufig Zahlen, Buchstaben und andere Schriftzeichen sind, die Sie aus gesprochnen Sprachen kennen.\\

\textbf{Wichtig}:

\begin{itemize}
	\item Es können aber auch Zeichen wie \# sein, die Sie eher nicht als Schriftzeichen verstehen würden.
	\item Außerdem ist ein Literal, das Sie als Zahl kennen (z.B. die Ziffer 2) in einer Programmiersprache nicht unbedingt eine Zahl. Es kann dort auch durchaus so verwendet werden wie ein Buchstabe in einer gesprochenen Sprache.
\end{itemize}

Und weil es diese beiden Unterschiede zu nicht-Programmiersprachen gibt, wird bei Programmiersprachen des öfteren nicht von Schriftzeichen, Buchstaben, Zahlen oder ähnlichem gesprochen, sondern von Literalen. Wenn Sie also eine Programmiersprache erlernen, dann müssen Sie zunächst lernen, welche Literale in dieser Sprache wofür verwendet werden.

\section{Virtuelle Objekte in Programmiersprachen}

In imperativen Programmiersprachen gibt es verschiedene Arten virtueller Objekte. Die einfachsten virtuellen Objekte werden dabei üblicherweise als Konstanten und Variablen bezeichnet. Konstanten sind dabei Sonderformen von Variablen, die sich bei der Programmierung dadurch von Variablen unterscheiden, dass Ihr Wert nur einmal festgelegt, aber für den Rest der Lebensdauer einer Variablen nicht mehr geändert werden kann.\\

\textbf{Funktionale Programmiersprachen}\index{Programmierung!funktional} zeichnen sich dadurch aus, \textbf{dass ALLE Variablen Konstanten sind}. \textbf{Java}\index{Programmiersprache!Java} ist seit der Version 8 ein Sonderfall: Hier können Sie weitgehend imperativ programmieren, also mit Variablen und Konstanten. Aber es gibt auch funktionale Anteile seit dieser Java-Version. In diesen sind Variablen dementsprechend immer Konstanten. \\

Die theoretische Grundlage zur funktionalen Programmierung nennt sich \textbf{Lambda-Kalkül}\index{Lambda-Kalkül}. Wenn Sie also in Programmierung über den Begriff Lambda (z.B. als Symbol für die Wellenlänge) stolpern und es nicht um physikalische Grundlagen geht, dann geht es meist um funktionale Programmierung. Das Lamda-Kalkül ist ein Ansatz, der Effizienz zum zentralen Dreh- und Angelpunkt der Programmierung erklärt und es ist aus der Perspektive der imperativen Programmierung nicht zu verstehen. Aufgrund der strikten Fokussierung auf effizienten Code ist funktionale Programmierung meist nicht leicht lesbar. Dafür ist es ein Programmierstil, der mit extrem wenig Code komplexe Probleme löst. Der Einstieg in funktionale Programmierung fällt den meisten Programmierern also schwer. Häufig behaupten sie deshalb, es handle sich um einen ganz unsinnigen Ansatz zu programmieren. Wenn Sie (Medien-)InformatikerIn werden wollen, aber dennoch diese Aussage von sich geben, dann zeigt das, dass Sie in diesem Bereich inkompetent sind. Das bedeutet nicht, dass der Einstieg unbedingt leicht wäre, er erfordert wie der Einstieg in jede(!) Art der Programmierung Zeit und Anstrengung.

\subsection{Primitive virtuelle Objekte}

In verschiedenen Programmiersprachen werden Sie auf den Begriff des \textbf{Primitive}\index{Primitive} bzw. des primitiven Datentyps stoßen. Das sind virtuelle Objekte, die nicht mehr weiter aufteilbar sind. Wenn wir beispielsweise eine Variable haben, unter der ein Buchstabe oder eine Zahl gespeichert ist, dann ist das ein Primitive. Wenn dagegen eine Variable auf eine Funktion verweist oder ein Objekt wie in Java oder auf beliebige andere zusammengesetzte virtuelle Objekte (also auch Datenstrukturen), dann handelt es sich nicht um einen Primitive.\\

Leider wird der Begriff des Primitives meist nicht genutzt, obwohl er für eine differenzierte Betrachtung von Programmiersprachen sehr praktisch ist. Deshalb haben Einsteiger häufig ein Problem damit, dass eine Variable in typisierten Sprachen mal auf eine einzelne Zahl, mal auf einen Text und mal auf eine Datenstruktur verweisen kann. Um hier eine klare Unterscheidung zu treffen wird in diesem Buch der Begriff des \textbf{Datentyp}s\index{Datentyp} ausschließlich für primitive Datentypen genutzt. Ebenso wird der Begriff \textbf{Variable}\index{Variable} nur für virtuelle Objekte genutzt, die ein Primitive sind.

\subsection{Deklaration einer Variablen}

Wenn wir durch eine entsprechende Programmzeile eine Variable erzeugen, wird das als \textbf{Deklarierung}\index{Deklarierung} bezeichnet. Bei der Deklarierung wird bei jeder imperativen Programmiersprache ein Bezeichner festgelegt.\\

In \textbf{statisch typisierten}\index{statisch}\index{Typisierung!statisch} Sprachen wie C, C++ und Java wird außerdem bei der Deklarierung durch Programmierer der sogenannte Datentyp festgelegt. \\

Der \textbf{Datentyp}\index{Datentyp} ist nichts anderes als die Interpretation des Wertes eines virtuellen Objekts, womit auch die für das virtuelle Objekt gültigen Operationen festgelegt werden. Zusätzlich wird ein Pointer angelegt. Außer bei der maschinennahen Programmierung passiert das durch die Programmiersprache. (Anm.: Wenn Sie es mit Programmiersprachen zu tun haben, die die manuelle und automatische Festlegung des Pointers bei der Deklaration einer Variablen erlauben, müssen Sie sicherstellen, dass Sie nicht versehentlich auf einen Speicherbereich referenzieren, der bereits durch ein anderes virtuelles Objekt genutzt wird. In diesem Teil des Buches werden wir damit aber nichts zu tun haben.)\\

Lassen Sie sich hier bitte nicht davon verwirren, dass die Interpretation des Wertes festgelegt wird, bevor ein Wert festgelegt wird. Der Grund ist recht simpel: Es wird hier nicht nur festgelegt, wie der Wert interpretiert werden soll, sondern auch wie viel Speicher für die Speicherung des Wertes verwendet werden soll.\\

Wenn Sie schon mit Java programmiert haben, dann können Sie jetzt vollständig nachvollziehen, warum es mehrere Datentypen für ganzzahlige Werte gibt.\\

Wenn eine Programmiersprache bei statisch typisierte Variablen eine Regelung für das Zugriffsrecht hat, dann muss dieses ebenfalls bei der Deklaration festgelegt werden. Damit haben Sie beispielsweise bei der klassenbasierten Objektorientierung wie in Java zu tun.\\

\begin{enumerate}
	\item Bei statisch typisierten, kompilierten Sprachen mit Regelung des Zugriffs wie Java sieht die Deklaration einer Variablen so aus:
	\begin{verbatim}
	zugriffsrecht datentyp bezeichner;
	\end{verbatim}
	Beachten Sie aber bitte, dass das Semikolon ein Zeichen ist, das in C, C++ und Java das Ende einer Programmzeile anzeigt. In anderen Sprachen werden andere Zeichen oder schlicht der Zeilenumbruch dafür genutzt.
	\item Bei Java gibt es eine Besonderheit, wenn Sie eine Konstanate deklararieren wollen. In dem Fall müssen Sie zusätzlich das Schlüsselwort \verb|final| nutzen:\\
	\\
	\textbf{Wichtig:} Es gibt in Java auch das Schlüsselwort \verb|static|, aber das bedeutet gerade nicht statisch bzw. konstant. Wir werden erst bei der Einführung in die objektorientierte Programmierung mit Java darauf zurück kommen.
	\begin{verbatim}
	zugriffsrecht final datentyp bezeichner;
	\end{verbatim}
	Beachten Sie aber bitte, dass das Semikolon ein Zeichen ist, das in C, C++ und Java das Ende einer Programmzeile anzeigt. In anderen Sprachen werden andere Zeichen oder schlicht der Zeilenumbruch dafür genutzt.
	\item Bei statisch komplierten Sprachen ohne Zugriffsregelung wie C sieht die Deklaration einer Variablen so aus:
	\begin{verbatim}
	datentyp bezeichner;
	\end{verbatim}
	\item Bei dynamisch typisierten Sprachen gibt es in aller Regel keine Deklaration, weil die Programmiersprache im Moment der Wertzuweisung zu einem neuen Bezeichner automatisch einen Pointer und einen Datentyp in Abhängigkeit vom Wert des virtuellen Objekts festlegt.
	\item Bei Sprachen, die virtuelle Obejkte ohne einen Wert nutzen gibt es ebenfalls keine Deklaration, sondern wir verwenden einen Bezeichner, ohne ihn vorher in irgend einer Weise eingeführt zu haben.
\end{enumerate}

Zur Erinnerung:\\

\textbf{Dynamisch typisierte}\index{dynamisch} Sprachen unterscheiden sich von \textbf{statisch typisierten}\index{statisch} Sprachen dadurch, dass bei Ihnen der Datentyp von der Sprache selbst verwaltet wird. Streng genommen bedeutet dynamisch typisiert lediglich, dass sich der Datentyp einer Variable im Laufe des Programmablaufs ändern kann. Deshalb wird häufig bei statisch typisierten Sprachen von einer Typsicherheit gesprochen, aber der Begriff ist unsinnig. Denn dynamisch typisierte Sprachen ändern den Datentyp einer Variablen nach festen Regeln. Und ob nun Sie als ProgrammiererIn einen Datentyp festlegen oder ob das Programm das nach festen Regeln tut: In beiden Fällen hat eine Variable zu jedem Zeitpunkt einen bestimmten Datentyp. Der Unterschied ist der: Bei statisch typisierten Sprachen können Sie direkt im Programm nachlesen, welchen Datentyp ein virtuelles Objekt hat. Wobei das auch nur dann gilt, wenn der Entwickler nicht zu chaotisch programmiert hat. Bei dynamisch typisierten Sprachen müssen Sie dagegen die Regeln lernen, nach denen die Sprache den Datentyp ändert bzw. festlegt, um zu erkennen, wann eine Variable welchen Datentyp hat.\\

\subsection{Initialisierung einer Variablen}

Der zweite Schritt bei der Arbeit mit einer Variablen ist die \textbf{Intialisierung}\index{Initialisierung}. Hier wird der Wert der Variablen festgelegt. In vielen statisch typisierten Sprachen kann eine Variable in einer Zeile deklariert und initialisiert werden. In einigen Sprachen müssen Sie dagegen alle Variablen zuerst explizit deklarieren, bevor Sie sie verwenden dürfen.\\

Die Initialisierung ist die erste Operation, die für Variablen (also virtuelle Objekte einer imperativen Programmiersprache) definiert ist. Im Gegensatz zu den meisten anderen Operationen ist diese also für alle Variablen unabhängig vom Datentyp gültig.

\begin{enumerate}
	\item Bei statisch typisierten, kompilierten Sprachen wie C, C++ oder Java sieht die Deklaration einer Variablen so aus:
	\begin{verbatim}
	bezeichner = wert;
	\end{verbatim}
	\item Bei dynamisch typisierten Sprachen wie Ruby gibt es keine Deklaration und wir initialisieren direkt:
	\begin{verbatim}
	bezeichner = wert;
	\end{verbatim}
	\item PHP wählt hier einen Sonderweg, weil Variablen hier durch ein \$-Zeichen ausgezeichnet werden. Aber ansonsten ist das System konsistent:
	\begin{verbatim}
	$bezeichner = wert;
	\end{verbatim}
	\item In PHP gibt es aber auch Variablen ohne führendes \$-Zeichen. Das sind Konstanten, die mit dem Schlüsselwort \verb|const| programmiert werden müssen:
	\begin{verbatim}
	const bezeichner = wert;
	\end{verbatim}
	\item Auch wenn es trivial erscheint, sei hier nochmal betont: Sprachen, die virtuelle Objekte ohne einen Wert kennen haben natürlich keine Initialisierung.
\end{enumerate}

\subsection{Kombination von Deklaration und Initialisierung}

In vielen etwas weniger streng definierten Sprachen kann die Deklaration und die Intialisierung virtueller Objekte in einer Zeile kombiniert werden. Dementsprechend ergeben sich u.a. die folgenden Varianten:

\begin{enumerate}
	\item \verb|zugriffsrecht final datentyp bezeichner = wert;|
	\item \verb|zugriffsrecht datentyp bezeichner = wert;|
	\item \verb|datentyp bezeichner = wert;|
\end{enumerate}

In Sprachen wie C und Pascal kann der \glqq{}Datentyp\grqq{} einer Variablen ein Pointer sein. Das bedeutet, dass wir dort nicht etwa eine Zahl, einen Text oder ein anderes konkretes Objekt abspeichern, sondern einen Verweis auf den Wert einer anderen Variablen. Das ermöglicht einen sehr effizienten Programmierstil für bestimmte Arten von Problemen, die tatsächlich recht häufig auftreten. Ein vollständiges Verständnis dafür können Sie aber im Grunde erst dann entwickeln, wenn Sie die Grundlagen der maschinennahen Programmierung beherrschen.\\

In objektorientierten Sprachen wie Java und JavaScript können Sie wichtige Anwendungsbereiche von Pointern durch alternative Methoden umsetzen. Wir kommen darauf zu sprechen, wenn es um die Programmierung von eigenen Datenstrukturen in Java geht.\\

\subsection{Zuordnung eines Wertes}

Recht häufig werden Sie von der \textbf{Zuordnung eines Wertes zu einer Variablen}\index{Variable!Zuordnung eines Wertes} hören. Das ist nichts anderes als die Änderung des Wertes einer Variablen. Es bedeutet, dass der Wert einer Variablen geändert wird.\\

Um eine Wertzuordnung zu \textbf{programmieren} müssen Sie genau das selbe wie bei der Initialisierung einer Variablen machen.\\

Auch wenn es Ihnen vielleicht schon zu den Ohren herauskommt: Wenn eine Sprache keine Wert von virtuellen Objekten hat, dann können Sie einem virtuellen Objekt in dieser Sprache natürlich auch keinen Wert zuordnen.

\subsection{Wert eines virtuellen Objekts}

Ein häufiges Missverständnis von Einsteigern besteht darin, das Wort \verb|Wert| mit einer Zahl gleichzusetzen. Das ist zwar insofern richtig als die meisten Computer mit Binärwerten arbeiten, aber es ist in sofern falsch als der Wert eines virtuellen Objekts (also auch der Wert einer Variablen) auch ein Buchstabe sein kann.\\

Sehr häufig programmieren wir keinen konkreten Wert (also eine Zahl oder einen Buchstaben), sondern eine Operation oder Funktion als Wert einer Variablen. Es gibt jetzt zwei Möglichkeiten:

\begin{itemize}
	\item In den meisten Fällen bedeutet das, dass die Operation oder Funktion zunächst ausgewertet wird, und dass das \glqq{}Ergebnis\grqq{} davon der Wert ist, der der Variablen zugeordnet wird.
	\item Es gibt aber noch eine zweite Möglichkeit, auf die wir an dieser Stelle noch nicht eingehen: Bestimmte Programmiersprachen erlauben es, dass eine Funktion unter einem Bezeichner gespeichert werden. Bei diesen Fällen wird eine solche Funktion als \textbf{first-class object}\index{first-class object} bezeichnet. Der Wert der Variablen ist dann nicht das Ergebnis der Funktion, sondern der Wert der Variablen ist die Funktion selbst. Momentan können Sie damit noch nichts anfangen, aber es ist wichtig, dass Ihnen klar wird, dass eine Funktion beim Programmieren nicht nur etwas ist, dass wir direkt \glqq{}ausrechnen\grqq{} lassen können.
\end{itemize}

Wenn wir uns mit Konzepten wie der der objektorientierten Programmierung beschäftigen, dann kann der Wert eines virtuellen Objektes auch eine Sammlung von anderen virtuellen Objekten, Verweise auf Dateien oder sogar zu vollständigen Programmen sein. Ein Beispiel dafür haben Sie gerade gesehen: In manchen Programmiersprachen ist der Wert einer Variable der Verweis auf eine Funktion unseres Programms.

\subsection{Anonyme virtuelle Objekte}

Stellen Sie sich für diesen Abschnitt vor, wir hätten einer Variable a, der wir den Wert eine Operation zugeordnet haben. (Sprich, wir haben eine Programmzeile wie \verb|a = y + z;|.)\\

Wenn der Rechner eine Operation (in unserem Beispiel die Summe von y und z) einer Programmiersprache auswertet, dann bleiben die Werte, die für diese Operation verwendet wurden (y, z) in aller Regel erhalten und werden nicht überschrieben. Auch das Ergebnis bleibt zumindest für eine kurze Zeit erhalten. Das bedeutet aber auch, dass das Ergebnis einer Operation irgendwo gespeichert werden muss.\\

Vielleicht denken Sie jetzt, dass das Ergebnis doch sofort als Wert der Variablen a zugeordnet wird, aber auf unterster Ebene steht das nicht eindeutig fest.\\

So trivial das klingt führt es also zu der Frage: Wo wird dieses Ergebnis gespeichert? Wenn eine Operation ausgewertet wird, haben wir ja noch keine Variable deklariert, die das Ergebnis als Wert zugeordnet bekommt. Ansonsten müsste der Rechner ja abstürzen, wenn eine Operation ausgeführt wird, ohne dass das Ergebnis einer Variablen zugeordnet wird.\\

Und das führt direkt zum Begriff der \textbf{anonymen Variable}\index{Variable!anonym}. Eine anonyme Variable ist also eine Variable, die einen Wert, einen Datentyp und einen Pointer aber noch keinen Bezeichner hat. Es gibt also einen Binärwert, der in einer Speicheradresse gespeichert ist und der von der Programmiersprache als ein Wert im Sinne eines Datentyps interpretiert wird.\\

Natürlich besteht der nächste Schritt darin, dass dieser Wert einer Variablen zugeordnet wird, aber zuvor wird er wie beschrieben in einem Speicherbereich abgelegt. Es gibt danach also zwei Möglichkeiten: 

\begin{enumerate}
	\item Entweder wird nun der Wert der (im Beispiel mit a bezeichneten) Variable mit dem Wert der anonymen Variable überschrieben.
	\item Oder der Pointer dieser Variablen wird auf die Adresse der anonymen Variablen geändert.
\end{enumerate}

\subsubsection{Aufgabe}

Sie wissen jetzt, was eine Zuordnung ist. Sie wissen auch, was eine anonyme Variable ist. Überlegen Sie sich nun, warum nach der Zuordnung des Wertes einer anonymen Variablen zu einer deklarierten Variable Speicher verschwendet wird. Wenn Ihnen das klar ist, recherchieren Sie, was der sogenannte \textbf{Garbage Collector}\index{Garbage Collector} ist.

\subsection{Schreibweise von Bezeichnern}

Je nach verwendeter Programmiersprache gibt es unterschiedliche Konventionen (Vereinbarungen) darüber, wie Bezeichner geschrieben werden. Diese unterscheiden sich von Sprache zu Sprache und müssen erlernt werden; es gibt keine logische Begründung, aus der sich die Konvention ergibt.

\section{Datentypen}

Sie wissen bereits, dass der Datentyp bei statisch typisierten Sprachen genau wie bei dynamisch typisierten Sprachen festlegt, wie die Programmiersprache den Wert einer Variablen interpretiert. Wenn Sie sich nun die Datentypen für ganze Zahlen bei Java ansehen, dann werden Sie feststellen, dass es dafür verschiedene Datentypen gibt:

\begin{enumerate}
	\item \verb|byte|
	\item \verb|short|
	\item \verb|int|
	\item \verb|long|
\end{enumerate}

Wenn also der Datentyp nur festlegen würde, wie die Werte einer Variablen interpretiert werden, dann würde eine Definition von vier verschiedenen Variablen, die alle ausschließlich ganzzahlige Werte speichern können keinen Sinn machen: Es sind alles ganze Zahlen, also müsste doch ein Datentyp reichen.\\

Tatsächlich gibt der Datentyp einer Variablen nicht nur vor, wie ein gespeicherter Binärwert an einer Addresse des Speichers interpretiert werden soll, sondern er gibt auch an, wie viele Bit für den Wert verwendet werden. Wenn Sie sich später mit maschinennaher Programmierung beschäftigen, wird Ihnen das vollständig klar werden. Für den Moment sei nur gesagt:

\begin{enumerate}
	\item Eine \verb|byte|-Variable in Java belegt ein Byte (also 8 Bit) im Speicher. Das entspricht einem Intervall von -128 bis 127.
	\item Eine \verb|short|-Variable in Java belegt zwei Byte (also 16 Bit) im Speicher.
	Das entspricht einem Intervall im Bereich von \(-3,2 \cdot 10^4\) bis \(3,2 \cdot 10^4\).
	\item Eine \verb|int|-Variable in Java belegt vier Byte (also 64 Bit) im Speicher. Das entspricht einem Intervall im Bereich von \(-2,2 \cdot 10^9\) bis \(2,2 \cdot 10^9\).
	\item Und eine \verb|long|-Variable in Java belegt acht Byte (also 128 Bit) im Speicher. Das entspricht einem Intervall im Bereich von \(-9,2 \cdot 10^18\) bis \(9,2 \cdot 10^18\).
\end{enumerate}

Wenn Sie also eine Highscore-Liste programmieren wollen und dafür eine Variable vom Typ int programmieren, dann sollten Sie sicher sein, dass die gespeicherten Werte nicht über 30 Mrd. steigen können. Im besten Falle erhalten Spieler sonst anstelle einer neuen Highscore plötzlich Werte im Bereich von -30 Mrd. Punkten, im schlimmsten Fall stürzt Ihr Programm ab.\\

\subsubsection{Aufgaben}

\begin{enumerate}
	\item Mit dem, was Sie bis jetzt über Datentypen wissen, müssen Sie die folgende Frage beantworten können: Warum kann \verb|String| kein Datentyp sein?
	\item Der Datentyp einer Variablen ist in C statisch. Das heißt, er kann nicht geändert werden. Überlegen Sie sich, was Sie tun könnten, wenn Sie den Datentyp einer Variablen ändern wollen, damit Sie mit der Variablen weiterarbeiten können. \\
	Hinweis: Es ist unmöglich den Datentyp einer Variablen zu ändern. Also muss es eine andere Möglichkeit geben.	
\end{enumerate}

Die Änderung des Datentyps einer Variablen wird als \textbf{Typecasting}\index{Typecasting} bezeichnet.

\subsection{Datentypen in C und Java}

Es gibt noch mehr als die hier aufgeführten Datentypen, aber selbst für Fortgeschrittene sollten diese genügen.\\ 

Bitte beachten Sie, dass Groß- und Kleinbuchstaben aber auch Zeichen wie der Unterstrich fest zugeordnet sind. Die drei Bezeichner \verb|bool|, \verb|Bool| und \verb|\_Bool| sind also weder das Selbe noch das Gleiche.

\subsection{Datentypen für ganze Zahlen}

Ähnlich wie bei Java gibt in diesem Bereich in C vier Datentypen, die danach unterteilt werden, wie viel Bit an Speicherplatz sie belegen. Leider ist die Bedeutung nicht identisch mit der in Java.

\begin{enumerate}
	\item \verb|short| hat in C 16 Bit Länge.
	\item \verb|int| und \verb|long| haben in C 32 Bit Länge.
	\item \verb|long long| (nicht mit dem einfachen long verwechseln!) hat in C 64 Bit Länge.
\end{enumerate}

Hier zum Vergleich nochmal die Datentypen in Java. Sie werden alle mit dem Wert 0 initialisiert. Eine explizite Initialisierung ist also nicht nötig:

\begin{enumerate}
	\item \verb|byte| in Java 8 Bit: -128 bis 127.
	\item \verb|short| in Java 16 Bit: \(-3,2 \cdot 10^4\) bis \(3,2 \cdot 10^4\).
	\item \verb|int| in Java 32 Bit: \(-2,2 \cdot 10^9\) bis \(2,2 \cdot 10^9\).
	\item \verb|long| in Java 64 Bit \(-9,2 \cdot 10^18\) bis \(9,2 \cdot 10^18\).
\end{enumerate}

Damit sind die einzigen ganzzahlige Datentypen, den Sie bei beiden Sprachen gleich nutzen können \verb|short| und \verb|int|. Wollen Sie dagegen besonders große Zahlen verwenden, müssen Sie differenzieren, ob Sie ein einfaches oder doppeltes long programmieren müssen.

\textbf{Programmierbeispiele}\\

Wenn Sie die Variable \verb|punktestand| für ein Fußballspiel programmieren wollen, dann lautet in Java die Deklaration und Intialisierung:\\

\verb|int puntestand;|

Hier könnten Sie zwei Dinge verwundern:

\begin{enumerate}
	\item Wenn Sie sich wundern, warum hier kein Wert zugeordnet wurde, dann lesen sie nochmal etwas weiter oben nach, mit welchem Wert ganzzahlige Variablen in Java immer initilisiert werden, außer wenn das vom Programmierer explizit überschrieben wird.
	\item Wenn Sie sich wundern, warum da kein Schlüsselwort für die Zugriffsbeschränkung steht, dann kommt hier ein kleines aber feines Detail: Es gibt eine Beschränkung des Zugriffs, für die es kein Schlüsselwort gibt. Sie werden später noch feststellen, dass die Feinheiten der Zugriffsbeschränkung in Java nicht systematisch sind, aber das ist bis zum Beginn der objektorientierten Programmierung irrelevant.
\end{enumerate}

\subsection{Operationen für ganzzahlige Variablen}

Die Operationen \verb|+|, \verb|-|, \verb|*| stehen für Addition, Subtraktion und Multiplikation und können problemlos angewendet werden. Das Ergebnis ist eine ganzzahlige Variable und so lange wir auf den Wertebereich achten, können wir zwei \verb|int|-Variablen problemlos mit einer dieser Operationen verknüpfen: Das Ergebnis wird wieder eine \verb|int|-Veriable sein.\\

Doch was ist mit der Division (programmiert als \verb|\|)? Die ist tatsächlich ein Problem, denn abgesehen von \verb|division by zero| gibt es hier ja noch Unklarheiten, wenn das Ergebnis eine ganzrationale Zahl ist: In statisch typisierten Sprachen können Sie davon ausgehen, dass eine Division zweier ganzer Zahlen immer ein ganzzahliges Ergebnis ist, bei dem die Stellen nach dem Komma schlicht unter den Tisch fallen. Also ergibt \verb|1 / 4 = 0| und nicht \verb|1 / 4 = 0.25|!\\

Allerdings müssen Sie hier bei jeder Programmiersprache wieder genau nachschlagen, wie ein solcher Fall ausgewertet wird und sich genau ansehen, was die Auswirkung davon ist. (Anm.: Es ist schon faszinierend, dass die Anhänger von statisch typisierten Sprachen darin kein Unsicherheitsproblem sehen, schließlich ist 1 : 4 nicht gleich 0.)\\

Andererseits habe ich noch einen Operator unter der Tisch fallen lassen, mit dem Sie sich in solchen Fällen absichern können: Es geht um den Modulo-Operator, der als \verb|%| programmiert wird. Die Sicherheit erreichen Sie in dem Fall z.B. wie folgt:

\begin{verbatim}
x = a / b;
y = a % b;
\end{verbatim}

Jetzt müssten Sie nur noch prüfen, ob y gleich 0 ist, um sicher zu gehen, dass \glqq{}nichts unter den Tisch fällt\grqq{}. Genau das ist aber der zentrale Nachteil von streng typisierten Sprachen gegenüber dynamisch typisierten Sprachen: In den meisten Fällen müssen Sie deutlich mehr Zeilen programmieren, ohne davon einen Mehrwert zu haben.\\

Hier nochmal die arithmetischen Operationen für ganzzahlige Variablen:

\begin{itemize}
	\item \verb|+| Addition
	\item \verb|-| Subtraktion
	\item \verb|*| Multiplikation
	\item \verb|\| Division
	\item \verb|%| Modulo
\end{itemize}

Es gibt noch eine Reihe weiterer Operationen, die für ganzzahlige Variablen definiert sind, aber dazu kommen wir nach der Einführung weiterer Datentypen, da das Ergebnis dieser Operationen ein Wert mit einem anderen Datentyp ist.

\section{Das erste Programm}

Wenn wir hier mit interpretierten Sprachen angefangen hätten, dann hätten wir die Programmzeilen direkt in den Interpreter eintragen können und dieser hätte umgehen die Ergebnisse ausgegeben. Bei C (bzw. C++) sowie Java haben wir es aber mit kompilierten Sprachen zu tun. Also besteht der erste Schritt darin, den Quellcode in einer Datei abzuspeichern, ihn zu kompilieren und die neue Datei zu starten. (Keine Sorge, das machen wir gleich Schritt für Schritt.)\\

Zur Wiederholung: Programme einer kompilierten Sprache werden vollständig in ein Format umgewandelt, das direkt vom Betriebssystem bzw. der Laufzeitumgebung der Programmiersprache ausgeführt werden kann. Diese Umwandlung wird als Kompilieren bezeichnet.

\subsection{Kleines Programm in C}

Wenn Sie entsprechend des Kapitels zur Vorbereitung des Rechners GCC installiert haben, dann können Sie direkt einen Editor öffnen und folgendes kurzes Programm eingeben. Speichern Sie es dann (idealerweise in einem Verzeichnis wie \verb|C:\a_meine_programme|) unter dem Namen \verb|helloMarvin.c|):\\

\begin{verbatim}
#include <stdio.h>
main()
{
	int a = 42;
	int b = 2 * a;
	printf("Die Antwort lautet: " + b );
}
\end{verbatim}

Wechseln Sie jetzt auf der Konsole in das Verzeichnis \verb|C:\a_meine_programme| und geben Sie dort \verb|gcc helloMarvin.c| ein und anschließend \verb|a| . Darauf wird eines der unter InformatikerInnen beliebtesten Zitate ausgegeben.\\

Was die include-Anweisung, das main() oder die geschweiften Klammern bedeuten schauen wir uns im Anschluss an die Besprechung der Datentypen an, aber vorher schauen wir und die gleiche Aufgabe in Java an:

\subsection{Kleines Programm in Java}

\begin{verbatim}
public class HelloMarvin{
	public static void main(String[] args)
	{
		int a = 42;
		int b = 2 * a;
		System.out.println("Die Antwort lautet: " + b );
	}
}
\end{verbatim}

Nachdem Sie diese Datei im Verzeichnis \verb|C:\a_meine_programme| unter dem Namen \verb|HelloMarvin.java| gespeichert haben, geben Sie \verb|javac HelloMarvin| ein und anschließend \verb|java HelloMarvin| .\\

Wie Sie sehen sind die Unterschiede zwischen beiden Programmen gar nicht so groß: Das eigentliche Programm ist bis auf ein Detail identisch. Nur drum herum steht eine ganze Menge Code, der Ihnen momentan unklar sein dürfte.\\

\subsection{Kleines Programm in Pascal}

\begin{verbatim}
program HelloMarvin(output);
var
a : integer;
b : integer;
begin{main}
a := 42;
b := 2 * a;
writeln('Die Antwort lautet: ' + b);
end{main}.
\end{verbatim}

\textbf{Aufgabe}:\\

Vergleichen Sie diesen Code mit dem in C und Java. Was ist der zentrale Unterschied bei der Programmierung der Variablen? (Nein, es ist nicht die Tatsache, dass der Datentyp hier \verb|integer| und bei den beiden anderen \verb|int| lautet.)

\subsection{Kleines Programm in PHP}

PHP kennen Sie ja bereits aus dem ersten Teil des Buches:

\begin{verbatim}
$a = 42;
$b = 2 * $a;
echo("Die Antwort lautet: " . b);
\end{verbatim}

\textbf{Aufgabe}:\\

Selbst wenn Sie es nicht mehr wissen, sollten Sie jetzt sofort einen Unterschied zwischen PHP einerseits und C, C++ sowie Java andererseits benennen können. Welcher ist das?

\subsection{Kleines Programm in Scheme}

Scheme ist ein Lisp-Dialekt und damit eine funktionale Sprache. Funktionale Sprachen sind wie beschrieben dafür gedacht, so effizient wie möglich zu programmieren. Design ist hier in aller Regel nicht umsetzbar. (Sprachen wie JavaScript nutzen zwar auch die funktionale Programmierung, aber dort ist es sehr wohl möglich, ein ansprechende Optik ins Programm zu integrieren.)

\begin{verbatim}
(define a 42)
(define b (* 2 a))
(display "Die Antwort lautet: ")
b
\end{verbatim}

\textbf{Aufgabe}:\\

Beschreiben Sie, welche Unterschiede Sie zwischen der funktionalen Programmierung und der imperativen Programmierung anhand dieses kleinen Beispiels erkannt haben.

\subsection{Kein sinnvolles Problem für PROLOG}

Die Teile unserer Programme, die wir in PROLOG umsetzen können, habe ich hier zusammengefasst. Wie Sie sehen fehlt da die Textausgabe. Aber das ist kein Wunder: PROLOG ist eine Sprache für logische Programmierung, das bedeutet, das PROLOG nicht dafür gedacht ist, um nett formulierte Ausgaben durchzuführen, sondern einzig dafür, aus einer komplexen Menge an Fakten mögliche Lösungen zu ermitteln. Wenn wir also unsere kleine Aufgabe in PROLOG umsetzen, dann ist das so, als wenn wir einen LKW nutzen, um eine Handvoll Staub aus der Wohnung zur Mülltonne zu transportieren. Die folgenden zwei Zeilen sind in PROLOG zwar möglich aber sowohl stilistisch schlecht als auch inhaltlich weitgehend sinnfrei.

\begin{verbatim}
a is 42.
b is 2 * a.
\end{verbatim}

Wir werden uns in einem späteren Kapitel ansehen, was es mit logischer Programmierung auf sich hat und dann werden Sie verstehen, warum dieses PROLOG-Programm so unsinnig ist.\\

Jetzt aber wieder zurück zu Datentypen in Programmiersprachen. Wir bleiben hier jedoch vorerst bei rein imperativen Programmiersprachen:

\section{Fortsetzung zu Datentypen}

Die Tatsache, dass der eigentliche Kern nahezu gleich ist, ist der Grund, aus dem ein Umstieg von Java auf C oder umgekehrt gar nicht so schwer ist. Das wiederum ist auch der Grund, warum ich hier beide Sprachen gemeinsam vermittle: Da beide Sprachen sehr gebräuchlich ist, ist es gut, wenn Sie beide beherrschen. Die größte Schwierigkeit beim Umstieg von Java zu C bzw. C++ ist die Pointerarithmetik und umgekehrt tun sich die meisten C bzw. C++ ProgrammiererInnnen schwer damit, darauf zu verzichten. Dabei bietet Java Ihnen da eine sehr angenehme Alternative an. Doch zu diesen Details später mehr.\\

Doch bevor wir uns weiter mit Datentypen auseinander setzen, kommen wir zu einem weiteren Thema, bei dem Sie ausnahmsweise auswendig lernen müssen und das für jede einzelne Programmiersprache aufs neue:

\subsection{Literale und reservierte Schlüsselwörter}

Oben haben Sie einfach Programmbeispiele abgetippt. Dabei haben Sie am Anfang dieses Kapitels gelernt, dass es für jede Programmiersprache zugelassene Literale und bei der Wahl der Bezeichner feste Regeln gibt, wie aus diesen Literalen die Bezeichner von virtuellen Objekten formuliert werden dürfen.

\subsubsection{Literale in C}

\subsubsection{Literale in Java}

Zunächst ist der Java Compiler im Stande alle UTF-16 Zeichen zu interpretieren. Das bedeutet, dass wir ein Java-Programm theoretisch als eine Folge von Unicodes in der Form /u0000 bis /uFFFF programmieren könnten.\\




\subsubsection{Literale in PHP}


\subsection{Datentypen für Buchstaben}

Sie mögen sich wundern, warum jetzt nicht die Datentypen für Fließkommazahlen folgen, aber das ist recht simpel: Für Rückmeldungen werden Sie im Regelfall Texte (also aus Buchstaben zusammengesetzte Ausgaben) benötigen. In den C, C++ und Java heißt der Datentyp zwar immer \verb|char|, aber in C und C++ hat er nur eine Länge von 8 Bit, während ein \verb|char| in Java 16 Bit belegt.\\

Damit haben wir auch hier wieder eine Situation, in der der gleiche Datentyp in nahe verwandten Programmiersprachen unterschiedlich umgesetzt wird: In C und C++ können wir in einer \verb|char|-Variablen Symbole aus der ASCII-Codierung speichern (also beispielsweise keine Umlaute), während wir in einer \verb|char|-Variablen in Java Symbole entsprechend der UTF-16-Codierung speichern und damit ohne weiteren Aufwand praktisch alle Zeichen von heute gesprochenen Sprachen nutzen können. 

Kontrolle\\

Deklarierung und Initialisierung von char-Variablen:\\


char einBuchstabe = `a`;

Wichtig: Im Gegensatz zu Strings, die wir uns erst im Rahmen der Datenstrukturen ansehen werden beginnen und enden char-Variablen mit einem einfachen Anführungszeichen.\\


\subsection{Datentypen für Wahrheitswerte}

Richtig gelesen: Es gibt einen Datentyp für Variablen, die nur unterscheiden, ob etwas wahr oder falsch ist. Solche Variablen nennt man auch boolesche Variablen. Den meisten Programmierern ist gar nicht klar, wie oft Sie mit diesen Variablen arbeiten, weil sie meist durch eine Operation erzeugt werden, deren Ergebnis sofort und danach nie wieder genutzt wird. Und in diesem Fall brauchen Sie keine Variable dieses Datentyps deklarieren.\\

Da das komplizierter klingt, als es ist, hier ein Beispiel: Sie können innerhalb eines Programms einfache wenn-dann-Abfragen wie die folgende programmieren: Wenn es kälter als 12 Grad ist, zieh dir einen Pullover an, sonst lass das. Hierzu wird ein Vergleich einprogrammiert, bei dem eine Variable mit dem Wert 12 vergleichen wird. Das Ergebnis ist eine boolesche Variable. Wenn Sie aber wie in diesem Beispiel direkt die Auswirkung programmieren (… dann zieh dir einen/keinen Pullover an), dann brauchen Sie das Ergebnis der Vergleichsoperation keiner expliziten Variablen zuordnen.\\

Boolesche Variablen können Sie übrigens nur dann deklarieren, wenn sie die Headerdatei stdbool.h am Programmanfang includen. Der Datentyp lautet dann bool.\\

\subsection{Datentypen für Fließkommazahlen}

Zur Erinnerung: Nutzen Sie wann immer möglich ganzzahlige Datentypen, wenn Sie programmieren, da Sie durch die Zahlen zur Basis 2 bei Divisionen von Zahlen zur Basis 10 schnell Rundungsfehler bzw. unschöne Zahlendarstellungen bekommen.\\

In diesem Bereich gibt es lediglich drei Datentypen, wobei zwei davon wieder identische Zahlenbereiche abbilden. Schlagen Sie diese Zahlenbereiche bei Bedarf nach.\\

•	float hat 32 Bit Länge
•	double und long double haben 64 Bit Länge\\

\subsection{Datentypen für aufzählbare Elemente}

Diese Datentypen sind recht spannend, weil sie es Ihnen ermöglichen, mit wenig Speicherbedarf relativ große Datenmengen abzubilden. Sie haben bei den bisherigen Datentypen jeweils den Fall gesehen, wo Sie einer Variablen einen konkreten Wert direkt zuordnen konnten. Jetzt lernen Sie einen Datentyp kennen, mit dem Sie einer Variablen einen Wert aus einer List indirekt zuordnen können.\\

Nehmen wir an, Sie haben eine Liste von Farben (wie auch immer Sie diese programmieren können). Und Sie möchten jetzt die Möglichkeit haben, dass eine Variable nur Einträge aus dieser Liste als Wert beinhalten kann. In diesem Fall lautet der Datentyp:\\

•	enum\\

\subsection{Datentypen für Pointer}

Pointer sind Datentypen, die nicht etwa den Inhalt einer Speicheradresse im Sinne von Zahlen, Buchstaben oder ähnlichem interpretieren, was in einer Codierungstabelle definiert wird, sondern die den Inhalt einer Speicheradresse als Speicheradresse verstehen.\\

Wenn Sie also einen Pointer haben, unter dessen Adresse der Wert 0x30 gespeichert ist, dann wird dieser Wert als die Adresse 0x30 „interpretiert“. Im Grunde sind Pointer (und die zugehörige Speicherarithmetik) also eine Möglichkeit, all die Speicherzugriffe zu realisieren, die wir sonst nur bei Assembler durchführen könnten. Mit Pointern können wir beispielsweise all die Datenstrukturen realisieren, die C selbst nicht mitbringt.\\

Allerdings werden wir uns im Rahmen dieser Veranstaltung nicht mit Pointern beschäftigen; wenn Sie mit C oder C++ nach diesem Kurs weiter arbeiten wollen, müssen Sie das noch nachholen. Bei Java dagegen gibt es keine Pointer, weil dort der direkte Zugriff auf den Speicher ausgeschlossen ist.












\emph{Wichtig}: Hier den PHP-Teil und den C-Teil sowie den imperativen Java-Teil integrieren.






Wichtig:\\


Alle Hinweise, die sich in diesem Abschnitt auf die Nutzung von Windows beziehen, gelten so für Windows 7. Bitte prüfen Sie ggf. selbständig, wie Sie unter 8 oder 10 vorgehen müssen. Allerdings sollten die Befehle in der Konsole auch dort bis auf die Verzeichnisstruktur von Windows vollständig identisch sein.\\


Das folgende ist für Windows-User, alle anderen nutzen bitte die entsprechenden Befehle und Programme ihres Betriebssystems. Um mit dem Programmieren zu beginnen, öffnen Sie bitte die sogenannte Eingabeaufforderung. (Bei anderen Betriebssystemen Console, Bash o.ä. genannt.)\\


Wenn bei den folgenden Aufgaben (Verzeichnis erstellen/ins Verzeichnis wechseln) Fehler passieren, dann haben Sie entweder keine ausreichenden Rechte, um diese Aufgabe an dem Rechner durchzuführen an dem Sie gerade sitzen oder Sie haben sich schlicht vertippt.\\


Mit dem Befehl mkdir C:/a\_prog erstellen Sie bitte in der Eingabeaufforderung ein Verzeichnis. (Sie können es auch anders nennen, aber wenn Sie es so benennen, wird es am Anfang Ihrer Verzeichnisübersicht auftauchen.)\\

Wechseln Sie jetzt mit dem Befehl cd C:/a\_prog in dieses neue Verzeichnis. \\


Starten Sie jetzt den Editor, den Sie im Bereich „Alle Programme“ unter Zubehör finden. Alternativ können Sie auch gerne einen einfachen Editor wie Notepad++ nutzen, der so wenige Komfortfunktionen bietet, dass er Einsteiger unterstützt anstatt Sie (wie die meisten IDEs) zu verwirren.\\


Hier ein kurzer C-Quellcode, den Sie bitte mittels des Editors im oben genannten Verzeichnis abspeichern. Nennen Sie das Programm beim Speichern am besten antwort42.c (wichtig ist nur, dass die Dateiendung .c ist) Dieser Code hat noch nichts mit der Programmierung des ARM-Prozessors zu tun, Sie können ihn also auf jedem Rechner programmieren und ausführen, der einen C-Compiler installiert hat.\\

\begin{verbatim}
\#include <stdio.h>

main() 
\{
	printf("Die Antwort lautet: 42");
\}
\end{verbatim}

C-Programm, das die Zeile „Die Antwort lautet: 42“ (ohne Anführungszeichen) auf der Konsole ausgibt.\\

Wichtig: Wenn Sie unbedingt mit einer IDE beginnen wollen, dann wird dieser Code wahrscheinlich nicht genügen und Sie werden zunächst z.B. ein Projekt erstellen u.ä. Da es hier jedoch mehr als genug Fehlerquellen gibt, aus denen ein Programm nicht läuft, rate ich Ihnen an dieser Stelle von der Nutzung einer IDE ab. (Diejenigen, die eine IDE nutzen werden einige zusätzliche Aufgaben in diesem Text finden. Diese Aufgaben dienen dazu, dass Sie die IDE tatsächlich nutzen und sich langfristig einen professionellen Stil angewöhnen.)\\

Wenn Sie das ignorieren, sollten Sie eine erweiterte Fassung des Quellcodes verwenden wie diejenige, die Sie im Abschnitt zur Deklaration und Initialisierung von Variablen (folgt weiter unten) finden können.

\section{Inhalt eines einfachen C-Programms}

Unser C-Programm besteht aus drei Teilen: \\

•	Zunächst wird eine sogenannte Headerdatei eingefügt. Was das im Detail bedeutet und wie es den Kompilierungsprozess beeinflusst, ist für das Verständnis des Programms nicht wichtig. Wie schon in den einleitenden Kapiteln erläutert nutzen wir bei der Entwicklung von Programmen in aller Regel Teile, die von anderen Entwicklern programmiert wurden. Und Header-Dateien sind ein Beispiel für solchen wiederverwendeten Code.\\

Ein Hinweis bezüglich des Namens: stdio steht schlicht für standard input out. Damit ist klar, was für Aufgaben diese Headerdatei in unser Programm einführt: Sie ermöglicht es uns unter anderem, Ausgaben auf dem Bildschirm zu erzeugen und Eingaben von der Tastatur anzunehmen.\\

(Hinweis für die Programmierung in Java: Wenn Sie in Java so etwas wie System.out.println() verwenden, dann greifen Sie auf stdout zu. Wenn Sie in Java die sogenannte Scanner-Klasse nutzen, dann verwenden Sie meist stdin. Dabei werden Sie in Java nicht die Bezeichnungen stdout und stdin verwenden, aber das was Sie benutzen ist genau das gleiche. Und stdio kombiniert schlicht stdin und stdout.)\\

•	Dann folgt die Funktion main(). Was Funktionen einer (imperativen) Programmiersprache sind, folgt im Abschnitt Operationen und Funktionen. Für den Moment sollten Sie sich lediglich merken, dass bei C, C++ und Java der Anfang eines Programms in der main()-Funktion steht.\\

•	Nach main() folgen drei Zeilen, wobei die erste und letzte schlicht eine öffnende und schließende Klammer enthält. Diese beiden Klammern besagen schlicht, dass alles, was sich zwischen Ihnen befindet Teil der main() ist. So wie eine Variable in der Mathematik als Stellvertreter für einen Wert verwendet werden kann, kann ein Funktionsname (wie hier main) für einen Reihe von Befehlen verwendet werden.\\

•	Damit bleibt nur noch die Frage, was denn diese Zeile tut, die mit printf beginnt. Das wird Teil mehrerer Aufgaben, die Sie demnächst lösen.\\

Kontrolle\\

Für diesen und einige der nachfolgenden Abschnitte gibt es keine explizite Kontrolle. Prüfen Sie bitte selbständig, ob Sie die Inhalte verstanden und umgesetzt haben. Beachten Sie bitte dabei, dass dieser Kurs für Einsteiger ohne Vorkenntnisse in der Programmierung erstellt wurde; Fortgeschrittene mögen bitte dennoch diese Abschnitte lesen, um festzustellen, wo ihre Erfahrungen von dem Abweichen, was hier beschrieben ist. Überlegen Sie dann bitte genau, worin diese Abweichung besteht, also ob es eine Vereinfachung für Einsteiger ist oder ob Sie sich unter einem Konzept tatsächlich etwas anderes vorstellen als das, was hier beschrieben ist. Beides ist selbstverständlich möglich.

\subsection{Ihr erstes C-Programm: Vom Quellcode zum Program Image}

Geben Sie jetzt in der Eingabeaufforderung den Befehl dir (bei anderen Betriebssystemen ls) ein, um zu sehen, welche Dateien sich in diesem Verzeichnis befinden. Sie sollten jetzt eine kurze Tabelle sehen, bei der unter anderem der Dateiname antwort42.c aufgeführt ist. Ist das nicht der Fall, dann haben Sie entweder den Quellcode gar nicht oder an einem anderen Ort abgespeichert.\\

Wie Sie wissen haben Sie damit Quellcode erzeugt, der noch nichts tut. Wechseln Sie jetzt bitte in die Eingabeaufforderung und geben Sie dort den Befehl gcc antwort42.c ein, um den Quellcode zu kompilieren.\\

Geben Sie nun nochmal dir (bzw. bei Linux ls) in der Eingabeaufforderung ein. Wie Sie sehen, befindet sich jetzt eine zweite Datei im Verzeichnis mit dem Namen a.exe . Links neben den Dateinamen befindet sich jeweils eine Zahl, die angibt, wie viele Byte jede Datei belegt.\\

Kontrolle\\

Warum hat unser Quellcode nur einen Umfang von 60 Byte, aber das Program Image hat mehr als 60.000 Byte? Woher kommt all dieser zusätzliche Inhalt?

\subsection{Über die Nutzung des gcc Compilers}

Wenn Sie nun den Buchstaben a eingeben und Enter drücken, passiert folgendes: Das Programm gibt den Satz Die Antwort lautet: 42 aus.\\

Frage: Warum heißt die Datei nicht antwort42.exe?\\

Die Antwort ist wieder einmal ganz simpel: Weil wir dem Compiler nicht gesagt haben, dass das kompilierte Programm unter dem Namen antwort42.exe gespeichert werden soll.\\

Frage: Was müssen wir tun, um einen Namen für die Ausgabedatei zu vergeben?\\

Wie so oft beim Verwenden von Programmen, die andere erstellt haben, müssen Sie jetzt suchen. GCC ist der Compiler, den wir verwenden und wenn Sie kein Linux-Nutzer sind, dann müssen Sie im Netz nach GCC, besser noch GCC HELP suchen. Ohne Englisch wird’s jetzt schwer, aber das ist in der Informatik immer so, also entweder lernen Sie die Sprache spätestens jetzt oder Sie werden dauerhaft massiv benachteiligt sein. Konkret suchen Sie nach so etwas wie einem gcc manual bzw. den gcc command options.\\

Kontrolle\\

(1)	Finden sie heraus, wie Sie den gcc dazu bringen können, Ihre Datei unter dem Namen antwort42.exe zu speichern. (Nein, es reicht nicht aus, wenn Sie die Datei einfach selbst umbenennen, nachdem sie vom gcc kompiliert wurde.)\\

(2)	Warum sollte der Dateiname nicht antwort 42.exe (also mit einer Leerstelle zwischen antwort und 42) heißen?\\

(3)	Angenommen Sie wollen dennoch einen solchen Dateinamen erhalten. Was müssen Sie tun, um das zu realisieren?\\

\subsection{Variablen}

Was die Register für die maschinennahe Programmierung sind, sind die Variablen für die imperative Programmierung. Deshalb beginnen wir die Einführung in die imperative Programmierung mit der Einführung von Variablen.\\

Wichtig:\\

Wie schon bei der Einführung in die Programmierung in PHP ist es wichtig, dass Sie an dieser Stelle verstehen, dass Variablen nur eine Möglichkeit sind, um Daten im Rahmen einer Programmiersprache zu verwalten. Die Programmierung mit Hilfe von Variablen ist dabei ein zentrales Kennzeichen aller imperativen Programmiersprachen, doch wie gesagt gibt es noch wesentlich mehr Paradigmen als nur das imperative.\\

Das folgende gilt so nur für kompilierte Sprachen (also u.a. für C, C++ und Java). Bei interpretierten Sprachen gelten zwar zum Teil dieselben Grundlagen, aber wie schon in den einführenden Kapiteln beschrieben legen wir dort den Datentyp einer Variablen nicht durch einen „Befehl“ fest, sondern der Interpreter der Sprache legt selbständig einen Datentyp fest bzw. ändert diesen dynamisch und wir können ihn im Regelfall nicht einsehen.\\

Eine Variable ist schlicht ein Bezeichner, über den wir auf einen Wert bzw. ein Zeichen verweisen können. Im Gegensatz zu einem Register haben wir hier aber große Freiheiten, wenn es darum geht, wie eine Variable bezeichnet werden soll. Wenn Sie also beispielsweise eine Variable einrichten wollen, um eine zurückgelegte Strecke zu speichern, dann können Sie diese Variable zurueckgelegteStrecke nennen und brauchen nicht auf kryptische Bezeichnungen wie R12 zurück zu greifen, so wie das bei der Nutzung der Register war. Es gibt zwar gewisse Einschränkungen, welche Zeichen Sie benutzen dürfen, aber so lange Sie nur Buchstaben verwenden, die auch im englischen Alphabet vorkommen, sollten Sie kein Problem haben. Bis auf das erste Zeichen können Sie auch Zahlen und bestimmte Sonderzeichen verwenden.\\

Das Arbeiten mit Variablen ist vor allem deshalb ein mächtiges Werkzeug, weil wir hier so tun können, als wenn der Computer doch mehr könnte als nur Zahlen aus dem Speicher zu laden, sie zu addieren und sie wieder im Speicher abzulegen. Diese Arbeitsschritte brauchen wir dann nicht mehr zu programmieren; sie werden quasi von der Programmiersprache (genauer gesagt von der Laufzeitumgebung oder dem Interpreter der Programmiersprache) erledigt.\\

Wie Sie jetzt wissen speichern Sie streng genommen keinen Wert in einer Variablen, sondern der Wert wird immer noch unter einer Speicheradresse abgelegt. Dennoch sprechen wir bei imperativen Sprachen davon, dass wir einer Variablen einen Wert zuordnen, dass wir unter einer Variablen einen Wert speichern, oder dass wir den Wert einer Variablen ändern. Die dem zugrundeliegenden Operationen, die wir in der maschinennahen Programmierung noch selbst programmieren mussten, sind jetzt irrelevant geworden.\\

Diejenigen von Ihnen, die bereits in einer imperativen Sprache programmiert haben werden jetzt einwenden, dass Sie einer Variablen nicht nur einzelne Zahlen und Zeichen zuweisen können. Das ist streng genommen falsch. Was Sie meinen ist, dass wir einem Bezeichner oder einem Namen z.B. bei einer Funktion mehr als nur einzelne Zahlen oder Zeichen zuordnen können. Ein String beispielsweise ist aber bereits eine Datenstruktur und keine einfache Variable mehr. Doch bevor wir uns um diese Fälle kümmern können, müssen wir zunächst klären, was eine Datenstruktur ist, die Sie allerdings bitte nicht mit Datentypen verwechseln, die Sie zuvor kennen lernen.\\

Kontrolle\\

Variablen sind wie Register, nur mit dem Unterschied, dass Sie praktisch keine Beschränkung in der Anzahl Variablen haben, die Sie verwenden wollen. Außerdem steht es Ihnen weitgehend frei, Variablen so zu nennen, wie Sie das wollen. An dieser Stelle wurde der Begriff der Variablen sehr strikt definiert und beispielsweise nicht als Bezeichner von Datenstrukturen verwendet.\\

Wichtig\\

Das ist nicht allgemeingültig, sondern es geht hier darum, dass Sie eine genaues Verständnis dafür bekommen, was eine Variable tatsächlich ist. Leider wird der Begriff für viele Dinge verwendet, selbst wenn diese Nutzung so nur für einzelne Programmiersprachen gilt. Wenn Sie dann eine andere Programmiersprache lernen, kommen Sie in Situationen, in denen Sie zum einen Variablen für bestimmte Dinge nicht nutzen können, bei denen Sie daran gewöhnt sind und zum anderen können Sie sie dann für etwas nutzen, was Ihnen vollkommen absurd vorkommt. Wenn Sie dagegen die strikte Definition des Begriffs Variable nutzen, den ich hier vorgestellt habe, wird Ihnen das deutlich seltener passieren.\\

\subsection{Statisch versus nicht-statisch}

In den einleitenden Kapiteln haben Sie ja bereits den Unterschied zwischen dynamisch und statisch typisierten Sprachen kennen gelernt. Aber es gibt eben nicht nur statische bzw. dynamische Datentypen, sondern noch wesentlich mehr, was bei einer Programmiersprache dynamisch sein kann. (Bei C ist das allerdings recht wenig.)\\

Dieser Unterschied zieht sich also quer durch die Programmierung und Sie werden dementsprechend immer wieder mit Sprachen zu tun haben, in denen Sie etwas programmieren, das entweder während des gesamten Programmablaufs gleichbleibt oder sich verändert bzw. sich verändern kann.\\

Leider gibt es hier jedoch keine einheitliche Nomenklatur. Das bedeutet, dass Sie nicht einfach im Handbuch zur Sprache nachschlagen können, welche Teile der Sprache nun statisch und welche dynamisch sind, bzw. wie Sie das jeweils festlegen können. \\

Bei Variablen ist es beispielsweise so, dass manchmal von konstanten Variablen und manchmal von statischen Variablen die Rede ist. Die Bezeichnung Konstante wie Sie sie aus der Mathematik kennen ist hier nicht zutreffend.\\

Erinnern Sie sich in diesem Bezug bitte daran, dass alles, was im Speicher eines Computers steht letztlich ein veränderlicher Wert ist: Es sind alles Daten, die jederzeit überschrieben werden können. Und der Quellcode, den Sie erstellen nutzt ja nur all das, was in einer Sprache bereits fest einprogrammiert ist. Ob diese Sprache (was bei C an vielen Stellen der Fall ist), dann einmalige Definitionen dauerhaft behält also statisch ist oder es zulässt, dass diese im Verlauf eines Programms beliebig geändert werden also dynamisch ist, hängt eben von der Sprache selbst ab und nicht davon, was Sie erwarten.\\

Kontrolle\\

Während Sie bei der maschinennahen Programmierung lediglich Werte in Register laden, dort verändern und dann im Speicher ablegen konnten, müssen Sie bei allen Elementen einer höheren Programmiersprache lernen, ob diese während des Programmablaufs geändert werden können oder nicht. Der Unterschied wird mit unterschiedlichen Begriffen bezeichnet.\\

Synonym für statisch wird konstant verwendet.\\

Synonym für dynamisch wird nicht-statisch, variabel, veränderlich und eine Reihe weiterer Begriff verwendet. \\

Verwechseln Sie bitte nicht eine Variable als Bezeichner für einen Speicherbereich, worüber wir hier reden, mit einer Variablen als dem eigentlichen und/oder veränderlichen Wert. Der Begriff der Variable, den ich hier vorstelle beinhaltet sowohl einen Bezeichner als auch einen Datentyp als auch einen Wert, der an einer Speicherstelle gespeichert ist.


\section{Deklaration, Initialisierung und der Scope}




Wichtig: Wenn Sie eine Variable deklarieren, wird sie bei den meisten Programmiersprachen mit einem Standardwert initialisiert. Wenn Sie eine Sprache regelmäßig nutzen, sollten Sie diesen Standardwert kennen.\\

Beispiele für die Deklaration:\\

Die folgende Zeile deklariert eine Variable mit dem Bezeichner eineZahl als Variable vom Typ ganzzahlig mit 32 Bit. Der Bezeichner „eineZahl“ darf in diesem Programm vorher noch nicht aufgetaucht sein.\\

int eineZahl;\\

Die folgenden zwei Zeilen sollen nur deutlich machen, dass eine Deklaration nichts ist, worüber Sie groß nachdenken müssten:\\

char grosserBuchstabe;
enum Farben;\\

Noch ein abschließender Hinweis: Das Semikolon zeigt dem Compiler an, dass hier eine Programmzeile zu Ende ist. Sie könnten also theoretisch auch diese Zeile nutzen, um alle drei Variablen zu deklarieren:\\

float pi; short bully; pointers zeiger;\\

Nur ist das eben schlecht lesbar. So programmieren nur Leute, die wollen, dass niemand Ihren Quellcode lesen kann (was aber nicht funktioniert) oder die nicht wissen, was sie tun.\\

Beispiele für die Initialisierung / Zuordnung eines Wertes:\\

Das folgende können Sie nur tun, wenn Sie die jeweiligen Variablen bereits initialisiert haben. Hier wird einer Variablen ein Wert zugeordnet:\\

eineZahl = 42;\\

Der Variablen eineZahl wird hier also der Wert 42 zugeordnet. Oben hatten wir eineZahl als int deklariert. Und da 42 eine ganze Zahl und innerhalb des Wertebereichs von int ist, können wir das tun.\\

Ist eine Variable erst einmal deklariert, können Sie ihr so oft neue Werte zuordnen, wie Sie wollen. Einsteigern fällt der Umgang mit Variablen deshalb manchmal etwas schwer, aber machen Sie sich da keine Gedanken, das wird Ihnen innerhalb kurzer Zeit in Fleisch und Blut übergehen; Sie müssen nur das tun, was fürs Lernen einer Programmiersprache das wichtigste ist: Programmieren, Programmieren und nochmal Programmieren.\\

Sie können bei C und vielen anderen Programmiersprachen auch eine Variable deklarieren und initialisieren. Das sähe dann so aus:\\

int nochEineZahl = 937;\\

In den Fällen, in denen Sie darauf angewiesen sind, mit Fließkommazahlen zu arbeiten, müssen Sie beachten, dass hier die anglo-amerikanische Notation für die Dezimaltrennung gilt. Umgangssprachlich ausgedrückt: Wo Sie bei Zahlen ein Komma verwenden, müssen Sie einen Punkt setzen! Hier das entsprechende Beispiel:\\

pi = 3.1415;\\

\subsubsection{Der Scope und das Ende von Variablen}

Wie Sie jetzt wissen, erzeugen Sie Variablen entweder durch eine Operation oder durch eine Deklaration. Und irgendwo müssen all die Variablen im Rechner verwaltet werden, damit sie nicht einfach überschrieben werden. Das ist Ihnen wahrscheinlich gar nicht klar: Sie wissen, dass Sie Variablen erzeugen, ihnen Werte zuordnen und diese ändern können. Aber über die Verwaltung des Speichers haben wir gar nicht gesprochen, denn das macht ja „die Programmiersprache“.\\

Folgendes Beispiel soll Ihnen das Problem vor Augen führen, das Ihnen bislang entgangen ist: Stellen Sie sich vor, Sie würden ein Programm verfassen. Alle Variablen, die dieses Programm erzeugt würden im Speicher erhalten bleiben. Zusätzlich arbeiten ja noch viele andere kleine und große Programme (als Teil des Betriebssystems) im Rechner, die auch massenhaft Variablen erzeugen. Und nehmen wir an, all diese Variablen würden ebenfalls im Speicher verbleiben. Was würde dann wohl innerhalb kürzester Zeit passieren?\\

…\\

Genau! Der Speicher wäre irgendwann voll und der Rechner würde gar nichts mehr tun oder damit beginnen, bereits durch alte Variablen belegten Speicher mit neuen Werten von anderen Variablen zu füllen. Kurz gesagt: Chaos.\\

Ein Teil der Lösung, um solches Chaos zu vermeiden nennt sich Garbage Collector, aber damit haben Sie bei der Programmierung nichts zu tun, denn es ist ein Teil der Laufzeitumgebung, der kontinuierlich prüft, welche Speicherbereiche von keinem Teil des Programms mehr genutzt werden. Diese Speicherbereiche gibt er dann wieder frei, sodass sie für andere Aufgaben verwendet werden können. Ein anderer Teil wird als Scope oder Rahmen bezeichnet und gehört zu den Dingen, die Sie fest einprogrammieren. Der Scope ist jeweils ein Bereich, in dem Bezeichner von Variablen gültig sind. Das selbe gilt für die Bezeichner von anderen Dingen wie Funktionen, über die wir noch nicht gesprochen haben.\\

Bei C wird der Scope durch geschweifte Klammern abgegrenzt.\\

Beispiel:\\

Innerhalb eines C Programms sehen Sie folgende Zeilen:\\

\begin{verbatim}
int main(void)
{
	int zahl1 = 1;
	int zahl2 = 2;
	return zahl1 + zahl2;
}

float addiere(float a, float b)
{
	float zahl1 = a;
	float zahl2 = b;
	return zahl1 + zahl2;
}
\end{verbatim}

Sie wissen jetzt zwar nicht, was Sie mit der Zeile float addiere(float a, float b) anfangen sollen, aber das spielt hier keine Rolle. Denn hier reden wir nur über den Scope und für den spielt das keine Rolle.\\

Da Sie jetzt wissen, was der Scope ist, wissen Sie auch, dass in diesem Programmfragment zweimal eine Variable vom Typ int deklariert wird, die mit dem Bezeichner zahl1 angesprochen wird. Sie wissen jetzt auch, dass beide Variablen nichts miteinander gemein haben, weil Sie jeweils in unterschiedlichen Scopes enthalten sind.\\

Nun lassen sich geschweifte Klammern jeweils beliebig verschachteln und wie immer gilt, dass bei jeder Programmiersprache individuell geregelt ist, ob eine Variable in den inneren Rümpfen eines Scope verwendet werden kann oder nicht.\\

Es ist ein guter Programmierstil jeweils am Anfang eines Rumpfes die Deklaration aller (nicht-anonynmen) Variablen durchzuführen, die innerhalb dieses Rumpfes verwendet werden, da man so doppelte Verwendungen ausschließt. \\

Kontrolle\\

So wie Sie bei der maschinennahen Programmierung mit Zahlen arbeiten, die in sogenannten Registern verarbeitet werden, arbeiten Sie bei der imperativen Programmierung mit sogenannten Variablen, die Sie deklarieren und initialisieren müssen.\\

Sie wissen, dass es im Gegensatz zu Speicherbereichen bei der maschinennahen Programmierung Standardwerte je nach Datentyp gibt, mit denen deklarierte Variablen automatisch initialisiert werden, und dass Sie diese Standardwerte kennen sollten, wenn Sie längere Zeit mit einer Sprache programmieren.\\

Innerhalb eines Scope darf bei C ein Bezeichner vor der Deklaration nicht auftauchen. Warum?\\

Übung\\

Damit Sie jetzt ein wenig Programmierpraxis bekommen, hier eine kleine Übung:
Starten Sie eine IDE oder einen Editor Ihrer Wahl, um ein C-Programm zu entwickeln.\\

(1) Nehmen Sie nun den Quelltext vom Anfang dieses Unterkapitels. Unterhalb dieses Absatzes finden Sie nun eine Fassung, in der eine Variable deklariert, initialisiert und ausgegeben wird.\\

Lassen Sie sich hier bitte nicht davon irritieren, dass nach dem Rumpf kein Semikolon auftaucht: Der Rumpf/Scope legt fest, in welchem Bereich Bezeichner definiert sind. Das Semikolon dagegen legt fest, wo ein Befehl endet.\\

Dann wären da noch die Einrückungen anzusprechen. Diese dienen in C der Lesbarkeit: Eine allgemein übliche Konvention lautet, dass die Zeilen innerhalb eines Rumpfes um drei Leerstellen eingerückt werden. Gerade wenn Sie mit gestaffelten Rümpfen (also Rümpfe in Rümpfen in Rümpfen in …) arbeiten, werden Sie das bei Änderungen zu schätzen wissen.\\

Beachten Sie bitte, dass es Programmiersprachen gibt, bei denen ein Scope ausschließlich durch solche Einrückungen festgelegt wird.\\

Der folgende Quellcode ist auch für IDEs geeignet, da er einige Ergänzungen enthält, die Sie jedoch erst dann verstehen können, wenn Sie den Abschnitt über Funktionen gelesen haben. Deshalb empfehle ich Ihnen, weiterhin mit einem Editor und dem Quellcode zu arbeiten, den Sie am Beginn des Kapitels finden können. \\

\begin{verbatim}
#include <stdio.h>
int main(void) {
	int eineZahl = 42;
	printf("Der Wert der Variablen ist: %d" , eineZahl);
	return 1;
}
\end{verbatim}

Das folgende gilt vorrangig für diejenigen von Ihnen, die eine IDE nutzen:\\

Auch wenn Sie das inzwischen wissen sollten, hier nochmal der Ablauf zum funktionierenden Programm: Nachdem Sie die Datei bzw. das Projekt gespeichert haben, müssen Sie es kompilieren und linken (dafür gibt es in IDEs eine Schaltfläche oder einen entsprechenden Menüeintrag). Zusammengefasst werden kompilieren und linken auch als build bezeichnet. Wenn Sie alles richtig abgetippt haben und die IDE richtig konfiguriert ist, erscheinen einige Zeilen in einem Fenster der IDE, die im Grunde nur besagen, dass kompilieren und linken erfolgreich verlaufen sind. Anschließend starten Sie das Programm.\\

(2) Auch wenn es nicht allzu spektakulär ist, folgt eine erweiterte Fassung des Quellcodes. \\

\begin{verbatim}
#include <stdio.h>
int main(void) {
	int eineZahl = 42;
	printf("Der Wert der Variablen ist: %d\n" , eineZahl);
	char buchstab = ´a´;
	printf("Der Buchstabe lautet: %c" , buchstabe);
	return 1;
}
\end{verbatim}

Sehen Sie sich an, welche Unterschiede es bei der Ausgabe gegenüber dem ersten Quelltext gibt und stellen Sie Vermutungen an, was diese Änderungen bewirken. Versuchen Sie dann basierend auf Ihren Annahmen das Programm erneut zu erweitern, indem Sie eine Variable vom Typ float mit dem Wert 15.293 hinzufügen und diese in einer weiteren Zeile ausgeben lassen.\\

Wenn Sie alles richtig gemacht haben, werden die folgenden Zeilen ausgegeben:\\

Der Wert der Variablen ist: 42
Der Buchstabe lautet: a
Die Fließkommazahl ist: 15.293000\\

(3) Versuchen Sie jetzt durch eine Internetrecherche herauszufinden, was Sie ändern müssen, damit in der letzten Zeile nicht 15.293000 sondern einfach nur 15.293 steht.\\

(4 - Schwierig) Und jetzt ändern sie das Programm so ab, dass in der letzten Zeile 15,293 steht. Wichtig: Die Aufgabe ist nur dann richtig gelöst, wenn Sie anschließend den Wert der Variablen von 15.293 auf beliebige andere Fließkommazahl ändern können und das Programm diese anderen Zahlen mit dem Komma anstelle des Punktes ausgibt. Für die Lösung ist es akzeptabel, wenn das Programm eine feste Anzahl an Stellen nach dem Komma ausgibt. Die Nachkommastellen brauchen hier also nicht vollständig oder richtig gerundet sein.\\

Wichtiger Hinweis\\

Wenn Sie mit dem Quellcode da oben ein wenig herumspielen, werden Sie feststellen, dass zwar viele Stellen genauso eingetippt werden müssen, wie das hier zu lesen ist. Aber andere Dinge können Sie relativ beliebig ändern und trotzdem gibt das Programm dieselbe Ausgabe aus.\\

Wenn Sie beispielsweise Ihr Programm so ändern, dass Sie alle Deklarationen an den Anfang des Rumpfes setzen, dann wird sich am Ablauf nichts ändern. Erinnern Sie sich noch, warum Sie das tun sollten?\\

\section{Operationen und Funktionen}

Bislang haben Sie des Öfteren davon gelesen, dass hier von Befehlen die Rede war. Wenn Sie mit Programmierern reden, dann werden Sie diese Bezeichnung dort eher selten hören. Vielmehr gibt es hier drei Begriffe, die Sie als Synonym für einen Befehl verstehen könnten, die Sie aber klar unterschieden müssen. In der imperativen Programmierung kommen zwei davon zum Einsatz: Operationen und Funktionen. Bei der objektorientierten Programmierung gibt es dann noch die sogenannten Methoden, die sich aber in der Programmierung nur durch ein Detail von Funktionen unterscheiden, das so trivial ist, dass es Ihnen wahrscheinlich nicht einmal auffallen wird, wenn es Ihnen niemand erklärt. Aber bleiben wir vorerst bei der imperativen Programmierung.\\

Operationen kennen Sie aus der Mathematik: Dort haben Sie die sogenannten Algebren kennen gelernt. Da aber die meisten Studienanfänger diese Lektion bereits wieder vergessen haben, folgt hier eine kleine Einführung… Alpers berüchtigte Algebra BROT: (In der Mathematik wird der Begriff einer Algebra anders verstanden; das hier ist ein Beispiel dafür, wie InformatikerInnen sich ein gutes Konzept der Mathematik ausleihen, um es für Ihre Zwecke zu nutzen... Also das, was alle Wissenschaften der Mathematik antun.)

\subsection{Die Algebra BROT}

Sie werden sich erinnern: In den einleitenden Kapiteln haben Sie etwas Verwirrendes gelesen: Demnach hat Mathematik eigentlich nur am Rande etwas mit Zahlen zu tun und vielmehr ginge es darum, neue Welten und Universen zu beschreiben. In den nächsten Absätzen werde ich Ihnen ein Beispiel dafür geben.\\

Zunächst beginnen einige InformatikerInnen seit einigen Jahrzehnten eine Argumentation damit, dass sie zunächst eine Menge definieren. Eine solche Menge definieren sie über die Eigenschaften, die alle Objekte haben, die Teil dieser Menge sind. Es geht dabei aber nicht darum, dass die Ausprägung der Eigenschaften beschrieben wird, sondern dass definiert wird, um welche Eigenschaften es sich handelt. Definitionen der verschiedenen Mengen von Zahlen haben Sie schon so oft gesehen, dass eine weitere Wiederholung unsinnig wäre. Also nehmen wir die Menge BROT. Ja, richtig gelesen, hier gibt’s keine Zahlen, hier gibt’s nur Brote. Und ja, das ist Mathematik oder zumindest das, was wir als InformatikerInnen von der Mathematik übrig lassen.\\

Trommelwirbel… und ein kurze Pause, damit Sie Ihrem Gehirn die Gelegenheit zur Beruhigung geben können.\\

Unsere Menge heißt also BROT. Und was wird darin sein? Natürlich jede Menge Brote. Und wie definieren wir so etwas? Genau, wir notieren, welche Eigenschaften ein Brot haben kann.\\

Boshafte Naturen hätten diese Menge so definiert, dass darin alle möglichen Dinge gesammelt werden würden, die nicht das Geringste mit Brot zu tun haben, denn verboten ist das nicht. Im realen Leben hören Sie ja auch pausenlos Bezeichnungen, die nicht dem naiven Eindruck entsprechen:\\

•	Nehmen wir eine Bezeichnung wie Gleichstellungsbeauftragte. Hier würden Sie naiv an einen Menschen (unabhängig vom Geschlecht) denken, dessen Aufgabe darin besteht Benachteiligungen auszuhebeln, die aufgrund des Geschlechts, der religiösen Zugehörigkeit oder anderer Aspekte bestehen, die aber nicht in der Leistung bzw. Leistungsfähigkeit der/des Einzelnen bestehen. Tatsächlich verbirgt sich hinter dieser Bezeichnung gelegentlich eine Frau, die ausschließlich die Belange von Frauen für Frauen vertritt. Dabei gehört es zur Allgemeinbildung, dass beispielsweise im erzieherischen Bereiche Männer aufgrund sexistischer Klischees häufig benachteiligt oder diffamiert werden.\\

•	Oder denken Sie an die sogenannten sozialen Netzwerke. Ein soziales Netzwerks besteht aus Menschen, die miteinander in Kotakt stehen. Webplattformen dagegen, die als soziale Netzwerke bezeichnet werden sind in aller Regel Sammelwerke, deren einziger Zweck darin besteht, ein allumfassendes Profil über die hier registrierten Nutzer zu erstellen. Dabei bedeutet allumfassend, dass selbst der Psychotherapeut eines Nutzers einer solchen Webplattform nach jahrelanger Therapie keinen derart tief gehenden Einblick in die Persönlichkeit und Lebensumstände des/der Betreffenden gewonnen haben kann.\\

Aber wieder zurück zum Thema.\\

Aufgabe:\\

Sammeln Sie einige Eigenschaften die bei verschiedenen Arten von Brot vorkommen. Wenn Ihnen nicht klar ist, was damit gemeint ist, dann denken Sie an eine Menge von Autos. Hier wäre Farbe eine mögliche Eigenschaft, denn abgesehen von einigen Prototypen hat im Regelfall jedes Auto wenigstens eine Farbe. Also nochmal: Was für Eigenschaften hat Brot? (Machen Sie sich hier keine Gedanken, wenn Sie eine Eigenschaft nicht in einem Wort zusammenfassen können, formulieren Sie die Eigenschaften ruhig zunächst als Sätze.)\\

Hier sind wir auch beim ersten Beispiel, was den Informatiker vom Programmierer unterscheidet: Der Programmierer geht jede Aufgabe konkret an und entwickelt für das konkrete Problem eine konkrete Lösung. Im Gegensatz dazu betrachtet ein Informatiker zunächst, wie das Problem auf abstrakter Ebene aussieht und nutzt dann ähnlich abstrakte Lösungsansätze, um einen Lösungsweg zu entwickeln, den er dann z.B. in Form eines Computerprogramms konkretisiert. Dieser Abstraktionsschritt wird auch als Modellieren bezeichnet, eine Methodik, die alle Naturwissenschaftler nutzen.\\

Wenn wir uns jetzt unsere Menge BROT ansehen, so lassen sich leicht einige Eigenschaften finden. Ein Brot kann saftig oder krümelig sein. Es hat eine Farbe, die häufig zwischen weiß und dunkelbraun schwankt, aber je nach Zusatzstoffen oder Zustand auch rötlich oder grünlich sein kann. Es hat eine Form irgendwo zwischen Fladen und Kasten. Es kann aus reinem Mehl bestehen, aus Mehl und verschiedenen Körnern, usw. Wenn Sie weiter überlegen, dann werden Ihnen hier noch viele Eigenschaften einfallen.\\

Sie könnten nun Variablen deklarieren, die jeweils einer Eigenschaft unserer Menge BROT entsprechen. Hier haben Sie auch schon ein Beispiel dafür, wie Mathematik und Programmieren sich in der Informatik treffen. Richtig spannend wird dieser Aspekt aber erst in der objektorientierten Programmierung.\\

Von der Menge zur Algebra\\

Das folgende ist keine präzise mathematische Einführung. Wenn Sie hier eine genaue Erklärung haben möchten, dann sprechen Sie bitte einen Mathematiker an.\\

Den Begriff Algebra haben Sie schon in der Schulzeit immer wieder gehört, aber wahrscheinlich nie eine Erklärung dazu gehört, also eine Antwort auf die Frage, was eine universelle Algebra oder eine algebraische Struktur ist. Eine algebraische Struktur besteht zum einen aus einer Menge (damit ist das gemeint, was wir soeben als Menge erarbeitet haben) und zum anderen aus einer beliebigen Anzahl Operationen. Das was eine algebraische Struktur dabei auszeichnet ist, dass jede dieser Operationen mit jedem Element der Menge der algebraischen Struktur nutzbar ist. Ein einfaches Beispiel für die Operation wäre die Addition und ein einfaches Beispiel für die Menge wären die natürlichen Zahlen: Egal auf welche zwei natürlichen Zahlen Sie eine Addition anwenden, das Ergebnis ist ebenfalls eine natürliche Zahl.\\

Operationen sind umgangssprachlich Tätigkeiten, es geht also bei der Definition einer algebraischen Struktur darum, eine Ansammlung von Objekten abzugrenzen, mit der bestimmte gleichartige Dinge getan werden können. Die Mathematik besteht dann darin, zu prüfen, welche weitergehenden Möglichkeiten sich aus den definierten Eigenschaften und Operationen ergeben und ob wir dabei spannende und elegante Abkürzungen finden können.\\

Das allerdings ist bei der Programmierung zunächst nicht relevant. Also zurück zu den Operationen. Lassen Sie uns nun zur Algebra BROT kommen, nachdem wir gerade die Menge BROT definiert haben. Richtig: Wir benutzen hier einen Bezeichner, namentlich BROT, um zwei unterschiedliche Dinge zu bezeichnen. Dürfen wir das? Klar, in der Mathematik dürfen wir das schon, in der Programmierung dagegen nicht. Der Grund ist einfach: Die Mathematik setzt den gesunden Menschenverstand voraus, also die Fähigkeit, beispielsweise zu erkennen, in welchem Umfeld ein Problem angesiedelt ist und daraus auf eine sinnvolle Lösung zu schließen. Die Programmierung dagegen… Nun Sie können diesen Satz in eigenen Worten beenden und können damit auch gleich auf die Antwort zur Frage schlussfolgern, warum ProgrammiererInnen nicht das gleiche wie InformatikerInnen sind.\\

Um also von der Menge BROT zur algebraischen Struktur BROT zu kommen, müssen wir jetzt also einige Tätigkeiten oder Anwendungen benennen, die mit einem Element der Menge Brot durchgeführt werden kann. Dabei spielt es keine Rolle, ob und wenn ja welchen Sinn die Anwendung dieser Operation macht. Es geht nur darum, ob sie mit jedem Element möglich ist.\\

Und hier nehmen wir den klassischen Wortwitz: Kann man Brot einfrieren? Na sicher kann man, Mann, Frau, Zombie, Echse, Spock oder wer auch immer das tun. Also haben wir unsere erste Operation: Einfrieren. Wobei so etwas eine 1-stellige Operation ist, was recht langweilig ist. (Die Stelligkeit einer Operation besagt etwas darüber, wie viele Elemente wir benötigen, um die Operation durchzuführen.)\\

Wie gesagt: Es ist nicht wichtig, ob diese Operation Sinn macht; wichtig ist ausschließlich, dass sie auf jedem Element der Menge (hier der Menge BROT) definiert ist. Die Aussage „Die Operation X ist definiert für Y.“ bedeutet hier nichts anderes, als dass es möglich ist, die Operation X mit dem Element Y durchzuführen. Wenn Y eine Menge ist, dann bedeutet diese Aussage, dass die Operation X ausnahmslos auf jedem Element der Menge Y ausgeführt werden kann. Sonst wäre nämlich unsere Kombination aus Menge und Operationen keine algebraische Struktur mehr.\\

Wenn Sie wie bei der Definition der Menge BROT jetzt die auf dieser Menge definierten Operationen notieren und beides gemeinsam aufschreiben, dann haben Sie schon die Algebra BROT definiert. Das folgende wäre also eine mögliche Definition der Algebra BROT, wobei hier P die Struktur bzw. das Modell ist und F die Operationen zusammenfassen:\\

BROT: \\

P = (Knusprigkeit | Farbe | Temperatur | Form)
F = (einfrieren | schneiden | bestreichen | essen | backen | stapeln)\\

Anmerkung: Formal werden Algebren ganz anders notiert, aber dazu sprechen Sie bitte den Mathematikdozenten Ihres Vertrauens an.\\

Von der algebraischen Struktur zur programmierten Operation\\

Sie fragen sich jetzt vielleicht, wie die Operationen ausgeführt werden können, denn darüber haben wir hier noch nicht gesprochen. Aber weil dies keine Einführung in die Mathematik ist, werden wir darauf auch nicht weiter eingehen. Vielmehr sollte dieser Teil dazu dienen, dass Sie eine Vorstellung davon bekommen, warum die Mathematik (bei diesem Beispiel speziell die Algebra) eine so wichtige Rolle in der Informatik übernimmt. Aber wir kommen jetzt wieder zur Programmierung zurück, wo wir uns ansehen werden, was Operationen bei einem Programm sind.\\

Operationen sind hier so etwas wie Aktivitäten, die für jeden Datentypen einzeln definiert sind. Um das mathematische Modell von eben zu nehmen: Stellen Sie sich jeden Datentyp als eine algebraische Struktur vor. Dann sind Sie auch direkt beim Verständnis davon, was nun eine Operation ist: Es ist das, was Sie mit jeder Variablen dieses Datentyps tun können. Operationen werden dabei gewissermaßen mit der Programmiersprache ausgeliefert, Sie können Sie also nicht einprogrammieren, sondern nutzen Sie als einfachste Befehlsformen.\\

Kontrolle\\

Datentypen und Operationen verhalten sich zueinander wie Mengen und Operationen in der Mathematik. Operationen stellen einfachste Befehle dar, mittels derer Sie Variablen ändern können.\\

Genau wie Datentypen sind Operationen in C statisch.\\

Aufgabe:\\

Überlegen Sie sich, was es bedeuten würde, wenn Operationen dynamisch wären. Vergessen Sie an dieser Stelle bitte nicht, dass es hier nicht um die Frage geht, ob das Sinn macht, sodern nur um die Frage, welche Folgen das hätte. Und da gibt es vieles, was Ihnen einfallen könnte.\\

\subsubsection{Mathematische Operationen}

Mathematische Operationen sind für ganzzahlige Datentypen und für Datentypen mit Fließkommazahlen definiert. Wichtig ist aber, dass Sie bei jeder Programmiersprache lernen, wie sich die Operationen genau verhalten. Bevor wir dazu kommen, hier die Übersicht: (a, b sind beliebige Variablen)\\

•	Addition: 	a + b
•	Subtraktion:	a – b
•	Multiplikation:	a * b
•	Division:	a / b
•	Modulo-Rest:	a \% b\\

Wichtig: Eine Operation wie a – b reduziert nicht (!) a um den Wert von b. Deshalb müssen Sie das Ergebnis einer Operation in aller Regel einer Variablen zuordnen. \\

Wiederholung: Wenn Sie tatsächlich die Variable a um den Wert der Variablen b reduzieren wollen, dann können Sie das tun, indem Sie das Ergebnis der Operation a – b der Variablen a zuordnen. Im Gegensatz zu dem, was Sie im Mathematikunterricht gelernt haben, macht also die folgende Programmzeile tatsächlich Sinn. (Wenn Sie das nicht verstehen, arbeiten Sie bitte nochmal den Abschnitt zu anonymen Variablen durch.)\\

a = a – b\\

Aufgabe: Begründen Sie mit Ihrem bisherigen Wissen aus diesem Kurs, warum die Operation a – b den Wert von a nicht reduziert.\\

Wichtig: Beachten Sie jedoch, dass jede Operation bei statisch typisierten Sprachen im Regelfall nur dann definiert ist, wenn beide Elemente vom gleichen Datentyp sind. Es kann Ihnen also passieren, dass Sie eine Operation, die für zwei unterschiedliche Datentypen definiert ist, dennoch nur mit Variablen des selben Datentyps durchführen können. Das ist für jede Programmiersprache individuell festgelegt; Sie können es nicht durch logische Schlussfolgerung feststellen, sondern müssen die Dokumentation der jeweiligen Sprache zu Rate ziehen.\\

Aufgabe 1: Prüfen Sie das nach, indem Sie die je zwei ganzzahlige und zwei Fließkomma-Variablen deklarieren und initialisieren. Dann programmieren Sie alle mathematischen Operationen, wenn beide Operanden vom gleichen Typ sind. Dann programmieren Sie sie, wenn der erste Operand ganzzahlig und der zweite Operand ein Fließkommawert ist. Abschließend umgekehrt. Sie müssten also zwanzig Zeilen Code mit der Berechnung und Ausgabe eines Ergebnisses programmieren. Sie brauchen das Ergebnis der Operationen keiner neuen Variablen zuzuordnen, sondern können die Operation anstelle einer Variablen in die Klammern von printf() eintragen.\\

(gehört zur Aufgabe) ACHTUNG: Dieser Code wird eine Vielzahl von Fehlermeldungen erzeugen und die Hauptaufgabe für Sie besteht darin, diese Fehlermeldungen zu lesen und sich zu überlegen, was zu dem Fehler geführt hat. Das können Sie aber nur dann, wenn Sie klar zwischen den einzelnen Datentypen abgrenzen und die Fehlermeldungen konzentriert lesen.\\

Und ja, so etwas kann Teil einer Prüfungsaufgabe sein.\\

Aufgabe 2: Überlegen Sie sich bei jedem Fehlerfall, was Sie im Quellcode ändern müssten, damit Sie eine Lösung bekommen, die dem entspricht, was Sie erwarten würden.\\

Aufgabe 3: Rechnen Sie die Ergebnisse nach (auch und gerade bei den Operationen, die keine Fehlermeldung erzeugt haben) und überlegen Sie sich bei den entsprechenden Fällen, warum das Ergebnis nicht mit dem übereinstimmt, was Sie jeweils erwarten würden. (Wenn Sie jeweils das Ergebnis erwartet haben, das der Rechner ausgegeben hat, dann notieren Sie, warum es nicht mit dem Ergebnis übereinstimmt, das Sie erhalten hätten, wenn Sie die arithmetischen Regeln aus dem Mathematikunterricht angewendet hätten.)

\section{Funktionen}

Den Begriff einer Funktion kennen Sie aus dem Mathematikunterricht. Naiv könnte man ihn so beschreiben: Eine Funktion wäre dann eine Sammlung von Operationen, die aus einem oder mehreren Elementen einer Menge ein Element einer anderen oder der selben Menge erzeugt. So können Sie beispielsweise eine Funktion auf einen Vektor anwenden, der die Länge des Vektors berechnet. Der Vektor ist Element einer Menge, die Länge des Vektors Teil einer anderen Menge.\\

Eine andere Erklärung einer Funktion ist die Abbildung eines Wertes einer Menge auf ein Element einer (ggf. anderen) Menge.\\

Die Mathematik beschäftigt sich nun weiter mit den Eigenschaften von Funktionen, Ähnlichkeiten und Unterschieden zwischen Gruppen von Funktionen und was sich daraus an weitergehenden Aussagen über diese Gruppen von Funktionen ableiten lässt. Also wenn Sie mich fragen, ist das wirklich spannend, gerade wenn man es auf komplexe reale Probleme anwendet, aber Sie wollen ja nur lernen wie man programmiert. Also schauen wir wieder auf den Funktionsbegriff in der Programmierung.\\

Und dort ist eine Funktion wieder etwas, das in aller Regel einen Bezeichner trägt (wobei auch sogenannte anonyme Funktionen möglich sind, aber die interessieren uns für den Moment nicht weiter) und dem Sie eine beliebige Anzahl an Variablen übergeben können. Wie viele das sind, wird bei der Programmierung einer Funktion festgelegt. Sie werden im Rahmen einer Funktion als Argumente bezeichnet, um sie sprachlich eindeutig von den Variablen eines Programms unterscheiden zu können. Denn die Funktion verarbeitet Kopien der Werte der Variablen (ohne den Wert der Variablen selbst zu ändern) und gibt einen eigenständigen Wert zurück. Und diese Kopien sind eigenständige Variablen, die als Argumente bezeichnet werden.\\

In C gilt, dass Funktionen statisch sind.\\

Wichtig: Auch wenn der Rumpf einer Variablen ein eigenständiger Scope ist, sollten Sie Variablenbezeichnungen, die Sie außerhalb einer Funktion eingeführt haben nicht innerhalb der Funktion verwenden, selbst wenn Sie mit einer Sprache arbeiten, die Ihnen das ermöglicht. Die Lesbarkeit und damit die Möglichkeit Ihr Programm später zu verbessern wird dadurch deutlich verschlechtert.\\

Es gibt noch Sonderformen von Funktionen: Funktionen benötigen nicht unbedingt Argumente, um ihre Aufgabe zu erfüllen. Und es ist möglich Funktionen zu programmieren, die keinen Rückgabewert erzeugen. Für die drei Varianten gibt es keine eigenständigen Bezeichnungen, wir sprechen unabhängig von der Variante grundsätzlich nur von einer „Funktion“. Und schließlich ist es noch möglich, Funktionen zu programmieren, die weder Argumente noch einen Rückgabewert haben. Am einfachsten ist das nachvollziehbar, wenn Sie an die main() denken. Aber andere Fälle sind ebenfalls denkbar. Im Alltag werden Sie hier beim Programmieren auch gar nicht differenzieren und wie alle Programmierer immer „nur“ von einer Funktion reden.\\

Sehen wir uns die drei ersten Fälle im Detail anhand von Beispielen an:
Variante a) Sie möchten eine Funktion haben, die Ihnen den größten gemeinsamen Teiler zweier Zahlen ausgibt. Ohne auf die Programmierung der Funktion einzugehen, können wir also sagen, dass wir eine Funktion haben, die zwei Argumente benötigt und einen Rückgabewert hat. Alle müssen vom Datentyp her ganzzahlig sein. In allgemeiner Form könnten Sie das dann so deklarieren:

\begin{verbatim}
int ggT (int argumente1, int argument2){ … }
\end{verbatim}

Diese Form ist standardisiert, Sie haben hier also nur bei der Bezeichnung der Argumente, bei der Wahl des jeweiligen Datentyps und beim Namen der Funktion gewisse Freiheiten. Um hier Missverständnisse zu vermeiden sei gesagt, dass in diesem Beispiel die Funktion nur deshalb genau zwei Argumente erhält, weil sie ja den größten gemeinsamen Teiler zweier Zahlen berechnen soll. In anderen Worten: Die Anzahl Argumente einer Funktion hängt davon ab, wie viele Argumente Sie verwenden wollen. Es gibt weder Vorschriften noch logische Gründe, die eine bestimmte Anzahl Argumente in irgend einer Form festlegen.\\

Die Deklaration da oben ist wie folgt zu lesen:\\

•	Das erste int gibt den Datentyp des Rückgabewerts an. Wir können hier problemlos int nehmen, wenn dieser Datentyp für die Argumente ausreicht, denn der ggT zweier Zahlen kann ja nie größer sein, als die Zahlen selbst. In Sprachen wie C kann eine Funktion nur genau einen Rückgabewert haben. Allerdings sind Sie hier nicht auf Variablen beschränkt, sondern können durchaus Datenstrukturen zurückgeben lassen. (Keine Sorge, Datenstrukturen haben wir noch nicht behandelt, aber sie folgen in Kürze.)
•	Dann folgt die Bezeichnung  bzw. der Name der Funktion. Hier haben wir die Abkürzung ggT für größter gemeinsamer Teiler gewählt. Aber die Bezeichnung steht uns frei. Wir hätten diese Funktion auch funktionAlpha927 nennen können; für die Programmiersprache macht das keinen Unterschied.
•	Nach dem Funktionsnamen folgt stets ein Klammernpaar, in dem die zu übergebenden Argumente jeweils mit dem benötigten Datentyp stehen. Wichtig: Diese Bezeichnungen sollten im Programm noch nicht verwendet worden sein. Denn sonst kann es passieren, dass die Funktion falsche Ergebnisse liefert.
•	Abschließend folgt der Rumpf der Funktion. In diesem stehen mehrere Programmzeilen, von denen die letzte mit dem „Befehl“ return beginnt. Hinter dem return muss eine Variable stehen, deren Datentyp derselbe ist, wie derjenige, den Sie für den Rückgabewert der Funktion definiert haben. Wie gewohnt gilt hier: Ob die Funktion das tut, was Sie wollen, hängt davon ab, ob Sie sie so programmiert haben oder nicht. Wenn Sie also den Funktionsrumpf so gefüllt haben, dass die Funktion etwas ganz anderes zurückgibt, dann beklagen Sie sich nicht, sondern konzentrieren Sie sich, um Ihre Denkfehler zu finden. Und hier auch ein ganz wichtiger Praxistipp: Wenn Sie schon seit Stunden programmieren, bewirkt eine Pause oft wahre Wunder. Oft hilft auch der Blick eines Nachbarn weiter.\\

Variante b) Die Funktion benötigt keinen Eingabewert. Hier ist das Klammernpaar hinter dem Funktionsnamen schlicht leer. Der Rest ist wie gehabt. Sie fragen sich, was solche eine Funktion soll? Nehmen wir an, Sie möchten eine Funktion haben, die Ihnen einen bestimmten Satz an Informationen über den Status Ihres Programms gibt. Dann brauchen Sie der Funktion beim Aufruf nicht mitzuteilen, welche Speicherstellen oder Variableninhalte Sie erhalten wollen, da es immer dieselben sind.\\

Variante c) Die Funktion hat keinen Rückgabewert. Dann dürfen Sie jedoch nicht einfach mit dem Namen der Funktion beginnen, sondern Sie benutzen den „Datentyp“ void. Eine solche Funktion haben Sie schon kennen gelernt: printf() benötigt zwar immer ein Argument, aber da dieses Argument dann auf dem Bildschirm ausgegeben werden soll und Sie keine Variablen ändern wollen, macht hier ein Rückgabewert nur selten Sinn.\\

Kontrolle\\

Im Quellcode, mit dem Sie bislang gearbeitet haben, kommen zwei Funktionen vor. Wie heißen diese und was tun Sie?\\

Ansonsten sollten Sie wissen, wie eine Funktion deklariert wird, was der Rumpf einer Funktion ist und welche Bedeutung das Schlüsselwort return hat. (Überlegen Sie beispielsweise, was mit Programmcode passiert, der innerhalb eines Funktionsrumpfes in der Zeile nach einem return steht. Und welche Bedingung muss der Rest der Zeile erfüllen, die mit return beginnt?)\\

Wenn etwas ein Schlüsselwort ist, dann bedeutet das schlicht, dass dieser Bezeichner eine feste Bedeutung in der einer Programmiersprache hat. Sie dürfen Schlüsselwörter also nur für genau die Aufgabe verwenden, die in der jeweiligen Sprache vorgesehen ist. Das Schlüsselwort return ist hier nur ein Beispiel. In aller Regel gibt es für jede Sprache eine Übersicht der Schlüsselwörter und meist ist diese Liste auch nicht allzu lang.\\

Aufgabe\\

Die folgende Aufgabe dient wieder vorrangig dazu, dass Sie lernen, mit den Fehlerausgaben des Compilers zurecht zu kommen. Denn Sie werden beim Programmieren sehr oft kleine Fehler machen, die erst beim Kompilieren erkennbar werden. Dann ist es wichtig, dass Sie mit den Fehlermeldungen des Compilers etwas anfangen können. Also über wir genau das. \\

Hinweis: In der Aufgabenstellung ist ein simpler Fehler eingebaut, der Sie etwas verwirren könnte, wenn Sie die Erklärungen zu statisch und dynamisch typisierten Sprachen aufmerksam gelesen haben. Programmieren Sie die Aufgabe dennoch bitte in der geschilderten Reihenfolge und führen Sie dann die folgenden Punkte aus:\\

•	Nachdem Sie das Programm fertig gestellt und kompiliert haben, kopieren Sie sämtliche Fehlermeldungen in ein Textdokument.
•	Beschreiben Sie in eigenen Worten, was die erste Fehlermeldung bedeuten könnte.
•	Schreiben Sie dann das auf, was Sie am Quellcode ändern wollen, um diese Fehlermeldung zu bereinigen.
•	Machen Sie so lange mit dieser Übung weiter, bis der Quellcode erfolgreich kompiliert wird und die gewünschte Ausgabe erfolgt.
Hier das Programm, das sie erstellen und dann korrigieren sollen: 
•	Deklarieren und initialisieren Sie vier Zahlen in einem C-Programm, denen Sie die Buchstaben a bis d zuordnen.
•	Deklarieren Sie dann eine Fließkomma-Variable mit dem Namen ergebnis1, der Sie die Funktion rechne(a, b, c, d) zuordnen.
•	Deklarieren Sie dann eine Fließkomma-Variable mit dem Namen ergebnis2, der Sie die Funktion rechne(b, a, c, d) zuordnen.
•	Programmieren Sie dann eine Ausgabe, in der die Werte von ergebnis1 und ergebnis2 auf den Bildschirm ausgegeben wird. (Sie können hier beliebig zusätzlichen Text einprogrammieren, wenn Sie das wollen.)
•	Erstellen Sie erst danach eine Funktion namens rechne(w, x, y, z), die x von w abzieht, dann y dazu zählt und z wieder abzieht. \\

Kontrolle\\

Wo lag der Fehler im hier beschriebenen Programm? Und warum sollte Sie das verwundern, wenn Sie daran denken, dass es sich bei C um eine kompilierte Sprache handelt?

\section{Wie eine Funktion vom Computer ausgeführt wird}

Die meisten Handbücher sparen sich diesen Teil, dabei ist er immens wichtig, damit Sie bestimmte fortgeschrittene Programmiermethoden verstehen und anwenden können.\\

Sie wissen jetzt, wie Sie eine Funktion programmieren, doch was tut der Rechner eigentlich, wenn er eine Funktion ausführt?\\

Um es etwas anschaulicher zu bekommen, nennen wir die Stelle eines Programms, von der aus eine Funktion aufgerufen wird funktionsaufrufende Stelle. Das ist kein Fachbegriff, aber es wird im Folgenden hilfreich sein, wenn wir hierfür einen definierten Begriff haben.\\

Wie Sie wissen werden die Argumente einer Funktion als eigenständige Variablen im Speicher abgelegt, damit sie entsprechend der Funktion geändert werden können, ohne an den ursprünglichen Variablen etwas zu ändern. Der praktische Nutzen besteht darin, dass Sie die Möglichkeit haben, die ursprünglichen Variablen zu ändern, aber genauso die Freiheit haben, Sie beizubehalten.\\

Nehmen wir nun an, unsere Funktion ist so wie eine Subroutine bei der maschinennahen Programmierung programmiert. Dann wird die funktionsaufrufende Stelle im LR, dem Link Register gespeichert. Anschließend arbeitet die Funktion wie eine Subroutine mit dem übergebenen Argument ihre Programmzeilen ab, speichert den Übergabewert in einer Speicherstelle oder einem freien Register und anschließend wird das Programm nach der funktionsaufrufenden Stelle wieder fortgesetzt. Der Nachteil dieser Methode besteht darin, dass Sie im Grunde immer eine Subroutine vollständig abarbeiten lassen müssen, bevor Sie die nächste aufrufen können. Sie können nicht ohne weiteren Programmieraufwand Subroutinen aus Subroutinen aufrufen und jeweils individuell mit neuen Argumenten arbeiten lassen.\\

Funktionen können aber wesentlich komplexer eingesetzt werden: Sie können aus einer Funktion heraus weitere Funktionen aufrufen und diese Aufrufe beliebig komplex staffeln. Und das auf eine Art und Weise, die im Gegensatz zu dieser Erklärung wie ein Kinderspiel ist. Im Gegensatz zur maschinennahen Programmierung, wo Ihnen lediglich ein Link Register zur Verfügung steht, wird hier bei jedem Funktionsaufruf quasi ein zusätzliches Link Register erzeugt. Außerdem wird jedes Argument, das bei einem solchen Funktionsaufruf übergeben wird eigenständig gespeichert.\\

In der Praxis bedeutet das, dass Sie alle Prozesse, die für sich abgeschlossen sind als eine Funktion programmieren können. Und auch wenn das bei kleinsten Programmen keinen großen Wert hat, ist es bereits bei Programmen mit vierzig oder mehr Programmzeilen eine unschätzbare Hilfe, um Übersichtlichkeit zu schaffen und mehrfachen Code zu vermeiden.\\

Eine ganz besonders mächtige Variante dieses Aufrufs von Funktionen aus anderen Funktionen lernen Sie in Kürze kennen. Die Rede ist von Rekursionen.

\subsection{Dynamische Funktionen und Funktionen als Argumente von Funktionen}

Auch wenn es in C keine dynamischen Funktionen gibt, sollten Sie sich mit diesem Thema auseinander setzen, weil es wichtig ist, dass Sie verstehen, was eine dynamische Funktion ist und wie Sie so etwas programmieren können. Das selbe gilt für Funktionen, die als Argument einer Funktion übergeben werden.\\

Sie wissen bereits, dass ein dynamischer Teil eines Programms während der Laufzeit (also zwischen Anfang und Ende eines Programms) geändert werden kann. Und bei nicht-statischen Variablen finden die meisten das auch einleuchtend. Aber wie sieht es bei dynamischen Funktionen aus? Da stellen sich den meisten die Nackenhaare auf und sie sagen, dass das doch unmöglich sein muss.\\

Sehen wir uns dazu nochmal an, wie eine Funktion programmiert wird: Sie definieren einen Namen, ggf. ein oder mehrere Argumente und den Datentyp eines Rückgabewertes. Und jetzt erinnern Sie sich daran, wie Sie eine Variable definieren: Sie legen einen Namen, sowie den Datentyp fest. Und jetzt fragen Sie sich selbst: Wenn es so leicht ist, den Inhalt einer Variablen zu ändern, warum sollten Sie dann nicht genauso leicht mittels einer Zuordnung den Rumpf einer Funktion austauschen? Bei C lautet die Antwort: Es ist eben so! (Was nicht gerade eine sehr befriedigende Antwort ist.)\\

Kommen wir nun zum zweiten Teil: Warum sollten Sie einer Funktion als Argument eine andere Funktion übergeben wollen? Ganz einfach: Stellen Sie sich vor, Sie haben einen Taschenrechner programmiert, der einige grundlegende Funktionen berechnen kann. Wenn Sie nun im Laufe von Jahren immer mehr Funktionen einprogrammieren, dann wird Ihr Programm immer unübersichtlicher. Wenn Sie dagegen die ganzen zusätzlichen Funktionen in einem externen Programmteil ablegen könnten, ließe es sich deutlich übersichtlicher gestalten, ohne dass der Funktionsumfang reduziert würde.

\section{Kontrollstrukturen}

Bis jetzt können Sie in etwa dieselben Programme entwickeln wie bei der maschinennahen Programmierung, als Sie wussten, wie Sie Register und Speicher verwenden können. Doch während wir dort kaum über diese Möglichkeiten hinausgegangen sind, kommen wir jetzt zu einem der ersten Werkzeuge, die den Vorteil der imperativen gegenüber der maschinennahen Programmierung ausmachen: Mit einer imperativen Programmiersprache können Sie ganz leicht den Ablauf des Programms in Abhängigkeit von Bedingungen steuern.\\

Bei der maschinennahen Programmierung war das einzige leicht nutzbare Mittel hierzu das Branchen. Eine Möglichkeit zum Branchen bei imperativen Programmiersprachen haben Sie schon kennen gelernt: Dort nutzen wir Funktionen für genau diese Aufgabe. Ein zentraler Unterschied besteht allerdings darin, dass Sie beliebig Funktionen aus Funktionen heraus aufrufen können, ohne dazu besondere Maßnahmen treffen zu müssen. \\

Doch wenn wir über Kontrollstrukturen reden, dann meinen wir damit etwas wesentlich mächtigeres als die Ausführung einer Subroutine in Abhängigkeit vom Wert einer Variablen: Wir reden über eine beliebig komplexe Verschachtelung unterschiedlichster Bedingungen, denen jeweils gänzlich andere Programmausführungen folgen. Außerdem erreichen wir über sinnvoll benannte Funktionen, dass unser Code wesentlich besser lesbar ist. Das wiederum ermöglicht es uns, Fehler wesentlich schneller zu finden und zusätzlich, sie mit geringerem Aufwand zu korrigieren.

\subsection{Wenn-Dann-Kontrolle}

Die einfachste Kontrollstruktur (engl. control flow) können Sie einsetzen, wenn Sie den Programmablauf in Form von einfachen wenn-dann (engl. if-then) Bedingungen strukturieren können. Und ja, diese wenn-dann-Bedingungen können Sie mit fast beliebiger Tiefe staffeln. Aber wir beginnen zunächst mit dem einfachen Fall.\\

Stellen Sie sich vor, Sie wollten dazu die Algebra BROT in ein Programm umwandeln: Sie würden dann über Variablen konkrete Eigenschaften eines einzelnen Brotes definieren, weil Sie in C noch keine andere Möglichkeit kennen gelernt haben. (Um ein Brot als ein einzelnes Datenobjekt zu programmieren, ist eine objektorientierte Programmiersprache Voraussetzung.) Wenn Ihre Aufgabe nun darin besteht, eine Sortierung zu programmieren, die ähnliche Brote in irgendeiner Form versammelt, dann könnten Sie das in Form einfacher wenn-dann-Vergleiche tun.\\

Doch dafür benötigen wir zunächst einige Operationen aus dem Bereich der booleschen Logik, allgemein als Vergleichsoperatoren bekannt. Diese entsprechen zum Großteil den Symbolen, die Sie aus dem Mathematikunterricht kennen. (kleiner als: <, größer als: >, usw.) Einzig den Gleichheitsoperator müssen Sie sich gesondert merken, denn im Gegensatz zur Mathematik nutzen wir hier nicht ein einzelnes, sondern ein doppeltes Gleichheitszeichen. Wie Sie schon gelernt haben ist das einfache Gleichheitszeichen in Sprachen wie C der Zuordnungsoperator, durch den Sie einer Variablen einen Wert zuordnen.\\

•	Prüfung auf Gleichheit von zwei Variablen: = = (zwei Gleichheitszeichen)\\

Diese Operatoren verwenden Sie genauso, wie Sie zuvor die einfachen Operatoren z.B. für die Addition verwendet haben, wo Sie anstelle von addiere a und b einfach a + b programmiert haben. Der Unterschied gegenüber den arithmetischen Operatoren besteht aber darin, dass das Ergebnis von Vergleichsoperatoren eine boolesche Variable ist. Und diese nutzen wir üblicherweise als anonyme Variablen. Kurz gesagt beschäftigt sich die boolesche Logik mit allen Fällen, in denen man zwischen wahr und falsch eindeutig unterscheiden kann.\\

Im Falle unserer wenn-dann-Kontrollstruktur können wir jetzt zum Beispiel unterschiedliche Programmabläufe in Abhängigkeit davon programmieren, ob ein Wert größer als ein anderer ist. Hier ein simples Beispiel in Pseudocode:\\

\begin{verbatim}
if ( a < b ) then { sortiere(a, b); }
else if (a > b) then { sortiere(b,a); }
else { printf(„Beide sind gleich.“); }
\end{verbatim}

Aufgabe:\\

(1)	Worin besteht der Unterschied zwischen dem eben aufgeführten und dem nun folgenden Algorithmus?

\begin{verbatim}
if ( a < b ) then { sortiere(a, b); }
else if (a == b) then { printf(„Beide sind gleich.“); }
else { sortiere(b,a); }
\end{verbatim}

(2)	Prüfen Sie mit einem C-Programm, ob (``Eins“ = = „Eins“) wahr ist. (Tipp: Sie können eine boolesche Variable nicht über printf() ausgeben lassen.\\

(3)	Begründen Sie, warum ein solcher Vergleich bei manchen Sprachen wahr (true) und bei manchen Sprachen falsch (false) ist.\\

(4)	Machen Sie sich bewusst, was geändert werden müsste, damit printf() eine boolesche Variable ausgeben kann und erweitern Sie Ihr Programm dann entsprechend. Wie schon bei einer früheren Aufgabe gilt hier wieder: Direkt ist das nicht möglich; Sie müssen sich eine Lösung für das Problem einfallen lassen, die Sie dann einprogrammieren müssen.\\

Aufgabe (schwer): \\

(1)	Ist eine solche Wenn-Dann-Struktur eigentlich statisch oder dynamisch? (Die Begründung ist das wichtige.)\\

(2)	Wenn Sie sich für eine der beiden Möglichkeiten entschieden haben: Was müsste gegeben sein, damit der andere Fall gilt?\\

Aufgabe (schwer):\\

Programmieren Sie die Funktion ggT(int a, int b), also die Funktion, die den größten gemeinsamen Teiler von a und b ausgibt.\\

Achtung: Diese Aufgabe ist deshalb schwer, weil Sie voraussetzt, dass Sie wirklich verstanden haben, welche Möglichkeiten Ihnen Funktionen bieten. Sollten Sie nach 15 Minuten Bedenk- und Probierzeit auf keine Idee gekommen sein, überspringen Sie die Aufgabe vorerst.\\

Kontrolle\\

Wann immer Sie bei einem Programm den Ablauf vom Zustand einzelner Variablen abhängig machen wollen, benutzen Sie einfache Wenn-Dann-Kontrollen. Bei C sehen die aufgrund Ihrer Entweder-Oder-Struktur etwas unübersichtlich aus, wenn hier viele Einzelfaktoren zur Entscheidung beitragen, aber Sie können natürlich mittels Funktionen mehr Eleganz in den Programmablauf bringen.

\section{Rekursionen}

Fragen Sie einhundert Studierende der Informatik, wo Sie zum ersten Mal Probleme beim Programmieren hatten und die Antwort wird lauten: Rekursionen. \\

Dabei sind Rekursionen eine unglaublich einfache Kontrollstruktur; Sie müssen nur (wie jeder ernstzunehmende Informatiker) wirklich verstanden haben, was eine Funktion ist. Das Wort Rekursion bedeutet übersetzt so viel wie „etwas, das immer wieder passiert“. Sie haben vielleicht schon von Schleifen gehört: Schleifen sind etwas, das Rekursionen ähnelt, aber im Gegensatz zu Rekursionen können Sie Schleifen nicht für jedes Problem nutzen, bei dem irgendein Programmteil sehr oft wiederholt werden soll. Für Schleifen gibt es den Fachbegriff der Iteration.

\subsection{Rekursionen im Seminarraum}

Bevor wir nun in die etwas formalere Erklärung eintauchen, was eine Rekursion ist und wie Sie sie programmieren können, zunächst ein anschauliches Beispiel für das, was bei einer Rekursion passiert:\\

Stellen Sie sich vor, Sie sitzen in einem Hörsaal und Ihnen ist langweilig. (Nein, Sie sollen es sich nur vorstellen… wehe, Ihnen ist wirklich langweilig.) In der Pause stellen sie fest, dass allen Ihren Kommilitonen langweilig ist und Sie vereinbaren ein Spiel: Jeder von Ihnen wird eine kleine Aufgabe erfüllen, wobei letztlich alle die gleiche Aufgabe haben. In der Programmierung würde man jetzt sagen, dass jeder von Ihnen eine Instanz einer Funktion ist: Sie sind alle unterschiedlich, sollen aber die gleiche Aufgabe erfüllen. Für Ihre Aufgabe benötigen Sie Stift und Papier. Beim Rechner wären dass die Register bzw. Adressen des Speichers.\\

Ihr Spiel (in der Programmierung wäre das der Algorithmus) nennt sich nun erzähle eine Geschichte und damit es lustig und überraschend wird, soll jeder von Ihnen nur ein Wort zur Geschichte hinzufügen. Hier hätten wir den Rumpf des Algorithmus bzw. den Rumpf des Programms in Pseudocode. Das könnte natürlich auch in einer Form notiert sein, die wie die Anleitung für ein Spiel verfasst ist, aber es ist wichtig, dass Sie lernen, Pseudocode und Programmcode wie normale Texte zu lesen.\\

\begin{verbatim}
erzaehleEineGeschichte(Wortform X)
{ wenn gilt, dass X = = Subjekt, dann
	{ denke dir ein Subjekt aus;
		notiere dieses Subjekt als Y;
		X ist jetzt Prädikat;
	}
	wenn das nicht gilt, aber wenn gilt X = = Prädikat, dann
	{  denke dir ein Prädikat aus;
		notiere dieses Prädikat als Y;
		X ist jetzt Objekt;
	}
	wenn beides nicht gilt (kurz: sonst), dann
	{  denke dir ein Objekt aus;
		notiere dieses Objekt als Y;
		X ist jetzt Subjekt;
	}
	wenn jetzt gilt (du hast noch eine/n NachbarIn, der nicht mitgemacht hat)
	{  fordere ihn/sie auf: erzaehleEineGeschichte(X);
		merke dir seine/ihre Antwort als Z;
		hänge Z an Y dran; 
	}
	antworte dem-/derjenigen, die dir die Aufgabe erzaehleEineGeschichte() gestellt hat mit Y;
}
\end{verbatim}

Aufgaben: Schreiben Sie diesen Pseudocode so um, dass er soweit wie möglich wie ein C-Programm aussieht. Da Sie ja beispielsweise keinen Datentyp für Wortformen haben, tun Sie einfach so, als wenn das ein Datentyp wäre. Ähnliches gilt für Funktionen wie denke dir ein … aus. Tun Sie hier einfach so, als wenn Sie eine Funktion hätten, die genau das tut.\\

Wie funktioniert dieses Spiel?\\

Und wie würde ein solcher Algorithmus im Speicher eines Computers aussehen?\\

Haben Sie schon eine Idee, was daran eine Rekursion ist?

\subsection{Rekursionen etwas formaler}

Eine Rekursion ist eine ganz normale Funktion, aber während die meisten Programmierer Funktionen im Grunde nur wie Subroutinen nutzen, kommt hier das volle Potenzial von Funktionen zum Einsatz: Sie werden nicht nur ein einziges Mal abgearbeitet, bevor sie einen Rückgabewert übergeben, sondern sie werden so lange einem aktualisierten Rückgabewert neu gestartet, bis eine bestimmte Bedingung erreicht ist. Erst dann wird der aktuelle Rückgabewert zurück gegeben. \\

Programmierer, die Rekursionen nicht vollständig verstehen werfen an dieser Stelle ein, dass das doch genauso ist wie bei einer Schleife, aber das ist falsch: Bei Schleifen ist diese Bedingung ein fester Wert, der zu Beginn des Schleifenaufrufs bekannt sein muss. Im Gegensatz dazu kann bei einer Rekursion die Bedingung bei jeder Wiederholung abgeprüft werden. Und diese Bedingung kann jede denkbare Bedingung sein; es muss kein Zahlenwert sein.\\

Um das zu verdeutlichen: Stellen Sie sich vor, Sie wollten ein Programm erstellen, das anhand eines Stadtplanes prüft, ob Sie innerhalb eine bestimmten Zeit zu Fuß ans Ziel kommen können. Keine Sorge, wir werden hier nicht im Detail erörtern, wie Sie so etwas programmieren können. Alles, was uns für den Moment interessiert ist die Frage, wie denn die Abbruchbedingung für eine Rekursion lauten würde, mit der wir ein solches Programm entwickeln würden. Und die wäre ja denkbar einfach: Computer, wenn du einen Weg gefunden hast, dann verrate ihn mir. Und jetzt dürfte auch dem letzten Schleifenvertreter klar sein, dass so etwas nicht mit einer Schleife programmierbar ist.

\subsection{So programmieren Sie Rekursionen}

Dann wollen wir einmal sehen, was Sie tun müssen, um Rekursionen zu programmieren…\\

Alles, was Sie für die Programmierung von Rekursionen beherrschen müssen sind wenn-dann-Kontrollen und Funktionen. Also haben Sie schon alle Kenntnisse an Bord, die Sie hierfür brauchen.\\

Das erste, was Sie benötigen, wenn Sie eine Rekursion entwerfen bzw. programmieren wollen ist die sogenannte Abbruchbedingung. Denn wie Sie gleich sehen werden ist eine Rekursion so strukturiert, dass Sie endlos laufen wird, wenn kein Kriterium einprogrammiert ist, durch das sie endet.\\

Nehmen wir einmal an, Sie wollen einen Zähler programmieren, der den Wert einer Variablen so lange erhöht, bis diese den Wert 100 erreicht hat. Diesen Zähler wollen wir nun in Form einer Rekursion realisieren, die wir als zaehler(int a) bezeichnen.
Die Abbruchbedingung lautet hier a < 100. Es wäre auch möglich, das Programm so zu entwickeln, dass die Abbruchbedingung a < 101 lautet. Letztlich kommt es auf die Programmierung an.\\

Also sieht unsere Rekursion so aus:

\begin{verbatim}
int zaehler (int a)
{ 
	if (a < 100) then
	{ 
		zaehler(a+1); 
	}
	return a;
}
\end{verbatim}

\subsection{Darum funktionieren Rekursionen}

Die meisten Einsteiger haben hier ein Problem, weil Sie denken, dass in der vierten Zeile (wo zaehler(a+1) aufgerufen wird) der Inhalt der Rekursion überschrieben würde. Das ist aber nicht so.\\

Deshalb werden wir uns ansehen, was passiert, wenn diese Rekursion mit dem Befehl zaehler(97); aufgerufen wird. Oben bei dem Beispiel mit den gelangweilten Studierenden haben Sie schon ein Beispiel dafür. Aber wir betrachten jetzt die Situation beim Programmablauf im Rechner. Dazu müssen Sie sich vor Augen halten, was der Computer (im Sinne der maschinennahen Programmierung) tut:\\

Beim Aufruf der Funktion wird eine Variable des Datentyps int erzeugt, was ja nichts anderes heißt, als dass (bei einem Cortex-M0) ein Speicherbereich von 32 Bit Länge für diese Variable reserviert wird. Nehmen wir an, dieser Speicherbereich wird an der Adresse 0x1000 reserviert. Der aktuelle Rekursionsaufruf bezeichnet diesen Speicherbereich mit a. Das ist deshalb kein Problem, weil er nichts davon weiß, dass alle anderen Rekursionsaufrufe ebenfalls einen Speicherbereich individuell mit a bezeichnen. Und die anderen Rekursionsaufrufe wissen ebenfalls nichts von den Bezeichnungen, die die anderen Rekursionsaufrufe benutzen. Erinnern Sie sich in diesem Zusammenhang bitte wieder an den Scope, denn über nichts anderes reden wir hier.\\

Nun wird in unserem Beispiel die Zahl 97, also hexadezimal 0x61 der neuen Variablen zugewiesen und somit unter der Speicheradresse 0x1000 gespeichert.\\

Nachdem nun die Funktion durch die Kontrollstruktur (if (97 < 100)…) dazu aufgefordert wurde, die Funktion zaehler(97+1) auszuführen passiert folgendes: Die Funktion zaehler(int a) wird mit dem Argument 98 aufgerufen. Für uns heißt diese Funktion zwar genauso wie die letzte Funktion zaehler(int a), aber im Rechner werden die beiden Aufrufe getrennt voneinander im Speicher abgelegt. Es handelt sich also um zwei Instanzen einer Funktion. Stellen Sie sich das so vor, als wenn Sie einen Index verwenden würden:\\

Der Aufruf zaehler(97); wird im Rechner als der Aufruf zaehler1(97) gespeichert.\\

Die aus zaehler1(97) aufgerufene Funktion zaehler(98) wird intern als zaehler2(98) verwaltet und ist damit eben nicht derselbe Funktionsaufruf wie zaehler1(97).\\

Dieser Aufruf sorgt also nicht dafür, dass die Adresse 0x1000 überschrieben wird, sondern weil es ein neuer Funktionsaufruf ist, wird wieder eine neue Adresse für das Argument der Funktion reserviert. Hier wäre das also z.B. die Adresse 0x1004. Und an 0x1004 wird nun das aktuelle Argument, also der Wert 0x62 (hexadezimal für 98) gespeichert.\\

Dann folgt der nächste Aufruf von zaehler(int a) eben mit dem Wert 99. Das Ganze geht so lange weiter, bis die Kontrolle if (a < 100) falsch ist. Und jetzt passiert folgendes:\\

Die aktuell aufgerufene Funktion zaehler(int a) (intern also zaehler4(100)) gibt nun den Wert 100 zurück. An wen dieser Wert zurückgegeben wird? Na an zaehler3(99), denn diese hatte ja zaehler(100) aufgerufen. Und was tut nun zaehler3(99)? Genau: Diese Funktion tut mit dem Rückgabewert von zaehler4(100) gar nichts und führt nur die verbliebene return-Zeile aus. Sie gibt also den Wert 98 an zaehler2(98) zurück.\\

Kontrolle\\

Überlegen Sie, was diese Rekursion ausgibt, wenn Sie „fertig“ ist. Das ist irgendwie nicht das, was wir uns vorgestellt haben, schließlich sollte sie doch bis 100 zählen. Da sollte Sie dann doch auch 100 ausgeben. \\

Lösen Sie deshalb zur Kontrolle das folgende Problem: Wie müssen Sie den Code anpassen, damit die Rekursion tatsächlich den Wert 100 ausgibt. \\

Keine Sorge, wenn Sie hier zunächst verzweifeln, das ist ganz normal. Dabei besteht die Lösung in zwei kleinen Anpassungen. Und wenn Sie die geschafft haben, dann beherrschen Sie eine der mächtigsten Kontrollstrukturen schlechthin.\\

Diese Aufgabe ist eine ganz einfache Rekursion, die Sie auch mit Hilfe einer Schleife hätten realisieren können. Deshalb folgt auch gleich eine Aufgabe, die nicht mit einer Schleife lösbar ist.\\

Aufgabe (schwer): Programmieren Sie eine Rekursion, deren Argument eine ganze Zahl (nennen wir sie einfach n) ist. Die Rekursion soll nun die Summe berechnen, die im Pascalschen Dreieck in der n‘ten Zeile steht.

\subsection{Schleifen}

Bei vielen Programmierkursen werden Rekursionen übersprungen und anstatt dessen die sogenannten for- und while-Schleifen in allen Varianten besprochen. Das ist komplett überflüssig, weil Sie alles, was Sie mit einer for- oder while-Schleifen machen können wesentlich eleganter mit einer Rekursion erledigen können. Dazu kommt, dass Sie für jede Programmiersprache detailliert lernen müssen, wie Sie eine dieser Schleifen programmieren müssen.\\

Also lassen wir das doch lieber gleich ganz.\\

Na gut, wenn Sie es unbedingt wollen, können Sie von mir aus auch Schleifen benutzen. Suchen Sie einfach im Netz danach. Minuspunkte gibt’s dafür nicht, aber Rekursionen müssen Sie in jedem Fall beherrschen.

\section{Datenstrukturen}

Denken Sie nochmal an die algebraische Struktur BROT. Und stellen Sie sich jetzt vor, Sie müssten verschiedene Elemente der algebraischen Struktur einzeln programmieren. Dann müssten Sie Unmengen an Variablen deklarieren und initialisieren. Das ist nicht nur sehr arbeitsaufwändig, es ist vor allem außerordentlich fehleranfällig. In objektorientierten Sprachen gibt es u.a. für diese Aufgabe die sogenannten Klassen, in C können wir uns lediglich der sogenannten Datenstrukturen bedienen, die aber auch schon ein recht mächtiges Mittel sind, um die Fehleranfälligkeit unseres Codes zu reduzieren.\\

Aber zunächst zur Frage, was eine Datenstruktur ist: Die naive Antwort lautet: Es ist eine strukturierte Methode, um Daten aufzubewahren. So naiv die Antwort, so wenig sagt Sie uns… Also versuchen wir es einmal anders: Stellen Sie sich vor, Sie hätten 95 Elemente, die eine konkrete Menge der Algebra BROT bilden. Oben hatten wir definiert, dass die folgenden Eigenschaften ein Element dieser Algebra auszeichnen: Knusprigkeit, Farbe, Temperatur, Form.\\

Wenn wir nur die Mittel zur Verfügung hätten, die Sie schon kennen gelernt haben, dann müssten wir jetzt die Variablen knusprigkeit1 bis knusprigkeit95, die Variablen farbe1 bis farbe95 usw. deklarieren und ihnen jeweils einen zum Wert passenden Datentyp zuordnen.\\

Aber das brauchen wir. Wir können alternativ dazu eine Datenstruktur deklarieren. In Veranstaltungen zu Algorithmen und Datenstrukturen lernen Sie verschiedene dieser Datenstrukturen kennen. Wichtig ist, dass Sie in einer Sprache wie C jede Datenstruktur selbst einprogrammieren können, auch wenn sie nicht Teil der Sprache ist.\\

Für den Anfang genügt es, wenn Sie als Datenstruktur ein Array nutzen, mittel- bis langfristig sollten Sie allerdings lernen, wann Sie am besten welche Datenstruktur nutzen. Gerade verkettete Listen und Bäume sind in der Praxis häufig wesentlich effizienter und eleganter als ein Array, auch wenn sich Ihre Nutzung nicht ganz so einfach erschließt. Schlechte Programmierer und Neulinge erkennen Sie daran, dass diese in C ausschließlich Arrays als Datenstruktur nutzen.
 
\subsection{Arrays}

Eine Einschränkung, der ein Array in der Sprache C unterliegt, besteht darin, dass alle Elemente den gleichen Datentyp haben müssen. Dadurch lässt sich ein Array sehr leicht auf der Maschinenebene umsetzen. Dazu ein Beispiel:\\

Wenn Sie ein Array deklarieren, dass 95 Einträge hat und dessen Einträge alle vom Typ int sind, dann passiert im Hintergrund folgendes: Da jede Variable vom Typ int 32 Bit bzw. 4 Byte belegt, werden jetzt 95 * 4 Byte, also 380 Byte bzw. 380 Adressen am Stück für unser neues Array reserviert. (Sie wissen schon: Beim Programmieren in C sehen wir davon nichts.)\\

Im nächsten Schritt müssen wir noch jeden Eintrag des Array mit einem Wert initialisieren. Bei manchen Programmiersprachen passiert das automatisch, bei C nicht.\\

Schauen wir uns einmal an, wie eine Initialisierung praktisch durchgeführt wird: Angenommen wir initialisieren nur den fünfzigsten Eintrag eines Array mit dem Wert 27 (hexadezimal 0x1B). Dann wird zunächst geprüft, wie die Startadresse des Arrays lautet. Nehmen wir an, diese lautet 0x2000. Nehmen wir weiter an, es handelt sich um ein int-Array, dass also jeder Eintrag 4 Byte bzw. 4 Adressen belegt.\\

Dann wird nun die Startadresse genommen (0x2000) und dazu werden 49 * 4 (das ist 196, bzw. hexadezimal 0x124) addiert. Damit befindet sich der fünfzigste Eintrag des Array unter Adresse 0x2124. Jetzt wird also an der Adresse 0x2124 der Wert 0x1B gespeichert.\\

Sie fragen sich, warum wir hier 49 * 4 und nicht 50 * 4 gerechnet haben? Ganz einfach: Da die Startadresse (hier abgekürzt als S, bzw. S + 0 * 4) die erste Adresse ist, an der Daten im Array gespeichert werden, ist S + 1 * 4 die Adresse, an der das zweite Element des Arrays gespeichert wird. Dementsprechend finden Sie das fünfzigste Element nicht an Adresse S + 4 * 50 sondern an Adresse \\

S + 4 * 49.\\

Die Initialisierung eines Array wird üblicherweise im Rahmen einer for-Schleife erledigt, aber wie Sie jetzt wissen, können Sie das in Form einer Rekursion erledigen.\\

Programmierung eines Array in C:\\

Für die Deklaration eines Arrays müssen Sie neben dem Datentyp der Einträge bei C von Beginn an wissen, wie viele Einträge das Array haben soll. Achtung: Sie dürfen hier von der Anzahl Elemente nichts abziehen, denn auch wenn das letzte Element die Nummer hat, die um eins kleiner als die Anzahl Elemente des Array ist, bleibt die Anzahl Elemente gleich.\\

In unserem Fall haben wir es also mit 95 Elementen im Array temperatur zu tun. (Sie wissen schon: Kann man Brot einfrieren?) Die Deklaration des Array sieht dann so aus:\\

\begin{verbatim}
int temperatur[95];
\end{verbatim}

Denn im Gegensatz zu Funktionen aber genau wie eine Variable muss ein Array wieder deklariert werden, bevor es initialisiert werden kann.\\

Die Initialisierung eines Elements des Array ist wieder eine Zuweisung, die so ähnlich aussieht wie die Zuweisung eines Wertes zu einer Variablen. Nehmen wir an, wir wollen dem 35’igsten Element des Array einen Wert von 17 zuordnen, dann sähe das so aus:\\

\begin{verbatim}
temperatur[34] = 17;
\end{verbatim}

Achtung: Vergessen Sie an dieser Stelle nicht, das wir bei der Zählung der Elemente eines Array bei 0 anfangen. Deshalb müssen wir hier nicht temperatur[35] den Wert 17 zuordnen, sondern wie soeben geschehen temperatur[34].\\

Aufgabe: Deklarieren und initialisieren Sie innerhalb eine C-Programms ein Array der Länge 20.\\

Die Elemente sollen vom Typ float sein.
Das erste Element soll den Wert 10.0 haben.\\
Jedes Element soll mit einem Wert initialisiert werden, der um 1.23 größer ist als der seines Vorgängers. (Das erste Element bekommt also den Wert 10, das zweite den Wert 11.23, das dritte den Wert 12.46, usw.)
Lassen Sie anschließend die Werte ausgeben, wobei nach jeweils zehn Werten ein Zeilenumbruch erfolgen soll.\\

Hinweis: Wenn Sie einen Fehler bekommen, bei dem so etwas wie array index out of bound steht, dann haben Sie einen Teil dessen ignoriert, was in diesem Abschnitt erklärt wurde.\\

Tipp: Erinnern Sie sich an den Modulo-Operator, dann können Sie sich einiges an Programmieraufwand sparen.

\subsection{Strings}

Diejenigen unter Ihnen, die schon ein wenig imperativ programmiert haben, werden bei der Einführung von Datentypen eingewandt haben, dass ein String doch ein Datentyp ist. Dort konnten Sie lediglich nachlesen, dass das falsch ist. Aber erst jetzt haben wir alles besprochen, was Sie wissen müssen, um zu verstehen, warum das falsch ist.\\

Denn ein String ist eine Datenstruktur. Genauer gesagt ist es ein Array vom Typ char und damit dynamisch.\\

Aufgabe:\\

Programmieren Sie ein solches char-Array, dass den Satz hello, world enthält und geben Sie den Inhalt des char-Array auf dem Bildschirm aus.\\

Statische Strings können Sie realisieren, indem Sie wie beim Quellcode am Anfang des Kapitels einen beliebigen Text zwischen doppelten Anführungszeichen setzen. („Die Antwort lautet: 42“)\\

Kontrolle\\

Ein auszugebener Text lautet Hallo, Weld (mit d statt t in Welt) und ist unter einem Bezeichner namens begruessung abgespeichert. Wie können Sie diesen Fehler korrigieren, wenn a) der Text als statischer String und b) als dynamischer String vorliegt?

\subsection{Verkettete Listen, Bäume u.a.}

Wenn Sie eine Veranstaltung wie Algorithmen und Datenstrukturen besuchen, fragen Sie sich nun, wie Sie denn beispielsweise eine verkettete Liste in C programmieren können? Das ist gar kein so großes Problem, allerdings kommen Sie hier um die Programmierung mit Pointern nicht herum. Über Pointer haben wir aber noch nicht gesprochen und werden das vorerst nicht tun, weil Sie zwar ein sehr mächtiges Mittel der Programmierung sind, aber im Grunde eine Einführung der maschinennahen Programmierung in die Sprache C. Zu diesem Zeitpunkt werden wir Sie deshalb noch nicht behandeln. 

\subsection{Zusammenfassung}

Sie haben jetzt alles gelernt, was Sie für die grundlegende Programmierung in C benötigen. Einige Arten von Programmen können Sie mit diesem Wissen aber noch nicht erstellen. Damit Sie Klarheit haben, welche Arten von Programmen Sie jetzt noch nicht erstellen können, kommt hier eine kleine Aufstellung wichtiger Fälle: (So beißen Sie sich nicht die Zähne an einer Aufgabe aus, die Sie noch nicht lösen können.)\\

•	Nutzereingaben: Sie wissen noch nicht, wie Sie Nutzereingaben z.B. über die Tastatur verarbeiten können. Wobei Ihnen hier nur ganz wenig Wissen fehlt.
•	Vernetzung: Sie wissen noch nicht, wie Sie ein C-Programm entwickeln sollen, das selbständig eine Verbindung über Netze aufbauen und nutzen kann. Selbst einfache Programme wie einen Instant Messenger können Sie also noch nicht programmieren.
•	Grafikbasierte Oberflächen: Sie wissen noch nicht, wie Sie grafische Elemente wie ein Menü oder Schaltflächen programmieren können.\\

Aber dennoch kennen Sie jetzt bereits alles, was Sie an Kernelementen der Sprache C benötigen, um ein beliebiges Programm für einen Cortex-M0 zu entwickeln. Jetzt kommen also die Bereiche, die nicht zum eigentlichen Kern der Sprache C gehören, die Sie aber innerhalb eines C-Programms benötigen, um einen Cortex-M0 zu programmieren.


%\part{Einführung in die objektorientierte Programmierung und Softwareentwicklung}

%Die Abschnitte dieses Teils müssen noch an die LeTeX-Syntax angepasst werden.
% Nachfolgender Abschnitt muss noch an Latex angepasst werden.
%\section{Java 8 - Grundlegende Struktur}

Bitte beachten Sie, dass alles, was Sie in diesem Buch zurzeit lesen können auf die Java-Version 8 bezieht. Zum Teil sind Abschnitte jedoch bewusst etwas allgemeiner gehalten, da die Feinheiten der Programmierung mit der Folgeversion geändert werden könnten. Das habe ich deshalb getan, um zu verhindern, dass Sie sich Methoden fest einprägen, die in einem oder zwei Jahren nicht mehr gelten. Außerdem sind Sie so gezwungen das zu tun, was bei Java oberste Pflicht ist: Nutzen Sie für die Details der Sprache die Java API! Ohne die Nutzung der Java API werden Sie auf eine Art programmieren, die früher oder später veraltet oder sogar ungültig ist.

Im Gegensatz zu Programmiersprachen wie C, PROLOG, Scheme und vielen anderen ist Java eine Middleware. Wie Sie also in Veranstaltungen zu Netzwerken oder Nachrichtentechnik lernen können ist Java von vornherein darauf ausgelegt, dass Sie Aspekte wie die Datenübertragung über ein Netzwerk nicht im gleichen Detailgrad programmieren müssen wie beispielsweise in einer Nicht-Middleware wie C. Die Vor- und Nachteile dieses Ansatzes sind Ihnen klar, wenn Sie beide Ansätze programmiert haben: Wenn Sie alle Details selbst programmieren, können Sie ein Höchstmaß an Effizienz erreichen, auch weil Sie die Spezifikation des jeweiligen Systems ausnutzen können. Wenn Sie eine Middleware nutzen, erhalten Sie eine in vielen Fällen noch ausreichend effiziente Lösung mit weniger Programmieraufwand. Außerdem helfen die in einer Middleware wie Java vorhandenen Standardisierungen, um Software in großen Teams zu entwickeln und neue Mitglieder in eine Team zu integrieren.

Als angehende (Medie-)informatikerInnen müssen Sie sich aber darüber bewusst sein, dass Sie in Java nicht die höchstmögliche Effizienz in diesem oder anderen Bereichen erreichen werden. Sie müssen sich insbesondere darüber bewusst sein, was das für ein in Java entwickeltes Programm bedeutet.

Wie Sie wissen, wenn Sie bereits in Java programmiert haben, handelt es sich um eine klassenbasierte objektorientierte und seit Version 8 zumindest teilweise funktionale Programmiersprache. In diesem Abschnitt werden wir uns die hierarchische Struktur von Java ansehen, die sich zum Teil identisch in der Hierarchie der Pakete und Klassen widerspiegelt.

\paragraph{Aufgabe}

Was bedeutet es für diejenigen Ihre Programme, die Sie in Java erstellen, dass es nicht möglich ist, in Java ein Höchstmaß an Effizienz zu erreichen? 

\paragraph{Aufgabe}

Was ist in anderen Programmiersprachen nötig, um dort dieses Höchstmaß zu erreichen?

\subsection{Die Java API}

So wie Java kontinuierlich geändert wird, wird auch die API von Java kontinuierlich überarbeitet. Es ist deshalb nicht sinnvoll, die API herunterzuladen (außer wenn Sie davon ausgehen müssen, des öfteren keine Internetanbindung zu haben, wenn Sie in Java programmieren wollen). Zurzeit finden Sie die Java API unter \url{http://docs.oracle.com/javase/8/docs/index.html}\index{Programmierung!Java 8!API}. Ältere APIs finden Sie, indem Sie schlicht die Ziffer im Link anpassen. Deshalb dürfte die API für Version 9 nach der Veröffentlichung unter \url{http://docs.oracle.com/javase/9/docs/index.html} zu finden sein.

Wenn Sie also in Bezug auf Java einen Link in diesem Buch finden, der nicht funktioniert, dann sollten Sie zumindest über den hier genannten Link zur entsprechenden Passage der API gelangen können.

\subsection{JVM - Java Virtual Machine Technology}

Das Fundament der Middleware Java besteht in der \textbf{JVM}, der Java Virtual Machine. Bislang haben Sie in diesem Buch lesen können, dass es die sogenannte JRE (Java Runtime Environment) ist, das dafür zuständig ist, Java-Programme zu betreiben. Die eigentliche Plattform, die Java-Programme betreibt ist jedoch die JVM, die aber Teil des JRE ist.

Eine \bold{Virtuelle Maschine}\index{Virtuelle Maschine} ist ein Programm, das einen Rechner simuliert. Virtuelle Maschinen sind ein praktisches Werkzeug, wenn Sie beispielsweise Betriebssysteme und Anwendungen ausprobieren wollen, denn Sie benötigen für den Betrieb keinen zusätzlichen Rechner. Stellen Sie sich beispielsweise vor, Sie sind dafür zuständig, in einem Unternehmen die Software zu warten. Sie haben dann Nutzer mit den unterschiedlichsten Systemen und den unterschiedlichsten Installationen auf diesen Systemen. Um also Änderungen zu planen müssten Sie entweder von jedem dieser Systeme ein Exemplar in Ihrem Büro haben oder Sie nutzen eine VM, für die Sie all die verschiedenen Systeme als individuelle Datei vorliegen haben. Das genügt zwar nicht für eine vollständige Prüfung darauf, ob Änderungen reibungslos durchführbar sind, aber es ist eine sinnvolle erste Prüfung.

Die JVM sorgt also dafür, das Java Programme auf einer Vielzahl von Systemen laufen können.

Wenn sie bereits Java-Programme entwickelt haben, dann wissen Sie, dass Sie Programme nach dem Kompilieren mit dem Befehl java starten können. Das bedeutet aber genauer gesagt, dass Sie mit den Befehl java die JVM starten und ihr ein kompiliertes Java-Programm als Argument übergeben.

Sehen wir uns nun den Befehl java etwas genauer an: Es gibt ihn in den zwei Versionen \emph{java} und \emph{javaw}.

\begin{itemize}
	
	\item \textbf{java} startet nicht nur die JVM und führt über diese das übergebene Java-Programm aus, sondern öffnet zusätzlich ein Konsolenfenster. Dieser Befehl ist also für die Entwicklung von Java-Programmen gut, weil Sie hier Rückmeldungen erhalten, die innerhalb der Anwendung nicht angezeigt werden.
	
	\item \textbf{javaw} öffnet im Gegensatz zum Befehl java keine Konsole. Sollte es allerdings zu Fehlern kommen, wegen denen die Anwendung nicht gestartet werden kann, dann öffnet sich ein Fenster, das eine Fehlermeldung anzeigt.
	
\end{itemize}

Nach java bzw. javaw können Optionen folgen, die die Ausführung der Anwendung beeinflussen. beispielsweise können Sie die JVM anweisen, die Anwendung so schnell wie möglich auszuführen. Eine andere Option ermöglicht es, dass die JVM so wenig Speicher wie möglich belegt.

Nun folgt entweder der Name der Klasse oder die Option \emph{-jar} gefolgt vom Dateinamen. Bei der Option -jar folgt also keine einzelne Klasse. Vielmehr haben Sie es hier mit einer Verwendung von java bzw. javaw zu tun, die Sie kennen lernen werden, wenn Sie in Netzen ein Deployment durchführen. Damit ist gemeint, dass Sie z.B. in einem Unternehmensnetzwerk ein Paket verteilen lassen und es auf mehreren Rechnern ausführen lassen, ohne an jeden einzelnen Rechner zu gehen. Das können beispielsweise Sicherheitsupdates sein, aber denken Sie jetzt nicht, dass nur bei Sicherheitsupdates ein Deployment Sinn macht: Jede Art von Installation innerhalb eines Unternehmensnetzwerkes bzw. bei verteilten Anwendungen sollte weitgehend automatisch auf allen betroffenen Systemen durchgeführt werden.

Abschließend können Sie noch ein oder mehrere Argumente eingeben, die als String-Array an die main()-Methode der Java-Klasse übergeben wird, die durch die java- bzw. javaw-Anweisung gestartet wird. (Auch bei den Paketen, die per -jar gestartet werden gibt es eine Java-Klasse, die als erstes gestartet wird.)

Mehr zur JVM, zu java und javaw finden Sie bei Java 8 unter \url{http://docs.oracle.com/javase/8/docs/technotes/guides/vm/index.html}. 

\subsection{Base Libraries}

Die ersten Bibliotheken von Java werden von Oracle als \textbf{Base Libraries} bezeichnet. Diese Base Libraries sind wie alle Bibliotheken von Programmiersprachen nichts anderes als Sammlungen von Programmteilen, die bestimmte Aufgaben erfüllen.

Diese Bibliotheken haben als einzige Gemeinsamkeit, dass Sie den eigentlichen Sprachkern ausmachen. Klassen aus den Base Libraries dienen also niemals dazu, um die Anzeige Ihrer Anwendung zu ändern. Sie bieten keinerlei Unterstützung bei der Interaktion mit Nutzern. (Einzig die Eingabe und Ausgabe von Zeichenfolgen ist hier enthalten.) Auch die Audio- und Videoverarbeitung (von Eingabe oder Ausgabe ganz zu schweigen) werden mit diesen Bibliotheken nicht realisiert.

Über die einzelnen Bibliotheken werden wir sprechen, wenn wir uns die übrigen Bestandteile des JRE bzw. des JDK angesehen haben.

\subsection{Integration Libraries}

Es folgen eine handvoll Bibliotheken, mit denen Sie als fortgeschrittene Java-Entwickler zu tun bekommen: Die Integrationsbibliotheken stellen z.B. Schnittstellen zu Datenbanken her, bei nur mittels SQL-Sprachen angesprochen werden können. Hier finden Sie auch die RMI-Bibliothek, mittels derer Sie auf Java-Programmteile zugreifen können, die auf einem anderen Rechner laufen. Auch wenn Sie mit diesen Bibliotheken anfangs nichts zu tun haben stellen sie also sehr nützliche Möglichkeiten zur Verfügung.

\subsection{User Interface Libraries}

Jetzt kommen wir zu den Bibliotheken, mit denen die meisten von Ihnen die Programmierung beginnen wollen: Die UILs stellen Ihnen eine Vielzahl an Klassen zur Verfügung, mittels derer Sie z.B. Ein- und Ausgabe von Sounds, die Anzeige einer grafischen Nutzeroberfläche und vielem anderen mehr realisieren können, das für aktuelle Programme unverzichtbar ist. Wissenschaftler und Systemadministratoren mögen durchaus auf solchen "`Spielkram"´ verzichten, aber für nahezu alle anderen Kundengruppen gilt: Gibt es keine grafische Nutzeroberfläche mit Soundfeedback und Mausinteraktion, dann bleibts ein Ladenhüter.

\subsection{Deployment}

Dieser Begriff wurde oben schon eingeführt. Es mag deshalb verwirren, dass JAR-Dateien nicht in diese Gruppe von Bibliotheken gehören, aber der Grund ist recht simpel: Sie können JAR-Dateien auch ohne Deployment verwenden. Die Bibliotheken des Deployment-Pakets bieten Ihnen die Möglichkeit an, von einzelnen Patches für eine Java-Anwendung bis hin zu einer vollständigen Java-Anwendung alle denkbaren Softwarepakete automatisisert zusammenstellen, zu komprimieren und an beliebig viele Zielrechner über ein Netzwerk zu verschicken, wo sie dann nach Vorgaben ebenso automatisiert installiert werden können. Der Vollständigkeit halber sei noch erwähnt: Natürlich muss dazu am jeweiligen Rechner nach der Installation des Betriebssystems einmalig eine entsprechende Konfiguration durchgeführt werden, aber das ist auch alles, was an jedem Zielsystem getan werden muss.

\subsection{Weitere Pakete im JDK}

Es folgen nur noch Pakete, die Teil des JDK, aber nicht des JRE sind.

Wenn Sie sich die entsprechende Übersicht in der Java API, genauer gesagt in den Java Docs ansehen, dann werden Sie feststellen, dass dort der Befehl java als ein Teil des JDK, genauer des Pakets "`Tools \& Tool APIs"´ aufgeführt werden. Der Grund ist recht simpel: Um als Entwickler die verschiedenen Bestandteile der Middleware aktiv einsetzen zu können, sind teilweise Anpassungen nötig.

Das ist die grobe Einteilung von Java in Paket- oder Bibliotheksgruppen. Sehen wir uns jetzt die einzelnen Gruppen im Detail an.

\section{Base Libraries im Detail}

Vielleicht fragen Sie sich, warum wir nicht zunächst einen Blick auf die VM werfen: Alles was hier relevant ist wurde bereits ober erklärt. Für die weiteren Details in der aktuellen Java-Version ist es wichtig, dass Sie einen Blick in die Java API werfen.

In den folgenden Abschnitten finden Sie grundlegende Informationen dazu, was im jeweiligen package zu finden ist, bzw. warum es dieses package gibt. Beachten Sie dabei, dass viele dieser Pakete wiederum in Pakete aufgeteilt sind.

\subsection{lang}

Die offizielle Erklärung dieses Paketes ist etwas schwammig. Demnach handelt es sich um essentielle Funktionalitäten der Sprache. Für die Frage, was dieses Paket nützt ist diese Antwort nicht eben hilfreich. Da hätte die Auskunft "`lang ist wichtig,"´ genauso viel oder wenig ausgedrückt.

Einige Klassen dieses Pakets dienen als sogenannte Wrapper Klassen, um Variablen, die in Java ja keine eigenständigen Objekte sind, als Objekte zu verpacken.

Einige Klassen dieses Packets stellen sicher, dass in Java "`Dinge"´, wie Class, Object, Throwable, Thread und ähnliche mehr existieren, die wir bei der Programmierung wie selbstverständlich nutzen.

Sollten Sie sich gefragt haben, warum Sie Klassen ohne Konstruktor programmieren können, können Sie in diesem Teil der API die Antwort nachlesen: Klassen vom Typ Class werden automatisch durch die JVM mittels eines Aufrufs der Methode defineClass(...) erzeugt.

Dazu stellt es auch die Möglichkeiten sicher, um entsprechend der Hierarchie von Threads Vererbung zu realisieren.

Dann gibt es hier Klassen, die dafür sorgen, dass Prozesse (also die tatsächliche Ausführung von Programmteilen) durch das Betriebssystem durchgeführt werden.

Verkürzt ließe sich also sagen, dass java.lang all das beinhaltet, was nötig ist, um einfachste Java-Klassen zu programmieren, aber keine Variablen im Sinne virtueller Objekte, die direkt im Speicher abgelegt werden können und den auf ihnen definierten Operationen.

\paragraph{Aufgabe}

Werfen Sie einen Blick in die Java-API des package java.lang und lösen Sie anhand der dortigen Angaben die folgenden Aufgaben:

\begin{enumerate}

	\item Was passiert, wenn ein NaN vom Typ Float mit einem Objekt vom Typ Float verglichen wird, der eine Zahl enthält?

	\item Neben dem Umgang mit NaN gibt es noch etwas überraschendes, wenn es um den Vergleich mit Zahlen geht. Es geht hier um die Zahl 0.0f. Was ist das?
	
	\item Es gibt die Methode sum(), um zwei Float-Objekte zu addieren. Wie sieht es mit anderen arithmetischen Operationen aus?
	
	\item Suchen Sie weiter und finden Sie alle Methoden, mittels derer Sie Operationen durchführen können, die für Variablen vom Typ float definiert sind.
	
	\item Welchen Sinn kann es haben, einzelne Werte als Float-Objekte (oder andere Variablen z.B. boolesche Variablen) als Objekte vom Typ der entsprechenden Wrapper Klasse zu verpacken?
	
	\item Finden Sie (anhand der Java API zu Float) heraus, was der größte und kleinste Wert ist, den eine float Variable haben kann. Geben Sie ihn als gerundete 10-er Potenz an. Was fällt Ihnen auf? (Bitte hier die beiden Felder NEGATIVE\_INFINITY und POSITIVE\_INFINITY ignorieren.)
	
	\item Und jetzt das gleiche für ein Objekt vom Typ Double.
	
	\item Berechnen Sie, wie viel Speicher eine Variable vom Typ float bzw. double mindestens belegt.
	
	\item Erzeugen Sie jetzt ein Objekt von jedem der beiden Typen, jeweils mit einem maximalen Wert und prüfen Sie, wie viel Speicher tatsächlich verwendet werden.
	
	\item Finden Sie anhand der API (egal ob für Float oder Double) heraus, wie eine Zahl in einen Text umgewandelt werden kann, der z.B. für System.out.println() nötig ist.
	
	\item Sie wollen ein Programm entwickeln, bei dem es auf außerordentliche Präzision bei mathematischen Operationen ankommt. Sie müssen also beispielsweise wissen, wenn es zu einem Overflow bei einer Operation kommt. (Denken Sie hier an ein Programm, dass ein Raumschiff auf dem Flug zum Mars steuert oder an die Steuerung eines Kernkraftwerks. Für beide Fälle würden Sie in der Praxis auf andere Programmiersprachen zurück greifen, aber hier geht es darum, dass Sie die Möglichkeiten von Java kennen lernen.) Prüfen Sie, welche der beiden Klassen Math und StrictMath Sie in diesem Fall nutzen würden. Beschreiben Sie, wie die Klasse Ihnen ein höhere Sicherheit gewährleistet, bzw. welche Methoden das tun. Erklären Sie außerdem, warum es dann noch die andere Klasse gibt.
	
	\item Sie haben gelernt, dass Sie Objekte vom Typ String zwar überschreiben können, aber dass Sie mit Ihr nicht umgehen können, als wäre es ein char[] (Array von char-Variablen). Schauen Sie sich nun die Methode delete() der Klasse StringBuffer an. Wie wird die Datenstruktur dieser Klasse allgemein genannt?

\end{enumerate}

Mit diesen Übungen haben Sie ein vertieftes Verständnis für das Package java.lang und damit für die Entwicklung grundlegender Java-Anwendungen gewonnen. Insbesondere können Sie jetzt mithilfe der Wrapper Klassen Aufgaben lösen, für die Sie früher eigene Klassen entworfen hätten. Dazu kommt, dass Sie je nach nötiger Präzision von mathematischen Aufgaben die passenden Methoden verwenden werden. Und Sie werden nie wieder eine eigene Variable für die Eulersche Zahl oder Pi verwenden.

% Nachfolgender Abschnitt muss noch an Latex angepasst werden.
%\chapter{Einführung in die Objektorientierung mit Java}
\section{11.	GUIs – Programmierung grafischer Nutzeroberflächen}
\subsection{11.1.	Möglichkeiten, um GUIs in Java zu programmieren}
\subsection{11.2.	Bestandteile einer GUI in Java}
\subsection{11.3.	Programmierung einer „inhaltsleeren“ GUI}
\subsection{11.4.	Struktur von Top-Level Containern}
\paragraph{11.4.1.	Die Containment Hierarchy}
\paragraph{11.4.2.	Die Root Pane}
\paragraph{11.4.3.	Die Layered Pane}
\paragraph{11.4.4.	Die Content Pane}
\paragraph{11.4.5.	Die Glass Pane}
\subsection{11.5.	Nachtrag zur leeren GUI}
\section{11.6.	Einfügen von Komponenten}
\subsection{11.6.1.	Textkomponenten}
\paragraph{11.6.1.1.	JTextField}
\paragraph{11.6.1.2.	JPasswordField}
\paragraph{11.6.1.3.	JTextArea}
\paragraph{11.6.1.4.	JLabel}
\section{11.7.	Einschub zu Layouts}
\section{11.8.	Mehr Komponenten}
\subsection{11.8.1.	Auswahlmöglichkeiten}
\paragraph{11.8.1.1.	Ausgewählte Methoden für Auswahlmöglichkeiten}
\paragraph{11.8.1.2.	Entweder-Oder – Der JToggleButton}
\paragraph{11.8.1.3.	Entweder-Oder, jetzt als Quadrat mit Häkchen – JCheckBox}
\paragraph{11.8.1.4.	Entweder-Oder, jetzt als Kreis mit Füllung – JRadioButton}
\paragraph{11.8.1.5.	Wähle eines aus vielen – JRadioButton-Gruppe und JList}
\paragraph{11.8.1.6.	Drop-Down-Menu – JComboBox}
\paragraph{11.8.1.7.	Anzeige und Interaktion mit umfangreichen Daten}
\paragraph{11.8.1.8.	Schieberegler – JSlider}
\paragraph{11.8.1.9.	JButton}
\section{11.9.	Der erste nicht-Top-Level Container – Die JScrollPane}
\section{11.10.	Wahrnehmung versus Realität}
\section{11.11.	Layout Manager}
\subsection{11.11.1.	Gruppierung, Orientierung und Ausrichtung}
\subsection{11.11.2.	Standard-Layouts}
\subsection{11.11.3.	Das Border Layout: 5 Bereiche}
\subsection{11.11.4.	Nachtrag zum FlowLayout}
\subsection{11.11.5.	GridLayout – Tabellarischer Aufbau mit gleichgroßen Zellen}
\subsection{11.11.6.	GridBagLayout – Tabellarischer Aufbau mit Zellen individueller Größe}
\section{11.12.	JPanel – „Zwischen“-Container für Komponenten}
\section{11.13.	Menüleiste, Quellcodebeispiele und weitere Aufgaben}
\section{15.3.	JDialog}
\section{15.4.	JOptionPane}

\chapter{Event-driven Development}
\section{12.1.	Events in Java}
\section{12.2.	Listener in Java}
\section{12.3.	Ein einfacher Listener für eine Schaltfläche}
\subsection{12.3.1.	Einschub: Strings-Objekte und Sonderzeichen}
\section{12.4.	Das ActionEvent – Unser erstes Event}
\subsection{12.4.1.	Ein oder mehrere Listener, viele Quellen}
\section{12.5.	Innere Klassen}
\section{12.6.	Zwischenstand}
\section{12.7.	Listener-Typen nach Komponenten}
\section{12.8.	Listener-Methoden generell}
\subsection{12.8.1.	Methoden für eine Vielzahl von Events}
\subsection{12.8.2.	Methoden, die direkt mit der Verwendung der GUI durch Nutzer zu tun haben}
\subsection{12.8.3.	Methoden, um Änderungen an der GUI zu verfolgen}
\subsection{12.8.4.	Sonstige Methoden}
\section{12.9.	Die 17 Java-Events und 18 Listener}
\subsection{12.9.1.	MouseEvent – Alles rund um die Maus}
\subsection{12.9.2.	ItemEvent – Wenn Nutzer etwas anwählen, um es zu aktivieren}
\subsection{12.9.3.	FocusEvent – Wenn Nutzer etwas anwählen, ohne dabei Eingaben durchzuführen}
\subsection{12.9.4.	Sonstige Events}
\section{12.10.	Zusammenfassung}
\section{Noch einarbeiten.. Nachtrag zu Kapitel 9 – Errata und häufig gestellte Fragen}
\subsection{Klassenvariablen, Instanzvariablen und mehr zur Variablendeklaration}
\subsection{Unerwartete Fehlermeldungen}
\section{13.	Mausereignisse}
\subsection{13.1.	Die abstrakte Klasse MouseAdapter}
\subsection{13.2.	Wenn eine Maustaste von Nutzern genutzt wurde}
\paragraph{13.2.1.	Aufgabe}
\subsection{13.3.	Wenn Nutzer die Maus bewegen}
\paragraph{13.3.1.	Aufgabe}
\subsection{13.4.	Interaktionen mit dem Mausrad}
\subsection{13.5.	Event-Methoden bei Ereignissen mit der Maus}
\section{14.	Timer}
\subsection{14.1.	Zeitabhängige Ausführung von Aufgaben in Java}
\subsection{14.2.	TimerTask}
\paragraph{14.2.1.	Aufgabe (klausurrelevant)}
\subsection{14.3.	Timer}
\paragraph{14.3.1.	Instanziierung eines Timers}
\paragraph{14.3.2.	Übergabe von TimerTasks an Timer}
\paragraph{14.3.3.	Einmaliger Aufruf eines TimerTask zu einem bestimmten Zeitpunkt}
\paragraph{14.3.4.	Wiederholte Aufrufe eines TimerTask nach festen Intervallen}
\subparagraph{14.3.4.1.	Aufgabe (Klausurrelevant)}

\chapter{15.5.	Datenströme und Dateien}
\section{15.5.1.	Event versus Datenstrom}
\subsection{15.5.2.	Besonderheit aller Datenströme}
\subsection{15.5.3.	Pufferung}
\section{18.	Java und Datenübertragungen}
\subsection{18.1.	Sockets – Klassen für den Datenaustausch über Netzwerke}
\subsection{18.2.	Instanziieren der Sockets für Client und Server}
\paragraph{18.2.1.	Mehrere Clients und die Verbindung zum Server}
\subsection{18.3.	Schließen der Verbindung}
\section{18.4.	Zugriff auf den Datenstrom}
\section{18.5.	Riding the data stream}
\subsection{18.6.	Spülen wir die Rest in der Toilette runter.}
\section{18.7.	Zusammenfassung}

\chapter{16.	Nebenläufige Programmierung}
\section{16.1.	Von der prozeduralen zur nebenläufigen Programmierung}
\section{16.2.	Beschränkung nebenläufiger Programmierung}
\section{16.3.	Zustände eines Thread}
\subsection{16.3.1.	Lebensdauer bzw. Gültigkeitsbereich}
\subsection{16.3.2.	Wartet auf den Startbefehl}
\subsection{16.3.3.	Nicht arbeitsfähig}
\section{16.4.	Thread und Runnable}
\section{16.5.	Daemon – Ein hilfreicher Geist}
\section{16.6.	Zeitweilige Freigabe des Prozessorkerns}
\section{16.7.	Vorzeitiges Beenden von sleep() und anderer Methoden}
\section{16.8.	Synchronisation und Monitore – Verhinderung von Inkonsistenzen}
\section{17. - Hier die entsprechenden Inhalte aus dem Kapitel zu Exceptions einbauen.}
\section{16.9.	Das Grauen? – Nebenläufigkeit und SWING}
Bislang haben Sie gelernt, wie Sie Programme entwickeln können, die eine oder mehrere Aufgaben lösen. Sie haben dabei einige Grundlagen der prozeduralen und der strukturierten Programmierung kennen gelernt und diese genutzt, um einfache Programme zu entwickeln, die aus Klassen zusammengesetzt waren.
Damit haben Sie gelernt, wie Sie die Logik eines Programms entwickeln können. Jetzt wird es Zeit, dass Sie lernen, wie Sie „Fenster“ programmieren können, über die Nutzer ihr Programms komfortabel bedienen können. Dabei gehen wir zwar nicht weiter auf Fragen der Usability  ein, aber Sie lernen die ersten Grundlagen dafür kennen. In Ihrem Studium gibt es keine Veranstaltung, die sich explizit der Usability widmet. Wenn Sie Ihre Kenntnisse in diesem Bereich vertiefen wollen, dann sprechen Sie bitte Ihre Dozenten in den Veranstaltungen des Mediendesigns darauf an.
Zunächst aber zu zentralen Begriffen, mit denen wir uns hier beschäftigen werden:
Immer wenn Nutzer eines Programms den Programmablauf verfolgen oder ändern wollen, muss ein User Interface (kurz UI) vorhanden sein. Ironischerweise ist eines der größten Probleme für Programmierneulinge, dass ein UI zum Programmieren gar nicht nötig ist und deshalb auch zunächst nicht angezeigt wird. Der Vollständigkeit halber sei hier betont, dass ein UI keinerlei Unterstützung für die Nutzung per Maus anzubieten braucht.
Um das Ganze komfortabler zu gestalten werden seit den 90er Jahren vorrangig grafische UI’s (kurz GUI) entwickelt. Hier können Nutzer mit der Maus viele Optionen ohne Einsatz der Tastatur anwählen. Außerdem ermöglichen es GUIs einstellbare Komponenten in logische Gruppen aufzuteilen und diese entsprechend darzustellen. Als Nutzer kennen Sie das bereits, weil Sie ständig über GUIs mit Computern interagieren. Diese Erklärung ist dennoch wichtig: Als Softwareentwickler müssen Sie sich bei der Entwicklung einer GUI  stets vor Augen halten, dass es hier irrelevant ist, was Sie sich bei den Strukturen Ihres Programms gedacht haben. Für eine GUI ist einzig relevant, dass Nutzer eine visuell erkennbare Logik vorfinden, die sie bei der Bedienung des Programms unterstützt.
Wichtig: Obwohl Sie eine GUI also mit den gleichen Mitteln entwickeln wie andere Teile eines Programms, spielen effiziente Algorithmen hier keine Rolle. Wichtig ist hier, dass Nutzer das, was sie zu sehen bekommen möglichst intuitiv bedienen können. Ob das der Fall ist hat nichts damit zu tun, wie gut Sie im Sinne der Informatik Software entwickeln können.
11.1.	Möglichkeiten, um GUIs in Java zu programmieren
Zur Zeit gibt es drei Bibliotheken, die Java für die Entwicklung von GUIs anbietet:
-	AWT (kurz für Abstract Window Toolkit) ist die älteste dieser Bibliotheken und bietet im Grunde nur rudimentäre Möglichkeiten: java.awt

-	Swing ist der Arbeitstitel von JFC (kurz für Java Foundation Classes), der bis heute für die JFC verwendet wird. Swing ist im Grunde nicht mehr für aktuelle GUIs ausreichend, u.a. da es für die Entwicklung von GUIs für Rechner mit horizontalem Display, Maus und Tastatur entwickelt wurde. Es bietet aber umfangreiche Möglichkeiten, um Formulare und Oberflächen für die Bild- und Soundbearbeitung zu entwickeln. Zum Teil werden hier auch Klassen des AWT verwendet. Name der Klassenbibliothek: javax.swing

-	JavaFX ist die neueste Bibliothek, um GUIs zu entwickeln. Wirklich neu ist es nicht mehr, wurde aber mit der Version 8 von Java erweitert.
Wir werden in diesem Kurs mit Swing arbeiten, weil wir uns hier nur grundsätzlich mit der Entwicklung von grafischen Nutzeroberflächen in Java beschäftigen wollen und JFC im Gegensatz zu AWT viele Elemente von GUIs so anzeigt, wie Nutzer das beim jeweiligen Betriebssystem gewohnt sind. (In der Java-API finden Sie hierfür den Begriff des look \& feel.) Zum Einstieg in die GUI-Entwicklung bietet es also mehr als ausreichende Möglichkeiten.
Beachten Sie bitte, dass HTML5 in Kombination mit CSS3 deutlich weitergehende Möglichkeiten bietet, um eine Nutzeroberfläche zu gestalten. Die Programmlogik können Sie dort mit Java, JavaScript, PHP, Python und vielen anderen Sprachen entwickeln, wobei JavaScript hier immer wichtiger wird. (Ein Grund, aus dem sich das in den Statistiken nicht niederschlägt besteht darin, dass bei der der Suche nach JavaScript häufig nicht nach Begriff JavaScript sondern z.B. nach Polyfill oder HTML5 gesucht wird.)
Anm. bezüglich java. vs. javax.
Hier wie an anderen Stellen fragen Sie sich vielleicht, was der Unterschied zwischen Klassenbibliotheken mit dem Präfix java. und denen mit dem Präfix javax. ist. Eine eindeutige Antwort ist nicht zu finden. Letztlich scheint es sich um ein Überbleibsel aus der Anfangszeit von Java zu handeln. Demnach wären Bibliotheken unter java. Teil des eigentlichen Kerns der Sprache und Bibliotheken unter javax. Erweiterungen dazu. Aber ob das tatsächlich zutrifft können wir leider nicht beantworten, denn selbst die ansonsten sehr gute Dokumentation (die „Java API“) gibt hierzu keine Auskunft.
11.2.	Bestandteile einer GUI in Java
Wie in HTML konstruieren Sie in den JFC Ansichten aus Elementen, die Containern genannt werden.
-	Top-Level-Container sind die Grundlage jeder Ansicht in Java: Alles, was gemeinsam angezeigt werden soll muss als Teil eines solchen Containers einprogrammiert werden. Top-Level-Container werden u.a. mit der Klasse javax.swing.JFrame bereitgestellt, die Sie deshalb für GUI-Klassen importieren können: Die meisten GUIs in Java sind eine Erweiterung der Klasse JFrame.

Zwei Top-Level-Container, mit denen wir uns nicht beschäftigen werden sind JApplet und JWindow. Instanzen von JApplet sind dafür gedacht, in Browsern genutzt zu werden, die ein Java-Plug-in nutzen. Und das ist eher selten der Fall, also macht es wenig Sinn, wenn wir hier darüber reden. Instanzen von JWindow unterscheiden Sich von JFrame-Instanzen dadurch, dass Sie keinen Rahmen bzw. keine Menüleiste haben. Somit brauchen wir sie hier nicht explizit besprechen.

-	Wenn Sie später ergänzende Fenster öffnen wollen, z.B. um Nutzer zu fragen, ob die Eingaben wirklich übernommen werden sollen, nutzen Sie dazu eine Instanz des Top-Level-Containers JDialog, der ebenfalls Teil des Package javax.swing ist. Beide (JFrame und JDialog) werden größtenteils identisch programmiert.

Es gibt einen entscheidenden Unterschied, den Sie sich merken müssen: JFrames können unabhängig von anderen Top-Level-Containern gestartet werden, ein JDialog ist dagegen immer abhängig von einem anderen Top-Level-Container: Wird dieser geschlossen, dann werden automatisch alle JDialog-Instanzen geschlossen, die „zu ihm gehören“. Und um die Verwirrung zu maximieren ist es möglich, eine GUI direkt als JDialog ohne eine zugehörige JFrame-Instanz zu programmieren.

-	Daneben gibt es noch nicht-Top-Level-Container, die wir erst später besprechen, weil sie vorrangig der Strukturierung von GUIs dienen.

-	Komponenten sind in Java Elemente, die innerhalb eines Containers genutzt werden können, um den Inhalt und das Aussehen einer GUI festzulegen. 

o	Komponenten erzeugen Sie, indem Sie eine Instanz einer der Klassen des Package java.swing erzeugen und es dann als Argument mit der Funktion add() an einen Top-Level-Container übergeben.

o	Jeder Top-Level-Container verfügt über die sogenannten Panes. Wenn Sie einem Top-Level-Container ein Objekt per add() hinzufügen, dann wird dieses tatsächlich nicht als Attribut der Instanz, sondern in einer seiner Panes abgelegt. Auf die Panes können Sie über getter- und setter-Methoden (z.B. von JFrame) zugreifen. Im Regelfall nutzen Sie Panes als anonyme Objekte und rufen, nachdem Sie auf sie per get...Pane() zugreifen, direkt eine Setter-Methode auf, um die Eigenschaften dieser Pane zu ändern. 

Hier ein Beispiel:

	Die ContentPane ist der Teil jedes Top-Level-Containers, der z.B. die Hintergrundfarbe einer GUI festlegt. Wenn Sie die Farbe des Hintergrunds einer GUI in Weiß ändern wollen, tun Sie das schlicht per getContentPane().setBackground(Color.white); wobei sie zuvor die Klasse java.awt.Color importieren müssen.

o	In diesem Kurs werden wir nur selten auf die Panes zugreifen. Dennoch sollten Sie wissen, dass sie existieren, weil Sie erst so verstehen, wie genau eine GUI in Java strukturiert ist. Und nur so können Sie sich später intensiv mit dem Thema beschäftigen.

-	Child und Parent sind Begriffe, die die Beziehung zwischen zwei Objekten einer GUI anzeigen. Ein Parent ist der Container, in dem das Child enthalten ist.
Anm.: Da in der Java-API im Package java.awt zwei abstrakte Superklassen mit den Namen Container und Component existieren und Component die Superklasse von Container ist, wird in Programmierbüchern zu Java häufig allgemein über beide als Komponenten gesprochen. Im Rahmen dieses Skripts vermeide ich diese Verallgemeinerung, um jeweils deutlich zu machen, ob wir gerade über Strukturlemente einer GUI (Container) oder über die bedienbaren Elemente (Komponenten) reden oder über etwas, das für beide gilt.
Ein zweites Konzept, das Sie bei der Entwicklung von GUIs verstehen müssen sind Events. Wir werden uns später ausführlich damit beschäftigen, aber zum Verständnis ist es gut, wenn Sie den Begriff schon jetzt kennen lernen.
Ein Event ist so etwas wie eine Nachricht, die kurz an einem schwarzen Brett aufgehängt wird. Ob jemand diese Nachricht liest und darauf reagiert ist unabhängig davon, dass es sie gibt. Das gleiche gilt für Events und das kennen Sie auch schon: Denken Sie an das letzte Mal, als Sie die linke Maustaste gedrückt haben und nichts passiert ist: Der Druck auf die Maustaste hat im Rechner ein Event erzeugt, aber in diesem Moment gab es keine Programmkomponente, die auf das Event reagiert hat. Events werden aus Sicht der Javaprogrammierung vom Betriebssystem erzeugt.
Ein Event kann beispielsweise die Informationen enthalten, dass die linke Maustaste gedrückt wird. Ein anderes Event ist die Information, dass der Mauspfeil sich an Position 250 x 910 des Displays befindet.
Das führt direkt zum Gegenstück von Events, den sogenannten Listenern. Ein Listener ist (um beim obigen Beispiel zu bleiben) jemand, der die ganze Zeit auf das Schwarze Brett starrt und bei einer ganz bestimmten Notiz reagiert. Um das kleine Beispiel mit dem MausEvent zu nehmen: Ein Listener könnte so programmiert sein, dass er dann reagiert, wenn die linke Maustaste losgelassen wird, während sich der Mauszeiger im Bereich einer Schaltfläche befindet. 
Bitte machen Sie sich das folgende klar: Es gibt wenigstens zwei Gründe, aus denen ein Programm nicht auf eine Nutzereingabe reagiert: Entweder wurde es gar nicht darauf programmiert oder das Betriebssystem ist gerade so ausgelastet, dass die Erzeugung des Events nicht stattfindet. Wenn Sie später Programme entwickeln, die über ein Netzwerk (wie das Internet) Daten austauschen, kommt damit eine dritte Störquelle hinzu. Denn wie Sie in der Veranstaltung Netzwerke und Internetsicherheit lernen, fallen ständig Teile von Datenübertragungen über Netzwerke aus, auch wenn Normaluser davon nichts merken.
Details zu Events und Listenern besprechen wir in einer Woche, vorerst sehen wir uns an, was Sie tun müssen, um eine GUI zu erzeugen.
11.3.	Programmierung einer „inhaltsleeren“ GUI
(Inhaltsleer steht hier in Anführungszeichen, weil jeder Top-Level-Container wie besprochen bereits die Panes beinhaltet.)
Um eine GUI in Java zu erzeugen, müssen Sie die folgenden Schritte durführen, die sie in einer eigenen Klasse programmieren:
-	Importieren Sie javax.swing.* und java.awt.*.

-	Programmieren Sie Ihre Klasse als Erweiterung von JFrame. 

Sie können natürlich auch innerhalb eines anderen Programms einfach ein JFrame-Objekt instanzieren, das Sie dann nutzen, um so ein anschaulicheres Beispiel zu haben. Beachten Sie aber bitte, dass dann die folgenden Beschreibungen nicht vollständig mit dem übereinstimmen, was Sie sehen werden.

-	Ihr Konstruktor beginnt mit super(String title), wobei title ein beliebiges String-Objekt ist, das als Titel der GUI verwendet werden wird.

-	In der main()-Methode erzeugen Sie eine Instanz Ihrer von JFrame abgeleiteten Klasse.

-	Dann legen Sie mittels der Methoden setSize(int x, int y) und setLocation(int x, int y) fest, wie groß das Fenster ist und an welcher Position des Bildschirms es erzeugt (aber noch nicht angezeigt) wird.

-	Das Fenster wird erst dann angezeigt, wenn Sie es mit dem Methodenaufruf setVisible(true) sichtbar machen.
Aufgaben:
-	Programmieren Sie eine solche inhaltsleere GUI und prüfen Sie, was passiert, wenn Sie einzelne Methodenaufrufe auskommentieren.

-	Starten Sie Ihr Programm aus der Kommandozeile. Beenden Sie es dann, indem Sie die entsprechende Schaltfläche der GUI anwählen. (Unter Windows das weiße X auf rotem Grund) Wenn Sie jetzt einen Blick in das Fenster mit der Kommandozeile werfen, werden Sie sehen, dass der GUI-Prozess immer noch läuft, denn dort können Sie keine neuen Anweisungen eingeben. Was meinen Sie ist der Grund dafür? 

Keine Sorge: Wenn Sie keine Veranstaltung zu Betriebssystemen oder zur Programmierung von Nebenläufigkeit besucht haben, dann werden Sie es nicht wissen. Aber eine Vermutung können Sie ja anstellen.
11.4.	Struktur von Top-Level Containern
Bislang haben Sie im Regelfall Programme in Java entwickelt, bei denen Sie wussten, wie und wo Instanzen von Objekten erzeugt wurden. Bei einer GUI gibt es jedoch eine Struktur, die vieles für Sie reguliert. Wenn Sie diese Struktur nicht kennen, dann können Sie nicht nachvollziehen, warum bestimmte Dinge so passieren, wie das der Fall ist. Dann ist Ihnen beispielsweise nicht klar, warum Sie mal etwas mit add() hinzufügen können und mal mit Gettern und Settern wie bei getContentPane().setColor() arbeiten müssen. Deshalb gehen wir in den folgenden Abschnitten auf genau diese Struktur ein.
Hier sei nochmal betont, dass alle in diesen Abschnitten beschriebenen Panes (stellen Sie sich darunter so etwas wie Container vor) erzeugt werden, wenn Sie einen Top-Level Container wie JFrame erzeugen. Wenn Sie also versuchen, eine dieser Panes selbst zu erzeugen, dann wird dann schlicht daran scheitern, dass sie bereits existieren. Fortgeschrittene Java-Entwickler tun das dagegen gelegentlich, wenn sie eine Pane ersetzen wollen. Dazu gehören allerdings Kenntnisse, die wir hier nicht besprechen.
11.4.1.	Die Containment Hierarchy
Mit diesem Begriff wird die grundlegende Abhängigkeit von Elementen einer GUI bezeichnet. In einfachen Worten: Es geht darum, welcher Container/welche Komponente in welchem anderen Container steckt.
Und hier müssen Sie sich eines merken: Jede Komponente und jeder Container kann nur Teil genau eines Containers sein. Wenn Sie also eine Schaltfläche oder eine andere Komponente erzeugen, dann können Sie diese nicht mehrfach verwenden. Sollten Sie also einen Bestätigen-Button programmieren und ihn im Quellcode an mehreren Stellen hinzufügen , dann wird die Schaltfläche tatsächlich nur an der Stelle eingefügt, an der Sie im Quellcode zuerst oder zuletzt hinzugefügt wurde.
Weiter oben wurde bereits darauf hingewiesen: Jede JFrame-Instanz verfügt über eine sogenannte Content-Pane, die z.B. die Hintergrundfarbe der GUI festlegt. Schauen wir uns einmal die Panes an, die jeder Top-Level-Container beinhaltet:
11.4.2.	Die Root Pane
Sie ist die strukturelle Grundlage jedes JFrame-Objekts. Innerhalb dieser Veranstaltung greifen wir nicht darauf zurück, da wir Ihnen vorrangig erklären wollen, wie Sie eine benutzbare GUI entwickeln können. Dazu widerum fügen Sie Komponenten und Container zu den übrigen Panes hinzu oder legen deren Layout fest. 
Die RootPane-Methoden sind dagegen getter und setter mit denen Sie jeweils auf die Eigenschaften der übrigen Panes zugreifen. Wenn Sie also später detailliert das Look-and-Feel Ihrer GUIs programmieren wollen, weil Sie beispielsweise der Meinung sind, ein einfaches x auf rotem Grund sei kein gutes Symbol für eine Schaltfläche zum Schließen eines Fensters, dann müssen Sie sich mit der RootPane auseinander setzen.
11.4.3.	Die Layered Pane
Diese Pane bietet spannende Möglichkeiten, um die Anordnung von Komponenten einer GUI individueller zu gestalten. Während Sie in diesem Kurs lernen, wie Sie sichtbare Komponenten auf einer Fläche nebeneinander anordnen können, bietet die JLayeredPane Ihnen Möglichkeiten, um sie vor- und hintereinander anzuordnen. Die JLayeredPane ist bereits vorkonfiguriert und sorgt z.B. dafür, dass ein Menü, das geöffnet wird vor anderen Komponenten einer GUI angezeigt wird. 
Gerade wenn Sie neue Möglichkeiten der GUI-Gestaltung untersuchen wollen, könnte diese Pane für Sie spannend werden.
11.4.4.	Die Content Pane
Im Rahmen dieses Kurses werden wir uns lediglich mit dieser Pane und der MenuBar beschäftigen. Alle Elemente, die in einem Top-Level-Container angezeigt werden, werden in einen dieser beiden Container eingefügt.
Beide werden im Layered Pane auf der default Ebene abgelegt. Das bedeutet, dass Änderungen an der Content Pane sich NICHT auf die Menu Bar auswirken (und umgekehrt).
11.4.5.	Die Glass Pane
Während Sie mit der Layered Pane die Anordnung von Komponenten in der Tiefe ändern können, können Sie mit der Glass Pane Bereiche der GUI abdecken und Events beeinflussen. Damit bietet die Glass Pane Ihnen sehr weitreichende Möglichkeiten. Ein veränderter Mauszeiger im Bereich einer Schaltfläche wäre da nur ein einfaches Beispiel.
Wie in Bezug auf die Layered Pane gilt auch in Bezug auf die Glass Pane, dass Sie sich hiermit auseinander setzen sollten, wenn Sie die Entwicklung von Java-GUIs vertiefen wollen. In diesem Kurs lassen wir sie außen vor.
11.5.	Nachtrag zur leeren GUI
Nachdem Sie nun wissen, woraus eine GUI besteht, lassen Sie uns noch zwei Konfigurationsschritte durchführen.
Da wäre zum einen die ausstehende Antwort auf die Frage, warum die Kommandozeile nicht wieder erscheint, nachdem wir das Fenster geschlossen haben. Diese Antwort hat etwas mit Nebenläufigkeit zu tun: Wenn wir ein Java-Programm starten, das ein Fenster öffnet, dann wird dafür im Betriebssystem ein Prozess gestartet. Damit dieser Prozess aber wieder beendet wird, müssen wir bei der Instanzierung des Top-Level-Containers (hier war das das JFrame-Objekt) einstellen, dass mit dem Schließen des Fensters auch der Prozess beendet werden soll. 
Aufgabe:
-	Dazu müssen wir noch die folgende Zeile ergänzen:

o	setDefaultCloseOperation(JFrame.EXIT_ON_CLOSE);

-	Zum anderen möchten Sie vielleicht die Hintergrundfarbe anpassen. Dazu ergänzen Sie einfach die folgenden Zeilen an der jeweils passenden Stelle:

o	import java.awt.Color;
o	getContentPane().setBackground(Color.WHITE);
Jetzt haben Sie eine GUI, die Sie nach Lust und Laune mit Bedienelementen und Inhalten füllen können.
Aufgabe:
-	Warum macht es Sinn, den Prozess der GUI nicht zu beenden, wenn das Fenster geschlossen ist?
11.6.	Einfügen von Komponenten
Wie besprochen erzeugen Sie Komponenten, indem Sie eine Instanz der entsprechenden Klasse aus der Klassenbibliothek von javax.swing erzeugen, konfigurieren und sie der GUI hinzufügen. Dafür nutzen Sie schlicht die Methode add() des Top-Level-Containers, auch wenn Sie dafür genauso den Aufruf getContentPane().add() verwenden können.
Sollte das nötig werden, dann können Sie eine Komponente mit der Methode remove() wieder entfernen.
Bevor wir uns eine Auswahl von Komponenten ansehen werden, sollten Sie zunächst einige Methoden kennen lernen, die alle Komponenten von Ihrer Superklasse erben:
-	void setBorder(Border) und Border getBorder() können genutzt werden, um den Rahmen einer Komponente festzulegen bzw. das entsprechende Border-Objekt abzufragen. Genau wie mit der Color-Klasse werden wir uns nicht weiter mit der Border-Klasse beschäftigen. Recherchieren Sie im Bedarfsfall, welche Möglichkeiten sie Ihnen bietet.

-	void setForeground(Color) und void setBackground(Color) sind nur scheinbar eindeutig; während setBackground() den Hintergrund färbt, legt setForeground() im Regelfall die Schriftfarbe fest. Color getForeground() und Color getBackground() sind dagegen selbsterklärend.

-	void setOpaque(boolean) und boolean isOpaque() legen fest, ob der Hintergrund der Komponente verdeckt oder transparent sein soll, bzw. geben an, ob das der Fall ist. (false entspricht einem transparenten Hintergrund) Alternativ können sie auch void setOpacity(float) verwenden. Der Wert muss zwischen 0.0 und 1.0 liegen. Dabei steht 0.0 für vollständige transparenz.

Ausblick: Weiterhin gibt es noch Getter und Setter für Font und Cursor sowie eine Reihe Methoden, die für Events und Listener wichtig sind. Dann wären da noch Methoden fürs Painting, die wichtig werden, wenn Sie graphische Elemente einer GUI individuell darstellen oder aktualisieren wollen. Aber auch das ist noch längst nicht alles. Für den Anfang soll es aber genügen.
11.6.1.	Textkomponenten
Die Klassenbibliothek javax.swing.text.* bietet mehrere Klassen an, die wir nutzen können, um verschiedenen Komponenten zu nutzen, mit denen längere Texte angezeigt und/oder geändert werden können. Wie gewohnt stellen wir hier nur eine Auswahl vor.
11.6.1.1.	JTextField
kann verwendet werden, wenn Nutzer nur kurze Texte wie einen Benutzernamen eingeben sollen. Wenn Sie bei der Instanzierung einen int-Wert übergeben, legen Sie damit die Länge des Feldes fest.
11.6.1.2.	JPasswordField
ist eine Unterklasse von JTextField. Im Gegensatz zu diesem maskiert es die Eingabe. Bitte beachten Sie, dass das nur Schutz gegen neugierige Kollegen bietet; handelsübliche Angriffssoftware kann weiterhin Tastatureingaben auslesen, da sie nicht die Anzeige, sondern die Tastaturanschläge ausliest.
11.6.1.3.	JTextArea
ist geeignet, wenn Nutzer längere Texte eingeben sollen.
Diese drei Felder sind für die Fälle einsetzbar, in denen ein Nutzer keine Vorgaben bekommen soll, mit denen seine Eingabe beschränkt wird. Auswahlmöglichkeiten folgen nach einem kurzen Einschub.
Aufgabe:
-	Erweitern Sie Ihre GUI um eine JTextArea mit rotem Hintergrund und gelber Schriftfarbe.

-	Probieren Sie aus, welche Möglichkeiten Sie bei der Nutzung dieses Feldes neben der reinen Texteingabe und –löschung Sie haben. (Copy, Paste usw.)
11.6.1.4.	JLabel
Neben den genannten Textfeldern gibt es die Klasse JLabel, die direkt von swing erbt und dafür gedacht ist, Beschriftungen zu ermöglichen. Ein Label kann sowohl Text als auch eine Grafikdatei sein.
Ein JLabel muss mit einem Argument instanziiert werden. Das kann ein String oder eine Grafikdatei sein, die als Instanz vom Typ Icon erzeugt wurde.
JLabel kennt die folgenden Methoden:
-	void setIcon(Icon) und void setText(String) sind selbsterklärend.

-	Mit setHorizontalAlignment(SwingConstant) können Sie festlegen, wie das Label ausgerichtet wird, wenn der Layout Manager des Parent-Containers das unterstützt: 
o	SwingConstant.LEFT
o	SwingConstant.RIGHT
o	SwingConstant.CENTER

-	Mit setHorizontalAlignment(SwingConstant) können Sie außerdem festlegen, in welcher Reihenfolge Grafik und Text innerhalb des Labels angeordnet werden:
o	SwingConstant.LEADING bewirkt, dass der Text links vom Bild angeordnet wird.
o	SwingConstant.TRAILING bewirkt das Gegenteil.
Nun wäre es naheliegend, nach Methoden wie setVerticalAlignment(SwingConstant) zu suchen und die existiert auch. Hier heißen die zusätzlichen Konstanten TOP und BOTTOM.
Hinweis: Es gibt weitere Komponenten, die diese Methoden unterstützen. Wir werden jeweils nicht darauf eingehen, schlagen Sie ggf. in der Java-API nach, ob es der Fall ist oder nicht.
Aufgabe:
-	Probieren wir das gleich aus. Erzeugen Sie ein JLabel, dem Sie bei der Instanzierung den String „Textfeld“ übergeben.

-	Fügen Sie dann dieses Objekt Ihrer GUI hinzu.

-	Macht es einen Unterschied, in welcher Reihenfolge Sie die Komponenten Ihrem Quellcode hinzufügen? Wenn ja, welcher Unterschied ist das und was können Sie daraus schlussfolgern.
Wenn Sie sehr testfreudig sind, werden Sie feststellen, dass die Textarea keine Scrollbalken besitzt. Scrollbalken gehören in den Bereich der Container. Die werden wir erst dann besprechen, wenn Sie eine Sammlung von Komponenten kennen gelernt haben, mit denen Sie die meisten Nutzerinteraktionen programmieren können.
11.7.	Einschub zu Layouts
Sie haben jetzt gelernt , dass wir nicht einfach beliebige Komponenten in eine GUI einfügen können, denn im schlimmsten Fall bewirkt das, dass einzelne dieser Komponenten gar nicht angezeigt werden. Das ist ein erster wichtiger Unterschied gegenüber HTML in Verbindung mit CSS: Wir müssen von vorneherein ein Layout festlegen.
Das Layout legen wir mit der JFrame-Methode void setLayout(Layout) fest. Als Argument übergeben wir hier einen Layout-Manager. Layout-Manager sind Klassen, die die Anordnung von Elementen in der GUI kontrollieren.
Layout-Manager sehen wir uns in Kürze im Detail an.
Aufgabe:
-	Legen Sie eine Instanz von FlowLayout als Layout-Manager für Ihre GUI fest.

-	Probieren Sie aus, was jetzt passiert, wenn Sie einen Text eingeben, der länger ist als es die Breite des Fensters zulässt.

-	Wie wirkt sich jetzt eine Änderung der Reihenfolge der beiden add()-Methodenaufrufe aus?

-	Legen Sie jetzt schwarz als neue Hintergrundfarbe für Ihre GUI fest und probieren Sie dann aus, ob und wenn ja wie es sich auswirkt, wenn Sie mittels setBackground() und setOpaque() für Ihre Komponenten unterschiedliche Hintergrundfarben transparent und nicht-transparent einstellen.
Sie sehen jetzt, dass die GUI wirklich nicht besonders gut aussieht, aber zumindest alle Komponenten enthält, die wir bislang programmiert haben. Damit können wir weitermachen.
11.8.	Mehr Komponenten
Wichtig: Bevor wir uns im Detail einige Komponenten ansehen, hier ein Hinweis, damit Ihre GUI-Klassen übersichtlich bleiben: 
Gehen Sie systematisch vor und kommentieren Sie! 
GUI-Klassen bestehen aus einer Vielzahl an Objekten, deren Zusammenhang meist nur schwer erkennbar ist. Sehen Sie sich dazu einmal die Beispiele an, die in den offiziellen Tutorien zu finden sind; selbst einfache GUIs haben schnell mehrere hundert Zeilen Code; ohne Kommentare ist es da nur schwer, sich zurecht zu finden.
-	Instanzieren Sie wenn möglich alle Top-Level-Container zuerst, dann alle Container und dann alle Komponenten. 

-	Kommentieren Sie alle Bereiche, sobald Sie sie programmieren, damit es für einen anderen Entwickler leicht ist zu erkennen, welche Teile Ihres Quellcodes für welche Teile der GUI verantwortlich sind.

-	Führen Sie dann die Konfiguration jeder einzelnen Instanz durch.

-	Kombinieren Sie erst danach die GUI aus den konfigurierten Objekten zusammen.

-	Die letzte Zeile enthält setVisible(true).
An dieser Stelle sei nochmal darauf hingewiesen, dass wir uns für den Augenblick nur damit beschäftigen, welche Elemente Teil unserer GUI werden. Das Aussehen dieser Elemente und ihre Funktionalität klären wir später.
Im Sinne der objektorientierten Softwareentwicklung entspricht dieses Vorgehen dem sogenannten Model-View-Controller Pattern (kurz MVC): 
-	Das Modell umfasst lediglich die Programmierung dessen, was angezeigt werden soll. 

-	Der View (Ansicht) umfasst ausschließlich das Aussehen dieser Elemente.

-	Und der Controller umfasst die eigentliche Programmlogik.
Bei der Entwicklung einer GUI müssen wir den Controller nochmal unterteilen: Zum einen wäre da die Programmlogik, die wir auch ohne eine GUI entwickeln. Hier geht es also um das, was das eigentliche Programm tut. Zum anderen wäre da die Logik, die wir als Event-Handling bezeichnen können. Hiermit meinen wir im Rahmen dieses Kurses das Abfragen von Events durch Listener, die dann Änderungen in der Ansicht oder Methodenaufrufe des eigentlichen Programms durchführen.
Fürs erste erfahren Sie hier also nicht, wie Sie eine Schaltfläche so programmieren können, dass dadurch etwas passiert; Sie erfahren vorerst nur, was Sie tun müssen, um eine Schaltfläche in der GUI anzeigen zu lassen.
11.8.1.	Auswahlmöglichkeiten
Es gibt eine Reihe an Möglichkeiten, Nutzer aus einer Menge an Möglichkeiten wählen zu lassen:
-	Es muss genau eine Zahl oder ein Begriff aus einer Menge ausgewählt werden.

-	Es darf eine beliebige Menge an Einträgen (also auch keiner) aus einer vorgegebenen Menge ausgewählt werden.

Anm.: In Java wird leider bei Menüs zwischen zwei Arten unterschieden, die im Grunde den gleichen Aufbau haben: Menüs, die innerhalb der GUI-„Fläche“ angezeigt werden, sind z.B. Instanzen von JList und Menüs, die als Teil der Menüleiste angezeigt werden, sind Instanzen von JMenu.
11.8.1.1.	Ausgewählte Methoden für Auswahlmöglichkeiten
Zusätzlich zu den Methoden, die Sie für Instanzen von JLabel kennen, können Sie die Anzeige von Auswahlmöglichkeiten noch durch die folgenden Methoden anpassen:
-	void doClick() simuliert, dass eine Komponente per Klick auf die linke Maustaste angewählt wurde.

-	boolean isSelected() gibt zurück, ob eine Komponente aktiviert ist. void setSelected(boolean) aktiviert oder deaktiviert sie.

-	setEnabled(boolean) stellt ein, ob eine Komponente angewählt werden kann oder nicht.

-	Zusätzlich zu den beiden set...Alignment()-Methoden kommen nun noch setHorizontalTextPosition(SwingConstant) und setVerticalTextPosition(SwingConstant) hinzu, die den Text im Verhältnis zum Icon anordnen.
Wichtig: Wenn Sie sich jetzt Sorgen machen, Sie müssten all diese Methodenaufrufe auswendig können, dann seien Sie beruhigt; bei der objektorientieren Softwareentwicklung ändern sich Bezeichnungen ständig. Wichtig ist, dass Sie die Methodik verstehen und umsetzen können, nicht dass Sie auswendig lernen. Investieren Sie also Ihre Zeit vorrangig darin, zu programmieren und nicht darin, Methoden- und Konstantennamen auswendig zu lernen.
11.8.1.2.	Entweder-Oder – Der JToggleButton
Wenn ein Nutzer sich zwischen zwei gegenteiligen Optionen entscheiden soll, ist ein JToggleButton die einfachste Lösung: Entweder der Schalter ist aktiviert oder deaktiviert, andere Möglichkeiten gibt es nicht.
11.8.1.3.	Entweder-Oder, jetzt als Quadrat mit Häkchen – JCheckBox
Die zweite Möglichkeit für diese Fälle ist eine JCheckBox. Programmiertechnisch funktioniert Sie genau wie ein JToggleButton (nicht zuletzt, weil dieser die Superklasse von JCheckBox ist), nur die Darstellung ist anders: Anstelle einer Schaltfläche erscheint ein Quadrat. Ist die JCheckBox aktiviert, dann erscheint in diesem Quadrat ein Häkchen.
11.8.1.4.	Entweder-Oder, jetzt als Kreis mit Füllung – JRadioButton
Als dritte Möglichkeit für Entweder-Oder-Entscheidungen hätten wir die JRadioButtons. 
11.8.1.5.	Wähle eines aus vielen – JRadioButton-Gruppe und JList
Allerdings gibt es einen Unterschied zwischen JCheckBox und JRadioButton: Wenn Sie (was wir noch nicht besprochen haben) mehrere JCheckBoxen gruppieren, dann können Nutzer beliebig viele davon aktivieren und deaktivieren. Bei JRadioButtons dagegen bewirkt die Gruppierung, dass nur genau eine Option aktiviert werden kann.
Eine solche Gruppe erhalten wir, indem wir eine Instanz von ButtonGroup erzeugen und dann alle Instanzen von JRadioButton mittels der Methode add() zu der Instanz von ButtonGroup hinzufügen.
Aufgabe:
-	Was denken Sie: Müssen wir nur die Instanz von ButtonGroup zu unserer GUI hinzufügen oder nur die einzelnen JRadioButtons oder alle?

-	Was wäre die logische Konsequenz jeder der drei Varianten basierend auf dem, was Sie bislang hier gelernt haben? Erörtern Sie das für und wider aller drei Varianten. Finden Sie die Lösung, die in Java angewendet wird logisch, nachvollziehbar oder unsinnig?

-	Warum macht diese Übung Sinn?
Kommen wir zur zweiten Variante, um Nutzer eine Option aus mehreren wählen zu lassen: Die JList<C> ist, wie Sie sich denken können eine generische Klasse. Im Gegensatz zu eine JRadioButton-Gruppe erfordert es wesentlich weniger Programmieraufwand, sie zu erzeugen: Anstelle für jeden Eintrag eine Instanz zu erzeugen, genügen die folgenden drei Zeilen:
C[] elem = { ... };    // Hier wird ein Array der Elemente erzeugt, die in der JList 
// angezeigt werden sollen.
JList<C> list = new JList<C>(elem);     // Hier wird die JList-Instanz mit den Einträgen erzeugt.
Beispiel: Erzeugung einer JList<C>
Anschließend muss die JList wie gewohnt zur GUI hinzugefügt werden.
Lassen Sie sich hier nicht davon irritieren, wenn nur ein Teil der Einträge angezeigt werden und andere scheinbar nicht erreichbar sind. Das liegt wieder daran, dass wir uns momentan zwar die Komponenten einer GUI ansehen, ihre Darstellung aber auf später verschieben.
Für Fortgeschrittene: Es ist ebenfalls möglich, eine JList zu implementieren, deren Einträge während der Laufzeit dynamisch geändert werden können. Dazu müssen Sie ein Objekt als Argument an den Konstruktor von JList übergeben, das ListModel<E> implementiert. Im Rahmen dieses Kurses werden wir darauf nicht eingehen.
11.8.1.6.	Drop-Down-Menu – JComboBox
Eine weitere Möglichkeit, Nutzer eines von mehreren Elementen auswählen zu lassen ist die sogenannten Kombinationsbox. Diese zeigt jeweils nur den aktuell ausgewählten Eintrag der Liste sowie eine Schaltfläche, um die übrigen Menüeinträge anzeigen zu lassen.
Bis auf die Bezeichnung der Klasse ist die Instanziierung einer JComboBox und das Hinzufügen zur GUI mit der einer JList identisch. Allerdings können Sie über die Methode void addItem(C) jederzeit Einträge hinzufügen oder sie mittels void removeItem(C) entfernen.
Aufgabe:
Stellen Sie eine Tabelle auf, in der Sie die Vor- und Nachteile von JToggleButton, JCheckBox, JRadioButton, JList und JComboBox im Verhältnis zueinander notieren.
11.8.1.7.	Anzeige und Interaktion mit umfangreichen Daten
An dieser Stelle sei noch kurz auf zwei Komponenten verwiesen, die Sie bei der Darstellung umfangreicher Datenmengen nutzen können: Wenn Sie eine Baumstruktur wie ein Dateiverzeichnis anzeigen wollen, sehen Sie sich die Klasse JTree an und für Tabellen ist JTable nützlich.
11.8.1.8.	Schieberegler – JSlider
Anstelle der Eingabe einer frei wählbaren Zahl können Sie auch einen Schieberegler programmieren, mit dem Sie Nutzern die Möglichkeit anbieten, einen Wert aus einem Intervall auszuwählen. Dabei können Sie auch festlegen, ob der Wert innerhalb des Intervalls in festen Schritten (z.B. nur ganze Zahlen, nur Zahlen, die durch ½ teilbar sind, usw.) oder frei wählbar sein soll.
Einen Slider initialisieren Sie mit JSlider(int min, int max, int value), wobei min und max die Grenzen des Intervalls darstellen und value der Wert ist, der am Anfang gewählt werden soll. Diesen können Sie weglassen. Wenn Sie einen JSlider ohne Intervallgrenzen instanziieren, so erhält er die Grenzen von 0 bis 100.
Mit der Methode int getValue() erhalten Sie den aktuellen Wert des Sliders.
11.8.1.9.	JButton
Abschließend sei noch auf die Klasse JButton verwiesen, mit der Sie eine Schaltfläche auf der GUI einblenden können, die einen Text und oder eine Grafik beinhalten kann. Da sie inzwischen die Erzeugung und Einbindung von Komponenten kennen, sei an dieser Stelle einfach allgemein auf die Java-API verwiesen.
11.9.	Der erste nicht-Top-Level Container – Die JScrollPane
Kommen wir jetzt zum ersten Container, den Sie nutzen können, um Komponenten anders darzustellen. Sie haben bereits einige Komponenten, die derart groß werden können, dass sie nicht mehr innerhalb des Top-Level Containers vollständig dargestellt werden können. Die Lösung dazu kennen Sie aus der täglichen Nutzung von Rechnern: Scrollbalken dienen dazu, einen Ausschnitt eines Bereichs zu verschieben, der nicht vollständig abgebildet werden kann.
In Java gibt es dafür den Container JScrollPane.
Nachdem Sie eine Komponente (im folgenden Codebeispiel einfach comp genannt) erzeugt haben, für die Sie Scrollbalken anzeigen lassen möchten, fügen Sie diese wie folgt einer JScrollPane hinzu: (comp steht hier für die Komponente, der Sie Scrollbalken hinzufügen wollen.)
JScrollPane scrollpane = new JScrollPane(comp, constantVertical, constantHorizontal);
Beispiel: Quellcode für das Einfügen von Scrollbalken
Anstelle der Variablen constantVertical und constantHorizontal müssen Sie nun noch jeweils eine Konstante der Klasse ScrollPaneConstants einfügen:
-	ScrollPaneConstants.VERTICAL_SCROLLBAR_AS_NEEDED

-	ScrollPaneConstants.VERTICAL_SCROLLBAR_ALWAYS

-	ScrollPaneConstants.VERTICAL_SCROLLBAR_NEVER
Die Bedeutung der drei Konstanten sollte klar sein; für den horizontalen Bildlauf gibt es die gleichen Konstanten. Dort beginnen sie mit HORIZONTAL statt VERTICAL.
Aufgabe:
-	Fügen Sie der TextArea einen senkrechten Scrollbalken hinzu.
Wichtig: Wenn Sie sich hier schwer tun, dann beachten Sie bitte, dass Sie bei der GUI-Programmierung die JTextArea der JScrollPane hinzufügen müssen, um Scrollbalken zum Textbereich in der GUI hinzuzufügen. In anderen Worten: Die Programmierlogik ist genau umgekehrt gegenüber der Gestaltungslogik oder der Wahrnehmung von Nutzern. Oder nochmal anders: Die Wahrnehmung ist hier das Gegenteil der Realität.
11.10.	Wahrnehmung versus Realität
Das ist ein guter Moment um etwas zum Thema Wahrnehmung zu sagen. Es gibt unterschiedliche Interpretationen dieses Begriffs. Hier verstehen wir Wahrnehmung als die Interpretation dessen, was ein Mensch mit seinen Sinnen perzipiert. Im Gegensatz dazu wird beispielsweise in der Veranstaltung „Wahrnehmung“ besprochen, wie Filme unterschwellige Botschaften übermitteln können, wobei Kameraperspektive, Schattenfall, Gruppierung von Personen im Bild und viele andere audiovisuelle Aspekte genutzt werden.
Zurück zur Wahrnehmung, wie Sie hier verstanden werden sollte. Dazu zunächst der eindeutige Begriff der Perzeption: Unter Perzeption wird beispielsweise die Fähigkeit des Auges verstanden, bestimmte Farben als Reize an das Gehirn weiterzuleiten. Im Gehirn spielen sich dann verschiedene Prozesse ab, die letztlich dafür sorgen, dass jeder von uns einen individuellen Eindruck davon hat, wie die Umgebung aussieht. Das ist hier mit Wahrnehmung gemeint. Dabei spielen fast ausschließlich Dinge eine Rolle wie zum Beispiel unsere individuelle Einstellung gegenüber verschiedenen Farben. Diese Einstellung ist für jeden von uns im Regelfall so umfassend, dass wir gar nicht begreifen, dass es sich um unsere individuelle Interpretation handelt. Deshalb sagen wir auch praktisch nie: „Ich finde das Wetter schön.“ Stattdessen sagen wir in aller Regel: „Es ist ein schöner Morgen.“
Das Problem mit Wahrnehmung in diesem Sinne ist, dass wissenschaftliche Forschung in diesem Bereich kaum möglich ist. So ist es zwar möglich, durch Forschungsgruppen empirisch zu klären, ob eine gewisse Farbzusammenstellung Betrachter eher anzieht als abstößt, aber die eigentlichen Prozesse, die sich dabei im Gehirn abspielen können leider nicht eindeutig gemessen werden. Deshalb sind die Ergebnisse von Studien auch häufig nur beschränkt, aber dafür innerhalb dieses beschränkten Bereichs aussagekräftig. 
Unabhängig davon müssen Sie sich bewusst sein, dass Sie einen Fehler machen, wenn Sie betonen, dass Ihre GUI doch genau richtig programmiert ist; Sie werten dann Ihre GUI entsprechend Ihrer individuellen Wahrnehmung und ignorieren damit, dass es keine eindeutig richtige oder falsche Wahrnehmung gibt. Und das gilt auch dann noch, wenn Sie sich bei einigen Leuten ein Feedback zu Ihrer GUI einholen, denn: Wahrnehmung ist immer individuell.
Wichtig: Wenn Sie denken, dass dieser Abschnitt irrelevant für die Programmierung von GUIs ist, dann wechseln Sie bitte in eine Ausbildung zum Fachinformatiker/zur Fachinformatikerin; Sie sind an einer Hochschule falsch. Allerdings ist die Auseinandersetzung mit diesem Bereich ein zentraler Bereich eines Designstudiums, somit müssen Sie sich zwar der Bedeutung von Wahrnehmung bewusst sein, aber die Einbeziehung der Wahrnehmung in die Gestaltung einer nutzerfreundlichen GUI ist Aufgabe von Designern. (Daraus resultiert auch das Motto Form Follows Function.)
11.11.	Layout Manager
Um überhaupt alle Elemente einer GUI anzeigen zu lassen haben wir oben das FlowLayout genutzt. Das ordnet alle Komponenten schlicht in Form einer endlosen Zeile an und fügt bei Bedarf Zeilenumbrüche ein. In der Einleitung haben Sie aber gelesen, dass die Qualität einer GUI davon abhängt, wie gut Sie von Nutzern bedient werden kann, die keine Schulung und kein Handbuch über die Bedienung haben. Also ist das FlowLayout nur eine Notlösung. In diesem Abschnitt lernen Sie deshalb Layout-Manager kennen, die mehr leisten als nur bei Bedarf Zeilenumbrüche zu integrieren.
11.11.1.	Gruppierung, Orientierung und Ausrichtung
Bevor wir hier auf die Vor- und Nachteile eingehen müssen wir drei Begriffe klären, die bei allen Layouts eine Rolle spielen:
Wie schon weiter oben angesprochen ist es wichtig, dass logisch zusammengehörige Komponenten auch in der GUI zusammen angezeigt werden. Das wird mit dem Begriff Gruppierung bezeichnet. Hier wird in gestalterischer Hinsicht noch zwischen impliziter und expliziter Gruppierung unterschieden. Explizit meint hier alles, was sichtbar ist, wie ein Rahmen oder einer Hintergrundfarbe für eine einzelne Gruppe. Implizit meint dagegen alle Varianten, bei denen eine Gruppe eher subtil erkennbar ist. Eine tabellarische Anordnung ohne sichtbaren Rahmen wäre ein Beispiel.
Bei der Orientierung geht es darum, wie das Display bei Nutzern ausgerichtet ist: Vertikal oder Horizontal. Leider werden Nutzeroberflächen meist nur für eine Orientierung entwickelt. Das ist deshalb von Nachteil, weil so entweder bei Nutzern von Rechnern oder bei Nutzern von Smartphones unschöne Balken zu sehen sind. Noch unschöner ist es dagegen, wenn die Orientierung bei der Entwicklung gänzlich ignoriert wird. Das Ergebnis sind dann beispielsweise GUIs, die Texte auf der kompletten Breite eines 20“-Displays anzeigen. Und die sind nur schlecht lesbar. Die beiden Orientierungen werden häufig als Landscape (horizontal) und Portrait (vertikal) bezeichnet.
Die Orientierung ist nicht mit der Ausrichtung zu verwechseln, denn wenn es um die Ausrichtung geht (engl. alignment), dann reden wir darüber, wie Elemente innerhalb von Containern angeordnet sind. Wie Sie schon gesehen haben müssen für die vertikale und die horizontale Ausrichtung teilweise unterschiedliche Methoden verwendet werden. Die Bezeichnungen hier sind in aller Regel top, bottom, left, right und center. Aber auch hier müssen Sie in aller Regel (auch das haben Sie schon gesehen) auf Konstanten einer Superklasse zugreifen, sodass Sie je nach Komponente unterschiedliche Konstanten für die gleiche Ausrichtung verwenden müssen. einheitliche Alignment-Konstanten (also so etwas wie Alignment.LEFT) gibt es dagegen nicht.
11.11.2.	Standard-Layouts
Leider gibt es kein Standard-Layout für alle Container, sondern jeder Container hat ein eigenes Standard-Layout. Natürlich können Sie diese Layouts auswendig lernen, aber hier empfehle ich Ihnen eher, einfach für jede GUI ein Layout festzulegen. Wenn das dann das Standard-Layout war, haben Sie zwar eine zusätzliche Programmzeile, aber Fehler können Sie dadurch nicht machen. Und außerdem ist so leichter erkennbar, welches Layout Sie verwenden.
Wichtig: Wenn Ihnen ein einzelnes Layout nicht genügt, dann können Sie über Container-Klassen weitere Layouts in Teilbereichen einer GUI realisieren. Dafür können Sie beispielsweise eine Instanz von JPanel (besprechen wir noch) nutzen. Fühlen Sie sich deshalb bitte nicht zu beschränkt durch die Möglichkeiten der folgenden Layouts.
In anderen Worten: Sie können Container beliebig in Containern verschachteln. Denken Sie dazu vielleicht an die Matroshka-Puppen, aber denken sie deshalb nicht, dass ein Container innerhalb eines anderen Containers genauso aufgebaut sein muss wie derjenige, in dem er steckt. Für Freunde der großartigen britischen Serie Dr. Who dazu ein Zitat, dass sie sicher kennen: „Es ist innen größer als außen.“
11.11.3.	Das Border Layout: 5 Bereiche
Das BorderLayout teilt die GUI in fünf Bereiche auf: Mitte, links, rechts, oben und unten. Sie fügen zwar eine Komponente wie gewohnt mit add() hinzu, aber die Konstanten haben größtenteils die Namen von Himmelsrichtungen:
-	BorderLayout.CENTER sollte selbsterklärend sein.

-	BorderLayout.WEST entspricht der Konstante LEFT bei anderen Alignments.

-	BorderLayout.EAST entspricht der Konstante RIGHT bei anderen Alignments.

-	BorderLayout.NORTH entspricht der Konstante TOP bei anderen Alignments.

-	BorderLayout.SOUTH entspricht der Konstante BOTTOM bei anderen Alignments.
11.11.4.	Nachtrag zum FlowLayout
Bei der Instanziierung des FlowLayouts können Sie eine Ausrichtung als Argument übergeben:
-	FlowLayout.LEFT, .RIGHT und .CENTER geben jeweils an, von wo aus die Komponenten in jeder Zeile angeordnet werden.
11.11.5.	GridLayout – Tabellarischer Aufbau mit gleichgroßen Zellen
Das GridLayout entspricht vom Aufbau her einer Tabelle: Instanziierung per GridLayout(int rows, int columns). Wenn nun Elemente hinzugefügt werden, dann werden sie bis zum Ende jeder Zeile Spalte für Spalte in der Reihenfolge eingeordnet, in der Sie mittels add() im Quellcode hinzugefügt werden. Anschließend geht es in der jeweils nächsten Zeile weiter. 
Die einzelnen Zellen der „Tabelle“ (die ohne Rahmen angezeigt wird) sind so groß wie die größte Komponente, die im GridLayout angezeigt wird. Werden mehr Elemente hinzugefügt, als das Grid Zellen hat, dann werden automatisch Spalten hinzugefügt.
11.11.6.	GridBagLayout – Tabellarischer Aufbau mit Zellen individueller Größe
Das GridBagLayout verspricht, dass Größe der Zellen im Gegensatz zu den Zellen des GridLayouts jeweils individuell konfiguriert werden können. Bei genauem Hinsehen wird aber schnell deutlich, dass das so nicht zutreffen kann, sondern dass lediglich jede Spalte und jede Zeile eine individuelle Größe haben kann. Das ist im Regelfall sinnvoller als der Einsatz des GridLayout. Dafür ist die Konfiguration eines GridBagLayouts komplexer als die eines GridLayouts. Werfen Sie bei Interesse einen Blick in die Java-API und die offiziellen Java-Tutorials.
11.12.	JPanel – „Zwischen“-Container für Komponenten
Um einen nutzbare GUI zu entwickeln genügt es im Grunde, wenn Sie einen Top-Level-Container mit einem Layout-Manager konfigurieren und diesem verschiedene Komponenten zuordnen. Doch gerade wenn Sie in Abhängigkeit vom Log-In von Nutzern ggf. verschiedene Komponenten ein- oder ausblenden wollen, ohne dass sich die Darstellung der GUI ansonsten ändert, dann macht es Sinn, wenn Sie Komponenten zunächst in Instanzen von JPanel einordnen und erst diese Instanzen zur GUI hinzufügen. Betrachten Sie Instanzen der Klasse JPanel einfach als unsichtbare Flächen, dann sollten Sie sie problemlos nutzen können.
An dieser Stelle sei noch auf die Methode setSize(int width, int height) verwiesen, mit der Sie jeder Komponente (und damit auch ein JPanel) eine feste Größe zuordnen können.
11.13.	Menüleiste, Quellcodebeispiele und weitere Aufgaben
In diesem Kapitel haben wir uns nicht mit Menüleisten auseinander gesetzt. Hierfür, sowie für umfangreiche Beispiele mit Quellcode sowie Aufgaben zum Thema möchte ich Sie auf die Folien von Prof. Plaß verweisen.
12.	Events – Wenn Unerwartetes passiert
Im letzten Kapitel dieses Skripts haben Sie gelernt, was eine GUI ist und wie Sie mit Containern und Komponenten eine GUI aufbauen können. Damit eine GUI aber Ihren Zweck erfüllt, der schlicht darin besteht, Nutzer auf komfortable Art und Weise bei der Bedienung einer Software zu unterstützen, muss sie auf Änderungen der Programmvariablen und auf Nutzereigaben reagieren.
Das wird mit Events und Listenern realisiert. In Java gibt es dazu entsprechende Klassen. Im Gegensatz zu dem, was Sie bislang gelernt haben programmieren Sie dafür aber keine Event-Klasse, sondern nutzen Event-Objekte, die scheinbar aus dem Nichts entstehen. Jedes Event-Objekt zeigt an, dass eine bestimmte Nutzerinteraktion stattgefunden hat.
Auch wenn wir das momentan nicht weiter ausnutzen werden sollten Sie sich merken, dass Events nicht nur die Position des Mauspfeils oder das Drücken einer Taste auf der Tastatur sein können. Auch Signale, die über ein Netzwerk an einen Rechner übermittelt werden, werden als Events bezeichnet. Wenn Sie das Eventhandling (Verarbeitung von Events durch ein Programm) beherrschen, können Sie also noch ganz andere Dinge programmieren als „nur“ die Verarbeitung von Nutzereingaben. Und wenn Sie Eventhandling umfassend verstanden haben, dann verstehen Sie damit eines der zentralen Konzepte von Betriebssystemen. 
In der Veranstaltung Informatik 3 werden Sie dieses Konzept unter einem anderen Namen kennen lernen. Was Ihnen dort als Interrupt begegnen wird ist genau das, was auf technischer Ebene Events auslöst. Sie werden dort auch lernen, wie Sie Interrupts programmieren. Doch lassen wir das für den Moment, schließlich sollen Sie ja erst einmal lernen, wie Sie mit Events umgehen können.
Was Sie programmieren müssen, um Events zu verarbeiten, sind die sogenannten Listener, die allesamt Interfaces sind. Sie werden in diesem Kapitel lernen, wie Sie solche Listener programmieren und in Ihre GUI-Klassen integrieren müssen, um dafür zu sorgen, dass Nutzereingaben verarbeitet werden.
12.1.	Events in Java
Sie wollen wissen, wie Sie ein Event in Java programmieren sollen? 
Kurze Antwort: Gar nicht. 
Lange Antwort: Da Events aus der Sicht eines Java-Programms vom Betriebssystem erzeugt werden, also quasi aus dem Nichts auftauchen, brauchen wir uns um ihre Erzeugung keine Gedanken zu machen: Wir programmieren einfach für jedes Event, das uns wichtig ist einen Listener und das wars.
Antwort für Fortgeschrittene: Wenn Sie bereits mit Testfällen bzw. den sogenannten Unit-Tests gearbeitet haben, dann können Sie mithilfe der Event-Klassen eine Nutzerinteraktion simulieren.
12.2.	Listener in Java
Um einen Listener zu programmieren , müssen wir für jedes Event, auf das unsere GUI reagieren soll drei Dinge tun:
-	Implementierung eines Listener-Interfaces. (Dazu müssen Sie java.awt.event.* importieren.)

-	Hinzufügen dieses Listeners mit der entsprechenden add...Listener(...Listener)-Methode, die für alle Komponenten definiert ist.
Diese zwei Schritte sind sehr einfach und mit einer Ausnahme haben Sie im Grunde jetzt alles verstanden, was Sie für die Programmierung einer GUI wissen müssen. Dieser Spezialfall, wird über die sogenannten inneren Klassen abgewickelt, mit denen wir uns in diesem Kapitel etwas später im Detail beschäftigen. Für den Moment merken Sie sich bitte einfach, dass eine innere Klasse eine Klasse ist, die genau wie die Methoden einer Klasse innerhalb des Klassenrumpfes einer anderen Klasse steht.
12.3.	Ein einfacher Listener für eine Schaltfläche
Stellen Sie sich vor, Sie wollen die GUI für einen Getränkeautomaten programmieren. Wenn die Schaltfläche für den Auswurf von Nutzern angewählt wird, soll ein Kaffee ausgeschenkt werden. Banale Kontrollstrukturen wie „Wurde Geld eingeworfen?“ lassen wir hier außen vor. Nun dauert es bei solchen Automaten in aller Regel einige Sekunde, bis die tatsächliche Ausgabe des Getränks beginnt. Sicherheitshalber sollte die Taste nach einmaligem Druck also für einige Sekunden deaktiviert werden.
Um diese Aufgabe zu lösen müssen Sie tatsächlich zwei Listener programmieren: Einen, der den Tastendruck registriert, um dann die Schaltfläche zu deaktivieren und einen, der in Abhängigkeit von einem Timer die Schaltfläche wieder reaktiviert.
Das zweite Event können wir erst dann programmieren, wenn wir uns mit Timern beschäftigt haben, was in vier Wochen passiert. Das erste Event dagegen ist recht einfach umzusetzen:
Aufgabe:
-	Erweitern Sie eine Ihrer GUIs um den JButton schalter, der mit „Drück schon!“ beschriftet ist.

-	Importieren Sie java.awt.event.* in Ihre GUI.

-	Rufen Sie die Methode addActionListener(ActionListener) von schalter auf.

-	Anstelle einer Variablen vom Typ ActionListener übergeben Sie der Methode eine anonyme Instanz vom Typ ActionListener.

-	Programmieren Sie in diese anonyme Klasse die Methode public void actionPerformed(ActionEvent e).

-	Programmieren Sie den Aufruf schalter.setEnabled(false) in diese Methode ein. 

-	Starten Sie Ihre GUI und sehen Sie sich an, wie der neue Schalter sich verhält.
Wie gesagt können wir momentan noch keine Zeitsteuerung realisieren, weil wir uns noch nicht mit den Zeitklassen von Java beschäftigt haben. Nun soll aber die Schaltfläche wieder reaktiviert werden. Da ist die einfachste Variante einen zweiten Schalter zu programmieren und diesen an denselben Listener zu binden.
12.3.1.	Einschub: Strings-Objekte und Sonderzeichen
Da wir es hier mit Nutzeroberflächen zu tun haben ist es wichtig, dass Sonderzeichen richtig angezeigt werden. Im Gegensatz zu HTML5 können wir hier bei Java Probleme bekommen. Deshalb müssen wir uns ansehen, wie wir in Java-String-Objekten Sonderzeichen so programmieren müssen, damit sie in jedem Fall richtig angezeigt werden.
In unserem Beispiel hatten wir das ü in „Drück mich!“, das für Probleme sorgen kann. Da die Websuche nach „unicode german characters“ leider nicht die Tabelle für deutsche Sonderzeichen herausspuckt, hier die sieben wichtigsten Unicode-Folgen, damit Sie deutsche Sonderzeichen in Texten nutzen können:
-	Ä bzw. ä per \u00c4 bzw. \u00e4
-	Ö bzw. ö per \u00d6 bzw. \u00f6
-	Ü bzw. ü per \u00dc bzw. \u00fc
-	ß per \u00df
Diese Escape-Sequenzen sollten Sie also in Zukunft immer dann nutzen, wenn Sie deutsche Sonderzeichen in Textstrings verwenden wollen. Allerdings können Sie mit Unicode-Folgen auch jedes Zeichen Ihres Quellcodes eingeben, nur wäre das doch etwas seltsam und mehr Sicherheit bringt es auch nicht (beim Abspeichern werden die Zeichen ohnehin als Unicode-Sequenzen abgespeichert); nur der Quellcode ist schlechter lesbar.
Aufgabe:
-	Wenn noch nicht geschehen, ändern Sie jetzt bitte jedes ü in \u00fc um. Richtig gesehen: Damit steht da nicht mehr „Drück mich!“ sondern „Dr\u00fcck mich!“. 
Und ja, das ist ausgesprochen schlecht lesbar. Deshalb werden die Aufgaben hier auch weiterhin nicht mit unicode-Sequenzen sondern in lesbarer Form mit Sonderzeichen formuliert. Beim Programmieren müssen Sie jedoch immer an die Übersetzung in Unicode-Folgen denken.
Nach diesem Einschub wieder zurück zu unserer Aufgabe, einen zweiten Schalter zu programmieren, der die erste Schaltfläche wieder reaktiviert.
12.4.	Das ActionEvent – Unser erstes Event
Doch wie machen wir das? Der Listener kann ja nicht unterscheiden, woher das Event gekommen ist, oder? So ein Event kommt doch vom Betriebssystem, wie soll Java da etwas drüber wissen?
Es gibt zwei Möglichkeiten, dieses Problem zu lösen. Hier die einfacher zu verstehende, aber dafür nur selten einsetzbare Version. (Die allgemeine Lösung lernen Sie im Abschnitt Innere Klassen kennen.)
Ein Event wie das Anwählen einer Schaltfläche wird zwar vom Betriebssystem erzeugt und ist aus unserer Sicht als Java-Programmierer „einfach da“, aber tatsächlich handelt es sich um ein Objekt vom Typ ActionEvent. Dieser Event-Typ gibt zwar nur eine Information über das erzeugende Objekt preis, aber das genügt uns schon. (Daneben gibt es auch an, wann es erzeugt wurde, aber das soll uns hier nicht interessieren.)
Oben haben wir einen Schalter mit der Beschriftung „Drück schon!“ programmiert. Und genau dieses „Drück schon!“ erhalten wir als Rückgabewert, wenn wir innerhalb des Rumpfes von actionPerformed(ActionEvent e) den Methodenaufruf e.getActionCommand() nutzen.
Aufgabe:
-	Erweitern Sie jetzt den Rumpf von actionPerformed() wie folgt:

o	Alles, was bislang den Rumpf bildet lassen Sie dann ausführen, wenn e.getActionCommand() gleich „Drück schon!“ ist. In diesem Fall soll der Schalter den Schriftzug „ausgeschaltet“ anzeigen.

o	In den else-Zweig fügen Sie bitte Methodenaufrufe ein, um den Schalter wieder aktivierbar zu machen. Außerdem ändern Sie hier die Beschriftung der Schaltfläche wieder in „Drück schon!“ um.

-	Und nun starten Sie die GUI. Was passiert?

-	Was passiert nicht? Und warum passiert es nicht? (Die Fortgeschrittenen unter Ihnen werden es schon vor dem Programmstart erkannt haben und sich gefragt haben, warum sie das einprogrammieren sollten.)
Wenn Sie aufgepasst haben, dann wussten Sie, dass einfach deshalb nichts passiert, weil die Schaltfläche deaktiviert wurde. Somit kann der else-Zweig momentan nicht erreicht werden. Alle andern wissen es jetzt... hoffentlich.
Wir müssen es also irgendwie schaffen, in den else-Zweig zu kommen. Aber das ist doch einfach: Wir müssen die Klasse nur so ändern, dass dieser Listener von irgendeiner anderen Schaltfläche genutzt werden kann. Natürlich darf es dann keine anonyme Instanz mehr sein, aber dazu kommen wir gleich.
Aufgabe:
-	Warum darf es dann keine anonyme Klasse mehr sein?
12.4.1.	Ein oder mehrere Listener, viele Quellen
Wenn ein Objekt ein Event erzeugt (wie in unserem Fall die Schaltfläche), dann wird dieses Objekt als Datenquelle bezeichnet. Ein „Empfänger“ wird dagegen als Datensenke bezeichnet. Der Einfachheit halber bleiben wir hier bei Quelle und Senke.
Am Anfang dieses Kapitels haben Sie gelernt, dass eine Komponente nur einmal in eine GUI eingebunden werden darf. Wenn Sie sich das gemerkt haben, dann fragen Sie sich vielleicht, wie denn ein Listener an mehreren Stellen „zuhören“ soll. Aber das ist ganz einfach: Ein Listener ist keine Komponente und deshalb gilt diese Einschränkung hier nicht. Wir können also unseren ActionListener mittels addActionListener() an mehrere Komponenten binden. Wir können auch mehrere Listener an eine Komponente binden. Somit können wir für jedes Event einen Listener programmieren und binden diesen an alle Komponenten an, für die er relevant ist.
Wenn Ihnen das zu schnell ging, hier einige möglichen Fälle:
-	Wir können einen Listener an eine Quelle binden und der Listener wirkt sich auch nur auf diese Quelle aus.

-	Wir können einen Listener an eine Quelle binden und der Listener wirkt sich auf eine andere Komponente aus.

-	Wir können einen Listener an eine Quelle binden und der Listener wirkt sich auf mehrere Komponenten aus.

-	Wir können einen Listener an eine Quelle binden und der Listener wirkt sich nicht nur auf Komponenten aus, sondern ändert durch entsprechende Methodenaufrufe etwas an unserem eigentlichen Programm. (Nicht vergessen: Eine GUI ist immer nur eine komfortable Oberfläche, mit der Nutzer auf das eigentliche Programm zugreifen.)

-	Bis auf den ersten Fall ersetzen Sie jetzt bitte die Worte „an eine Quelle binden“ durch „an mehrere Quellen binden“. Damit hätten wir insgesamt sieben Möglichkeiten. 
Aufgabe:
-	Hier eine Kontrollfrage zum Verständnis von Eventhandling: Warum macht die Möglichkeit „Ein Methodenaufruf unseres Programms ändert eine Listener-Instanz“ keinen Sinn? (Das wäre die Umkehrung der vierten Möglichkeit.)
Bis auf den ersten Fall darf (wie eben schon geschrieben) der Listener keine anonyme Instanz sein. Ändern wir also zunächst unseren Quellcode, sodass der Listener an mehrere Quellen gebunden werden kann. Das bringt uns direkt zum nächsten neuen Thema:
12.5.	Innere Klassen
Denn bislang haben wir nur Klassen programmiert, die unabhängig voneinander agieren. Jetzt haben wir aber die Situation, dass die Instanz einer Klasse (gemeint ist hier die Instanz von ActionListener) auf Variablen der Instanz einer anderen Klasse zugreifen soll (gemeint ist hier die Instanz von JButton).
Wichtig: Sollten Sie bei den folgenden Aufgaben mit Fehlermeldungen auftauchen, bei denen Worte wie „... a static variable from a non-static context ...“ auftauchen, dann lesen Sie bitten den folgenden Absatz ganz genau durch. Es handelt sich hier zwar um eine Wiederholung von Inhalten, die wir schon in P1 behandelt haben, aber es ist leicht, diese Inhalte zu vergessen.
Bei der Programmierung von GUIs mit inneren Klassen müssen Sie einen Ansatz hinter sich lassen, der bei der Programmierung mit Java sehr komfortabel ist. Es handelt sich um eines der zentralen Konzepte der Objektorientierung. Dabei können Sie Klassen programmieren, die Sie dann beliebig häufig instanziieren. Dazu haben Sie wahrscheinlich ausschließlich Instanzvariablen genutzt. Das sind Variablen, die für jedes Objekt derselben Klasse individuelle Werte annehmen können. Wenn wir mit inneren Klasen arbeiten, bedeutet das dagegen, dass wir vorrangig mit Klassenvariablen arbeiten. Das sind Variablen, die für alle Instanzen einer Klasse gleich sind. Die oben genannte Fehlermeldung kommt nun dadurch zustande, dass Sie im Programmcode (wenigstens) eine Klassenvariable individuell für ein einzelnes Objekt mit einem Wert belegen und das darf nicht funktionieren.
Aufgabe:
-	Formulieren Sie in eigenen Worten (ja! schriftlich!), warum es nicht funktionieren darf, dass eine Klassenvariable für ein Objekt dieser Klasse individuell geändert wird.
Zurück zu unserem Thema der GUI-Programmierung mit inneren Klassen.
Wie Sie wissen kann jede Methode einer Klasse auf die Variablen dieser Klasse bzw. der eigenen Instanz zugreifen. Also müssten wir es „nur“ schaffen, eine Klasse zu haben, die die Besonderheit aufweist, dass die Teil einer anderen Klasse ist. Denn so können die Instanzen dieser Klasse auch auf die Variablen der anderen Klasse zugreifen, die z.B. als private deklariert sind.
Und das geht: Die sogenannten inneren Klassen werden innerhalb des Klassenrumpfes einer anderen Klasse programmiert, von dieser äußeren Klasse instanziiert und können dann auf die Variablen eben dieser äußeren Klasse zugreifen. Aber Sie erkennen es schon an der Betonung des Klassenbegriffs: Im Gegensatz zum letzten Beispiel arbeiten wir nun (fast) ausschließlich mit Klassenvariablen.
Und bevor Sie grübeln, was Sie bei inneren Klassen anders programmieren müssen als bei „normalen“ Klassen: Der einzige Unterschied ist der, dass diese als private deklariert werden müssen. Alles andere bleibt gleich.
Aufgabe:
-	Kopieren Sie die anonyme Instanziierung von ActionListener vollständig aus dem Methodenaufruf addActionListener() heraus und fügen Sie sie als private final class dem Klassenrumpf Ihrer GUI-Klasse hinzu.

-	Benennen Sie diese innere Klasse als Umschalter. (Wie gewohnt gilt: Natürlich können Sie hier eine beliebige Bezeichnung wählen, aber bleiben Sie für diese Aufgabenstellung dabei, damit Sie die folgenden Schritte so umsetze können, wie hier beschrieben.)

-	Ändern Sie jetzt den Rest des Quellcodes so ab, dass alle Instanzvariablen zu Klassenvariablen werden. Am Ende befindet sich dann in der main()-Methode nur noch die Instanziierung der GUI.

-	Vergessen Sie dabei nicht, dass Sie das Argument von addActionListener() noch ändern müssen. Schließlich soll hier eine Instanz von Umschalter verwendet werden.
12.6.	Zwischenstand
Sie haben jetzt zwei Varianten kennen gelernt, um Eventhandling, also die Verarbeitung von Nutzereingaben zu verarbeiten:
-	Wenn ein Event nur durch eine einzelne Komponente erzeugt wird und „dort“ direkt verarbeitet werden soll, können Sie sie als anonyme Klasse instanziieren. Für die Fehlerkorrektur ist das allerdings unübersichtlich.

-	Wenn wenigstens ein Event von mehr als einer Komponente erzeugt oder auf eine andere Komponente zugreifen kann, dann müssen wir den Listener als innere Klasse programmieren. Das bedeutet aber auch, dass wir die GUI effektiv über Klassenvariablen programmieren.
Aufgabe:
-	Beschreiben Sie erneut in eigenen Worten für beide eben genannten Fälle (ein Event und mehrere Komponenten / ein Event wird von anderer Komponente erzeugt als die auf die es sich auswirkt) individuell:
o	Warum muss ein Listener dann als innere Klasse programmiert werden? 
o	Und warum müssen dann die Instanzvariablen der GUI-Klasse zu Klassenvariablen „umgewandelt“ werden?
Der Rest dieses Kapitels besteht aus einer Übersicht darüber, welchen Listener Sie für welche Komponenten verwenden müssen. Wie schon oben geschrieben fügen Sie sie mit Methoden hinzu, die alle mit dem Wort add gefolgt vom Klassennamen des Listeners beginne und wie alle Methoden mit den üblichen runden Klammern enden. Das Argument ist dann der Listener bzw. eine Instanz einer Klasse, die diesen Listener erweitert.
Für die Programmierung beliebiger Listener stehen in Java 17 Event-Klassen und 18 Listener-Klassen zur Verfügung. In den folgenden Abschnitten können Sie jeweils aus unterschiedlichen Perspektiven nachsehen, welchen Listener bzw. welche Listener-Methode Sie am besten nutzen können.
12.7.	Listener-Typen nach Komponenten
In den meisten Fällen werden Sie das Eventhandling über eine der folgenden drei Methoden an eine Komponente anbinden:
-	addActionListener(ActionListener l)
dieser wird dann aufgerufen, wenn Nutzer irgendeine Interaktion mit der Komponente durchgeführt haben.

-	addChangeListener(ChangeListener l)
dieser wird dann aufgerufen, wenn sich der Wert einer Komponente geändert hat.

-	addItemListener(ItemListener l)
dieser wird dann aufgerufen, wenn eine Komponente aktiviert oder deaktiviert wird.
12.8.	Listener-Methoden generell
Wir kommen jetzt zu einer Vielzahl an Methoden, die Sie implementieren können, um die Reaktion der GUI auf Events zu steuern.
Da Sie bei jeder dieser Methoden über das Argument erkennen können, in welchem Listener Sie sie implementieren müssen, sind diese nicht explizit aufgeführt. Und da aus dem Listener jeweils folgt, mit welcher add...Listener(...Listener)-Methode Sie ihn an eine Komponente binden müssen, gilt hier dasselbe. Es kann allerdings sein, dass aus der Beschreibung nicht eindeutig hervorgeht, für welche Komponenten dieser Listener definiert ist.
Methoden, die nur intern verwendet werden (also nicht für die Programmierung einer GUI durch einen Softwareentwickler gedacht sind) sind hier aus naheliegenden Gründen nicht aufgeführt.
12.8.1.	Methoden für eine Vielzahl von Events
actionPerformed(ActionEvent e)
kennen Sie bereits: Wenn Nutzer eine Interaktion z.B. mit einem JButton durchführt, wird actionPerformed() aufgerufen.
12.8.2.	Methoden, die direkt mit der Verwendung der GUI durch Nutzer zu tun haben
focusGained(FocusEvent e) und focusLost(FocusEvent e)
werden dann aufgerufen, wenn Nutzer ein Element z.B. per Tabulatortaste oder Maus anwählen. Der Unterschied zu itemStateChanged() ist nur scheinbar subtil und muss beachtet werden.
windowGainedFocus(WindowEvent e) und windowLostFocus(WindowEvent e)
gelten wie die beiden focus...()-Methoden aber für Top-Level-Container anstelle von Komponenten.
itemStateChanged(ItemEvent e) und stateChanged(ChangeEvent e)
wird dann aufgerufen, wenn Nutzer ein Element einer Komponente bzw. eine Komponente aktivieren oder deaktivieren.
stateChanged() tritt also in den Fällen auf, in denen z.B. ein JToggleButton, eine JCheckBox o.ä. aktiviert bzw. deaktiviert wird. itemStateChanged() tritt dagegen dann auf, wenn z.B. ein Element einer Liste wie der JList aktiviert oder deaktiviert wird.
inputMethodTextChanged(InputMethodEvent e) und textValueChanged(TextEvent e)
treten auf, wenn Nutzer eingegebenen Text geändert haben.
keyPressed(KeyEvent e), keyReleased(KeyEvent e) und keyTyped(KeyEvent e)
können genutzt werden, um zu erfassen, welche Taste(n) Nutzer gedrückt haben, bzw. wann sie getippt und wieder losgelassen werden, während der Fokus auf einer Komponente liegt.
Fortgeschrittene Entwickler könnten keyTyped() z.B. nutzen, um Sonderzeichen bei der Eingabe zu erkennen und diese durch die entsprechend Unicode-Sequenz zu ersetzen, um so Fehler bei der Datenübertragung oder -anzeige zu vermeiden.
mouseClicked(MouseEvent e), mousePressed(MouseEvent e) und mouseReleased(MouseEvent e)
werden aufgerufen, wenn eine Maustaste gedrückt, noch immer gehalten oder wieder losgelassen wird. (Die erste und dritte Methode wird also einmalig, mousePressed() dagegen kontinuierlich ausgelöst. Behalten Sie das im Hinterkopf, damit Sie Ihre GUI nicht durch ausbremsen.) Welche Maustaste das ist, lässt sich über bestimmte Konstanten ermitteln. Werfen Sie im Bedarfsfall einen Blick in die Java API zu MouseEvents.
mouseEntered(MouseEvent e) und mouseExited(MouseEvent e)
geben an, ob der Mauspfeil in den Bereich der Komponente oder daraus heraus gesteuert wurde. 
mouseDragged(MouseEvent e) und mouseMoved(MouseEvent e)
(Sonderfall: Hier heißt der Listener MouseMotionListener und NICHT MouseListener)
Zwar werden beide Methoden durch die Bewegung des Mauspfeils im Bereich einer Komponente ausgelöst, aber mouseDragged() nur dann, wenn gleichzeitig eine Maustaste gedrückt ist, mouseMoved() nur dann, wenn dabei KEINE Maustaste gedrückt ist.
mouseWheelMoved(MouseWheelEvent e)
wird naheliegender Weise aufgerufen, wenn Nutzer das Mausrad bewegen.
12.8.3.	Methoden, um Änderungen an der GUI zu verfolgen
adjustmentValueChanged(AdjustmentEvent e)
wird aufgerufen, wenn z.B. die Größe einer GUI geändert wird.
componentAdded(ContainerEvent e) und componentRemoved(ContainerEvent e)
treten naheliegenderweise dann auf, wenn eine Komponente zur GUI hinzugefügt oder daraus entfernt wird.
componentHidden(ComponentEvent e) sowie componentMoved(ComponentEvent e), componentResized(ComponentEvent e) und componentShown(ComponentEvent e)
werden jeweils dann aufgerufen, wenn eine Komponente ein- oder ausgeblendet bzw. bewegt oder verformt wird.
windowActivated(WindowEvent e), windowDeactivated(WindowEvent e) sowie windowOpened(WindowEvent e), windowClosing(WindowEvent e), windowClosed(WindowEvent e) und windowIconified(WindowEvent e), windowDeiconified(WindowEvent e)
bzw. windowStateChanged(WindowEvent e)
werden Sie im Regelfall nicht nutzen. Hier geht es um Methodenaufrufe, die z.B. dann durchgeführt werden, wenn Nutzer die drei Tasten zum Minimieren, Maximieren oder Schließen eines Fensters verwenden.
12.8.4.	Sonstige Methoden
eventDispatched(AWTEvent e)
hat etwas mit nebenläufiger Programmierung zu tun.
12.9.	Die 17 Java-Events und 18 Listener
Zur Erinnerung: Alle Events und Listener sind Klassen bzw. Interfaces des Pakets java.awt.event. Diese Auflistung soll Ihnen nur als Nachschlagewerk dienen, verfallen Sie bitte nicht auf die Idee, hier etwas auswendig lernen zu wollen.
12.9.1.	MouseEvent – Alles rund um die Maus
MouseEvents fassen mehrere Fälle zusammen:
-	Der Mauspfeil wird in eine Komponente bewegt oder aus dieser heraus.
-	Der Mauspfeil befindet sich im Bereich einer Komponente.
-	Eine Maustaste wird gedrückt oder losgelassen oder gedrückt und wieder losgelassen.
Wie Sie sehen sind MouseEvents in JButtons also detaillierter als ActionEvents, mit denen lediglich signalisiert wird, dass eine Schaltfläche angewählt wird.
Es gibt zu diesen Events zwei verschiedene Listener: MouseMotionListener müssen (!) Sie nutzen, wenn es um Events geht, die etwas mit der Bewegung der Maus zu tun haben. In allen anderen Fällen benötigen Sie die MouseListener.
Tutorial: https://docs.oracle.com/javase/tutorial/uiswing/events/mouselistener.html 
Tutorial: https://docs.oracle.com/javase/tutorial/uiswing/events/mousemotionlistener.html 
12.9.2.	ItemEvent – Wenn Nutzer etwas anwählen, um es zu aktivieren
Sie kennen z.B. mit Menüeinträgen Elemente einer GUI, die zwar von Nutzern ausgewählt werden, um Funktionen der GUI aufzurufen, die aber nicht dazu da sind, um tatsächliche Eingaben durchzuführen.
Tutorial: https://docs.oracle.com/javase/tutorial/uiswing/events/itemlistener.html 
12.9.3.	FocusEvent – Wenn Nutzer etwas anwählen, ohne dabei Eingaben durchzuführen
FocusEvents zeigen an, dass Nutzer z.B. eine TextArea angewählt haben, ohne dort Text einzutragen. Hier geht es also nicht um die Frage, welche Elemente Nutzer gerade bedienen, um Eingaben durchzuführen, sondern welche Komponenten und Container der GUI sie verwenden.
Tutorial: https://docs.oracle.com/javase/tutorial/uiswing/events/focuslistener.html
12.9.4.	Sonstige Events
-	ComponentEvent und ContainerEvent sind die Superklassen aller GUI-bezogenen Events. 
-	InputEvents sind die Superklassen aller Events, die durch Nutzereingaben erzeugt werden. 
-	HierarchyEvents werden intern verwendet, um anzuzeigen, wie einzelne Komponenten und Container voneinander abhängen. 
-	PaintEvents werden ebenfalls intern verwendet, Sie sind die Ursache dafür, dass Nutzereingaben bei der Grafikbearbeitung so angezeigt werden, wie Nutzer das erwarten. 
-	InvocationEvents stehen in Beziehung mit nebenläufiger Programmierung.
-	InputMethodEvents haben leider eine missverständliche Bezeichnung, da sie Informationen darüber bereitstellen, welche Texte Nutzer z.B. in einer TextArea eingeben. Da für uns im Regelfall nur von Belang ist, welcher Text in einem Textfeld enthalten ist und wir dafür eine Methode haben, können wir InputMethodEvents ignorieren.
-	Ähnliches gilt für TextEvent, das uns schlicht anzeigt, dass eine Änderung an einem Text stattgefunden hat.
-	Ein KeyEvent tritt jedesmal auf, wenn Nutzer eine Taste drücken.
-	MouseWheelEvents treten bei der Nutzung des Mausrades auf.
-	WindowEvents zeigen an, dass ein Fenster sich geändert hat. Diese Events werden intern verarbeitet.
12.10.	Zusammenfassung
Bis zu diesem Kapitel haben Sie in Java Programmabläufe programmiert, indem Sie Instanzen von Klassen instanziiert haben und dann Methoden dieser Instanzen mit Argumenten aufgerufen haben.
Sie haben jetzt gelernt, dass Sie eine interaktive GUI entwickeln, indem Sie Klassen programmieren, die als Listener bezeichnet werden. In diesen Klassen müssen Sie wie gewohnt Methoden implementiert, deren Namen jedoch fest vorgegeben sind. Der Methodenaufruf wird dann aber nicht wie gewohnt durchgeführt. Vielmehr werden Implementierungen dieser Listener mithilfe von add...Listener()-Methoden an die entsprechenden Komponenten der GUI gebunden. Der eigentliche Methodenaufruf wird dagegen nicht von Ihnen programmiert, sondern automatisch jedes Mal durchgeführt, wenn das entsprechende Event eintritt.
Dazu müssen Sie
-	die Container und Komponenten als Klassenvariablen programmieren,

-	dann Listener als innere Klassen programmieren, die die verschiedenen Eingaben von Nutzern erfassen, die wichtig sein können, indem Sie die relevanten Methoden der Listener auf die entsprechenden Komponenten der GUI zugreifen lassen,

-	und danach diese Listener mit der passenden add...Listener()-Methode an die Komponente binden, durch die Sie ausgelöst werden sollen.
oder indem Sie für einzelne Komponenten Listenern als anonyme Klassen programmieren. Nur wenn es möglich ist, alle Listener als anonyme Klassen zu implementieren, macht es Sinn, das so zu tun.
Hausaufgabe 1:
-	Am Ende von Kapitel 2.3 ist die Rede von einer zweiten Schaltfläche, über die die Schaltfläche mit der Beschriftung „Drück schon!“ wieder aktiviert werden kann. Programmieren Sie diese zweite Schaltfläche in Ihre GUI.

-	Diese Schaltfläche soll die erste Schaltfläche wieder aktiviert werden und dafür sorgen, dass sie die Beschriftung „Drück schon!“ anzeigt.

-	Dazu darf kein zusätzlicher Listener implementiert werden; das Deaktivieren und Reaktivieren muss über den selben Listener realisiert werden.

-	Wenn Sie gut mitgearbeitet haben, sind Sie mit der reinen Programmierung in zwei Minuten fertig. Also lassen Sie sich ruhig Zeit.
Bitte bearbeiten Sie außerdem die Hausaufgaben 2 – 5.b von Prof. Plaß.
Nachtrag zu Kapitel 9 – Errata und häufig gestellte Fragen
In der letzten Veranstaltung gab es einige Punkte, zu denen ich noch ein paar Hinweise geben möchte, bzw. die häufig angesprochen wurden:
Klassenvariablen, Instanzvariablen und mehr zur Variablendeklaration
Hier hatte ich mich darauf konzentriert, Ihnen zu erklären, worin der Unterschied liegt und wann Sie ihn in Ihrem Quellcode ignoriert haben. Allerdings hatte ich dabei vergessen darauf hinzuweisen, dass Sie Klassenvariablen (genau wie Klassenmethoden) mit dem Schlüsselwort static deklarieren.
Wenn Sie also Variablen im Klassenrumpf deklarieren, sind sie NICHT automatisch Klassenvariablen, sondern nur und ausschließlich dann, wenn Sie sie zusätzlich mit dem Schlüsselwort static deklarieren.
Daneben war einigen unklar, was es mit dem Schlüsselwort final auf sich hat. Wenn Sie etwas in Java als final deklarieren, dann machen Sie es damit zu etwas, das einer Konstanten sehr ähnelt: Der Wert einer final Variable darf nur einmal festgelegt und dann nicht mehr geändert werden, eine final Methode darf nicht von einer erbenden Klasse überschrieben werden, usw.
Wenn also eine Java-Fehlermeldung besagt, dass Sie eine Variable als final deklarieren müssen, dann bedeutet das, dass diese sich nicht ändern darf.
Unerwartete Fehlermeldungen
Zum Teil kam es zu Fehlermeldungen im Bezug auf ein JButton-Objekt, an das ein ActionListener „angebunden“ werden sollte. Die Ursache war in allen Fällen ein Konfigurationsfehler, durch den Ihre IDE auf ein „altes Java“ also eine JDK für Java 7 oder früher zugriff, um Ihren Quellcode zu kompilieren. Hier gibt es eine Vielzahl an Ursachen.
Variante a: Wenn Sie Ihre alten Java-Programme weiter nutzen wollen, dann sollten Sie die alte JDK bzw. die alte IDE nicht deinstallieren. Stattdessen können Sie folgendes tun: Installieren Sie zusätzlich (wenn noch nicht geschehen) das JDK 8 (bzw. 1.8). Anschließend installieren Sie Eclipse in ein neues Verzeichnis, um die bestehende Installation nicht zu überschreiben. Achten Sie außerdem darauf, dass Sie ein neues workspace-Verzeichnis anlegen, um eine strikte Trennung zwischen bisheriger und neuer Installation zu erreichen. Wenn Sie eine andere IDE als Eclipse nutzen, ist nicht sicher, ob dieses Verfahren so funktioniert. Eclipse hat gegenüber vielen Programmen (zumindest unter Windows) den Vorteil, dass es in einem eigenständigen Verzeichnis installiert wird, in dem auch die gesamte Konfiguration gespeichert wird. Im Zweifelsfall können Sie also diverse Eclipse-Installationen auf Ihrem System jeweils in einem eigenen Verzeichnis betreiben, die allesamt vollständig unabhängig voneinander sind. Abgesehen von der Verschwendung von Speicherplatz hat das keinen Nachteil. Fortgeschrittene ändern je nach Bedarf die Konfiguration, aber es dauert natürlich ein wenig, um da alle Bereiche zu kennen.
Variante b: Wenn es Ihnen nicht wichtig ist, dass Ihre alten Programme weiter laufen und Sie Eclipse für keine andere Programmiersprache nutzen, dann deinstallieren Sie zunächst Eclipse und dann sämtliche Java-Installationen. Starten Sie nun den Rechner neu und installieren Sie dann zuerst das JDK für Java 1.8 und danach eine IDE.
Variante c (vorrangig für Linux): Es gibt für einige Programmiersprachen sogenannte Versionsmanager. Wenn Sie eine solche Software nutzen, dann können Sie darüber komfortabel zwischen den verschiedenen Versionen von Programmiersprachen wechseln. Alternativ dazu könnten Sie auch ein Skript erstellen, um den Wechsel zwischen den Versionen zu realisieren.


13.	Mausereignisse
Wir kommen hier nochmal zum Themengebiet Events und Listener. Wie im entsprechenden Kapitel beschrieben gibt es in Java für Events, die durch Mausinteraktionen von Nutzern erzeugt werden können drei Listener, den MouseListener, den MouseMotionListener und den MouseWheelListener. Doch obwohl sie unterschiedliche Namen haben, werden die beiden ersten mit der Methode addMouseListener() an ein Element der GUI gebunden, während es für den MouseWheelListener die Methode addMouseWheelListener() gibt. Um diese Details zu umschiffen werden wir uns dagegen die abstrakte Klasse MouseAdapter ansehen, die alle drei Listener zusammenfasst.
Kurz gesagt: Obwohl Events durch Mausinteraktionen also genau nach dem gleichen Prinzip verarbeitet werden wie alle anderen Events, sind Sie nicht ganz so simpel zu programmieren: Wir können bzw. müssen drei verschiedene Listener dafür nutzen, wenn wir nicht den Umweg über die abstrakte Klasse MouseAdapter wählen.
In anderen Kursen werden Sie zunächst die beiden konkreten Listener (MouseListener und MouseMotionListener) kennen lernen und dann erfahren, welche Methoden die jeweils enthalten. Da Sie in den meisten Fällen aber beide benötigen und der MouseWheelListener natürlich auch nicht zu verachten ist, überspringe ich das und stelle Ihnen eine Vorgehensweise vor, die immer funktioniert und das Ganze vereinfacht. Außerdem müssen Sie beim MouseAdapter nicht alle Methoden überschreiben, sondern nur diejenigen, die Ihnen wichtig sind. Damit sparen Sie sich Arbeit und vermeiden unnötige Fehlermöglichkeiten.
13.1.	Die abstrakte Klasse MouseAdapter
Wie Sie (hoffentlich) noch wissen, sind abstrakte Klassen Klassen, die Sie über extends in Form einer eigenständigen Klasse erweitern müssen, um nutzbare Klassen zu erhalten.
Das bedeutet, dass Sie für den von mir vorgeschlagenen Weg eine Klasse z.B. mit MyMouseListener extends MouseAdapter programmieren müssen. Nennen Sie sie jedoch keinesfalls MouseWriter, da MouseAdapter bereits die Listener-Klasse MouseListener verwendet. Auch wenn es am Namen nicht zu erkennen ist, ist die abstrakte Klasse MouseAdapter tatsächlich eine abstrakte Listener-Klasse. Wir erhalten mit MyMouseListener also einen vollwertigen MouseListener, den wir mittels container.addMouseListener(new MyMouseListener()); an ein beliebiges Container-Objekt anbinden können.
Beachten Sie bitte noch Folgendes: Es gibt keine unterschiedlichen Events für unterschiedliche Maustasten, dafür (wie auch z.B. fürs Zählen der Mausklicks) gibt es aber Methoden, die in der Klasse MouseEvent und MouseWheelEvent implementiert sind. Wir werden uns diese Fälle am Ende des Kapitels kurz ansehen.
Wichtig: Denken Sie aber bitte in allen Fällen daran, dass Sie jeden Listener an genau einen Container binden. Wenn Nutzer die Maus aus dem Bereich eines Containers bewegen, in dessen Bereich sie die Maustaste gedrückt haben, dann wird das Loslassen der Maustaste im Regelfall nicht (!) an diesen Container übermittelt. Das ist ein Sonderfall, den Sie bei keinem Listener beachten mussten, mit dem Sie bislang zu tun hatten.
13.2.	Wenn eine Maustaste von Nutzern genutzt wurde
Hier müssen wir zwischen vier Fällen unterscheiden:
-	Wenn eine Aktion erfolgen soll, nachdem Nutzer die Maustaste gedrückt und (!) wieder losgelassen haben, können wir das entsprechende Verhalten in 
void mouseClicked(MouseEvent d) implementieren.

Wichtig: mouseClicked() wird unter Umständen nicht aufgerufen, wenn Nutzer gleichzeitig die Maus bewegen.

-	Im zweiten Fall geht es darum, dass Nutzer eine Maustaste gedrückt haben und sie dauerhaft halten, während der Mauspfeil sich im Bereich des Containers befindet. Wenn in diesem Fall eine Reaktion des Programms erfolgen soll, müssen wir sie in der Methode 
void mousePressed(MouseEvent e) implementieren. 

Wichtig: mousePressed() wird unter Umständen dann nicht aufgerufen, wenn Nutzer gleichzeitig die Maus bewegen.

Ebenfalls wichtig: Im Gegensatz zu mouseClicked()wird die Methode im gesamten Zeitraum immer wieder aufgerufen, in dem Nutzer die Maustaste gedrückt halten. Sie sollten hier also keinesfalls etwas programmieren, das nur einmalig ausgeführt werden darf.

-	Beim dritten Fall geht es darum, dass die Maus bewegt wird, während eine Maustaste gedrückt wird. In diesem Fall implementieren Sie das entsprechende Verhalten in 
void mouseDragged(MouseEvent e).

(Diese Methode ist ein Grund, warum ich Ihnen die Programmierung mittels MouseAdapter erkläre: Währen die anderen drei Fälle aus MouseListener stammen, stammt dieser aus MouseMotionListener. Wenn Sie also ohne MouseAdapter programmieren würden, müssten Sie diesen Fall mit einem getrennten Listener verarbeiten.)

-	Der letzte Fall tritt dann ein, wenn Nutzer die Maustaste wieder loslassen. Hier kommt 
void mouseReleased(MouseEvent e) zum Einsatz.

Wichtig: Dabei dürfen Sie nicht voraussetzen, dass Nutzer die Maustaste über dem gleichen Container loslassen, über dem Sie sie gedrückt haben. Sie dürfen ebenfalls nicht voraussetzen, dass Nutzer die gleiche Maustaste loslassen, die sie zuletzt gedrückt haben. Es wird zwar nur selten vorkommen, dass Nutzer zuerst die linke Maustaste drücken und gedrückt halten, dann die rechte drücken und gedrückt halten und dann die linke Maustaste zuerst wieder loslassen, aber wenn Sie nur genug Nutzer haben, wird es solche Fälle früher oder später geben. Wie Sie sehen müssen Sie sich bei der professionellen Entwicklung von GUIs mit teilweise absurd erscheinenden Problemen beschäftigen.

13.2.1.	Aufgabe
Sie haben ein Bedienelement in Form eines nach oben zeigenden Pfeils in Ihrer GUI implementiert. So lange Nutzer dieses Element mit der Maus anwählen (also den Mauspfeil darüber „stehen“ lassen und eine Maustaste drücken), soll eine Variable des Programms kontinuierlich erhöht werden. Denken Sie dabei an so etwas wie die Steuerung einer Zapfsäule: So lange Nutzer die entsprechende Taste drücken, soll Benzin ins Motorrad gefüllt werden.
-	Warum müssen Sie das mit mousePressed() realisieren und können nicht mouseClicked() verwenden?
13.3.	Wenn Nutzer die Maus bewegen
Auch hier haben wir es wieder mit vier Fällen zu tun, wobei Sie einen schon im vorigen Abschnitt kennen gelernt haben:
-	Wenn etwas in genau dem Moment passieren soll, in dem Nutzer den Mauspfeil in den Bereich eines Containers bewegen, implementieren Sie dafür 
void mouseEntered(MouseEvent e).

Wichtig: Diese Methode wird genau einmal aufgerufen, nämlich dann, wenn der Mauszeiger in den Bereich des Containers bewegt wird. So lange er sich dort befindet, wird die Methode nicht nochmal aufgerufen.

-	Im zweiten Fall wird die Maus bewegt, während sie sich über einem Container befindet. Hier implementieren Sie void mouseMoved(MouseEvent e).

-	Den dritten Fall kennen Sie bereits. Dabei geht es um mouseDragged(), also die Methode, die so lange aufgerufen wird, wie die Maus sich über einem Container befindet, während sie bewegt wird Nutzer und wenigstens eine Maustaste drücken.

-	Der vierte Fall tritt dann ein, wenn der Mauszeiger den Bereich eines Containers verlässt. Hier implementieren Sie void mouseExited(MouseEvent e).

Wichtig: Es gibt also keine Methode, die dann aufgerufen wird, wenn der Mauszeiger sich im Bereich eines Containers befindet, aber dabei nicht bewegt wird. Sie müssen sich also in diesem Fall eine Lösung einfallen lassen, aber das sollte Ihnen mittlerweile nicht mehr schwerfallen.
13.3.1.	Aufgabe
Machen Sie sich klar, was Sie alles tun müssen, damit Nutzer ein Element Ihrer GUI tatsächlich verschieben können. Für Nutzer sieht es so aus, als wenn Sie solche Elemente in dem Moment an den Mauszeiger binden, in dem sie die linke Maustaste drücken, gedrückt halten und die Maus bewegen. Dabei kommt zwar mouseDragged() zum Einsatz, aber Sie müssen hier Ihre Kenntnisse aus den Bereichen Grafikprogrammierung und Bildbearbeitung zusammen mit der Programmierung von GUIs kombinieren, um dieses Verhalten einzuprogrammieren.
Überlegen Sie weiterhin, was sie grundsätzlich tun müssten, wenn es möglich sein soll, dass Nutzer durch mittels drag-and-drop ein grafisches Element der GUI rotieren lassen können.
Und stellen Sie sich jetzt noch den Fall vor, in dem der Mauszeiger in diesem Fall nicht mehr weiter bewegt werden soll, weil die Mausbewegung nur noch vorgeben soll, wie schnell das grafische Element rotiert. Wenn Sie das nicht programmieren, dann würde die Rotation in dem Moment enden, wenn der Mauszeiger den Bereich des Containers verlässt.
13.4.	Interaktionen mit dem Mausrad
Wenn Nutzer das Mausrad benutzen, dann gibt es nur eine Methode, um die Sie sich kümmern müssen: void mouseWheelMoved(MouseWheelEvent e)
Wichtig: Beachten Sie bitte, dass im Gegensatz zu allen anderen Methoden beim MouseAdapter (bzw. beim MouseWheelListener) in diesem Fall kein MouseEvent, sondern ein MouseWheelEvent auftritt.
13.5.	Event-Methoden bei Ereignissen mit der Maus
Die einzelnen Methoden werde ich hier nicht im Detail erläutern, bitte schlagen Sie im Bedarfsfall in der Java API nach. An dieser Stelle werden Sie lediglich erfahren, welche Möglichkeiten Sie durch die Methoden erhalten:
-	Mittels getButton() erhalten Sie eine Konstante des Eventobjekts, die angibt, welche Maustaste gedrückt wurde.
-	Mittels getLocationOnScreen(), getPoint(), getX(), getY(), getXOnScreen(), getYOnSchreen() können Sie Informationen zu der Position erhalten, an der das Event aufgetreten ist.
-	Mittels translatePoint(int x, int y) können Sie das Event verlagern.
-	Mittels getModifiersEx() und getMouseModifiersText(int modifiers) können Sie Informationen dazu erhalten, welche Tasten beim Event gedrückt wurden. (Bsp.: Shift-Taste wurde zusammen mit einer Maustaste gedrückt.)
-	Mittels isPopupTrigger() erhalten Sie einen Hinweis darauf, ob durch das Event auch ein PopUp Menü erzeugt wurde.
-	Außerdem gibt es noch paramString(), der für die Identifikation des Events genutzt werden kann. (Für unsere Zwecke irrelevant.)
Wichtig: Wenn eine Maus mehr als drei Tasten hat, dann gibt es für die zusätzlichen Tasten keine Konstanten in Java. Sie können also mit dem, was Java direkt anbietet nur die linke, rechte und mittlere Maustaste abfragen. Besonderheiten von Gamingmäusen und 3D-Mäusen behandeln wir hier nicht.
Bei MouseWheelEvents gibt es die folgenden sechs Methoden. Beachten Sie bitte, dass die Rotation nicht in Grad oder Länge gemessen wird, sondern in ganzzahligen Klicks. Achten Sie bitte beim Blick in die Java API darauf, dass der Rückgabewert zwar häufig vom Typ int oder long ist, aber tatsächlich eine Konstante zurück gegeben wird, mit der Sie dann komfortabler weiter arbeiten können.
-	Die Anzahl Klicks erhalten Sie mittels getWheelRotations() und getPrecisionWheelRotation(), die sich nur im Datentyp des Rückgabewerts unterscheiden.
-	getScrollAmount() und getUnitsToScroll() gibt die Anzahl an „Einheiten“ an, um die ein Element wie eine Scrollpane pro Klick weiterbewegt werden soll.
-	getScrollType() gibt mittels entsprechender Konstanten zurück, ob Zeilenweise oder Seitenweise gescrollt werden soll.
-	Wie gewohnt haben wir auch hier wieder eine Methode, die für unsere Zwecke irrelevant ist, da sie sich auf die Identität des Eventobjekts bezieht: paramString()
Wichtig: Beachten Sie bitte, dass die Maustaste, die sich unterhalb des Mausrades befindet über getButton() von MousEvent abgefragt werden kann. Sie hat sowohl technisch als auch programmiertechnisch nichts mit dem Mausrad bzw. dem MouseWheelEvent zu tun.
14.	Timer
Bei fast allem, was Sie bislang programmiert haben, haben Sie Software im Grunde genommen nach dem Prinzip entwickelt, dass aus dem Quellcode genau zu erkennen ist, wann welche Operationen und Methoden aufgerufen werden, bzw. in welcher Reihenfolge sie ausgeführt wurden. Einzige Ausnahme sind Events und Dateioperationen gewesen, denn dort hängt es ja beispielsweise von den Eingaben von Nutzern ab, wann welche Methoden aufgerufen werden.
Ein weiterer Bereich, bei dem quasi ohne Ihr Zutun Methoden aufgerufen werden sind die sogenannten Timer. Timer-Instanzen sind Objekte, die in Abhängigkeit vom Systemtakt Methoden aufrufen. Leider vermittelt Ihnen die Bezeichnung Timer den Eindruck, dass diese Objekte in Abhängigkeit von der Zeit aktiv werden, aber das kann gar nicht der Fall sein. Die Gründe dafür verstehen Sie, wenn Sie Informatik 3 aufmerksam verfolgen. Für diejenigen von Ihnen, die Informatik 3 nicht belegen bzw. noch nicht belegt haben, hier eine kurze Zusammenfassung:
Wie schnell ein Programm ausgeführt wird hängt vor allem davon ab, wie schnell der Prozessor des Systems arbeitet. Diese Geschwindigkeit wird in Arbeitsschritten pro Sekunde angegeben. Bei einem Prozessor mit 3 GHz können pro Sekunde 3 Mrd. Arbeitsschritte pro Sekunde, bzw. 3 Mio. Arbeitsschritte pro Millisekunde ausgeführt werden. Wenn wir also ignorieren, dass ein Prozessor nicht nur unser Programm ausführt, sondern noch eine Vielzahl weiterer Programme und wir außerdem ignorieren, dass in den meisten Programmzeilen mehr als eine Aufgabe zu bewältigen ist, dann gehen wir davon aus, dass ein Rechner mit einem solchen Prozessor für ein Programm mit bis zu 3 Mrd. Zeilen innerhalb einer Sekunde bearbeiten kann. 
In Informatik 3 und in Theoretischer Informatik lernen Sie, wie Sie wesentlich genauer feststellen können, wie viele Aufgaben (in beiden Veranstaltungen Instruktionen oder Operationen genannt) der Rechner bewältigen muss, um ein Programm zu durchlaufen. Außerdem lernen Sie dort, dass wir im Regelfall nicht mit absoluten Zahlen sondern mit Größenordnungen arbeiten, um den Aufwand zu vergleichen, den zwei unterschiedliche Programme erzeugen.
Aber wie gesagt würden wir dann immer noch außeracht lassen, dass neben unserem Programm noch eine Vielzahl weiterer Programme im Rechner aktiv sind, die ebenfalls alle vom Prozessor ausgeführt werden müssen. Darum ist es streng genommen unmöglich mit einem Computer präzise festzustellen, wie viel Zeit vergangen ist. Sie glauben mir nicht und denken, dass man doch nur ein Programm entwickeln müsste, das ständig misst, wie viele Programme wie viele Aufgaben vom Prozessor bearbeiten lassen? Auch das ist keine Lösung, weil wir dann noch ein Programm bräuchten, dass auch den Arbeitsaufwand dieses Kontrollprogramms messen würden. Und dieses Programm bräuchte dann wieder eine Kontrollprogramm und so weiter und so fort.
Also können wir streng genommen in Java nichts programmieren, das unser Programm in Abhängigkeit von der Zeit steuert. In den meisten Fällen sind die Schwankungen durch unterschiedliche Auslastungen des Prozessors aber so marginal, dass es für menschliche Nutzer nicht möglich ist, den Unterschied zu bemerken. Also tun wir bei der Programmierung so, als wenn wir Methoden in festen zeitlichen Intervallen ausführen können. Sie müssen sich aber bewusst sein, dass es immer wieder dazu kommen wird, dass diese Intervalle nicht mit absoluter Präzision eingehalten werden.
Wir werden später im Semester noch die sogenannte nebenläufige Programmierung behandeln. Dort werden Sie Fälle kennen lernen, in denen Sie gar keine zeitliche Sicherheit mehr haben: Sie werden dort selbst programmieren müssen, dass bestimmte Ereignisse nur dann ausgeführt werden dürfen, wenn andere Ereignisse schon stattgefunden haben. Die nebenläufige Programmierung ist nämlich die erste Methode, um Aufgaben auf mehrere Prozessorkerne eines Rechners zu verteilen. Das bringt neue Herausforderungen mit sich, aber darum kümmern wir uns wie gesagt später.
14.1.	Zeitabhängige Ausführung von Aufgaben in Java
Wenn wir in Java einen Programmteil in festen zeitlichen Abständen oder zu einem späteren Zeitpunkt ausführen lassen wollen, dann können wir dazu die Klasse Timer benutzen. Wir können dadurch zum ersten Mal Programme realisieren, in denen einzelnen Aufgaben unabhängig voneinander ausgeführt werden. Nehmen wir an, Sie wollen ein Spiel in Java entwickeln, bei dem Spieler aus einem Gefängnis ausbrechen müssen. In festen zeitlichen Abständen sollen nun bestimmte Bereiche der Spielfläche von Wächtern kontrolliert werden. Mit Zeitsteuerung können Sie also festlegen, welcher Wächter welchen Bereich zu welchem Zeitpunkt kontrolliert.
Die Aufgabe, die wir zeitlich gesteuert ausführen lassen wollen müssen wir als eine Unterklasse von TimerTask programmieren.
Beide Klassen sind im Paket java.util enthalten, wir müssen also java.util.Timer und java.util.TimerTask in jeder Klasse importieren, in der wir zeitabhängig Aufgaben ausführen wollen.
Da die Klasse, die TimerTask erweitert programmiert werden muss, bevor Sie vom Timer verwendet werden kann, schauen wir sie uns auch zuerst an:
14.2.	TimerTask
Eine Klasse, die TimerTask erweitert muss die Methode public void run() implementieren. Später wird diese Methode von der Timer-Klasse aufgerufen. Um einen TimerTask zu programmieren, denken Sie nochmal an die Implementierung von Listenern zurück: Dort mussten Sie alles, was im Falle eines bestimmten Events passieren sollte in einer Methode programmieren. Aber während Sie bei den verschiedenen Events immer genau wissen (oder nachschlagen) mussten, wie diese Methoden hießen, haben Sie bei TimerTasks kaum Denkarbeit: Dort heißt diese Methode immer run().
Außerdem gibt es noch zwei Methoden, die Sie bei TimerTasks nicht überschreiben oder aufrufen dürfen, da beide bestimmte Funktionen haben: Mit cancel() kann ein Timer einen TimerTask beenden. scheduledExcecutionTime() gibt dagegen den Zeitpunkt der letzten Ausführung des TimerTasks zurück. 
Ansonsten programmieren Sie TimerTasks genau wie jede andere Klasse mit Feldern und Methoden. Denken Sie aber bitte daran, dass die von Ihnen programmierte Methodenaufrufe eines TimerTasks ausschließlich aus der Methode run() heraus aufgerufen werden können.
14.2.1.	Aufgabe (klausurrelevant)
Nehmen wir an, Sie haben eine Klasse Ausbruch, die die Logik unseres Spiels umsetzen soll. Das Spielfeld ist als ein zwei-dimensionales Array spielfeld[][] gespeichert, das als private deklariert ist. Denken Sie hier an eine simple Matrix, also an x/y-Koordinaten.
Programmieren Sie jetzt die Klasse Waechter als TimerTask. Programmieren Sie sie so, dass ein Waechter bei jedem Aufruf von run() den gleichen Bereich kontrolliert, dass dieser Bereich aber beim ersten Aufruf von run() zufällig bestimmt wird. 
Tipp: Waechter soll auf spielfeld[][] direkt zugreifen können. Dazu müssen Sie Waechter als eine besondere Art von Klasse bzw. in einem bestimmten Bereich des Programms programmieren. Wie heißt diese Art von Klassen?
14.3.	Timer
Im Gegensatz zum TimerTask werden Sie keine Klasse programmieren, die Timer erweitert. Sie erzeugen schlicht einen Timer und rufen eine von drei Methoden auf. Schauen wir uns das im Detail an:
14.3.1.	Instanziierung eines Timers.
Eine Instanz von Timer erzeugen Sie mit new Timer(). Sie können darüber hinaus noch true oder einen String oder einen String und true als Argumente beim Aufruf übergeben. Tatsächlich empfehle ich Ihnen die Instanziierung mit new Timer(true), weil das bewirkt, dass Ihr Timer und der TimerTask automatisch beendet werden, wenn die Instanz beendet wird, die den Timer instanziiert hat. Sonst haben sie den gleichen Effekt wie bei einer GUI, bei der Sie den Methodenaufruf setDefaultCloseOperation(JFrame.EXIT_ON_CLOSE); vergessen haben: Das aufrufende Programm existiert nicht mehr und die GUI wird auch nicht mehr angezeigt, aber sie läuft immer noch weiter. Instanziieren Sie einen Timer also mit dem Argument true, dann haben Sie die Sicherheit, dass er nur so lange als Aufgabe im Speicher des Rechners existiert, wie Ihr Programm existiert.
Und seien Sie nicht verwundert: Der TimerTask wird nicht bei der Instanziierung an den Timer übergeben. Tatsächlich können Sie einem Timer mehrere TimerTasks übergeben.
14.3.2.	Übergabe von TimerTasks an Timer
Um einen TimerTask an einen Timer zu binden nutzen Sie eine der beiden Methoden schedule() und scheduleAtFixedRate(). Leider sind die Namen nicht gut gewählt, denn beide können genutzt werden, um einen TimerTask in regelmäßigen Abständen auszuführen. Der Unterschied ist folgender: Angenommen, ein TimerTask wurde nicht pünktlich ausgeführt (weil der Rechner zu viel zu tun hat). Dann bedeutet das, dass TimerTask, der per schedule() übergeben wurde übersprungen wird. Bei scheduleAtFixedRate() wird er dagegen so oft ausgeführt, wie er übersprungen wurde. Im Falle unseres Spiels Gefängnisausbruch wäre das also für den TimerTask Waechter sinnlos. In Fällen wo es dagegen wichtig ist, das ein Task auf jeden Fall jedes Mal ausgeführt wird, wenn er ausgeführt werden sollte, auch wenn das vielleicht deutlich verspätet passiert, ist scheduleAtFixedRate() die passende Methode. Werfen Sie ggf. einen Blick in die Java API, wenn Sie damit zu tun haben. Hier werden wir uns nur schedule() ansehen.
14.3.3.	Einmaliger Aufruf eines TimerTask zu einem bestimmten Zeitpunkt
Mit timer.schedule(TimerTask t, Date start) können Sie dem Timer timer einen TimerTask übergeben, der zum Zeitpunkt start einmalig ausgeführt wird. So könnten Sie beispielsweise für Ihre Silvesterparty automatisch ein Feuerwerk starten lassen, wenn Sie die Steuerung in Java programmieren. Schlagen Sie ggf. in der Java API näheres zur Klasse Date nach. (Leider wird dort der Zeitpunkt in Millisekunden nach dem 1.1.1970 programmiert.)
Mit timer.schedule(TimerTask t, long delay) können Sie dagegen festlegen, wie lange es dauern soll, bis der TimerTask einmalig aufgerufen wird. Nehmen wir an, Sie wollen ein Programm entwickeln, mit dem Sie sich auf Klausuren vorbereiten. Dann müssen Sie nur noch berechnen, wie viele Millisekunden z.B. 90 Minuten sind, und könnten nach dem dem Start des Programms ein automatisches Ende für Eingaben einprogrammieren. Genauso würden Sie es auch programmieren, wenn der Spieler des Gefängnisausbruchs innerhalb einer vorgegebenen Zeit fertig werden muss.
14.3.4.	Wiederholte Aufrufe eines TimerTask nach festen Intervallen
Wenn Sie beim Aufruf zusätzlich noch eine weitere long-Variable programmieren (z.B. timer.schedule(TimerTask t, long delay, long n)), dann wird der TimerTask nach dem delay alle n Millisekunden aufgerufen.
14.3.4.1.	Aufgabe (Klausurrelevant)
Warum darf der Aufruf bei unserem Spiel nicht timer.schedule(new Waechter(), m, n); lauten?
Tipp: Dabei sollen m und n beliebige Zahlen sein und timer eine Instanz von Timer und new Waechter() eine legale Instanziierung unserer TimerTask-Klasse, aber das ist für die Fragestellung irrelevant. Es geht hier also nicht um die die Syntax. 


15.	Dateien und Dialoge
Wichtig: Aus Zeitgründen ist dieses Kapitel nicht abgeschlossen. Sie erfahren hier die Hintergründe zum Dateizugriff und anderen Operationen, bei denen Sie auf Ressourcen zugreifen, die nicht Teil Ihres Quellcodes sind. Dieses Verständnis ist für die Programmierung (siehe Skript von Prof. Dr. Plaß) z.B. von Dateizugriffen wichtig. Deshalb sollten Sie es lesen, bevor Sie mit der eigentlichen Programmierung beginnen. Zurzeit kann nicht sichergestellt werden, dass dieses Kapitel abgeschlossen werden wird.

Bevor wir uns die Themen Dateien und Dialoge ansehen, werfen wir einen Blick zurück auf das, was Sie bislang kennen gelernt haben, damit Sie verstehen, welchen Mehrwert dieses Kapitel Ihnen bietet.
Im ersten Teil des Kurses haben Sie gelernt, einfache Programme in Java zu entwickeln, wobei dieselben Kenntnisse auch in wesentlich umfangreicheren Programmen genauso verwendet werden. Sie haben dabei zum einen die prozedurale und strukturierte Programmierung kennen gelernt, auch wenn Sie wahrscheinlich beide Begriffe nicht zuordnen können. Dazu haben Sie die klassenbasierte objektorientierte Programmierung in Java erlernt.
Schauen wir uns diese drei Ansätze einmal etwas genauer an, damit Sie verstehen, was sie bedeuten, sodass Sie später erkennen können, ob Sie in einer anderen Programmiersprache einfach nur eine andere Syntax erlenen müssen (sprich, ob dort andere Zeichen und Befehle verwendet werden, aber letztlich das gleiche Konzept umgesetzt wird), oder ob Sie tatsächlich etwas komplett neues erlernen müssen.
15.1.	Programmierparadigmen
Zunächst zum Begriff der Überschrift: Ein Programmierparadigma beschreibt einen Ansatz bzw. eine Sammlung von Konzepten, nach denen Software entwickelt werden kann. Programmiersprachen sind dann immer eine Umsetzung von Teilen dieser Konzepte. Welche Konzepte das sind ist einzig die Entscheidung der Entwickler einer Programmiersprache. Häufig bezeichnen Entwickler von Programmiersprachen dann Dinge mit diesen Konzepten, die ihnen nicht vollständig entsprechen. So wird Java beispielsweise als objektorientierte Sprache bezeichnet. Tatsächlich handelt es sich um eine klassenbasierte objektorientierte Sprache. (Auf den Unterschied gehe ich gegen Ende dieses Abschnitts ein.) Teilweise werden auch Konzepte wie Datenstrukturen in Sprachen falsch bezeichnet. So gibt es in PHP die sogenannten Arrays, die aber mit tatsächlichen Arrays nicht viel gemein haben. Diese Konzepte können Sie in Veranstaltungen wie Algorithmen und Datenstrukturen oder Software Engineering kennen lernen.
Die verschiedenen Ansätze, die Sie in diesem Abschnitt kennen lernen sind allesamt Programmierparadigmen:
Wenn Sie prozedural programmieren, dann tun Sie nichts anderes als Variablen zu definieren (besser gesagt zu deklarieren), diesen Variablen Werte zuzuordnen und diese Werte mittels der sogenannten Operationen abzuändern. Dies ist einer der Programmieransätze, die als imperative Programmierung zusammengefasst werden. 
Hier wie an anderer Stelle werden prozedurale Programmiersprachen unter anderem in statisch und dynamisch typisierte Sprachen unterteilt: Bei statisch typisierten Sprachen wie Java werden einer Variablen vom Entwickler feste Datentypen zugeordnet, die sich im Programmablauf nicht ändern können. Die Vor- und Nachteile haben Sie kennen gelernt.
Im Gegensatz dazu werden die Datentypen bei dynamisch typisierten Sprachen von der Programmiersprache selbst (genauer gesagt von der jeweiligen Laufzeitumgebung) verwaltet und können sich jederzeit ändern. Es spielt nun keine Rolle, welche der beiden Arten Sie persönlich bevorzugen; wenn Sie eine der beiden Varianten ablehnen, mögen Sie in einzelnen Sprachen gute Software entwickeln können, aber mit guter Informatik haben Ihre Kenntnisse nichts zu tun.
Bevor wir zur Erklärung der imperativen Programmierung kommen, hier noch kurz etwas dazu, was die strukturierte Programmierung ist. Es steckt schon im Namen und Sie werden sich wundern, warum es hierfür überhaupt eine eigene Bezeichnung gibt: Bei strukturierter Programmierung arbeiten wir mit Strukturen wie Schleifen.
Wie Sie sehen kann also eine Sprache mehrere Konzepte umsetzen, es gibt also nicht etwa Programmiersprachen, die entweder prozedural oder strukturiert sind, sondern viele Sprachen, die wenigstens eines von beidem sind.
Jetzt zur imperativen Programmierung: Imperative Programmierung beinhaltet stets, dass Sie in einem Programm festlegen, wie der Computer arbeiten soll. Somit sind prozedurale und strukturierte Programmiersprachen immer imperativ. Was hier fehlt und was Ihnen bei jeder Fehlersuche den Tag ruiniert, ist die Antwort darauf, wozu er arbeiten soll, bzw. welchen Sinn die einzelnen Teile eines Programms haben. Dennoch gehen fast alle Menschen davon aus, dass Programmierung und imperative Programmierung das gleiche wären. Und das ist falsch.
Wenn Sie dagegen in einer Programmiersprache festlegen, was der Computer tun soll, ohne zu sagen, wie er diese Aufgabe im Detail erfüllen soll, dann wird das als deklarative Programmierung bezeichnet. Eine Art der deklarativen Programmierung wird zurzeit in mehreren  häufig verwendeten Programmiersprachen umgesetzt: Die funktionale Programmierung. Die Basis der funktionalen Programmierung ist das sogenannte Lambda-Kalkül. In Java wird es seit Version 8 unterstützt und in JavaScript schon deutlich länger.
Dies wird keine Einführung ins Lamda-Kalkül, deshalb sei an dieser Stelle nur so viel gesagt: Weil wir nicht mehr im Detail programmieren, wie der Computer arbeiten soll, sind funktionale Programme wesentlich kürzer als imperative Programme. Das Verhältnis kann durchaus bei 1:5 liegen. Sprich: Was Sie in Java ohne funktionale Programmierung in 200 Zeilen ausdrücken, kann mit funktionaler Programmierung durchaus in 40 Zeilen passen.
Leider setzt sich die funktionale Programmierung erst langsam im kommerziellen Bereich durch, sodass die meisten Einführungen für Einsteiger kaum geeignet sind. Dennoch sollten Sie sich hiermit auseinander setzen, da es wahrscheinlich ist, dass diese Programmierweise sich mehr und mehr durchsetzen wird: Sie mag schwer zu erlernen sein, ist dafür ausgesprochen außerordentlich effizient und hilft massiv bei der Vermeidung von Fehlern, sodass die Fehlerbehebung deutlich vereinfacht wird.
Wichtig: Sollte das noch nicht deutlich geworden sein: Ohne eine Einführung ins Lamda-Kalkül bzw. die funktionale Programmierung können Sie Java 8 genau wie JavaScript nicht in vollem Umfang nutzen. Sie können dann schlicht nicht verstehen, warum bestimmte Dinge so zu programmieren sind, wie das der Fall ist.
Jetzt noch ein Hinweis bezüglich dem, was Sie als Objektorientierung in Java kennen gelernt haben: Genau gesagt handelt es sich hier um die klassenbasierte Objektorientierung. Das ist aber NICHT „die“ objektorientierte Programmierung. So gibt es beispielweise in JavaScript die prototypbasierte Objektorientierung. Dabei erzeugen Sie Objekte nicht als Klassen, sondern direkt aus Funktionen.
Und damit wären wir bei der zentralen Warnung dieses Abschnitts: Wenn Sie glauben, dass Sie ein Programmierkonzept kennen, weil Sie etwas mit dessen Namen in einer Programmiersprache kennen gelernt haben, dann liegen Sie im Regelfall falsch.
15.2.	Übergang zum aktuellen Kapitel
Nachdem Sie also in P1 die grundlegende Programmierung klassenbasierter objektorientierter Programme erlernt haben, haben Sie bei den beiden letzten Terminen gelernt, wie Sie zu einem solchen Programm eine grafische Nutzeroberfläche programmieren können.
In dieser Veranstaltung werden Sie noch einige Themen rund um GUIs kennen lernen. Heute schauen wir uns Dateien und Dialoge an.
Ein Dialog ist eine Sonderform von GUI, aber für die Programmierung und Nutzung gibt es hier kaum neues.
Wenn wir dagegen über Dateien reden, dann verlassen wir den Bereich der GUI-Programmierung, denn hier beschäftigen wir zum zweiten Mal nach den Events mit etwas, das zum Teil außerhalb des eigentlichen Java-Programms liegt: Beim Eventhandling haben wir mit Objekten gearbeitet, die quasi aus dem Nichts entstanden sind. Zur Erinnerung: Wenn wir es etwas naiv betrachten, dann sind Events Objekte in Java, die vom Betriebssystem erzeugt werden.
Wenn wir auf Dateien zugreifen, dann nutzen wir erneut etwas, das wir nicht selbst in Java programmieren, denn die Daten, die letztlich auf der Festplatte oder einem anderen Medium gespeichert oder von dort ausgelesen werden, sind ja nicht Teil unserer Programmierung. Diesen Teil müssen Sie grundsätzlich verstehen, weil es hier eben nicht nur darum geht, wie wir die Daten unseres Programms in einer Datei speichern oder von dort aus lesen. Vielmehr erfahren Sie hier alles darüber, die Daten zwischen einem Java-Programm und beliebigen anderen Programmen oder Computern ausgetauscht werden.
Tatsächlich kommen wir hier in einen Bereich, der seinen Ursprung im Bereich der Nachrichtentechnik hat: Wir greifen auf Datenströme zurück. Was das ist und wie wir damit umgehen ist der zweite Teil der heutigen Veranstaltung. Das kann aber für die gesamte Thematik nur eine erste Einführung sein.
Der Rest dieser Veranstaltung unterteilt sich in drei Bereiche: Zum einen wären da verschiedene Aspekte der GUI-Programmierung. Wir haben noch keine Menüs programmiert und die Bearbeitung von Computergrafiken oder Bildern mit der Maus werden Teil der nächsten Veranstaltungen sein. Zusätzlich kommen wir zur Timer-Klasse, mit der wir zeitabhängige Interaktionen programmieren können.
Der zweite große Bereich der Veranstaltung behandelt das Themengebiet der Algorithmen und Datenstrukturen. Hier geht es um eine der zentralen Kompetenzen aller InformatikerInnen: Die Antwort auf die Frage, wie wir Daten möglichst effizient im Programm verwalten können. Nicht-InformatikerInnen verwenden hier in aller Regel eine von zwei Lösungen: Arrays und Datenbanken. Und das kann sehr schnell dazu führen, dass Programme sehr langsam werden.
Der letzte Bereich greift zwei Punkte auf, die wir leider nicht in der nötigen Tiefe behandeln können. Zum einen wäre da die nebenläufige Programmierung, bei der Sie bewusst die vielen Prozessorkerne eines Rechners nutzen. Zum anderen kommen wir nochmal auf das Thema Datenaustausch zu sprechen: Dieses Mal geht es nicht um Dateien, sondern um den Datenaustausch mit anderen Rechnern.
Fassen wir das zusammen: Nachdem Sie im ersten Semester gelernt haben, lauffähige Programme in Java zu erstellen, besprechen wir momentan verschiedenes, damit diese Programme auch von Nutzern genutzt werden können. Anschließend sehen wir uns einige Punkte an, deren Beherrschung InformatikerInnen von Programmierern unterscheiden.
15.3.	JDialog
Wie Sie schon den früheren Erläuterungen entnehmen konnten programmieren Sie einen Dialog mit der Klasse JDialog und gehen dabei fast genauso vor, als wenn Sie einen JFrame programmieren würden.
Es gibt zwei Unterschiede: 
-	Bei der Instanziierung mit new JDialog(parent, title, modal) übergeben Sie dem Konstruktor über das Argument parent den Top-Level-Container, an den dieser Dialog gebunden wird. Damit sorgen Sie dafür, dass der Dialog geschlossen wird, wenn das parent-Fenster geschlossen wird.

-	Weiterhin ist modal eine boolean-Variable, die besagt, ob es sich um einen modalen oder nicht-modalen Dialog handelt. Modale Dialoge blockieren alle anderen Fenster; Nutzer müssen also zunächst den Dialog beenden, bevor sie mit den übrigen Fenster arbeiten können.
15.4.	JOptionPane
JOptionPanes sind im Grunde ebenfalls Dialoge, aber sie bieten ausschließlich vorgegebene Ansichten an, die durch statische Methoden aufgerufen werden. Obwohl sie also deutlich beschränkter als alle bisherigen GUIs sind, müssen Sie hier ein ganz anderes Kochrezept nutzen, als das bisher der Fall war. 
Nach der Instanziierung erzeugen Sie keine Instanzen von Komponenten, sondern geben über die entsprechenden Methoden von JOptionPane an, welcher Text angezeigt werden soll, welche und wie viele Schaltflächen anzeigt werden sollen, usw.
Mehr Details zu JDialog und JOptionPane finden Sie im Skript von Prof. Plaß.
15.5.	Datenströme und Dateien
Damit kommen wir zum zentralen Thema dieses Termins. Letzte Wochen haben Sie gelernt, dass Sie in Java mit Objekten arbeiten können, die Sie nicht selbst instanziiert haben, sondern die scheinbar aus dem Nichts kommen.
Aus Sicht der Java-Programmierung erzeugen wir ja stets Objekte, die wir dann über Methodenaufrufe ändern oder nutzen. Und auch wenn wir bei Events diese Objekte nicht selbst erzeugt haben, blieb doch dieses Grundprinzip.
Bei Datenströmen gibt es allerdings etwas neues, das außerdem recht anspruchsvoll ist. Deshalb müssen wir uns zunächst ansehen, was der Unterschied zwischen dem ist, wodurch ein Event erzeugt wird und dem, was wir als Datenstrom nutzen.
15.5.1.	Event versus Datenstrom
Bei einem Event tun wir so, als wenn es sich um eine einzelne klar abgegrenzte Nachricht handelt. Auf der Systemebene mögen es tatsächlich mehrere Signale sein, die über eine Datenleitung „wandern“, aber für uns ist das egal: Wir haben hier ein Event und das ist für sich abgeschlossen. Also brauchen wir die Sachen auch nicht genauer zu betrachten.
Bei einem Datenstrom sieht das anders aus, so wie es aus dem Namen hervorgeht: Es handelt sich um einen Strom, also um eine Abfolge von Signalen, die nicht per se abgegrenzt sind. Das einzige was hier abgegrenzt ist (aber auch das wieder nur für uns als Java-Programmierer), sind die Datenpakete, aus denen sich der Strom zusammensetzt. Diese Datenpakete sind einzelne Bytes, also jeweils acht Bit am Stück.
15.5.2.	Besonderheit aller Datenströme
Daraus folgt, dass wir bei Datenströmen ebenfalls nicht voraussetzen dürfen, dass das Ende eines Stromes klar abgegrenzt ist; vielmehr gehen wir hier programmiertechnisch vor, als wenn wir ein Telefonat führen: Dort öffnen wir einen Kommunikationskanal, indem wir eine Leitung öffnen (Hörer abheben / grüne Hörertaste anwählen), dann einen Nummer eingeben und darauf warten, dass die Gegenseite abhebt. Und irgendwann, ohne dass der Zeitpunkt im Vorfeld klar wäre, trennen wir (oder die Gegenstelle) den Kommunikationskanal. (Hörer auflegen / rote Hörertaste drücken) Wie wir den Kommunikationskanal in der Zwischenzeit nutzen ist von diesem Ablauf komplett unabhängig. (Aktuelle Technologien wie LTE oder asynchrone Kommunikation wie E-Mail u.a. lassen wir hier mal außen vor.)
Der Ablauf bei Datenströmen ist konzeptionell derselbe: Sie erzeugen eine Instanz einer Datenstromklasse und wenn der Datenaustausch beendet ist, „beenden“ Sie sie wieder.
Hier wird in aller Regel zwischen Klassen für den Empfang von Daten und den Versand von Daten unterschieden. Im Gegensatz zu dem, was Sie bei einem Telefonat erleben, nutzen Sie hier zwei unterschiedliche Objekte: Eines, über das Sie „sprechen“ und ein anderes, über dass Sie zuhören. Obwohl: So abwegig ist das ja nicht: Sie sprechen schließlich mit dem Mund und hören mit den Ohren; ein gemeinsames Organ zum Sprechen und Hören haben Sie auch nicht.
Der Datenaustausch besteht dann im Versenden oder Empfangen von Paketen, die die Größe eines Byte haben. Wenn Sie also eine Bilddatei in Ihr Javaprogramm laden wollen, finden also tatsächlich mehrere Tausend Leseoperationen statt, bei denen jeweils ein Byte der Bilddatei gelesen wird. Aber keine Sorge, wir arbeiten ja mit strukturierter Programmierung, also müssen Sie das nur einmal programmieren.
Dieser Ablauf ist für alle Arten von Datenaustausch zwischen einem Java-Programm und „Dingen“ außerhalb von Java identisch. Die verwendeten Klassen sind es allerdings nicht unbedingt. Auch die Methodenaufrufe und Argumente unterscheiden sich zum Teil deutlich. Doch wenn Sie das Grundprinzip verstanden haben, wird Ihnen die Programmierung deutlich leichter fallen.
Aufgabe:
-	Bislang haben Sie String-Objekte immer als eine Einheit betrachtet. Tatsächlich handelt es sich dabei um Arrays von Character-Variablen. Erklären Sie in eigenen Worten wie ein String als Datenstrom strukturiert wäre.

Tipp: Das hat noch nichts mit einer konkreten Programmieraufgabe zu tun; hier geht es um die Frage, ob Sie wirklich verstanden haben, was ein Datenstrom ist, was ein String-Objekt ist und warum daraus direkt folgt, wie ein String-Objekt direkt in einen Datenstrom übertragen werden kann.
15.5.3.	Pufferung
Neben dem Konzept des Datenstroms müssen wir uns nun das Konzept der Pufferung ansehen: So wie wir nicht wissen, wann ein Datenstrom als Ganzes beendet ist, können wir im Regelfall auch nicht mit absoluter Genauigkeit wissen, wann ein Datenpaket beginnt und wann es endet. Mehr über die Gründe erfahren Sie in Veranstaltungen wie KT und NWI. In dieser Hinsicht möchte ich Ihnen die Wahlpflichtkurse Nachrichtentechnik 1 und 2 bei Prof. Mores ans Herz legen: Diese sind zwar außerordentlich anspruchsvoll, aber wenn Sie hier bestehen, dann verstehen Sie wirklich, was eigentlich passiert, wenn Daten übertragen werden.
Wenn Sie in wenigen Wochen mit Grafiken und Bildern in Java arbeiten, wird Ihnen das sogenannte dubble Buffering begegnen. Das ist eine Pufferung, bei der ein Bild, das aus einem Datenstrom erzeugt wird zunächst vollständig zwischengespeichert wird. Wenn es angezeigt werden soll, ist es also sofort vollständig zu sehen.
Aufgabe:
-	Wie wird es denn sonst angezeigt, wenn es direkt aus einem Datenstrom am Bildschirm angezeigt wird? Oder anders gefragt: Wenn in einer GUI ein Bild aus einem Datenstrom erzeugt wird, der über ein Netzwerk „einfließt“, was passiert dann unter Umständen beim Bildaufbau, das Nutzer stören würde.


16.	Nebenläufige Programmierung
Praktisch alle aktuellen Rechner (das schließt auch Smartphones mit ein) verfügen über Prozessoren, die mehrere Kerne besitzen. Wenn wir eine naive Perspektive einnehmen, dann gehen wir davon aus, dass unsere Programme mit jedem zusätzlichen Prozessorkern auch automatisch entsprechend schneller werden. Wenn das der Fall wäre, würde ein Programm auf einem Rechner mit zwei Kernen doppelt so schnell laufen, wie auf einem Rechner mit einem Kern, auf einem System mit vier Kernen würde es nochmal doppelt so schnell laufen usw. Aber wie gesagt: Das ist eine naive Perspektive, die mit der Realität nichts gemein hat.
Tatsächlich müssen wir bei der Programmierung in Java bewusst steuern, dass Aufgaben auf verschiedene Kerne verteilt werden. Wir brauchen dabei zwar nichts darüber zu wissen, wie viele Kerne vorhanden sind, aber auch so haben wir es hier mit einigen Problemen zu tun, die Ihnen bislang bei der Programmierung in Java nicht begegnet sind.
Wenn wir nun ein Programm entwickeln, bei dem wir bewusst die Aufgaben auf verschiedene Kerne verteilen, dann wird das als nebenläufige Programmierung bezeichnet. In diesem Zusammenhang wird Ihnen der Begriff des Thread begegnen: Jeder Teil eines Programms, der so programmiert wird, dass er unabhängig von anderen Teilen des Programms auf einem Prozessorkern ausgeführt werden kann, wird (u.a. in Java) als Thread bezeichnet. Wenn Sie also etwas über Multithreading hören, dann geht es um nichts anderes als um nebenläufige Programmierung: Ein Programm, in dem mehrere Threads unabhängig voneinander ausgeführt werden.
Im Englischen spricht man übrigens von der concurrency, wenn man über nebenläufige Prozesse redet.
Das offizielle Tutorial zu nebenläufiger Programmierung in Java finden Sie unter https://docs.oracle.com/javase/tutorial/essential/concurrency/ 
16.1.	Von der prozeduralen zur nebenläufigen Programmierung
Wenn Sie mit der nebenläufigen Programmierung beginnen, dann ist der einfachste Weg, dass Sie sich ein beliebiges Programm nehmen und es darauf untersuchen, welche Teile des Programms unabhängig ausgeführt werden können. Diese gliedern Sie dann bei Java in eine Klasse aus, die das Interface Runnable implementiert oder von der Klasse Thread erbt. 
Wichtig:
Im Folgenden werden wir den Begriff des Thread für beide Fälle verwenden. So lange hier also nicht explizit angegeben ist, dass etwas nur für Implementierungen von Runnable oder nur für Erweiterungen von Thread gilt, gilt es für beide Fälle gleichermaßen.
Ohne es zu wissen haben Sie das schon getan, dann ein TimerTask implementiert Runnable. Alles, was Sie bei Timern über die Programmierung von TimerTasks kennen gelernt haben, können Sie also vollständig für die Programmierung von Threads übernehmen. Einziger Unterschied: Während ein TimerTask über einen Timer gescheduled wird, starten wir einen Thread auf etwas andere Art und Weise. Außerdem werden Sie noch einige zusätzliche Methoden kennen lernen, wie beispielsweise sleep(), mit der Sie einen Thread pausieren lassen können.
Schauen wir uns einige Grundlagen im Detail an: Nehmen wir an, Sie sind als Softwareentwickler in einem multinationalen Konzern angestellt, der zum Teil Hunderte Filialen in einzelnen Ländern hat. Sie entwickeln den Teil einer Software, in dem es um die Warenverwaltung geht. Das bedeutet, dass Sie Methoden programmieren, die beispielsweise die Preise aller Waren in jeder Filiale eines Landes oder einer Region anpassen. Ein typischer Fall wäre so etwas wie die Änderung der Umsatzsteuer. Gehen wir einfach davon aus, dass die Umsatzsteuer in einem Land von 25% auf 27 % erhöht wird. Lassen wir einmal das Thema Datenbanken und Effizienz außen vor und tun so, als wenn die Preise alle Waren in einem Array gespeichert wären.
Mit dem, was Sie bislang kennen würden Sie jetzt also eine Iteration programmieren, die den Preis ohne Umsatzsteuer mit 1,27 multipliziert und durch 1,25 dividiert und das Ergebnis wieder im Array abspeichert. Das sähe also ungefähr so aus:
for (int i = 0; i < warenPreise.length; i++) { warenPreise[i] = warenPreise[i] * 1.27 / 1.25; }

Stellen wir uns nun vor, wir hätten es mit insgesamt 300 Mio. Preisen zu tun und jeder Prozessorkern läuft mit drei GHz. Schauen wir uns nun an, wie lange es dauert, diese Iteration auszuführen:
-	int i = 0;		
// Eine Operation

-	i < warenPreise.length 	
// Pro Preis macht das zwei Operationen: Das Auslesen der Länge des Arrays und den Vergleich, also insgesamt 600 Mio. Operationen.

-	i++			
// Pro Preis eine Operation: 300 Mio. Operationen.

-	warenPreise[i]		
// Auslesen der 300 Mio. Werte aus dem Array macht 300 Mio. Operationen.

-	* 1.27			
// 300 Mio. Multiplikationen ergibt wieder 300 Mio. Operationen.

-	/ 1.25			
// 300 Mio. Divisionen ergibt wieder 300 Mio. Operationen.

-	warenPreise[i] = ...	
// 300 Mio. Speicherungen von Werten im Array macht 300 Mio. Operationen.
Wenn wir das Summieren, kommen wir auf 2,1 Mrd. Operationen. Ein Prozessorkern mit 3 GHz, der sonst nichts zu tun hat, benötigt dafür also rund 0,7 Sekunden. Das klingt nach wenig. Nun müssen Sie aber davon ausgehen, dass in Ihrer Unternehmenssoftware viele Tausend von solchen Iterationsaufrufen wie unserem praktisch zur gleichen Zeit gestartet werden: Wenn die alle einen vergleichbaren Aufwand erzeugen und wir immer noch so tun, als wenn wir nur einen Prozessorkern zur Verfügung haben, dann müssen wir auf das Ende unserer Preisanpassung mehrere Minuten oder Stunden warten. Also müssen wir bewusst mit mehreren Prozessorkernen arbeiten, wann immer das möglich ist.
Außerdem arbeitet Java ohnehin grundsätzlich mit mehreren Threads. Denken Sie hier schlicht an Events bzw. Listener bei einer umfangreichen GUI: Da nicht bekannt sein kann, wo Nutzer als nächstes mit der GUI interagieren, müssen alle Listener quasi gleichzeitig und ununterbrochen bereit sein, bei einem Event aktiv zu werden. Und da im Hintergrund ein Betriebssystem mit Hunderten Prozessen aktiv ist, von weiteren Programmen im Hintergrund ganz zu schweigen, wäre es ein nicht akzeptabler Fehler, wenn GUIs nicht nebenläufig implementiert wären. Nur haben wir uns das bislang noch nicht angesehen.
Zusammenfassung:
Wenn wir Programme so entwickeln, dass Sie alle Kerne eines Prozessors ausnutzen, dann wird das als nebenläufige Programmierung bezeichnet. Bei kleinen Programmen ist der Gewinn im Regelfall nicht zu bemerken, aber sobald wir es mit Hunderten Aufgaben zu tun haben, die gleichzeitig ausgeführt werden, ist es ein schwerer Mangel, Software nicht nebenläufig zu programmieren.
Doch bevor wir uns das ansehen, sollten wir zunächst prüfen, in welchen Fällen wir Aufgaben nebenläufig ausführen lassen können.
16.2.	Beschränkung nebenläufiger Programmierung
Schauen wir uns dazu das Beispiel von oben an: Dort werden Einträge eines Array geändert. Wenn wir prüfen wollen, ob ein Programmteil nebenläufig gelöst werden kann, dann müssen wir nur prüfen, ob er sich in einzelne Schritte unterteilen lässt, die nicht voneinander abhängen. Bei unserem Beispiel ist das der Fall: Weil ja die Änderung des einzelnen Preises unabhängig davon durchgeführt wird, wie die Änderung der anderen Preise durchgeführt wird, können wir jede Änderung als eigenen Thread ausführen lassen.
Hier zwei Beispiele, die zeigen, dass nicht alles bzw. nicht alles so leicht nebenläufig gelöst werden kann: 
-	Bildkompression: Wenn wir ein Bild komprimieren, dann bedeutet das, dass wir nicht jeden Bildpunkt so speichern, wie er ursprünglich war, sondern dass wir ein Verfahren anwenden, bei dem das Bild noch eine ausreichende Qualität hat, aber deutlich weniger Speicher verbraucht. Wenn wir hierzu die JPEG-Kompression anwenden, dann hängt der Wert vieler Bildpunkte von einer Berechnung ab, bei der die benachbarten Bildpunkte einbezogen werden. Da aber auch deren Wert zum Großteil wiederum von Berechnungen abhängt, bei der ihre Nachbarn einbezogen werden, können wir die Kompression aller Bildpunkte nicht vollständig nebenläufig durchführen. Allerdings werden bei JPEG immer Gruppen von Bildpunkten gemeinsam berechnet, sodass wir zwar nicht für jeden einzelnen Bildpunkt, aber dafür eben für mehrere Bildpunkte gemeinsam einen Thread starten können.

-	Rekursion: Vergegenwärtigen Sie sich, wie die Werte des Pascalschen Dreiecks berechnet werden. Da werden jeweils zwei bereits berechnete Werte verwendet, um einen neuen Wert zu berechnen. Also können wir hier keine nebenläufige Programmierung nutzen, denn alle berechneten Werte sind ja von der Berechnung anderer Werte abhängig. 

(Hinweis bezüglich dieser Aufgabe, der nichts mit der nebenläufigen Programmierung zu tun hat: Sie können aber dennoch die Berechnung des Pascalschen Dreiecks sehr effizient programmieren. Dazu müssen Sie sich nur vergegenwärtigen, dass in jeder Zeile nur ein neuer Wert berechnet wird. So ist der erste und letzte Wert jeder Zeile immer die 1. Ab der dritten Zeile ist der zweite und der vorletzte Wert immer die 2. Ab der fünften Zeile ist der dritte und der drittletzte Werte immer die 3. Mit diesen Tipps, ein wenig Nachdenken und mit Hilfe eines Arrays für Zwischenergebnisse sollten Sie im Stande sein, eine effiziente Rekursion für das Pascalsche Dreieck zu programmieren. Aber wie gesagt: Das hat nichts mit nebenläufiger Programmierung zu tun, sondern es ist angewandte Informatik.)
Zusammenfassung:
Teile von Programmen können nur dann nebenläufig programmiert werden, wenn Sie vollkommen unabhängig voneinander ausgeführt werden können. Die Schwierigkeit besteht also darin, zu erkennen, welche Teile eines Programms nebenläufig programmiert werden können.
An dieser Stelle noch ein Hinweis: Immer wenn in der Literatur von leichtgewichtigen Prozessen die Rede ist, sind damit Teile eines Programms gemeint, die nebenläufig ausgeführt werden.
Nachdem wir jetzt eine grundlegende Vorstellung davon haben, warum nebenläufige Programmierung Sinn macht und wann wir Sie anwenden können, schauen wir uns an, welche Zustände ein Objekt vom Typ Thread haben kann.
16.3.	Zustände eines Thread
Auch das ist wieder etwas Neues: Bei nebenläufiger Programmierung haben die einzelnen Threads einen Zustand. Um das zu verstehen, stellen Sie sich bitte die Objekte eines objektorientieren Programms als Arbeitnehmer vor und Methodenaufrufe als Anweisungen an diese Menschen. Wenn Sie dieses Bild vor Augen haben, dann wird das Prinzip schnell klar: Ein Arbeitnehmer, der gerade eine Aufgabe ausführt, muss diese Aufgabe abbrechen, um eine andere Aufgabe auszuführen. Mit Zustand ist hier gemeint, ob und warum der Arbeitnehmer etwas tun kann oder eben nicht.
Wenn Sie kein Problem mit dieser Vorstellung haben (lassen wir philosophische oder politische Aspekte einmal außen vor), dann haben Sie verstanden, was die Grundidee der Objektorientierung ist, auch wenn wir dort eher von Maschinen bzw. virtuellen Einheiten sprechen.
Die folgenden Abschnitte behandeln zum Teil mehrere Zustände, die einfach deshalb gruppiert sind, weil die Auswirkung gleich, aber die Ursache für den Zustand unterschiedlich ist. Die Namen der Zustände, die in den einzelnen Abschnitten hervorgehoben sind, sind gleichzeitig Konstanten. Um zu prüfen, in welchem Zustand ein Thread sich befindet, können Sie die Methode getState() auf jedem Thread aufrufen.
16.3.1.	Lebensdauer bzw. Gültigkeitsbereich
Da wir hier über virtuelle Objekte reden, haben wir wie bei Variablen und Datenstrukturen einen Zeitraum, innerhalb dessen das virtuelle Objekt existiert. Wenn es erzeugt wurde, hat es den Zustand NEW, wenn es beendet wurde, hat es den Zustand TERMINATED.
Diese Zustände sind für uns im Regelfall belanglos, sollten Sie allerdings von einem Thread die Rückmeldung erhalten, dass er in einem dieser Zustände ist, dann bedeutet das, dass er noch nicht bzw. nicht mehr ausgeführt wird und auch nicht ausgeführt werden kann.
16.3.2.	Wartet auf den Startbefehl
Wenn ein Thread erzeugt wurde und es nichts gibt, wegen dem er seine Aufgabe(n) noch nicht anfangen kann, befindet er sich im Zustand RUNNABLE. Dieser Zustand tritt nur bei Erweiterungen der Klasse Thread auf.
16.3.3.	Nicht arbeitsfähig
Jetzt gibt es noch drei Zustände, wegen denen ein Thread nicht aktiv werden kann:
-	Zunächst gibt es die Situation, dass er auf eine Eingabe wartet, die noch nicht berechnet wurde. (Denken Sie an das Beispiel mit der Berechnung des Pascalschen Dreiecks.) In diesem Fall befindet er sich im Zustand BLOCKED. 

Genauer gesagt gibt dieser Zustand an, dass der Thread auf einen synchronisierten Thread wartet. Kurz gesagt ist Synchronisation eine Abstimmung im Programmablauf, bei der wir zwei Aufgaben miteinander abstimmen, die voneinander abhängen. Stellen Sie sich beispielsweise einen Videostream vor: Der Audio- und Videoanteil des Streams sollen „gleichzeitig“ abgespielt werden. Töne sollen also genau zum richtigen Zeitpunkt abgespielt werden. Methoden, die dazu dienen werden als Synchronisation bezeichnet.

-	Dann können wir einen Thread auch anweisen, dass er erst dann ausgeführt werden soll, wenn ein anderer Thread seine Aufgabe erfüllt hat, egal wie lange das dauert. Ein Thread, der aus diesem Grund wartet befindet sich im Zustand WAITING.

-	Ähnlich wie WAITING ist der Zustand TIMED_WAITING. Hier wartet der Thread maximal eine bestimmte Zeit lang darauf, dass ein anderer Thread seine Aufgabe erfüllt.
Zusammenfassung:
Die Schwierigkeit bei der Programmierung von Threads besteht also darin, zu verhindern, dass Threads dauerhaft im Zustand BLOCKED oder WAITING sind. 
Vorweg schon der Hinweis: Wenn ein Thread (das kann nur bei Erweiterungen der gleichnamigen Klasse passieren) nicht von NEW zu RUNNABLE wechseln, dann haben wir einen dummen Fehler gemacht. Allerdings können Sie den noch nicht kennen, weil wir uns dazu zunächst noch die Klasse Thread ansehen müssen.
16.4.	Thread und Runnable
Bevor wie zur Klasse Thread und dem Interface Runnable kommen, hier noch ein kurzer Hinweis: Threads können zwar nicht untereinander kommunizieren, aber in Java kann ein Objekt, das einen Thread erzeugt hat Methoden dieses Threads aufrufen. Diese Objekt kann also die einzelnen Threads koordinieren.
Jede Klasse, die Runnable implementiert bzw. Thread erweitert muss eine Methode namens 
public void run() enthalten, aber alles, was in diesem Bezug relevant ist, kennen Sie bereits aus TimerTask.
Erweiterungen von Thread müssen zusätzlich über die Methode start() von NEW in RUNNABLE versetzt werden. Wenn Sie das vergessen, existiert der Thread zwar als virtuelles Objekt, aber er ist nicht ausführbar. Das bedeutet, dass Sie zunächst mehrere Threads als Erweiterung von Thread programmieren können und sie dann nahezu zeitgleich „aktivieren“ können. Wenn Ihnen das nicht sinnvoll erscheint, denken Sie an die GUI-Programmierung zurück: Dort konnten Sie zunächst alle Komponenten der GUI erzeugen, bevor Sie die GUI per setVisible(true) anzeigen lassen. Wenn Sie das nicht machen, könnten Nutzer auf einem langsamen System sehen, wie die GUI Stück für Stück aufgebaut wird, bzw. wie hinzugefügte Komponenten das Layout ändern.
Wichtig: start() und run()können nicht von einem Thread selbst aufgerufen werden. Vielmehr muss das Objekt, das den Thread instanziiert hat diese Methoden aufrufen.
16.5.	Daemon – Ein hilfreicher Geist
Während in den monotheistischen Religionen Dämonen böse Geister sind, steht dieser Begriff in der griechischen Antike für hilfreiche Geister. Und was sollte ein Thread immer sein? Genau, ein Programmteil, der eine sinnvolle Aufgabe übernimmt, so lange es Sinn macht, dass er Sie ausführt.
Sie werden sich noch erinnern, dass Sie Instanzen von Timer mit new Timer(true) erzeugen sollten. Wenn wir das tun, dann existieren sowohl der Timer als auch die bei ihm scheduleten TimerTasks höchstens so lange, wie das Objekt existiert, das den Timer instanziiert hat.
Das gleiche haben Sie schon am Beginn des Semesters kennen gelernt, als es um die Konstante JFrame.EXIT_ON_CLOSE ging: Ein top-level-Container, der diese Konstante nicht (!) per setDefaultCloseOperation() erhalten hat, existiert als Prozess weiter im Systemspeicher, auch wenn das Fenster vom Nutzer geschlossen wurde.
Threads kennen eine Methode public void setDaemon(boolean b), mit der wir einen Thread zu einem Daemon machen können. (Dazu muss das Argument true sein.)
16.6.	Zeitweilige Freigabe des Prozessorkerns
Sie haben oben Zustände kennen gelernt, bei denen ein Thread aktiv ist, aber nichts tut, z.B. weil er auf die Ergebnisse eines anderen Threads wartet. Was Sie dort nicht erfahren haben ist, dass Threads in diesen Zuständen dennoch einen Prozessorkern blockieren und das macht natürlich wenig Sinn.
Wichtig: Das gilt nicht für einen Thread im Zustand BLOCKED. Ist ein Thread in diesem Zustand, dann steht der Prozessorkern, den er zuvor belegt hat für andere Threads zur Verfügung.
Um einen wartenden Thread aus dem Prozessorkern zu „entfernen“ können wir die Methoden 
public static void sleep(long milliseconds, int nanaseconds) nutzen.  
Die Methode public static void yield() können wir ähnlich wie sleep() verwenden. Anstatt hier einen Zeitraum vorzugeben, nachdem der Thread wieder an der Reihe ist, wird er bei yield() quasi ans Ende der Reihe gestellt. Stellen Sie sich dazu vor, dass alle Threads in einer Reihe anstehen und von der Java Laufzeitumgebung auf die verfügbaren Prozessorkerne verteilt werden.
16.7.	Vorzeitiges Beenden von sleep() und anderer Methoden
Dieser Abschnitt wurde nur der Vollständigkeit halber eingefügt. Wenn Sie sich später intensiver mit der nebenläufigen Programmierung beschäftigen, sollten Sie ihn durcharbeiten, um einen ersten Einblick in fortgeschrittene Methoden der nebenläufigen Programmierung zu erhalten.
Es kann natürlich auch sein, dass wir einen schlafenden oder anderweitig blockierten Thread wieder reaktivieren wollen. Dazu rufen wir public void interrupt() auf. Um diesen Fall zu beherrschen müssen Sie sich mit dem Exception Handling auskennen, das wir bislang noch nicht besprochen haben, was wir in diesem Kapitel auch nicht mehr tun werden. Die entsprechende Exception ist die InterruptedException.
Wir können nun beispielsweise innerhalb der run()-Methode unseres Thread mit der isInterrupted()-Methode Fallunterscheidungen einfügen, um die Ausführung im Falle eines Interrupts anzupassen.
Wir können obendrein dafür sorgen, dass der Interrupt nur vorübergehend ausgeführt wird. Der Thread befindet sich dann anschließend wieder im vorigen Zustand:
-	Wenn ein Interrupt nur dazu führen soll, dass der Thread einmalig ausgeführt, aber dann bis zum Ende des ursprünglich festgelegten Zeitraums weiterschlafen soll, dann verwenden Sie die Methode public static boolean interrupted().

Wichtig: Diese Methode müssen Sie sehr aufmerksam verwenden. Sie gibt zwar zurück, ob ein Thread interrupted wurde. Aber zusätzlich bewirkt sie, dass der Thread anschließend als nicht-interrupted gilt. Nehmen wir an, Sie rufen mehreren Methoden von run() auf und bei mehreren davon verwenden Sie interrupted() anstelle von isInterrupted(), um zu prüfen, ob sie im Falle eines Interrupt ausgeführt werden soll. Dann wird nur die erste dieser Methoden innerhalb des Interrupts tatsächlich ausgeführt: Da interrupted() bewirkt, dass der Thread anschließend nicht mehr als interrupted gilt, gibt interrupted() anschließend false aus.
16.8.	Synchronisation und Monitore – Verhinderung von Inkonsistenzen
Sie haben oben gelernt, dass nur solche Teile von Programmen nebenläufig programmiert werden können, die tatsächlich unabhängig voneinander ausgeführt werden können. Zur Erinnerung: Wenn Sie das ignorieren, erhalten Sie Threads die blockiert sind. Allerdings (und das lernen Sie gleich kennen) gibt es auch andere Fälle, in denen Threads blockiert sind.
Jetzt kommen wir zu einem Fall, in dem mehr als ein Thread Variablen oder Datenstrukturen ändern und dadurch logische Fehler erzeugen. Diese Fälle sind wesentlich schlimmer als blockierende Threads, da es keine Programmiertechnik gibt, die das Auftreten eines solchen Falles zeigt. In anderen Worten: Wenn ein solcher Fehler auftritt, haben Sie einen Fehler im Programm, aber das Programm arbeitet weiter als wenn alles in Ordnung wäre.
Ein Beispiel kennen Sie, wenn Sie schon Erfahrung mit Datenbanken gesammelt haben. Es gehört in den Bereich der sogenannten Inkonsistenz. Aber zur Erklärung können wir auch ein einfaches Programm nutzen. Stellen Sie sich dazu vor, dass Sie eine int-Variable haben, die von einem Thread gelesen wird. Der Thread stellt nun einige Berechnungen an und überschreibt dann die Variable.
Wenn Sie verstanden haben, wie nebenläufige Programmierung funktioniert, dann erkennen Sie das Problem sofort. Für alle anderen zur Erinnerung: Nebenläufige Programmierung bedeutet, dass wir Programmteile unabhängig voneinander arbeiten lassen. Das bedeutet, dass wir mit den bislang besprochenen Mitteln keine Möglichkeit haben, zu verhindern, dass ein zweiter Thread unsere int-Variable überschreibt, während der erste Thread noch seine Berechnung durchführt. Hat der erst Thread dann seine Berechnung durchgeführt und überschreibt die Variable, dann ist damit die Variable inkonsistent, sprich ihr Wert stimmt nicht mehr. Der Grund ist simpel: Der zweite Thread hat zwar seine Berechnung durchgeführt, aber das Ergebnis wurde ohne weitere Prüfung vom ersten Thread überschrieben.
Sie haben nun zwei Möglichkeiten, um solche Fälle zu vermeiden: 
-	Sie können zum einen eine Methode mit dem Schlüsselwort synchronized gegen die gleichzeitige Ausführung durch mehrere Threads sperren.

Bsp.: Die Methode aendereX() kann nur von einer Instanz der nachfolgenden Klasse Nebenlaeufig gleichzeitig ausgeführt werden. Man könnte also sagen, dass dieses Schlüsselwort nebenläufige Programme partiell in prozedurale Programme verwandelt.

class Nebenlaeufig extends Thread {
	...
	private synchronized void aendereX(int y){
		if (y > 0)
		{x += y;}
		else
		{x -= <;}
	}
}

-	Wenn Sie dagegen nur Teile von Methodenaufrufen synchronisieren wollen, geht auch das. Sie verwenden dann die Methode synchronized(Object o) { ... }, der Sie ein beliebiges Objekt übergeben. Dieses Objekt ist dann so etwas wie eine Semaphore oder ein Token. Beide Begriffe stehen für Objekte, die nur einmal existieren. Und nur der-/diejenige, der es hat darf etwas tun. Im Bereich der Netzwerke gibt es beispielsweise den sogenannten Token-Ring. Hier darf nur der Rechner eine Datenübertragung im Netz durchführen, der das Token hat. Oder denken Sie an Diskussionskreise, in denen immer nur ein Gesprächsteilnehmer zurzeit etwas sagen darf. Das ist die-/derjenige, die einen beliebigen Gegenstand in der Hand hält. Machen Sie sich also nicht zu viele Gedanken darüber: Diese Objekt hat mit der eigentlichen Ausführung des Rumpfes von synchronized() nichts zu tun. Definieren Sie einfach ein beliebiges Objekt, beispielsweise eine Variable oder nutzen Sie this und nutzen Sie es ausschließlich als Argument für synchronized().
16.9.	Das Grauen? – Nebenläufigkeit und SWING
Im Grunde würde dieses Kapitel mit dem vorigen Abschnitt enden, wenn die Entwickler von Java bei GUIs mit SWING keine Ausnahme gemacht hätten: Alle Elemente einer GUI in SWING werden durch den sogenannten event dispatching thread verwaltet. Einzig die Methoden repaint(), revalidate() und invalidate() sind davon ausgenommen.
Die zentrale Auswirkung für uns besteht darin, dass wir bei Komponenten einer GUI keine Threads so nutzen dürfen, wie wir das gerade erst gelernt haben.
Weiterhin dürfen wir bei GUIs ausschließlich Implementierungen von Runnable verwenden. Erweiterungen von Thread sind also verboten.
Und außerdem müssen wir einen Thread jetzt z.B. mit 
SwingUtilities.invokeLater(new Runnable() {...} ); zum event dispatching thread hinzufügen. Wenn wir das getan haben, wird der neue Thread quasi ans Ende der bereits vorhandenen Threads der GUI angefügt und dann dort verwaltet.
Wollen wir eine Methode unseres Thread aufrufen, dann dürfen wir das ausschließlich dann tun, wenn der event dispatching thread uns dazu eine Möglichkeit bietet. Ein direktes Aufrufen wie wir das bislang getan haben ist verboten!

17.	Fehler und nicht-prozedurale Abläufe
Bei der Programmierung in einer beliebigen Sprache können vier Fälle auftreten, die Einsteiger als Fehler bezeichnen würden. Wenn wir genauer hinsehen, können wir noch wesentlich genauer differenzieren, aber für diesen Kurs soll die grobe Einteilung in vier Fälle genügen.
17.1.	Syntax Error – Syntaktische Fehler
Egal in welcher Programmiersprache Sie programmieren, diese Fehler werden Sie immer wieder machen, schlicht, weil es menschlich ist, sich zu vertippen. Mit dem Begriff des Syntax Fehlers bezeichnen wir alle Abweichungen von der Rechtschreibung einer Programmiersprache. (Das schließt auch solche Dinge ein wie das Vergessen eines Semikolons am Ende einer Programmzeile.)
Syntaktische Fehler sind die Fehler, die im Grunde am leichtesten zu lösen sind, weil Sie dazu führen, dass der Compiler eine Fehlermeldung auswirft.
17.2.	Semantische Fehler – „Warum macht der das denn nicht?“-Fehler
Diese Fehler sind für uns als Softwareentwickler die schwierigsten Fehler, aber genau wie syntaktische Fehler liegt die Ursache für diese Situationen bei uns.
Denn während wir bei einem syntaktischen Fehler etwas eingegeben haben, dass der Rechner bzw. die Programmiersprache nicht ausführen kann, haben wir bei einem semantischen Fehler etwas anderes programmiert, als das, was wir programmieren wollten.
Die Ursache dafür liegt im zentralen Unterschied zwischen menschlicher Sprache, Mathematik und Programmiersprachen:
In menschlichen Sprachen haben die meisten Sätze mehrere mögliche Bedeutungen, die sich aus dem Kontext ergeben, in dem der jeweilige Satz genannt wurde. Auch unterschiedliche Betonung der einzelnen Wörter kann sich eine andere Bedeutung ergeben.
In der Mathematik dagegen ist die Betonung von Satzteilen irrelevant. Dafür hat ein Satz bzw. eine Zeile eines mathematischen Textes häufig eine wesentlich größere Anzahl an möglichen Bedeutungen. Nehmen wir die Zeile 3 + 7 = 10. Diese Zeile besagt schlicht: Wenn wir von einer Art von Objekten drei Stück nehmen, von der gleichen Art nochmals sieben, dann haben wir insgesamt zehn davon. Ob wir dabei über Trecker, Punkte in Flensburg oder Knochenbrüche reden ist für die Mathematik irrelevant. Diese annähernd unbegrenzte Möglichkeit der Interpretation eines Satzes in der Mathematik bereitet uns bereits große Probleme, weil Sie eben nicht direkt der Art entspricht, wie wir uns alltäglich unterhalten. Das liegt aber schlicht daran, dass wir uns im Alltag über mehr oder weniger konkrete Themen sprechen, während die Mathematik auch und insbesondere nach allgemeingültigen Gesetzmäßigkeiten sucht.
Programmiersprachen stellen uns dagegen vor das genau umgekehrte Problem: Hier hat jeder Satz eine ganz präzise definierte Bedeutung. Deshalb führt jedes Programm genau die Aufgabe/n aus, mit denen es programmiert wurde. Aber da wir nicht daran gewöhnt sind, dass jeder Satz exakt eine Bedeutung hat, neigen wir dazu, uns mit einem ungefähren Verständnis von Teilen einer Programmiersprache abzufinden. Das führt dann dazu, dass wir in vielen Fällen Programme entwickeln, die ungefähr das tun, was wir wollen. Früher oder später tritt dann aber ein Fall auf, der aufgrund unserer Programmierung eben nicht so abgearbeitet wird, wie wir das wollen. Der Fehler kommt dann schlicht dadurch zustande, dass wir nicht genau genug gearbeitet haben. Der erste Fall semantischer Fehler, mit dem die meisten Einsteiger zu tun bekommen tritt bei der Iteration über ein Array auf.
17.3.	Interrupts – Unterbrechungen des aktuellen Programmablaufs
Wie Sie in Informatik 3 gelernt haben entsteht ein Interrupt durch die unterschiedlichsten Einwirkungen einer beliebigen Komponente des Rechners. Beispiele wären das Drücken einer Taste auf der Tastatur sein oder ein Signal, das die Netzwerkkarte empfängt.
Nun gibt es zwei Möglichkeiten: Unser Programm soll auf einen Interrupt reagieren oder eben nicht. Wenn wir darauf reagieren, dann sprechen wir von der Behandlung eines Interrupts. In Java kennen Sie das als Events.
Wenn wir also Interrupts behandeln, dann bedeutet das in den meisten Programmiersprachen, dass wir den Programmablauf an der aktuellen Stelle pausieren, einen anderen Programmbereich abarbeiten und später wieder an unsere aktuelle Position zurückkehren, damit das Programm hier fortgesetzt wird. Da Java keinen direkten Zugriff auf die Ressourcen des Rechners zulässt, haben wir es hier auch nicht mit Interrupts in diesem Sinne zu tun: Um auf die Eingaben von Nutzern zu reagieren nutzen wir Klassen, die solche Interrupts als Event-Objekt zur Verfügung stellen. Wie Sie das tun müssen haben Sie kurz nach Beginn des Semesters kennen gelernt.
17.4.	Exceptions – Ausnahmen von der Regel
(Noch Einbauen: Murphies Law als anschauliche Verallgemeinerung für Exceptions)
In den vorigen Abschnitten ging es um Situationen, bei denen wir entweder einen Schreib- oder einen Denkfehler gemacht haben. Dann folgten Situationen, in denen wir unser Programm auf Einflüsse von außen reagieren ließen. Exceptions werden von Einsteigern für syntaktische Fehler gehalten, aber das ist falsch. Denn Exceptions sind ein Mittel, mit dem wir Abweichungen vom normalen Programmablauf in unser Programm integrieren können.
Java gehört zu den Sprachen, die Exception Handling, also die Behandlung von Exceptions anbieten. Und hier ist auch gleich die Verwandtschaft zu den Interrupts: Es handelt sich nicht etwa um Fehler, die nicht auftreten dürfen, sondern es handeln sich um Fälle, in denen etwas passiert, das so zwar vorkommen kann, aber eben eine Ausnahme ist. Tritt eine solche Ausnahme bei einer Methode auf, dann sagen wir, dass die Methode eine Exception wirft. In diesem Fall haben wir also keinen Rückgabewert.
Gleich vorweg: Wann immer eine Methode eine Exception wirft müssen wir mit try/catch arbeiten. Dabei schreiben wir den Programmteil, der eine Exception werfen kann in den try-Rumpft und ersten für jede mögliche Exception einen catch-Rumpft, wobei catch genau wie eine Methode von einem Klammernpaar gefolgt wird, das ein Argument enthält. In diesem Fall ist das Argument eben der Name der Exception und eine Variable, mit der wir das Exception-Objekt im catch-Rumpf ansprechen können.
Exceptions sind also mehr als nur Kontrollstrukturen und sind wichtig, sobald wir mit Vererbung anfangen. Nehmen wir dazu ein Beispiel, das Sie in der letzten Hausaufgabe kennen gelernt haben. Dort hatten Sie mit einer Methode zu tun, die die sogenannte InterruptedException werfen konnte. Diese Exception tritt, um es in einfachen Worten zu beschreiben dann auf, wenn ein Thread gerade nicht ausgeführt werden darf.
Da jeder Thread in Java nach dem Aufruf von start() eine solche Exception werfen kann, müssen wir bei Aufrufen von späteren Methoden immer mit try/catch arbeiten. Nehmen wir dazu das Beispiel der letzten Hausaufgabe. Dort sollte auf einen Thread die Methode stop() aufgerufen werden, um ihn zu beenden.
Hier eine einfache Lösung:
...
try{
	thread1.stop();
}
catch (InterruptException e){}
...

In diesem Fall haben wir effektiv programmiert, dass der Methodenaufruf stop() nicht ausgeführt wird, wenn er zu einer InterruptException führt. Ansonsten können Sie im catch-Rumpf wie gewohnt alles programmieren, was Sie wollen. Außerdem können Sie verschiedene Methoden auf Exception-Objekten aufrufen. Werfen Sie dazu bitte wie gewohnt einen Blick in die Java API.
Einsteiger verwenden hier häufig die Methode getMessage(), mit der sie sich eine detaillierte Meldung zur Exception ausgeben lassen können. Das sollten Sie jedoch nur dann tun, wenn Sie nicht genau verstehen, warum die Exception geworfen wurde. Im fertigen Programm müssen Sie alle möglichen Exceptions durch das Programm lösen lassen.
Aber wie schon eingangs beschrieben müssen Sie nicht unbedingt ein echtes Exception Handling durchführen: Wenn die Exception für den Programmablauf irrelevant ist, dann nehmen Sie einen try/catch wie oben gezeigt.

18.	Java und Datenübertragungen
Im Kapitel „Dateien und Dialoge“ haben Sie gelernt, was Datenströme sind. Wenn Sie hier nicht mehr ganz sicher sind, worum es sich dabei handelt oder nicht sofort erkennen, warum das für die Übertragung von Daten über das Internet wichtig ist, dann arbeiten Sie bitte nochmal die entsprechenden Seiten im Kapitel zu Dateien und Dialogen durch. Denn die gleichen Grundlagen, die für den Zugriff auf Dateien gelten sind auch in diesem Kapitel gültig.
Allerdings ist die Nutzung eines Datenstroms im Falle der Datenübertragung über das Internet (bzw. Netzwerke im Allgemeinen) etwas komplexer als beim Zugriff auf Dateien. In der Veranstaltung NWI hören Sie gerade die Grundlagen der Datenübertragung in Netzwerken, weshalb an dieser Stelle nichts weiter über Themen wie Package Loss, Three-Way-Handshake und ähnliches erläutert wird. Aber auch diese Themen müssen Sie verinnerlicht haben. Sollten hier noch Unklarheiten bestehen, dann setzen Sie sich bitte nochmal an Ihre Unterlagen zu NWI.
Die Begriffe Server und Client dagegen werden wir hier nochmal betrachten, weil Sie streng genommen nicht in den Bereich von NWI fallen: Beide nutzen Netzwerke, um darüber Daten auszutauschen und leider gibt es ein häufiges Missverständnis, wonach sowohl Server als auch Client Rechner in einem Netzwerk sind. Wie gesagt handelt es sich hier um ein Missverständnis: Es gibt zwar Rechner, die speziell als Server verkauft werden. Aber dabei handelt es sich um Geräte, die eine besondere Sicherheit bieten. Diese haben dann beispielsweise eine USV (unterbrechungsfreie Stromversorgung), also eine Komponente, die sicherstellt, dass der Server auch dann weiter aktiv bleibt, wenn ein Stromausfall für eine begrenzte Zeit auftritt. Wie gesagt ist die Absicherung gegen Stromausfälle nur eine Absicherung gegen Risiken, für die Rechner ausgelegt sein können, die als Server verkauft werden.
Server und Clients sind tatsächlich Programme oder Programmteile, die auf Rechnern laufen. Ob sie ein Netzwerk nutzen, um Daten auszutauschen oder beide sich auf dem gleichen Rechner befinden ist dafür irrelevant. Was beide unterscheidet ist Ihre Arbeitsweise: Ein Client ist ein Programmteil, der Daten anfordert, wobei häufig auch von der Anforderung eines Dienstes gesprochen wird. Das ist jedoch aus Sicht der Softwareentwicklung falsch: Der Server führt zwar ein Programm aus, das als Dienst bezeichnet werden kann, aber er liefert eben nicht den Dienst an den Client aus, sondern er liefert die Daten aus, die durch die Durchführung des Dienstes entstanden sind. Es mag zwar trivial klingen, ist aber für uns als Softwareentwickler ein essentieller Unterschied. Nicht zu vergessen: Zusätzlich ist ein Client im Stande, vom Server Daten anzunehmen. Auch das mag trivial klingen, aber da wir sicherstellen müssen, dass der Client diese Funktion erfüllt, dürfen wir es nicht vergessen. Der Server dagegen ist ein Programmteil, der auf Anfragen von anderen Programmen wartet, diese Anfragen abarbeitet und als Antwort Daten an diese anderen Programme sendet.
Ein Sonderfall wäre ein Server, der lediglich den Eingang von Nachrichten annimmt. Sollten Sie so etwas aus Sicherheitsgründen entwickeln wollen, dann beachten Sie bitte, dass das nicht bedeutet, dass gar keine Antwort an den Rechner übertragen wird, auf dem der Client betrieben wird. Die Grundlagen von TCP/IP gelten hier unverändert, denn Sockets basieren darauf.
Wenn wir also in Java Programme entwickeln wollen, die Daten an andere Rechner übertragen sollen (z.B. für online Games, Bankensysteme, Fernsteuerungen usw.), dann müssen wir uns mit der Fragen beschäftigen, wie wir ein Programm in Java um einen Client bzw. um einen Server erweitern können. Über die Frage, wie wir die Sicherheitsprobleme lösen, die Sie in NWI grundsätzlich kennen lernen, werden wir hier nicht sprechen. Das bedeutet nicht, dass diese Probleme irrelevant wären, aber hier müssen Sie selbst aktiv werden, um sich die nötigen Kenntnisse anzueignen.
18.1.	Sockets – Klassen für den Datenaustausch über Netzwerke
Wie gewohnt nutzen wir in Java eine bzw. mehrere Instanzen von Klassen, um auf Datenströme über ein Netzwerk zuzugreifen. In diesem Kapitel sehen wir uns dazu verschiedene Socketklassen an. Das sind Klassen, die es uns ermöglichen, Daten per TCP/IP zu übertragen.
Wichtig:
Bitte überlesen Sie das nicht: Sockets zu nutzen bedeutet automatisch TCP/IP zu nutzen. Also gelten hier alle Vor- und Nachteile sowie alle weiteren Spezifikationen und Funktionalitäten, die auch für TCP/IP gelten. Wir haben also keine Möglichkeit, um die beispielsweise die zeitlichen Nachteile von TCP/IP auszugleichen. Wenn Sie also auf derartigen Humbug verfallen, wie die Behauptung, Sie könnten mit Sockets eine Echtzeitanwendung erstellen, dann ist das gleichbedeutend mit der Behauptung, Sie könnten in Java ein Programm erstellen, das schneller arbeitet als die Lichtgeschwindigkeit: Da TCP/IP aufgrund des ARQ für die Übertragung einzelner Pakete durchaus mehrere Sekunden benötigen kann, müssten Sie für eine Echtzeitanwendung ein Java-Programm mit Sockets verfassen, das die eingehenden Daten zu einem Zeitpunkt verarbeitet, der vor dem Empfang der Daten liegt. Da würden sowohl Einstein als auch Heisenberg den Kopf ob Ihrer Naivität schütteln.
Bevor wir loslegen können brauchen wir zwei Angaben:
-	Die IP-Adresse des Servers.
-	Den Port, auf dem der Server auf eingehende Anfragen wartet.
Zur Erinnerung:  Die Ports 0 bis 1024 sind für bestimmte Protokolle reserviert, wenn wir also eine eigene Anwendung mit Netzwerkanbindung entwickeln wollen, sollten wir prüfen, ob wir einen dieser Ports benutzen können (weil wir das entsprechende Protokoll nutzen werden) oder eben einen anderen Port (max. 65535) verwenden.
Die beiden Klassen, die wir benötigen, um einen Client und einen Server zu programmieren sind Socket und ServerSocket. Wenn Sie eine sichere Datenübertragung realisieren möchten, dann besteht der erste Schritt darin, die Klassen SSLSocket und SSLServerSocket zu verwenden. In diesem Kapitel werden wir uns mit beiden nicht auseinander setzen, da es hier lediglich um die Grundlagen gehen soll, mit denen wir überhaupt eine Datenübertragung realisieren können. In der Praxis müssen Sie dagegen die SSL-Varianten nutzen. Beide erben von Socket bzw. ServerSocket, sodass Sie dort alles weiter nutzen können, was Sie hier lernen.
Leider ist die klare Trennung in Client und Server bei Java nicht konsistent eingehalten worden. Hier benötigen wir zwar eine Instanz von Socket für jeden Client und eine Instanz von ServerSocket für den Server, aber serverseitig wird für jede Verbindung zu einem Client eine Instanz von Socket erzeugt. Der eigentliche Datenaustausch wird dann auf beiden Seiten mit einer Instanz von Socket realisiert.
Das ist leider nicht die einzige Inkonsistenz, wenn wir über Datenübertragungen mit Sockets sprechen. Der Grund dürfte darin liegen, dass Sockets aus der Urzeit von Java stammen: Sie wurden bereits mit der Version 1.4 eingeführt. Wir müssen hier also damit zurechtkommen, dass die Programmierung nicht so konsistent erfolgen kann, wie wir uns das wünschen würden. Aber keine Sorge: Bei der Programmierung sind die Auswirkungen nicht so umfangreich, wie Sie das jetzt befürchten könnten.
18.2.	Instanziieren der Sockets für Client und Server
Da eine Instanz von Socket eine Exception wirft, wenn unter der angegebenen IP-Adresse kein Server erreichbar ist, müssen wir zunächst den Server implementieren. Das passiert in zwei Schritten: 
-	Im ersten Schritt erzeugen wir als Server unseres Programms eine Instanz von ServerSocket, die zwei Argumente erhält: Den Port, auf dem der Server auf eingehende Anfragen wartet und die Anzahl Clients, für die der Server maximal gleichzeitig erreichbar sein soll. Bezüglich der Anzahl Clients machen Sie sich bitte das Problem des distributed denial of Service-Angriffs (kurz DDoS) bewusst, das Sie in NWI kennen lernen können.

ServerSocket server = new ServerSocket(portNum, queueLength);

-	Wir haben jetzt zwar eine Instanz von ServerSocket, aber so wie eine GUI erst dann bedienbar ist, wenn wir sie sichtbar gemacht haben, müssen wir unseren Server auch zunächst auffordern, tatsächlich auf eingehende Anfragen zu hören. Lassen Sie sich bitte nicht davon irritieren, dass wir dazu eine eigene Instanz von Socket im Serverteil des Programms erzeugen. Der Grund besteht wie oben beschrieben darin, dass eine ServerSocket-Instanz die Anfragen nicht im Sinne eines Servers selbst verarbeitet. Bei Java ist auch das die Aufgabe von Instanzen von Socket. Achten Sie deshalb darauf, Bezeichner für die Socket-Instanzen zu verwenden, die eindeutig sind.

Socket serverSocket = server.accept();
Da wir jetzt einen Server haben, dem Clients Anfragen stellen können, kommen wir als nächstes zur Instanziierung eines Clients. Für den benötigen wir wie besprochen die IP-Adresse des Servers und nochmal die Portnummer des Servers. Um IP-Adressen und URLs als Argument bei der Instanziierung unseres Socket-Objekts programmieren, müssen wir auf die Klasse InetAddress zurückgreifen. Und hier haben wir die zweite Inkonsistenz, mit der wir bei der Programmierung von Sockets zu tun bekommen: Es gibt keinen Konstruktor für InetAddress. Die Gründe liegen vermutlich im Versuch, das sogenannte Fabrik-Pattern aus dem Bereich der Software Architektur umzusetzen, aber da Software Architektur ein Spezialbereich des Software Engineering ist und Sie bislang noch keine grundlegende Einführung ins Software Engineering haben, lassen wir die Details an dieser Stelle außen vor.
Um eine gültige InetAddress-Instanz zu erhalten, und einen Client-Socker zu instanziieren, können wir folgenden Weg nutzen, der für unsere aktuelle Anwendung vollkommen ausreicht: (Port 12345 ist hier rein willkürlich ausgewählt.)
String host = „localhost“;
int serverPort = 12345;
InetAddress serverAddress = InetAddress.getByName(host);
Socket clientSocket = new Socket(serverAddress, serverPort);
oder kurz:
Socket clientSocket = new Socket(InetAddress.getByName(„localhost“), 12345);
Wenn Sie eine IP-Adresse verwenden wollen (nehmen wir hier als Beispiel die 127.0.0.1), dann programmieren Sie das genauso wie eine URL. In dem Fall wäre also String host = „127.0.0.1“; bzw.
Socket clientSocket = new Socket(InetAddress.getByName(„127.0.0.1“), serverPort);
Damit haben wir jetzt alles vorbereitet, um Anfragen vom Client an den Server und Antworten vom Server zu Client zu schicken.
Anmerkung:
-	Beachten Sie bitte, dass in diesem Kapitel keine Hinweise bezüglich des Exception Handlings auftauchen. Tatsächlich müssen Sie jedoch in diesem Bereich eine Vielzahl von Exceptions behandeln, was vorrangig mit den Problemen zusammenhängt, die Sie in NWI kennen lernen.
18.2.1.	Mehrere Clients und die Verbindung zum Server
Nun ist es allerdings weitestgehend sinnlos, nur eine Verbindung von einem Client zu einem Server aufzubauen. Wenn Sie es nicht selbst schaffen, mittels Multithreading mehrere Threads mit einem Server zu verbinden, dann werfen Sie einen Blick auf diese Diskussion: http://stackoverflow.com/questions/10131377/socket-programming-multiple-client-to-one-server Dort finden Sie mehrere Lösungen, die zum Teil sehr kurz ausfallen. Dadurch sollte es Ihnen leicht fallen, diese Lösungen in Ihre Programme zu übernehmen.
18.3.	Schließen der Verbindung
Dieser Abschnitt ist kurz aber sehr wichtig: Da wir durch die verschiedenen Sockets Prozesse gestartet haben, die im Hintergrund Systemressourcen belegen ist es extrem wichtig, die verschiedenen Sockets am Ende wieder zu schließen. Dazu nutzen wir wie schon bei der Dateiübertragung wieder die Methode close().
18.4.	Zugriff auf den Datenstrom
Die einfachste Anwendung von Datenübertragungen nutzen die meisten von Ihnen mehrfach jeden Tag: Instant Messenger sind Programme, über die Nutzer eine Textnachricht an einen oder mehrere Empfänger bzw. in einen Chatroom senden können. Weiterhin zeigen diese Programme Textnachrichten anderer Nutzer an. Sehen wir uns dazu an, wie wir unsere Sockets nutzen können, um über ein Netzwerk konkrete Texte zu übertragen.
Wenn Sie sich an dieser Stelle fragen, wie Sie denn hier die Anzeige der Texte realisieren sollen, dann erinnern Sie sich bitte daran, wie Sie eine GUI mit einem Textfeld erzeugt haben. Denn genau so können Sie natürlich auch einen Instant Messenger programmieren: Sie erweitern schlicht Ihre GUI um einen Client und programmieren einen Server, der die Texte an andere Clients weiterleitet.
An dieser Stelle kann es etwas unübersichtlich werden, weil wir hier nicht mehr zwischen Client und Server unterscheiden: Das folgende sind Abläufe, die bei beiden identisch programmiert werden, weil auf beiden Seiten eine Instanz von Socket aktiv ist.
Wie Sie wissen erwarten sowohl der Server als auch der Client Eingaben und erzeugen Ausgaben. Die Ausgaben des Clients sind seine Anfragen und diese Anfragen sind die Eingaben des Servers. Umgekehrt sind die Antworten des Servers seine Ausgaben und gleichzeitig die Eingaben des Clients. Um das in den Codebeispielen zu verdeutlichen werden in diesem Kapitel die Variablenbezeichner clientRequest und serverReply verwendet. Um dann noch zu verdeutlichen, ob es sich um zu versendenden oder empfangene Daten handelt, wird jeweils noch ein In oder Out an den Bezeichner gehängt, sodass wir bei clientRequestIn und clientRequestOut sowie serverReplyIn und serverReplyOut landen. Bei den Datenströmen kommt dann noch die Endung Stream dazu usw.
Wichtig:
Sie müssen dabei daran denken, dass Client und Server jeweils Teil einer eigenen Klasse sein sollten. Die Instanzen dieser Klassen können später dank TCP/IP über beliebige Netzwerke Daten austauschen. In diesem Fall mag es zwar möglich sein, beide in einer Klasse zu programmieren, aber ich empfehle Ihnen ausdrücklich, für jeden eine individuelle Klasse zu programmieren, wovon die eine all das enthält, das auf der Clientseite wichtig ist und die andere all das, was auf der Serverseite wichtig ist. Wenn beide gemeinsame Eigenschaften und Methoden haben, dann lagern Sie diese wie gewohnt in eine gemeinsame Superklasse aus.
Weiter im Thema:
Bevor wir jedoch auf den Datenstrom zugreifen können, wie wir das bei der Programmierung von Zugriffen auf Dateien gemacht haben, müssen wir zunächst Datenstromobjekte erzeugen. Denn Sockets sind nur dazu da, um die Verbindung zwischen Server und Client aufzubauen bzw. zu verwalten. Auf den Datenstrom, der über diese Verbindung fließt greifen wir mit Objekten vom Typ InputStream und OutputStream zu.
Instanzen erhalten wir über die Methodenaufrufe getInputStream() und getOutputStream(), die für Socket definiert sind. Setzen wir das in unserem Beispiel um:
-	Beim Server:

InputStream clientRequestInStream = serverSocket.getInputStream();
OutputStream serverReplyOutStream = serverSocket.getOutputStream();

-	Beim Client:

InputStream serverReplyInStream = serverSocket.getInputStream();
OutputStream clientRequestOutStream = serverSocket.getOutputStream();
Diese Bezeichner erscheinen für den Praxiseinsatz etwas lang, aber dafür haben sie den Vorteil, dass kein Missverständnis möglich ist. Gerade wenn Sie bei der Entwicklung einer Client/Server-Anwendung zwischen beiden häufig hin- und herspringen, wird Ihnen das helfen, Fehler zu vermeiden. Am Ende wird es sich also auszahlen und ist eindeutig praxisrelevant.
Damit haben wir bis jetzt die Verbindung zwischen Server und Client aufgebaut und außerdem eine Zugriffsmöglichkeit auf den Datenstrom geschaffen. Jetzt müssen wir nur noch Daten übertragen.
18.5.	Riding the data stream
Die Klasse, um Eingaben aus einem Datenstrom auszulesen kennen Sie schon: Scanner.
Im Gegensatz dazu haben Sie Ausgaben für Nutzer bislang durch eine von zwei Varianten erzeugt:
-	Variante 1: 
Sie haben per System.out.println() eine Ausgabe direkt in der Konsole erzeugt.

-	Variante 2:
Sie haben eine Text-Komponente einer GUI geändert und dadurch indirekt die entsprechende Textausgabe erzeugt.
Beides kommt für unseren Datenstrom nicht in Frage. Dort verwenden wir die Klasse PrintWriter.
Jetzt zeigt sich auch der Vorteil der etwas langen Bezeichner für unsere Datenstromobjekte. Denn hier ist der Code, um Texte auf den Datenstrom zu schreiben bzw. von dort zu lesen:
-	Hier für den Server:
Scanner clientRequestIn = new Scanner(clientRequestInStream);
PrintWriter serverReplyOut = new PrintWriter(serverReplyOutStream);

-	Hier für den Client:
Scanner serverReplyIn = new Scanner(serverReplyInStream);
PrintWriter clientRequestOut = new PrintWriter(clientRequestOutStream);
Alles weitere (für Textübertragungen) funktioniert genauso, wie Sie es bislang mit der Klasse Scanner bzw. System.out kennen gelernt haben. Es mag jetzt etwas ungewohnt sein, Texte als Datenströme zu übertragen und dabei die gleichen Methoden zu verwenden, die Sie vorher nur für die Ausgabe auf der Konsole verwendet haben, aber das wird sich bald geben.
Viel besser ist, dass Sie mit einem PrintWriter, der auf einen Datenstrom zugreift nicht nur Texte, sondern auch vollständige Objekt per print() bzw. println() übertragen können. (Werfen Sie dazu einen Blick in die Java API.)
Wichtig:
Beachten Sie bitte, dass Sie die Ströme am Ende mit close() schließen müssen. Das muss insbesondere passieren, bevor Sie die zugehörige Instanz von Socket schließen.
18.6.	Spülen wir die Rest in der Toilette runter.
Gelegentlich wird es vorkommen, dass Sie die Reste einer Datenübertragung nicht mehr verarbeiten wollen. In diesem Fall müssen Sie sie mit dem Methodenaufruf flush() auf der jeweiligen Scanner-Instanz verwerfen, um Inkonsistenzen oder Exceptions zu vermeiden.
18.7.	Zusammenfassung
Um eine TCP/IP-Datenübertragung durchzuführen müssen Sie:
-	Einen ServerSocket als Server-Teil des Programms
und server- wie clientseitig je eine Socket-Instanz für die Kommunikation erzeugen.
-	Der Socket auf der Serverseite muss per accept() Annahmen registrieren,
bevor der Clientsocket instanziiert wird.
-	Auf beiden Seiten benötigen wir dann je ein Datenstromobjekt für den Input und eines für den Output, die über einen Methodenaufruf auf dem jeweiligen Socket erzeugt werden.
-	Um dann Daten zu senden bzw. empfangen nutzen wir Instanzen von Scanner bzw. PrintWriter.
-	Manchmal müssen wir per flush() einen Datenstrom leeren.
-	In jedem Fall müssen wir die Verbindung mit close() auf jeder dieser Instanzen schließen. Die Reihenfolge ist dabei wichtig: Sie ist genau umgekehrt wie beim Verbindungsaufbau.
Alles Weitere sind Details, die vom Anwendungsfall abhängen und natürlich das Exceptionhandling.
Sie beherrschen jetzt die Grundlagen, um jede denkbare vernetzte Anwendung zu entwickeln, die Ihnen in den Sinn kommt. Es gibt lediglich zwei Fälle, in denen Sie noch einmal zusätzliche Kenntnisse benötigen, die Sie sich jetzt allerdings recht schnell selbst aneignen dürften: Die Datenbankanbindung z.B. mit JSP oder JDBC und Audioverarbeitung. Ersteres ist Teil der Veranstaltung RDB, letzteres können Sie von Herrn Plaß aus den vorigen Semestern erhalten.
Weiterhin gibt es noch Vorgehensweisen, die bei großen Softwareprojekten die Fehleranfälligkeit senken und die Entwicklungsdauer reduzieren. Damit beschäftigen Sie sich in der Veranstaltung Software Engineering. Hierzu gehören insbesondere die sogenannten Unit Tests, die zum Entwicklungsmuster des Test-Driven-Development (kurz TDD) gehören.


% Nachfolgender Abschnitt muss noch an Latex angepasst werden.
%\chapter{Anhänge}
\section{Anhang A – (Mathematische) Grundlagen für die Programmierung}
\subsection{Modulare Arithmetik}
\subsection{Computerspeicher und Zahlenbereiche}
\paragraph{Programmierung ohne signed Integer}
\paragraph{Programmierung mit signed Integer}
\paragraph{Overflow und Carry}
\paragraph{Übersicht: Signed und unsigned Integer für 3-Bit-Computer}
Wenn Sie Computer programmieren müssen sie verschiedene mathematische Grundlagen beachten, da Sie die Grundlage für die Funktionsweise von Computern darstellen. Dieser Anhang kann eine vollwertige Mathematikvorlesung nicht ersetzen, aber darum geht es hier auch gar nicht; vielmehr soll Ihnen dieser Abschnitt dazu dienen, den Übergang von der häufig rein abstrakten Mathematik hin zur etwas konkreteren Anwendung beim Programmieren zu erleichtern.
A.1. Modulare Arithmetik
Modulare Arithmetik spielt bei der Programmierung von Computern immer dann eine zentrale Rolle, wenn Sie direkt auf Speicher zugreifen können. (Deshalb wird dieses Thema auch in der Veranstaltung Mathematik 2 für Media Systems behandelt.) Daraus resultiert, dass es ein essentieller Aspekt im Bereich der IT-Sicherheit ist, sowohl bei Angriffsszenarien (z.B. Viren) als auch Verteidigungstechniken.
Die Modulare Arithmetik kennen Sie seit der Kindheit, auch wenn Sie vielleicht heute zum ersten Mal den Begriff hören bzw. lesen. Ein einfaches Beispiel ist die Uhrzeit: Nehmen wir an, es ist gerade 18 Uhr. Wenn Sie berechnen, wie spät es in 80 Stunden sein wird, dann wenden Sie das an, was in der Mathematik Modulare Arithmetik genannt wird:
Zunächst teilen Sie die 80 durch 24 (oder 12) und addieren dann den Rest zur aktuellen Uhrzeit und teilen diesen dann ggf. wieder durch 24 und behalten nur den Rest: 
80 : 24 = 3, Rest 8
18 + 8 = 26
26 : 24 = 1, Rest 2
Lösung: Wenn es jetzt 18 Uhr ist, dann wird es in 80 Stunden 2 Uhr sein.
Wie Sie sehen ist Modulare Arithmetik im Grunde etwas ganz einfaches. Wenn Sie in der Mathematikvorlesung gefehlt haben sollten, kommt hier noch eine kurze Wiederholung der Grundlagen, die Sie im Bereich der Programmierung benötigen:
Die modulare Arithmetik gehört zu den Teilgebieten der Algebra und der Zahlentheorie. Wenn Sie sich intensiv in die zugehörigen mathematischen Grundlagen einarbeiten möchten, dann müssen sie sich mit Begriffen wie Ringen beschäftigen. Im Rahmen der Programmierung werden Sie mit modularer Arithmetik mit ganzen Zahlen zu tun haben.
Wie Sie oben gesehen haben, ist bei der modularen Arithmetik im Regelfall der Rest einer Division der interessante Teil. Dieser Rest wird als Modulo bezeichnet. Um das Modulo zu berechnen können sie in vielen aber längst nicht allen Sprachen das Prozentzeichen % als Operation genauso verwenden, wie Sie sonst das Divisionszeichen für die entsprechende Operation verwenden würden. Als Ergebnis erhalten Sie dann das Modulo.
Warnung: Aus alter Gewohnheit passiert es leicht, dass Sie anstelle Modulo mit Division rechnen. Vergessen Sie nicht: 42 Modulo 7 ist 0. Oft genug passiert es Studierenden, dass sie hier 6 als Ergebnis notieren, weil Sie an 42 durch 7 denken.
Für den Rest gibt es noch den Fachbegriff des Residuums. Sollten Sie also in Fachbüchern darüber stolpern, wissen Sie jetzt, was gemeint ist.
Sie werden in diesem Bereich auch auf den Begriff der Restklasse treffen. Wie so oft in der Mathematik kann eine Restklasse nicht für sich stehen, sondern sie macht erst dann Sinn, wenn sie gemeinsam mit einem Modulo verwendet wird. Es gibt für jedes Modulo genauso viele Restklassen, wie das Modulo groß ist. Eine einzelne Restklasse beinhaltet alle Werte, die nach Anwendung der Modulo-Operation mit einem Modulo dasselbe Ergebnis ergeben. 
Beispiel: Beim Modulo 4 erhalten wir vier Restklassen:
•	R0 enthält all die Werte, die Modulo 4 eine 0 ergeben: { …, -8, -4, 0, 4, 8, 12, … } Das ist also diejenige Restklasse, die diejenigen Zahlen beinhaltet, die bei Division keinen Rest ergeben.
•	R1 enthält all die Werte, die Modulo 4 eine 1 ergeben: { …, -9, -5, -1, 1, 5, 9, 13, … }
•	usw.
Wenn Sie genau hinsehen, werden Sie merken, dass nur bei der R0 der Abstand zwischen allen Elementen der Restklasse gleich vier ist. Das mag eine triviale Feststellung sein, aber sie ist nicht unwichtig, denn auch hier unterläuft einigen Studierenden der Fehler, mit der modularen Arithmetik wie mit der Division umzugehen.
Auch wenn Sie bei der Uhrzeit davon umgangssprachlich sagen, dass 13 Uhr gleich 1 Uhr ist, gilt hier natürlich nicht Gleicheit im mathematischen Sinne. Um diesen Unterschied hervorzuheben gibt es den Begriff der Kongruenz: Formal korrekt wäre es zu sagen: 13 ist kongruent zu 1 Modulo 12.
Kongruenz wird durch ein „Gleichzeichen“ mit drei Linien dargestellt: 
13 ≡ 1 ( mod 12 )
Eigenschaften der Modulo Arithmetik: Wenn Sie mit Modulo rechnen, gelten Kommutativität, Assoziativität und Distributivität. Das bedeutet u.a., dass Sie jeden einzelnen Wert durch das Modulo kürzen dürfen und nicht zuerst die Multiplikationen und Additionen berechnen müssen.
Beispiel: 
27 * ( 39 + 15 ) ist kongruent Modulo 16 zu 
11 * ( 7 – 1 ) ist kongruent Modulo 16 zu
11 * 6 ist gleich 66 ist kongruent Modulo 16 zu
4.
Wie Sie sehen können Sie sich bei modularer Arithmetik also viel Arbeit (und insbesondere einen Taschenrechner) sparen, wenn Sie schlicht sauber kürzen, bzw. die Restklassen im Kopf behalten. Beispiel: 15 ≡ -1 mod 16
Bevor wir uns nun ansehen können, was das bei der Programmierung bedeutet, müssen wir uns mit Zahlenbereichen auseinander setzen
A.2. Computerspeicher und Zahlenbereiche
Wie Sie wissen, ist der Arbeitsspeicher jedes Computers beschränkt. Vielleicht wenden Sie jetzt ein, dass doch jeder Speicher beschränkt ist. Dazu lässt sich allerdings sagen, dass Sie durchaus einen Bandspeicher nutzen können, der kontinuierlich mit Speicherbändern gefüttert werden kann und von daher quasi einen Endlosspeicher darstellt. Aber das nur am Rande.
Aber da Sie in der Schule gelernt haben, dass bereits die ganzen Zahlen endlos sind, sind Ihnen die Konsequenzen bereits bei einfachsten Computerprogrammen gar nicht bewusst: Eine Zahl wie 0,3 kann ein Computer nicht richtig speichern oder verarbeiten. Und eine Zahl wie 5 Millionen kann ein 32-Bit-Computer im Grunde auch nicht speichern. Ein Computer hat auch keine Methode, um eine Zahl wie -7 zu speichern, denn er kennt keine negativen Zahlen.
Formulieren wir es einmal anders:
•	8-Bit-Computer kennen nur die Zahlen von 0 bis 256.
•	16-Bit-Computer kennen nur die Zahlen von 0 bis 65536.
•	32-Bit-Computer kennen nur die Zahlen von 0 bis 4294967296.
•	Usw.
Alle Zahlen, die außerhalb dieses Bereichs liegen müssen in einer anderen Form repräsentiert werden.
Wenn wir nun einen Computer programmieren, dann gibt es zwei Standardmethoden, um Zahlen zu speichern, ohne dass wir uns aus dem Bereich bewegen, den der Computer kennt: Signed und Unsigned Integers. Signed bezeichnet hierbei die Tatsache, dass wir mit Vorzeichen rechnen. Wir bekommen so also negative Zahlen, obwohl der Computer sie nicht kennt. Es ist deshalb wichtig, dass Sie sich vergegenwärtigen, dass diese Unterscheidung eine willkürliche Festlegung durch uns als Programmierer ist, von der der Computer nichts weiß und auf die er dementsprechend auch nicht reagieren kann. Dementsprechend müssen wir alle möglichen Konsequenzen selbst abgreifen und können uns nicht auf eine Unterstützung durch den Rechner verlassen. (Auf die Konsequenzen gehen wir gleich ein.)
Um Rechenbeispiele zu bekommen, werden wir in diesem Abschnitt voraussetzen, dass wir es mit einem 3-Bit-Computer zu tun haben. Damit ergibt sich, dass dieser die folgenden acht Zahlen kennt bzw. darstellen kann. Die Prinzipien, die Sie gleich kennen lernen gelten in dieser Form für alle binär-basierten Computer.
000
001
010
011
100
101
110
111
A.2.1. Programmierung ohne signed Integer
Nun können wir als Entwickler diesen Zahlen Werte zuordnen. Bleiben wir dazu bei dem Fall, in dem wir ohne signed Integer arbeiten. Und hier gibt es gleich die erste Schwierigkeit: Beginnen wir bei 0 oder bei 1? Die Konsequenz daraus können Sie aus folgender Tabelle entnehmen: (Jedes Mal, wenn Sie es mit Arrays zu tun haben, werden Sie genau deshalb konzentriert sein müssen.)
Die 1. Zahl lautet:	000	und heißt 0
Die 2. Zahl lautet:	001	und heißt 1
Die 3. Zahl lautet:	010	und heißt 2
Die 4. Zahl lautet:	011	und heißt 3
Die 5. Zahl lautet:	100 	und heißt 4
Die 6. Zahl lautet:	101 	und heißt 5
Die 7. Zahl lautet:	110 	und heißt 6
Die 8. Zahl lautet:	111 	und heißt 7
Wie Sie sehen, haben wir es mit den 8 Zahlen von 0 bis 7 zu tun. Das kann bereits zu Fehlern führen, so zum Beispiel, wenn Sie bei einem Array mit acht Einträgen programmieren, dass der Rechner bis zum Eintrag Nr. 8 etwas durchführen soll: Das führt zu einem Fehler, weil der 8. Eintrag die Nr. 7 hat. Dagegen kennt er keinen Eintrag mit der Nummer 8!
Bei der maschinennahen Programmierung kommt noch eine andere Fehlerquelle hinzu: Dort müssen Sie des Öfteren einzelne Bits (auf 1) setzen oder (auf 0) resetten. Es ergeben sich dann folgenden Szenarien:
•	Sie müssen das Bit Nummer 3 setzen: Dann müssen Sie den binären Wert 1000, also die Zahl 8 programmieren.
•	Sie müssen das 3. Bit setzen: Dann müssen Sie den binären Wert 100, also die Zahl 4 programmieren.
•	Sie müssen die Zahl 3 programmieren: Dann müssen Sie den binären Wert 11 programmieren.
•	Sie müssen das 3. Element von etwas programmieren. Und hier kann Ihnen ausschließlich die Dokumentation des zu programmierenden Systems weiterhelfen, um zu entscheiden, welcher der drei Fälle zutrifft. Mit der Methode try and error werden Sie hier im Regelfall gnadenlos scheitern.
Diese Aufstellung sollte Ihnen also verdeutlichen, dass Sie hier hoch konzentriert an die Arbeit gehen müssen. Im Gegensatz zu höheren Programmiersprachen können Sie diese Fälle in aller Regel nicht über logische Schlussfolgerungen, sondern ausschließlich über genaues Wissen lösen.
Wichtig: So lange Sie sich im Bereich der bitweisen Programmierung befinden, werden die Begriffe set und reset als Synonyme für das Setzen auf 1 (set) bzw. 0 (reset) verwendet. Bei der Programmierung von Mikroprozessoren müssen Sie an dieser Stelle aufpassen, denn dort wird auch das Neustarten eines Programms als Reset bezeichnet. Und das hat in aller Regel nichts mit dem setzen eines einzelnen Bit auf den Wert 1 zu tun.
A.2.2. Programmierung mit signed Integer
Nun wollen wir aber häufig mit negativen Zahlen arbeiten. Wenn Sie sich die binäre Darstellung der Zahlen (also die einzige Form, die der der Rechner wirklich kennt) ansehen, könnten Sie auf die Idee kommen, dass Sie einfach das höchstwertige Bit für ganze Zahlen auf 0 und für negative Zahlen auf 1 setzen. Probieren wir es also aus:
000	ist 0.
001	ist 1.
010	ist 2.
011	ist 3.
111	ist -3.
110	ist -2.
101	ist -1.
100	ist dann negativ 0 und was soll das sein?
Also müssen wir uns ein anderes System überlegen. Es gibt mehrere Ansätze, für uns ist aber das sogenannte 2er Komplement die Standardlösung dieses Problems. Das 2er-Komplement berechnen Sie, indem Sie sämtliche Bits eines Wertes invertieren, das MSB (kurz für Most Significant Bit bzw. Höchstwertigstes Bit) in eine 1 umwandeln und dann eine 1 addieren.
Beispiel: Sie wollen das 2er Komplement zu 11 berechnen:
11 ist 1011 in binärer Schreibweise.
Wenn Sie jedes Bit invertieren, ergibt das 0100.
Wenn Sie das MSB als 1 setzen, ergibt das 10100.
Wenn Sie eine 1 addieren, ergibt das 10101 und damit haben Sie die Darstellung von -11 in Form des 2er Komplements.
Wichtig: Sie müssen noch darauf achten, dass Sie das Ergebnis entsprechend der Bittigkeit des Rechners angeben. Nehmen wir an, Sie wollen auf einem 8-Bit-Rechner die -11 darstellen, dann müssen Sie darauf achten, dass die Zahl eben 8 Stellen hat. In diesem Fall wären die -11 also nicht 10101, sondern 10000101.
Und? Ist es Ihnen aufgefallen? Da oben steht, dass 11 gleich 1011 ist. Damit Sie in solchen Fällen wissen, welche Zahl in Binärdarstellung und welche in Dezimaldarstellung notiert ist, wird manchmal (aus Faulheit aber nur selten) über einen Index die Basis der Zahlen angegeben. Formal korrekt wäre es also 1110 = 10112 zu schreiben.
Hexadezimale Zahlen werden häufig durch das voranstellen der Kombination 0x markiert. 10112 = 0xb
Wenn wir nun in unserem 3-Bit-Rechner negative Zahlen per 2er Komplement darstellen oder besser gesagt die Zahlen mit führender 1 als negative Zahlen im Sinne des 2er Komplements interpretieren wollen, dann ergibt sich folgende Aufstellung:
000	ist 0.
001	ist 1.
010	ist 2.
011	ist 3.
100	ist -4.
101	ist -3.
110	ist -2.
111	ist -1.
Kontrolle
Wie Sie sehen haben wir jetzt eine negative Zahl mehr als positive Zahlen.
Außerdem scheint die Reihenfolgen im negativen Bereich falsch zu sein: Die 100 ist eine größere negative Zahl als die 101. Aber wenn Sie die Zahlen in Form eines Kreises (also entsprechende der Modularen Arithmetik) anordnen, dann werden Sie sehen, dass das nur eine „optische“ Täuschung ist.
A.2.3. Overflow und Carry
Wie weiter oben erläutert weiß der Rechner (genauer gesagt der Mikroprozessor des Rechners) nichts über signed und unsigned; für ihn besteht die Welt nur aus dem, was wir als unsigned Integer nennen und im Grunde kennt er nicht einmal das. Im Kern kann er außerdem nur addieren, aber das macht er mit einer für uns nicht mehr nachvollziehbaren Geschwindigkeit: Innerhalb einer Sekunde führen aktuelle Intel- und AMD-Prozessoren zwischen 2 und 4 Milliarden Operation aus. Und wenn es sich um Mehrkernprozessoren handelt, dann sind es dementsprechende Vielfache von 2 bis 4 Milliarden Operationen.
Denken Sie an die Grundlagen der modularen Arithmetik: Dort wird ja, wenn das Modulo erreicht wird, das Ergebnis wieder auf 0 zurück gesetzt. Und das passiert natürlich auch bei Prozessoren, denn deren Speicher haben ja ein Modulo: Das ist der größte unsigned Integer, der entsprechend der Bittigkeit des Rechners darstellbar ist.
Wenn sie die modulare Arithmetik verstanden haben, dann wissen Sie, dass dieses Modulo sich nicht ändert, wenn wir mit signed Integers rechnen. Dieser Satz irritiert Sie? Am Ende dieses Unterabschnitts sollte das klar werden.
Aufgabe:
Nehmen wir wieder unseren 3-Bit-Rechner: Welche größte Zahl kennt dieser?
Sie sollten erst dann weiter lesen, wenn Ihnen klar ist, warum diese Frage falsch gestellt ist. Wenn Ihnen nicht klar ist, dass sie falsch gestellt ist, dann haben Sie eine ganz essentielle Aussage ignoriert, die in diesem Buch und diesem Kapitel immer wieder formuliert wurde: Der Rechner arbeitet mit Hilfe von Zahlenkolonnen, die wir als Einsen und Nullen interpretieren. Und wie Sie gerade eben erst lesen konnten, kennt der Rechner nur diese Einsen und Nullen, aber er führt keinerlei Interpretation davon durch. Das tun nur wir als Nutzer eines Computers. Weiterhin legen wir dann willkürlich Gruppen von Einsen und Nullen zusammen und interpretieren auch diese Gruppen wieder vollkommen willkürlich. Auch davon „weiß“ der Computer nicht das Geringste. Dementsprechend gibt es gar keine größte Zahl, die der 3-Bit-Computer kennt.
Jetzt aber zurück zu den zentralen Begriffen des Overflow und des Carry.
Nehmen wir an, Sie arbeiten mit unsigned Integer, interpretieren die Zahlenfolgen des Rechners ausschließlich als positive Zahlen, die genauso geordnet sind wie die natürlichen Zahlen (inkl. der Null). Und Sie lassen den Rechnern mit diesen Zahlen das tun, was er kann: Sie lassen ihn addieren. 
Aufgabe:
Was ist dann das Ergebnis von 7 + 1?
Wenn Sie jetzt acht antworten: Schämen Sie sich! Wie groß war doch noch die größte unsigned Zahl, die dieser Computer darstellen kann? Genau: Das war die Zahl sieben. Anders ausgedrückt: Unser Rechner rechnet modulo acht. Und wenn Sie jetzt an die modulare Arithmetik denken, was ist dann sieben plus eins modulo acht? Genau: Null. Immer, wenn so etwas passiert, also wenn der Prozessor eine Operation durchführen soll, bei der er als Ergebnis einer unsigned Operation eine Zahl darstellen soll, die zu groß für seinen Zahlenbereich ist, dann nennt man das einen Overflow.
Wenn Sie ein Programm in einer maschinennahen Sprache oder einer Sprache wie C bzw. C++ entwickeln, müssen Sie beachten, ob ein Overflow auftreten kann und ggf. das Programm entsprechend anpassen. Overflows spielen im Bereich der IT-Sicherheit eine wichtige Rolle.
Aufgabe:
Wenn er die Möglichkeit dazu hat, wird ein Computer immer einen Overflow anzeigen. In welchem Fall ist ein Overflow kein Problem, auf das Sie in Ihrem Programm eingehen müssen?
Es gibt einen zweiten Fall, um den Sie sich programmiertechnisch in bestimmten Fällen kümmern müssen. Nach der Überschrift haben Sie es sich wohl schon gedacht: Man redet hier vom Carry. Ein Carry ist im Falle der Rechnung mit signed Zahlen das gleiche, was ein Overflow für unsigned Zahlen ist.
Aufgaben:
Beim Übergang zwischen welchen zwei Zahlen tritt bei unserem 3-Bit-Computer ein Carry auf?
Und wann ist ein Carry kein Problem?
A.2.4. Übersicht: Signed und unsigned Integer für 3-Bit-Computer
Abschließend fürs spätere Nachschlagen nochmal alle Zahlen für signed und unsigned in einer Übersicht. Eventuell hilft Ihnen diese Tabelle, wenn Sie bestimmte Zusammenhänge noch nicht vollständig verstanden haben.
Binärfolge	interpretiert als
unsigned	signed
Die 1. Zahl lautet:	000		0
Die 2. Zahl lautet:	001		1
Die 3. Zahl lautet:	010		2
Die 4. Zahl lautet:	011		3
Die 5. Zahl lautet:	100 		4		-4
Die 6. Zahl lautet:	101 		5		-3
Die 7. Zahl lautet:	110 		6		-2
Die 8. Zahl lautet:	111 		7		-1
000				  0
001				  1
010				  2
011				  3




\renewcommand{\indexname}{Stichwortverzeichnis}		% Legt den Titel des Stichwortverzeichnisses fest.
\addcontentsline{toc}{chapter}{Stichwortverzeichnis}
\printindex

\end{document}