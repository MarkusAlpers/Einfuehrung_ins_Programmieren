
\subsection*{Zielgruppe und Vorwort}

Dieses Buch habe ich erstellt, um Studierenden der Studiengänge Media Systems (entspricht Medieninformatik an anderen Hochschulen) und Medientechnik an der HAW Hamburg den Einstieg ins Programmieren zu erleichtern. Deshalb finden sich hier teilweise Anmerkungen für die Studierenden der beiden Studiengänge, die aber in ähnlicher Form für Studierenden der Informatik und der Elektrotechnik gelten. Da es jedoch so formuliert ist, dass es für Studienanfänger ohne Programmiererfahrung geeignet ist, kann jede/r Studierende es gut nutzen, um sich in die Programmierung einzuarbeiten. Wenn die Version 1.0 abgeschlossen ist wird es eine \textbf{grundlegende Einführung} in die Konzepte (Paradigmen), der maschinennahen, der imperativen, der funktionalen, der relationalen und der objektorientierten Programmierung in den Ausprägungen protoypbasiert und klassenbasiert sein. Zusätzlich behandelt es den Einstieg in die Entwicklung verteilter Anwendungen.\\

Wie alle Lehrbücher für Studierende setzt es eines voraus: Wenn Sie es nutzen wollen, dann funktioniert das dann, und ausschließlich dann, wenn Sie zusätzlich zum Lesen zwei Dinge tun: Zum einen müssen Sie ständig kontrollieren, ob Sie jeden \textbf{neuen Begriff wirklich verstanden} haben und prüfen, wie er im Zusammenhang mit dem bisher Gelernten steht und zum anderen \textbf{müssen Sie tatsächlich programmieren}.\\

Was Sie hier nicht finden sind zum einen alle Varianten der Programmierung, die im Kern aus der Elektrotechnik entstanden sind oder für deren Verständnis Sie die Grundlagen kontinuierlicher Systeme beherrschen müssen. In der Informatik werden diese Bereich als \textbf{Technische Informatik}\index{Informatik!Technische Inf.} bezeichnet. Das schließt beispielsweise die Programmierung von Steuer- und Regelsystemen, also insbesondere \textbf{SPS}e\index{Systeme!SPS} und \textbf{FPGA}s\index{Systeme!FPGA} ein.\\

Damit sind wir auch schon bei einem ersten Missverständnis das zwischen InformatikerInnen einerseits und NaturwissenschaftlerInnen, IngenieurInnen und TechnikerInnen (kurz \textbf{INT-Akademiker}) existiert: Was in der Informatik als \textbf{Technische Informatik}\index{Informatik!Technische Inf.} bezeichnet wird ist alles, was die\\
übrigen drei als Informatik kennen. Diese gehen deshalb in aller Regel von der irrigen Vorstellung aus, das InformatikerInnen Programmierung meinen, wenn sie von \textbf{Praktischer Informatik}\index{Informatik!Praktische Inf.} reden. Tatsächlich haben beide (Praktische Informatik und Programmierung) kaum etwas miteinander zu tun. An dieser Stelle sei deshalb (vorrangig für Informatikstudierende) betont:\\

Dies ist eine \textbf{Einführung ins Programmieren, nicht in die Praktische Informatik}. Es wird zwar immer wieder Hinweise auf die Praktische und Theoretische Informatik geben, aber vorrangig ist und bleibt dies eine\\
Einführung ins Programmieren.\\

Die \textbf{systemnahe Programmierung}\index{Programmierung!systemnah} und die Programmierung von \textbf{Parallelprozessoren}\index{Programmierung!parallel} sowie die Implementierung von \textbf{Protokolle}n\index{Protokoll} für die Datenübertragung über Netzwerke entfallen ebenfalls. Dennoch werden Sie in diesem Buch zumindest einen Einblick in die Grundlagen der systemnahen Programmierung erhalten, da diese Systeme die Grundlage für alle Programmieransätze darstellen, die Sie hier kennen lernen können. Hier gilt dasselbe, was schon im letzten Absatz galt: INT-Akademiker kennen in aller Regel nur die systemnahe Programmierung und alle Konzepte, die sich direkt daraus ableiten lassen und die von InformatikerInnen mit dem Oberbegriff \textbf{Technische Informatik} bezeichnet werden. Es gibt jedoch auch Programmierkonzepte, die damit nicht mehr verständlich sind und für die es nötig ist, sich wesentlich grundlegender und abstrakter mit der Programmierung zu beschäftigen. Dazu kommen wir im zweiten Teil dieses Buches.\\

Fragen des \textbf{Software Engineering}\index{Software Engineering} werden zwar angerissen und es gibt Hinweise auf typische Missverständnisse, eine grundlegende Einführung ins Software Engineering kann dieser Band jedoch nicht ersetzen. Wie der Titel dieses Buches klar ausdrückt, geht es hier ums Programmieren. An den entsprechenden Stellen werden Sie aber entsprechende Hinweise auf Bücher und Themengebiete finden, damit Sie ggf. wissen, wonach Sie für weiterführendes Wissen suchen müssen. Ob sie sich nun zunächst in die Programmierung oder ins Software Engineering stürzen wollen, bleibt Ihnen überlassen; beides hat seine Vor- und Nachteile. In der Informatik\\
müssen Sie aber in jedem Fall beides durcharbeiten, um auch nur in\\
Ansätzen gute Software entwickeln zu können.\\

Der Grund ist simpel: Bei der \textbf{Programmierung}\index{Programmierung} geht es darum, Konzepte zur Lösung eines Programms in eine Sprache zu übersetzen, die ein Computer ausführen kann. Beim \textbf{Software Engineering}\index{Software Engineering} geht es dagegen darum, gute Konzepte zu entwickeln, die in Programmiersprachen übersetzt werden können. Wer nur eines von beidem beherrscht entwickelt häufig Programme, die niemandem nützen oder nützliche Konzepte, die niemand (in Form eines Computerprogramms) nutzen kann. Deshalb gibt es in jedem brauchbaren Kurs zur Programmierung Auszüge Teile, die eigentlich in den Bereich des Software Engineering gehören\footnote{Weshalb es nur wenige Kurse zur Programmierung gibt, die nach Ansicht dieses Autors brauchbar sind.}. Umgekehrt enthält jeder brauchbare Kurs zum Software Engineering Teile zur Programmierung\footnote{Weshalb auch hierfür aus Sicht dieses Autors nur wenig Brauchbares auf dem Markt ist.}.\\

Zusätzlich müssen Sie jedoch in jedem Fall noch die Grundlagen der Praktischen Informatik erlernen, um hochwertige Software zu erstellen. Diese können Sie im Bereich der Algorithmik (genauer \textbf{Algorithmen und Datenstrukturen}\index{Algorithmen und Datenstrukturen}, \textbf{Algorithmendesign}\index{Algorithmendesign} sowie \textbf{Algorithmik}\index{Algorithmik}) erlernen.\\

Aufgrund der häufigen \textbf{Änderungen bei aktueller Software} kann dieses Buch nur beschränkt Unterstützung bei Installations- und Konfigurationsfragen bieten. Hier bleibt zu hoffen, dass die Entwickler der einzelnen Sprache bzw. zusätzlicher Software eine ausreichende Dokumentation auf Ihrer Webpage bereitstellen.\\

Nochmal in anderen Worten: Ein häufiges Missverständnis besteht darin, dass Programmierung und Informatik bzw. Programmierung und Praktische Informatik miteinander verwechselt werden. Denn \textbf{Programmierung}\index{Programmierung} ist lediglich die Umsetzung einer Idee mit Hilfe einer Sprache, die einem Computer befiehlt, \textbf{was er tun soll}. Ob sie auch festlegt, \textbf{wie er das tun soll} ist eine ganz andere Frage. Einführungen in die Programmierung, die hier nicht deutlich werden sind der Grund für eine Vielzahl von Missverständnissen rund um die Programmierung.\\

Um überhaupt zu programmieren, müssen Sie also lediglich wissen, wie die Befehle und Befehlsstrukturen einer Sprache aussehen. Im Kern ist das also nichts anderes als das Erlernen einer gesprochenen Sprache. Doch so wie es selbst für das Erlernen nahe verwandter gesprochener Sprachen eben nicht ausreicht, nur die Übersetzung einzelner Wörter zu erlernen, genügt es für die kompetente Beherrschung von Programmiersprachen\\
nicht, sich nur grundsätzlich damit beschäftigt zu haben: So wie Sie eine gesprochene Sprache tatsächlich in Gesprächen benutzen müssen, um sie zu erlernen, müssen Sie eine Programmiersprache benutzen, indem Sie eine Vielzahl an Programmen damit entwickeln.\\

\textbf{Wichtig:}\\
Fähige (Medien-)InformatikerInnen sind nicht automatisch guten ProgrammiererInnen. Wenn Sie verstanden haben, was der Unterschied zwischen Informatik und Programmieren ist, dann wird es sie wundern, dass es\\
überhaupt Menschen gibt, die diese Aussage bezweifeln.\\

Die \textbf{Informatik}\index{Informatik} dagegen setzt sich mit der Frage auseinander, wie und ob eine bestimmte Idee besonders elegant und effizient umgesetzt werden kann. \textbf{Ob für die Umsetzung der Idee ein Computer nötig ist, ist zweitrangig.} Aber da Computer die Stärke haben, dass Sie langweilige Aufgaben mit einer für uns unfassbaren Geschwindigkeit ausführen, sind Sie das Werkzeug Nummer 1 für die Informatik. Zumindest ist nach Ansicht dieses Autors die Durchführung von 4 Milliarden Additionen pro Sekunde unfassbar schnell. Zum Vergleich: Würden alle Menschen auf dieser Welt gleichzeitig eine Addition zweier Zahlen mit bis zu 20 Stellen durchführen und für die Berechnung sowie das Aufschreiben nur zwei Sekunden brauchen, dann wären sie alle gemeinsam genauso schnell wie ein einzelner Prozessor, der in einem handelsüblichen Computer steckt.\\

Hier ein Beispiel, mit dem sich Informatikstudierende gegen Ende des Bachelorstudiums auseinander setzen: Sie fahren in den Skiurlaub und wollen mal das Skifahren ausprobieren. Nun könnten Sie die Skier kaufen oder mieten. Da Sie ja nicht wissen, ob Ihnen Skifahren wirklich Spaß macht, wäre es unsinnig, gleich am ersten Tag das Geld für den Kauf auszugeben. Aber auch am zweiten Tag wäre es nicht unbedingt sinnvoll, denn wer weiß, ob Sie am dritten Tag noch Lust dazu haben. Die Frage lautet also: Wann macht es Sinn, die Skier zu kaufen? Das ist ein Beispiel für einen Bereich, der in der Informatik als \textbf{online-Algorithmen}\index{Algorithmus!online} bezeichnet wird. Der zugehörige Bereich der Praktischen Informatik heißt \textbf{Algorithmendesign}\index{Algorithmendesign}.\\

Online-Algorithmen dienen beispielsweise dazu, die Verwaltung des Speichers einer Festplatte zu organisieren oder um die Mitarbeiter für Kassen in einem Supermarkt einzuplanen. Bei online-Algorithmen geht es immer um die Frage: Wie bereite ich mich am besten auf eine Situation vor, von der ich noch nicht genau weiß, wie sie aussehen wird? Und wie Sie sehen hat das zunächst einmal nichts mit Programmieren zu tun.\\

Sie möchten ein Beispiel, bei dem es um Programmierung und online-\\
Algorithmen geht? Dann haben Sie leider noch nicht verstanden, was der Unterschied zwischen Praktischer Informatik und Programmieren ist. Also weiter mit den online-Algorithmen: Denken Sie beispielsweise daran, was bei ebay kurz vor Ende einer Versteigerung passiert. Oder wie wäre es mit einem online-Händler wie amazon, kurz vor Weihnachten: Wann wie viele Menschen versuchen werden, auf eine einzelne Seite zuzugreifen, kann niemand wissen. Aber mit fähigen Informatikern kann jeder sich darauf so gut wie möglich vorbereiten. Hier sind wir übrigens auch schon ganz nahe an einem aktuellen Forschungsgebiet der (Medien-)Informatik: Bei \textbf{Big Data}\index{Big Data} handelt es sich um einen Bereich, in dem aus gigantischen Datenmengen versucht wird, Gruppen von Personen mit gemeinsamen Eigenschaften herauszufiltern und dann Prognosen über deren zukünftiges Verhalten zu schließen. Hier sind wir ganz nahe am Gebiet des \textbf{Datenschutzes}\index{Datenschutzes} mit ganz konkreten Auswirkungen auf das alltägliche Leben. Denn Big Data führt zum Teil zu absurden Abläufen: Wenn Sie beim Ausfüllen eines Kreditantrages online mehrfach Einträge ändern, dann wird alleine deshalb der Zinssatz erhöht oder der Antrag abgelehnt. Die Begründung stammt direkt aus dem Bereich angewandten Big Data und lautet so: Menschen mit niedrigem Bildungsniveau vertippen sich häufiger als Menschen mit hohem Bildungsniveau. Menschen mit hohem Bildungsniveau haben in aller Regel eine höheres Einkommen, längere Arbeitsverhältnisse und eine geringere Wahrscheinlichkeit, arbeitslos zu werden. Das lässt sich verkürzen zu: Menschen mit höherem Bildungsstand sind in aller Regel zuverlässiger bei der Rückzahlung von Krediten. Wenn wir jetzt noch die erste Aussage hinzunehmen, lautet die Schlussfolgerung nach den Prinzipien des Big Data: Wer sich öfter vertippt wird häufiger Probleme haben, einen Kredit zurückzuzahlen. Also wird der Zinssatz erhöht oder der Kredit gleich ganz verweigert. Und nein, das ist kein Scherz, sondern findet so Anwendung bei online-Finanzdienstleistern.\\

Wieder zurück zu den online-Algorithmen: In der BWL wird dieser Teil der Informatik als \textbf{Logistik}\index{Logistik} bezeichnet, wobei der Bereich der online-\\
Algorithmen deutlich mehr umfasst als nur Logistik. BWLern ist dabei in aller Regel leider nicht bewusst, dass es sich hier um einen Bereich handelt, der eine Kernkompetenz der Informatik ist. Das führt dazu, dass in der Forschung zur BWL teilweise Forschungen betrieben werden, um Probleme zu lösen, für die die Informatik längst eine Lösung bereit hält. Denken Sie bitte dennoch nicht in Kategorien wie Schuld: Es ist ein grundsätzliches Problem, dass zu viele Akademiker nur in den Kategorien der eigenen Disziplin denken und kaum den Austausch mit anderen Disziplinen suchen. Das ändert sich zwar in einigen wenigen Fällen, doch häufig kommt es dabei nicht zu einem vollwertigen Austausch, sondern es wird versucht, die jeweils andere Disziplin der eigenen anzupassen. Ein erster Schritt, um das nicht zu tun besteht darin, sich darüber auszutauschen, was Begriffe der jeweiligen Disziplin bedeuten. Ein "`gutes"´ Beispiel haben bereits kennen gelernt: Was INT-Akademiker als Informatik bezeichnen ist für Informatiker in aller Regel nur der Teilbereich der \textbf{Technischen Informatik}\index{Informatik!Technische Inf.}. So lange solche Missverständnisse nicht ausgeräumt sind und AkademikerInnen unterschiedlicher Disziplinen die Kompetenz des/der jeweils anderen nicht anerkennen, ist ein echter Austausch nicht möglich. Und dann sind auch umfassende Projekte mit Beteiligung unterschiedlicher wissenschaftlicher Disziplinen nicht möglich.\\
 
Sie studieren \textbf{Medientechnik}\index{Medientechnik} und fragen sich, was das mit Ihrem Studium zu tun hat? Ganz einfach: Auch in Ihrem Bereich werden Computer eingesetzt und Sie werden nicht immer eine fertige Lösung in Form eines Computerprogramms vorfinden. Also müssen Sie im Stande sein, einfache Probleme mit einem Computer selbst zu lösen. Und wenn die Probleme zu komplex werden, dann müssen Sie im Stande sein, InformatikerInnen zu erklären, worin das Problem besteht, denn sonst können die keine Lösung für Sie entwickeln. Wie wäre es beispielsweise mit einem Programm, mit dem Sie Ihre Schaltskizzen für das E-Technik-Labor erstellen können, die Sie außerdem online speichern und abgeben können, ohne sie als Mailanhang verschicken zu müssen? Wenn Sie die Veranstaltung "`Programmieren 1"' (entspricht Teil I dieses Buches) erfolgreich absolvieren, dann werden Sie im Stande sein, solche Programme selbst zu entwickeln.\\

Sie studieren \textbf{Media Systems}\index{Media Systems} und fragen sich, warum Sie einen Kurs mit dem Titel "`Einführung ins Programmieren"' (kurz PRG) belegen sollen,\\
wenn Sie doch schon die Veranstaltung "`Programmieren 1"' (kurz P1) belegen? In der Veranstaltung P1 lernen Sie die Entwicklung von Programmen mit einer imperativen und klassenbasierten objektorientierten Programmiersprache. In PRG konzentrieren wir uns dagegen auf einen anderen Bereich: Die Entwicklung verteilter Anwendungen. Wann immer Sie eine App nutzen, nutzen Sie eine verteilte Anwendung. Wann immer Sie ein Programm nutzen, das das Internet oder ein anderes Netzwerk voraussetzt, nutzen Sie eine verteilte Anwendung. Normalerweise ist das Stoff für ein Masterstudium, aber ich habe diese Veranstaltung so konzipiert, dass sie für Einsteiger ohne Vorkenntnisse geeignet ist. Denn so bekommen Sie einen Eindruck, wie all die verschiedenen Veranstaltungen Ihres Studiums ein sinnvolles ganzes ergeben.\\

Hier nochmals meine Bitte: Wenn Sie in diesem Skript Fehler finden (was bei einem Dokument dieser Länge unausweichlich der Fall ist), Sie weitergehende Fragen haben oder Ergänzungsvorschläge, dann senden Sie diese bitte an mich: \url{markus.alpers@haw-hamburg.de}

\subsection*{Work in Progress}

Der Begriff Work in Progress bedeutet, dass eine Arbeit noch nicht abgeschlossen ist und somit in Teilen unvollständig und in anderen Teilen unnötig detailliert ist. Das gilt zurzeit für dieses Buch: Zum einen habe ich verschiedene Passagen noch nicht abgeschlossen, zum anderen habe ich verschiedene Texte, die ich ursprünglich für unterschiedliche Kurse erstellt hatte hier zusammengefasst. Da diese Zusammenführung noch nicht abgeschlossen ist, wird es Passagen geben, in denen Sie nahezu identische Aussagen und Erklärungen erneut finden werden. Auch wenn Ihnen solche Passagen auffallen, schicken Sie mir bitte eine E-Mail.
