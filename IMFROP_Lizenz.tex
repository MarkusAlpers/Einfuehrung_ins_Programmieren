\subsection*{Hinweis bezüglich diskriminierender Formulierungen}

In diesem Text wurde darauf geachtet Formulierungen zu vermeiden, die diskriminierend verstanden werden können. Im Sinne der Lesbarkeit wurden dabei Formulierungen wie "`Informatiker und Informatikerinnen"'\\durch "`InformatikerInnen"' (mit großem i) ersetzt. An anderen Stellen habe ich Formen wie eine/einer durch eineR zusammengefasst. Hier berufe ich mich auf den Artikel "`Sprache und Ungleichheit"' der Bundeszentrale für politische Bildung, kurz BpB, vom 16. April 2014, insbesondere auf den Absatz "`Zum Umgang mit diskriminierender Sprache"', online abrufbar unter:\\

\url{http://www.bpb.de/apuz/130411/sprache-und-ungleich}\\
\url{heit?p=all}\\

Sollten Sie dennoch Formulierungen entdecken, die diesem Anspruch nicht entsprechen, möchte ich Sie bitten, mir eine entsprechende Nachricht zu senden, denn es ist mir wichtig, Ihnen mit diesem Buch eine wertvolle Unterstützung beim Start in die faszinierende Welt der Informatik zu bieten. Das sollte nicht durch verletzte Gefühle in Folge missverständlicher Formulierungen torpediert werden. \\

Sie erreichen mich unter \url{markus.alpers@haw-hamburg.de}. 

\subsection*{Hinweis zur Lizenz}

Dieses Buch wird in Teilen unter der Lizenz \emph{CC BY-SA 3.0 DE} veröffentlicht. Das bedeutet, dass Sie die entsprechenden Teile z.B. kopieren dürfen, so lange der Name des Autors erhalten bleibt. Sie dürfen diese auch in eigenen Werken weiterverwenden, ohne dafür z.B. eine Lizenzgebühr zahlen zu müssen. Dennoch müssen Sie auch hier bestimmte Bedingungen einhalten. Eine davon besteht darin, dass eine solche Veröffentlichung ebenfalls unter dieser Lizenz erfolgen muss. Sinn und Zweck solcher Lizenzen besteht darin, dass geistiges Eigentum frei sein und bleiben soll, wenn derjenige, der es erschaffen hat das wünscht. Und es ist mein Wunsch, dass so viele Menschen wie möglich von den Erklärungen in diesem Text profitieren.\\

Der vollständige Wortlaut der Lizenz ist auf folgender Seite nachzulesen. Dort erfahren Sie dann auch, welche Bedingungen einzuhalten sind:\\

\url{https://creativecommons.org/licenses/by-sa/3.0/de/}\\

Alle Teile des Buches, die ich unter der Lizenz \emph{CC BY-SA 3.0 DE}\\
veröffentliche enthalten am Anfang diesen Abschnitt "`Hinweise zur Lizenz"´. Wenn Sie einen Teil finden, in dem diese "`Hinweise zur Lizenz"´ nicht zu finden ist, dann dürfen Sie für den persönlichen Gebrauch dennoch Kopien davon anfertigen und Sie dürfen diese Kopien außerhalb von kommerziellen Projekten frei verwenden.\\

\subsection*{Hinweis zur Verwendbarkeit in wissenschaftlichen Arbeiten}

Bitte beachten Sie dabei aber, dass die Verwendung dieses Textes im Rahmen wissenschaftlicher Publikationen zurzeit aus anderen Gründen problematisch ist: Wie viele andere Quellen, die frei im Internet verfügbar sind, wurde auch dieser Text bislang nicht durch einen nachweislich entsprechend qualifizierten Lektor verifiziert. Damit genügen Zitate aus diesem Band streng genommen noch nicht den Ansprüchen wissenschaftlicher Arbeiten.\\

Bitte beachten Sie außerdem, dass dieses Buch eine Konvention nutzt, die in wissenschaftlichen Arbeiten verpönt ist: Wenn in einer wissenschaftlichen Arbeit ein Begriff hervorgehoben wird, dann wird dazu kursive\\
Schrift verwendet. In diesem Buch verwende ich dagegen Fettdruck, da es vielen Menschen schwer fällt, einen kursiv gedruckten Begriff schnell zu finden und ich mir wünsche, dass Sie es möglichst effizient auch als Nachschlagewerk nutzen können.