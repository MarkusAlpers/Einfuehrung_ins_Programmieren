\section{5.8.	Einstieg in HTML 5}
\subsection{5.8.1.	Das ist HTML}
\subsection{5.8.2.	Erster HTML-Quellcode}
\paragraph{5.8.2.1.	Tags und Container}
\subsection{5.8.3.	Struktur eines HTML-Skripts}
\paragraph{5.8.3.1.	HTML-Referenz: W3Schools}
\paragraph{5.8.3.2.	Aktuelles zu XHTML und HTML5}
\section{5.9.	Grundaufbau einer Webpage}
\subsection{5.9.1.	Aufgaben}
\subsection{5.9.2.	Die Doctype Definition}
\paragraph{5.9.2.1.	Aufgabe}
\subsection{5.9.3.	Die drei Container jedes HTML-Dokuments}
\subsection{5.9.4.	Attribute}
\subsection{5.9.5.	Tags}
\paragraph{5.9.5.1.	Aufgabe}
\subsection{5.9.6.	Übersicht über alle Tags}
\subsection{5.9.7.	Zusammenfassung zu den drei Containern}
\paragraph{5.9.7.1.	Aufgabe}
\section{5.10.	Barrierefreiheit, Internationalisierung und Lokalisierung}
\subsection{5.10.1.	Barrierefreiheit}
\paragraph{5.10.1.1.	Aufgabe}
\subsection{5.10.2.	Internationalisierung (kurz i18n) und Lokalisierung (kurz l10n)}
\section{5.11.	Der head-Container, Meta-Daten und Attribute}
\subsection{5.11.1.	Verwendung von Escape-Sequenzen}
\section{5.12.	HTML5: Anführungszeichen sind optional}
\section{5.13.	Strukturen von HTML5-Dokumenten}
\subsection{5.13.1.	header, footer, main und aside}
\subsection{5.13.2.	article und section}
\subsection{5.13.3.	h1 bis h5}
\subsection{5.13.4.	p}
\subsection{5.13.5.	nav}
\section{5.14.	Polyfills}
\subsection{5.14.1.	Einbindung von Polyfills - <script>-Container}
\section{5.15.	Zusammenfassung}
\subsection{5.15.1.	Hausaufgabe}
\section{5.16.	Hyperlinks}
\subsection{5.16.1.	Anker}
\paragraph{5.16.1.1.	Aufgabe}
\subsection{5.16.2.	Links}
\paragraph{5.16.2.1.	Aufgabe}
\subsection{5.16.3.	Verlinkungen als expliziter Download}
\paragraph{5.16.3.1.	Hausaufgabe}
\subsection{5.16.4.	URLs – absolute und relative Adressen}
\section{5.17.	Formulare}
\subsection{5.17.1.	Elemente eines Formulars}
\subsection{5.17.2.	Das name-Attribut und das id-Attribut – Sonderfälle in Formularen}
\subsection{5.17.3.	Container für Formulare}
\paragraph{5.17.3.1.	Formularfelder für Texteingaben}
\paragraph{5.17.3.2.	Formularfelder für Zahleneingaben}
\paragraph{5.17.3.3.	Formularfelder für Datums- und Zeitangaben}
\paragraph{5.17.3.4.	Auswahlmöglichkeiten}
\paragraph{5.17.3.6.	Dateien hochladen}
\subsection{5.17.4.	Container für die Gruppierung und Zuordnung von Eingabeelementen}
\subsection{5.17.5.	Zusammenfassung}
\section{5.18.	Multimediale Inhalte einfügen}
\subsection{5.18.1.	Bilder}
\subsection{5.18.2.	<figure> und <figurecaption>}
\subsection{5.18.3.	<figurecaption> für Fortgeschrittene}
\section{5.19.	Weitere multimediale Formate}
\subsection{5.19.1.	Einbindung eigener und frei verfügbarer Videodateien}
\subsection{5.19.2.	Anpassungsmöglichkeiten für den Video-Player}
\subsection{5.19.3.	Einbindung von geschützten Inhalten (Stichwort: DRM)}
\subsection{5.19.4.	Der <audio>-Container}
\subsection{5.19.5.	Close Captions, Untertitel, Einbindung von Webcams usw.}
\subsection{5.19.6.	Hinweis bezüglich Flash und ähnlichen Formaten}
\subsection{5.19.7.	Hausaufgabe}
\section{5.20.	Weitere Formatierungen und Möglichkeiten in HTML}
\subsection{5.20.1.	Spoiler und andere ausklappbare Texte}
\subsection{5.20.2.	Zeitangaben}
\subsection{5.20.3.	Hervorhebung von Texten}
\subsection{5.20.4.	Unterdrückung von Übersetzungen für Textpassagen}
\paragraph{5.20.4.1.	Vererbung von Attributen}
\subsection{5.20.5.	Aufzählungen (Ordered und Unordered Lists)}
\subsection{5.20.6.	Glossare (Description Lists)}
\subsection{5.20.7.	Tabellen (table)}
\subsection{5.20.8.	Microdata}
\paragraph{5.20.8.1.	Programmierung von Microdata}
\paragraph{5.20.8.2.	Festlegung des Objekttyps}
\paragraph{5.20.8.3.	Eigenschaften von Objekten}
\paragraph{5.20.8.4.	Umfangreiche Microdata am Beispiel einer Person mit Adressangabe}
\paragraph{5.20.8.5.	Validator für Microdata}
\section{5.21.	Zusammenfassung}
\subsection{5.21.1.	Mehr Text braucht die Welt … und Zeilenumbrüche}
\subsection{5.21.2.	Hyperlinks – Verbindungen zwischen Elementen}
\subsection{5.21.3.	Entitys - Umlaute und andere Sonderzeichen}
\subsection{5.21.4.	Deutsche Umlaute ohne Entities}
\subsection{5.21.5.	Hervorhebungen}
\subsection{5.21.6.	Formulare}
\paragraph{5.21.6.1.	Nutzereingaben in HTML programmieren}
\paragraph{5.21.6.2.	Überblick über Formularelemente}
\paragraph{5.21.6.3.	Das id-Attribut – Variablen in HTML}
\subsection{5.21.7.	Universalattribute}
\subsection{5.21.8.	Labels – Beschriftungen für input-Container}
\subsection{5.21.9.	CSS – Cascading Style Sheets}
\paragraph{5.21.9.1.	Für Rückgeschrittene – Präprozessoren: ACSS und SASS}
\subsection{5.21.10.	CSS und unterschiedliche Displayformate}
\subsection{5.21.11.	Schriftsätze und –farben – CSS/font}
