\section{Darüber reden wir in diesem Kurs:}

Wahrscheinlich haben Sie bereits den Begriff des Algorithmus an der ein oder anderen Stelle in Bezug auf Programme gehört. Auch mit Datenstrukturen wie Arrays haben Sie schon zu tun gehabt, wenn Sie ein wenig programmiert haben. Deshalb lautet eine der häufigsten Fragen zu Beginn eines Kurses über Algorithmen und Datenstrukturen: \glqq{}Was gibt's denn da noch zu lernen? In Java gibt es doch haufenweise Datenstrukturen, die können wir doch nutzen. Und Algorithmen sind doch auch klar...\grqq{}\\

Diese Frage offenbart aber ein klares Missverständnis: \textbf{Datenstrukturen}, die wir in Programmiersprachen angeboten bekommen, erfüllen jeweils einen bestimmten Bedarf, den viele Programmierer haben. Sie erfüllen aber eben längst nicht jeden Bedarf und wenn wir gute Software entwickeln wollen, dann müssen wir im Stande sein, selbst eine Datenstruktur zu programmieren, die für ein neues Problem gut geeignet ist. Und umgekehrt müssen wir im Stande sein, zu erkennen, ob eine Datenstruktur, die in einer Programmiersprache enthalten ist schon alles anbietet, was wir für eine Aufgabe brauchen. Schließlich wäre es nicht sehr intelligent, es selbst zu programmieren, dass es bereits genau so gibt.\\

Das führt dann gleich zum nächsten Missverständnis: Informatik ist eben nicht Programmierung, sondern Informatik ist die Wissenschaft, in der wir nach Methoden suchen, um schwierige Probleme zu lösen. Wenn die Probleme einfach sind, kann sie ja jeder lösen, also schauen wir uns die schweren Probleme an. Und wenn es schon eine Lösung gibt, dann wäre es Zeitverschwendung etwas eigenes zu entwickeln, das auch nicht besser ist. Der Lösungsweg, den wir dabei für ein Problem finden wird als Algorithmus bezeichnet. Und damit sind wir bei dem Grund, aus dem wir uns mit Datenstrukturen und Algorithmen beschäftigen: \textbf{Algorithmen} sind Beschreibungen einer Problemlösung und Datenstrukturen beschreiben, wie wir die dabei verwendeten Daten speichern. Nur gemeinsam machen die beiden Sinn.\\

Um das auch nochmal explizit zu ergänzen: Programmierung ist dann die Umsetzung von Algorithmen und Datenstrukturen in eine Programmiersprache. Wer also Programmieren kann, aber kein Informatiker ist, der kann zwar einen Computer dazu bringen, etwas zu tun, ist aber oft nicht im Stande zu entscheiden, wie brauchbar die programmierte Lösung für unterschiedliche Fälle ist. Schlimmer noch: Reine Programmiere versuchen teilweise Lösungen zu programmieren, bei denen InformatikerInnen nach einer kurzen Prüfung feststellen, dass eben diese Lösung nicht funktionieren kann. (Z.B. weil sie in 9 von 10 Fällen bis zum Ende des Universums dauert.) Viele reine ProgrammiererInnen neigen außerdem dazu, entweder eine Speziallösung zu programmieren, die in den meisten Fällen gänzlich sinnlos ist oder sie entwickeln eine derart umfangreiche Lösung, dass damit auch lauter Aufgaben gelöst werden können, die niemals vorkommen werden. Letzteres ist allerdings auch ein typischer Fehler von InformatikerInnen. Sie neigen außerdem häufig dazu, sich nicht das Problem anzusehen, sondern sie versuchen krampfhaft, eine Lösung zu entwickeln, die sie mit ihrer Lieblingsprogrammiersprache hinbekommen.\\

Damit kommen wir zu einem dritten Begriff, den alle InformatikerInnen beherrschen: Wenn wir über \textbf{Komplexität} reden, dann meinen wir damit in aller Regel einen Maßstab, mit dem wir die Qualität verschiedener Algorithmen vergleichen können. Der bekannteste Maßstab für Komplexität ist die sogenannte O-Notation. Sie ist ein Maßstab, mit dem wir vergleichen können, wie viel Zeit verschiedene Algorithmen benötigen, die eine unbekannte Menge an gleichartigen Daten bearbeiten. Aber wir können uns auch die Komplexität des Speicherbedarfs verschiedener Algorithmen ansehen. Das ist schon alleine deshalb interessant, weil wir natürlich sicher stellen müssen, dass unser Programm nicht mehr Speicher belegt, als im Computer vorhanden ist.\\

Einer der wichtigsten Bereiche, in denen wir mit Algorithmen und Datenstrukturen zu tun haben sind die sogenannten Betriebssysteme. Am Ende dieses Kurses werden wir uns deshalb ansehen, was Betriebssysteme tatsächlich sind. All diejenigen von Ihnen, die bei Betriebssystemen sofort an Windows, iOS oder Linux denken werden sich hier umstellen müssen: Wie an vielen anderen Stellen im Studium reden wir hier nicht über die Konfiguration oder Nutzung eines Betriebssystems, sondern darüber, was die EntwicklerInnen eines Betriebssystems beachten müssen, bzw. darüber, was ein Betriebssystem leistet.\\

Die Inhalte dieses Kurses können Sie in den unterschiedlichsten Bereichen der Softwareentwicklung aber auch in der Administration nutzen. Denn diejenigen von Ihnen, die die Inhalte dieses Kurses gut verinnerlichen, werden später im Stande sein, mit nur wenig Aufwand verhältnismäßig schnell Lösungen zu entwickeln, die möglichst gut geeignet sind, um eine Aufgabe durch eine Computer oder ein Computernetzwerk zu lösen. Aber es geht noch darüber hinaus: Auch logistische Prozesse (z.B. eine bedarfsgerechte Planung von Verkehrsmitteln im Öffentlichen Personennahverkehr) basieren auf den Grundlagen von Algorithmen und Datenstrukturen.\\

Weiterführende Kurse laufen unter Namen wie Algorithmendesign, Algorithmik oder auch Graphentheorie.